%%%%%%%%%%%%%%%%%%%%%%%%%%%%%%%%%%%%%%%%%%%%%%%%%%%%%%%%%%%%%%%%%%%%%%%%%%%%%%%%
%   Copyright 2009, Oracle and/or its affiliates.
%   All rights reserved.
%
%
%   Use is subject to license terms.
%
%   This distribution may include materials developed by third parties.
%
%%%%%%%%%%%%%%%%%%%%%%%%%%%%%%%%%%%%%%%%%%%%%%%%%%%%%%%%%%%%%%%%%%%%%%%%%%%%%%%%

\chapter{Tests and Properties}
\chaplabel{tests}

\note{Tests and properties are not yet supported.

Examples in this chapter are not tested nor run by the interpreter.}

\note{
Ticket \#248

By default, tests actually in a given component are run.
Do not run the run function while running tests.
Overloading the same name with test and non-test functionals is not
allowed.
To test multiple components:

{\small\tt fortress test blah1.fss blah2.fss ...}
}

\note{Introduce another keyword \KWD{testhelper}
for auxiliary functions in tests.
The \KWD{testhelper} codes need to be in APIs.}

\note{
A limited form of tests are supported by the interpreter as follows:

To help make programs more robust,
the \KWD{test} modifier can appear on a top-level function definition
with type ``\EXP{() \rightarrow ()}''.
Functions with modifier \KWD{test} must not be overloaded with
functions that do not have modifier \KWD{test}.
The collection of all functions in a program that include modifier \KWD{test}
are referred to collectively as the program's \emph{tests}.
Tests can refer to non-tests but
it is a static error for a non-test to refer to any test.
Tests can refer to non-tests but non-tests must not refer to any test.

For example, we can write the following test function:
\input{\home/basic/examples/TestFactorial.tex}


When a program's tests are \emph{run},
each top-level function definition with modifier \KWD{test}
is run in textual program order.
}

Fortress supports automated program testing.
Components and APIs can declare \emph{tests}.
Test declarations specify explicit finite collections over which the
test is run.
Components, APIs, traits, and objects can declare \emph{properties}
which describe boolean conditions that the enclosing construct is expected
to obey.  Tests and properties may modify the program state.

\section{The Purpose of Tests and Properties}
To help make programs more robust, Fortress programs are allowed
to include special constructs called
\emph{tests} and \emph{properties}.  Tests consist of test data
along with code that can be run on that data.
Properties are documentation used to describe the behavior
of the traits and functions of a program; they can be thought of
as comments written in a formal language.
For each property, there is a special function that can be
called by a program's tests to ensure that the property holds for specific
test data.


A fortress includes hooks to allow programmers to run a specific test
on an executable component, and to run all tests on such a component.
A particularly useful time to run
the tests of an executable component is at component link time;
errors in the behavior of constituent components can be caught
before the linked program is run.


\section{Test Declarations}
\seclabel{test-decls}

\begin{Grammar}
\emph{TestDecl} &::=& \KWD{test} \emph{Id} \texttt[\emph{GeneratorClauseList}\texttt]
\EXP{=} \emph{Expr} \\

\emph{GeneratorClauseList} &::=& \emph{GeneratorBinding}(\EXP{,} \emph{GeneratorClause})$^*$\\

\emph{GeneratorBinding}
&::=& \emph{BindIdOrBindIdTuple}\EXP{\leftarrow}\emph{Expr} \\

\emph{GeneratorClause}
&::=& \emph{GeneratorBinding}\\
&$|$& \emph{Expr} \\

\emph{BindIdOrBindIdTuple}
&::=& \emph{BindId}\\
&$|$& \texttt{(} \emph{BindId}\EXP{,} \emph{BindIdList} \texttt{)}\\

\end{Grammar}

A test declaration begins with
\KWD{test} followed by an identifier, a list of zero or more generators
(described in \secref{generators}) enclosed in square brackets, the
token \EXP{=}, and a subexpression.
When a test is \emph{run} (See \secref{basicops}), the
subexpression is evaluated in each extension of the enclosing environment
corresponding to each point in the cross product
of bindings determined by the generators in the generator list.

For example, the following test:
\note{The fortify has some problems here...}
% test fxLessThanFy[x <- E, y <- F] = assert (f(x) < f(y))
\begin{Fortress}
{\tt~}\pushtabs\=\+\( \KWD{test} \VAR{fxLessThanFy}[{x \leftarrow E}, {y \leftarrow F}] = \VAR{assert} (f(x) < f(y))\)\-\\\poptabs
\end{Fortress}
checks that, for each value $x$ supplied by generator $E$,
and for each value $y$ supplied by generator $F$,
\EXP{f(x)} is less than \EXP{f(y)}.


\section{Other Test Constructs}
The \KWD{test} modifier can also appear
on a function definition, trait definition, object definition,
or top-level variable definition. In these contexts, the modifier indicates
that the program construct it modifies can be referred to by a test.
Functions with modifier \KWD{test} must not be overloaded with
functions that do not have modifier \KWD{test}, and
traits with modifier \KWD{test} must not be extended
by traits or objects that do not have modifier \KWD{test}.
The collection of all constructs in a program that include modifier \KWD{test}
are referred to collectively as the program's \emph{tests}.
Tests can refer to non-tests but
it is a static error for a non-test to refer to any test.

For example, we can write the following helper function:
% test ensureApplicationFails(g,x) = do
%   applicationSucceeded := false
%   try
%     g(x)
%     applicationSucceeded := true
%   catch e
%     Exception => ()
%   end
%   if applicationSucceeded then
%     fail "Application succeeded!"
%   end
% end
\begin{Fortress}
{\tt~}\pushtabs\=\+\( \KWD{test} \VAR{ensureApplicationFails}(g,x) = \;\KWD{do}\)\\
{\tt~~}\pushtabs\=\+\(   \VAR{applicationSucceeded} \ASSIGN \VAR{false}\)\\
\(   \KWD{try}\)\\
{\tt~~}\pushtabs\=\+\(     g(x)\)\\
\(     \VAR{applicationSucceeded} \ASSIGN \VAR{true}\)\-\\\poptabs
\(   \KWD{catch} e\)\\
{\tt~~}\pushtabs\=\+\(     \TYP{Exception} \Rightarrow ()\)\-\\\poptabs
\(   \KWD{end}\)\\
\(   \KWD{if} \VAR{applicationSucceeded} \KWD{then}\)\\
{\tt~~}\pushtabs\=\+\(     \VAR{fail}\;\hbox{\rm``\STR{Application~succeeded!}''}\)\-\\\poptabs
\(   \KWD{end}\)\-\\\poptabs
\( \KWD{end}\)\-\\\poptabs
\end{Fortress}

The library function \VAR{fail} displays the error message provided
and terminates execution of the enclosing test.

\section{Running Tests}

When a program's tests are \emph{run}, the following actions are
taken:
\begin{enumerate}
\item All top-level
definitions, including constructs beginning with modifier \KWD{test}, are
initialized. A test declaration with name $t$ declares a function named $t$
that takes a tuple of parameters corresponding to the variables bound in
the generator list of the test declaration.
For each variable $v$ in the generator list
of $t$, if the type of generator supplied for $v$ is
\EXP{\TYP{Generator}\llbracket\OPR{\alpha}\rrbracket}
then the parameter in the function corresponding to $v$ has type
\EXP{\mathrm{\alpha}}. The return type of the function is \EXP{()}.

Each such function bound in this manner is referred to as a
\emph{test function}.
Test functions can be called from the rest of the program's tests.

\item Each test $t$ in a program is run in each extension of
$t$'s enclosing environment with a point in the cross product
of bindings determined by the generators in the test's generator list.
\end{enumerate}


\section{Test Suites}
\seclabel{test-suites}

In order to allow programmers to run strict subsets of all tests
defined in a program, Fortress allows tests to be
assembled into \emph{test suites}.

The convenience object \TYP{TestSuite} is defined in \library.
An instance of this object contains a set of test functions that can all
be called by invoking the method \VAR{run}:

% test object TestSuite(testFunctions = {})
%
%   add(f: () -> ()) = testFunctions.insert(f)
%
%   run() =
%     for t <- testFunctions do
%       t()
%     end
% end
\begin{Fortress}
{\tt~}\pushtabs\=\+\( \KWD{test}\;\;\KWD{object} \TYP{TestSuite}(\VAR{testFunctions} = \{\})\)\\[4pt]
{\tt~~}\pushtabs\=\+\(   \VAR{add}(f\COLON () \rightarrow ()) = \;\VAR{testFunctions}.\VAR{insert}(f)\)\\[4pt]
\(   \VAR{run}() =\)\\
{\tt~~}\pushtabs\=\+\(     \KWD{for} t \leftarrow \VAR{testFunctions}\;\;\KWD{do}\)\\
{\tt~~}\pushtabs\=\+\(       t()\)\-\\\poptabs
\(     \KWD{end}\)\-\-\\\poptabs\poptabs
\( \KWD{end}\)\-\\\poptabs
\end{Fortress}

Note that all tests in a \TYP{TestSuite} are run in parallel.

\section{Property Declarations}
\seclabel{property-decls}

\begin{Grammar}
\emph{PropertyDecl} &::=& \KWD{property} \options{\emph{Id} \EXP{=}}
\options{$\forall$ \emph{ValParam}} \emph{Expr} \\

\emph{ValParam} &::=& \emph{BindId}\\
&$|$& \texttt(\option{\emph{Params}}\texttt)\\

\emph{Params}
&::=& (\emph{Param}\EXP{,})$^*$ \options{\emph{Varargs}\EXP{,}} \emph{Keyword}(\EXP{,}\emph{Keyword})$^*$\\
&$|$& (\emph{Param}\EXP{,})$^*$  \emph{Varargs}\\
&$|$& \emph{Param}(\EXP{,} \emph{Param})$^*$\\

\emph{VarargsParam} &::=& \emph{BindId}\EXP{\COLONOP}\emph{Type}\EXP{...} \\

\emph{Varargs} &::=& \emph{VarargsParam}\\

\emph{Keyword} &::=& \emph{Param}\EXP{=}\emph{Expr} \\

\emph{PlainParam} &::=& \emph{BindId} \option{\emph{IsType}} \\
&$|$& \emph{Type} \\

\emph{Param} &::=& \emph{PlainParam}\\

\end{Grammar}

Components and APIs may include \emph{property}
declarations, documenting
boolean conditions that a program is expected to obey.
Syntactically, a property declaration begins with
\KWD{property} followed by an optional identifier followed by the token
\EXP{=}, an optional value parameter, which may be a tuple, preceded by the
token \EXP{\forall}, and a boolean subexpression.
In any execution of the program, the boolean subexpression is expected to
evaluate to true at any time for any binding of the property declaration's
parameter to any value of its declared type.
When a property declaration includes an identifier,
the property identifier is bound to a function whose parameter
and body are that of the property, and whose return type is \TYP{Boolean}.
A function bound in this manner is referred to as a \emph{property function}.


Properties can also be declared in trait or object declarations. Such
properties are expected to hold for all instances of the trait or object
and for all bindings of the property's parameter. If a property in a trait
or object includes a name, the name is bound to a method whose parameter
and body are that of the property, and whose return type is
\TYP{Boolean}.
A method bound in this manner is referred to as a \emph{property
  method}. A property method of a trait $T$ can be called,
via dotted method notation, on an instance of $T$.

Property functions and methods can be referred to in a program's tests.
If the result of a call to a property function or method is not
\VAR{true}, a \TYP{TestFailure} exception is thrown. For example, we can
write:
% property fIsMonotonic = FORALL(x: ZZ, y: ZZ) (x < y) -> (f(x) < f(y))
\begin{Fortress}
{\tt~}\pushtabs\=\+\( \KWD{property} \TYP{fIsMonotonic} = \forall(x\COLON \mathbb{Z}, y\COLON \mathbb{Z})\; (x < y) \rightarrow (f(x) < f(y))\)\-\\\poptabs
\end{Fortress}
% test s : Set[\ZZ\] = {-2000, 0, 1, 7, 42, 59, 100, 1000, 5697}
% test fIsMonotonicOverS [x <- s, y <- s] = fIsMonotonic(x,y)
% test fIsMonotonicHairy [x <- s, y <- s] = fIsMonotonic(x, x^2 + y)
\begin{Fortress}
{\tt~}\pushtabs\=\+\( \KWD{test} s \mathrel{\mathtt{:}} \TYP{Set}\llbracket\mathbb{Z}\rrbracket = \{-2000, 0, 1, 7, 42, 59, 100, 1000, 5697\}\)\\
\( \KWD{test} \VAR{fIsMonotonicOverS} [x \leftarrow s, y \leftarrow s] = \VAR{fIsMonotonic}(x,y)\)\\
\( \KWD{test} \VAR{fIsMonotonicHairy} [x \leftarrow s, y \leftarrow s] = \VAR{fIsMonotonic}(x, x^{2} + y)\)\-\\\poptabs
\end{Fortress}
The test \VAR{fIsMonotonicOverS} tests that function \VAR{f} is
monotonic
over all values in set \VAR{s}. The test \VAR{fIsMonotonicHairy}
tests that \VAR{f} is monotonic with respect to the values in \VAR{s}
compared to the set of values corresponding to all the ways in which
we can choose an element of \VAR{s}, square it,
and add it to another element of \VAR{s}.
