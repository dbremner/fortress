%%%%%%%%%%%%%%%%%%%%%%%%%%%%%%%%%%%%%%%%%%%%%%%%%%%%%%%%%%%%%%%%%%%%%%%%%%%%%%%%
%   Copyright 2009, Oracle and/or its affiliates.
%   All rights reserved.
%
%
%   Use is subject to license terms.
%
%   This distribution may include materials developed by third parties.
%
%%%%%%%%%%%%%%%%%%%%%%%%%%%%%%%%%%%%%%%%%%%%%%%%%%%%%%%%%%%%%%%%%%%%%%%%%%%%%%%%

\subsection{Reduction Variables}
\seclabel{reduction-vars}

\note{Reduction variables are not yet supported.}

To perform computations as locally as possible, and avoid the need to
synchronize in the middle of relatively simple \KWD{for} loops,
Fortress gives special treatment to \emph{reductions}.  We say that an
operator \EXP{\odot} is a \emph{reduction operator} for type $T$ if
$T$ is a subtype of
%Monoid[\T, ODOT\]
\(\TYP{Monoid}\llbracket{}T, \odot\rrbracket\), which implies that
\EXP{\odot} is an associative binary infix operator on $T$ (see
\secref{monoids-groups-rings-fields} for details about the
\TYP{Monoid} trait).

We say that a variable \VAR{l} is a \emph{reduction variable} reduced using
the reduction operator $\odot$ for a particular thread group if it
satisfies the following conditions:
\begin{itemize}
\item Every assignment to \VAR{l} within the thread group is of the
  form \EXP{l \mathrel{\oplus}= e}, where exactly one operator $\oplus$ or its
  group inverse $\ominus$ (see below) is used in these assignments.
\item The value of \VAR{l} is not otherwise read within the thread group.
\item The variable \VAR{l} is not a free variable of a functional.
  This includes the fields of the receiver object in a method
  definition.
\end{itemize}
Other threads which simultaneously reference a reduction variable
while a loop is running may see an arbitrary value for that variable.
Any updates performed by those threads may be lost.  The association
of terms in the reduction is arbitrary and guided by the loop
generators (\secref{defining-generators}).  The order of terms in the
reduction is the natural order of elements in the generator.  When the
operator is commutative, the order of terms is also arbitrary.

Several common mathematical operators are declared to be monoids in
\library.  These include \EXP{+}, $\cdot$, \EXP{\wedge}, \EXP{\vee},
and \EXP{\xor}.  If a type $T$ extends
%Group[\T, OPLUS, OMINUS\]
\EXP{\TYP{Group}\llbracket{}T, \oplus, \ominus\rrbracket}
(see \secref{monoids-groups-rings-fields} for details about the
\TYP{Group} trait)
then reduction expressions of the following form are also permitted:
%x OMINUS= y
\begin{Fortress}
\(x \mathrel{\ominus}= y\)
\end{Fortress}
Such expressions may be transformed into an algebraic equivalent such
as the following:
%x OPLUS= Identity[\OPLUS\] OMINUS y
\begin{Fortress}
\(x \mathrel{\oplus}= \TYP{Identity}\llbracket\oplus\rrbracket \ominus y\)
\end{Fortress}

The semantics of reductions enables implementation strategies such as
the one used in OpenMP~\cite{OpenMP}: A reduction variable \VAR{l} is
assigned \EXP{\TYP{Identity}\llbracket\oplus\rrbracket}
at the beginning of each iteration.  The
original variable value may be read ahead of time, resulting in the
loss of parallel updates to the variable which occur in other threads
while the loop is running.  When all iterations are complete, the
initial value of the reduction variable and values of the variable at
the end of each implicit thread are reduced and the result is assigned
to the reduction variable as the group completes.

In the following example, \VAR{sum} is a reduction variable:
\note{I manually rendered the array type.}
%arraySum[\nat x\](a:ZZ64[x]):ZZ64 = do
%    sum:ZZ64 := 0
%    for i<-a.indices do
%        sum+=a[i]
%    end
%    sum
%end
\begin{Fortress}
\(\VAR{arraySum}\llbracket\KWD{nat}\mskip 4mu plus 4mu{x}\rrbracket(a\COLONOP\mathbb{Z}64[x])\COLONOP\mathbb{Z}64 = \;\KWD{do}\)\\
{\tt~~~~}\pushtabs\=\+\(    \VAR{sum}\COLONOP\mathbb{Z}64 \ASSIGN 0\)\\
\(    \KWD{for}  \null\)\pushtabs\=\+\(i\leftarrow{}a.\VAR{indices} \KWD{do}\)\\
\(        \VAR{sum}\mathrel{+}=a[i]\)\-\\\poptabs
\(    \KWD{end}\)\\
\(    \VAR{sum}\)\-\\\poptabs
\(\KWD{end}\)
\end{Fortress}

The following demonstrates the use of a reduction variable \VAR{prod}
in a parallel \KWD{do}:
\note{The $\mathrel{\cdot}$ is manually edited.}
%prod : ZZ64 := 1
%do
%    prod DOT= g(y)
%also do
%    prod DOT= f(x)
%    prod DOT= g(x)
%also do
%    h(x,y)
%end
\begin{Fortress}
\(\VAR{prod} \mathrel{\mathtt{:}}\mskip 4mu plus 4mu\mathbb{Z}64 \ASSIGN 1\)\\
\(\KWD{do}\)\\
{\tt~~~~}\pushtabs\=\+\(    \VAR{prod} \mathrel{\cdot}= g(y)\)\-\\\poptabs
\(\KWD{also}\;\;\KWD{do}\)\\
{\tt~~~~}\pushtabs\=\+\(    \VAR{prod} \mathrel{\cdot}= f(x)\)\\
\(    \VAR{prod} \mathrel{\cdot}= g(x)\)\-\\\poptabs
\(\KWD{also}\;\;\KWD{do}\)\\
{\tt~~~~}\pushtabs\=\+\(    h(x,y)\)\-\\\poptabs
\(\KWD{end}\)
\end{Fortress}

The operator associated with a reduction variable can be invoked in
several places during evaluation of a thread group.  Most obviously,
it can be invoked anywhere the reduction operation occurs textually in
the code for that group.  The operator can also be invoked multiple
times implicitly and in parallel after the completion of some or all
of the threads in the group.  The group as a whole does not complete
normally until all these operator invocations have completed normally
and the reduced result has been assigned to the reduction variable for
use in subsequent computation.  If any operator invocation completes
abruptly, it is treated just as if an ordinary implicit thread in the
group completed abruptly.
