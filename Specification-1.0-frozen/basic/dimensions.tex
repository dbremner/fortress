%%%%%%%%%%%%%%%%%%%%%%%%%%%%%%%%%%%%%%%%%%%%%%%%%%%%%%%%%%%%%%%%%%%%%%%%%%%%%%%%
%   Copyright 2009, Oracle and/or its affiliates.
%   All rights reserved.
%
%
%   Use is subject to license terms.
%
%   This distribution may include materials developed by third parties.
%
%%%%%%%%%%%%%%%%%%%%%%%%%%%%%%%%%%%%%%%%%%%%%%%%%%%%%%%%%%%%%%%%%%%%%%%%%%%%%%%%

\chapter{Dimensions and Units}
\chaplabel{dimunits}

\note{Dimensions and units are not yet supported.

The examples in this chapter are not tested nor run by the interpreter.}


\begin{Grammar}
\emph{TypeRef} :&:=& \emph{DimType}\\

\emph{DimType} &::=& \emph{DimRef} \\
&$|$& \emph{TypeRef} \emph{DimRef}\\
&$|$& \emph{TypeRef} $\csdot$ \emph{DimRef} \\
&$|$& \emph{TypeRef} \EXP{/} \emph{DimRef}\\
&$|$& \emph{TypeRef} \TYP{per} \emph{DimRef}\\
&$|$& \emph{TypeRef} \emph{UnitRef}\\
&$|$& \emph{TypeRef} $\csdot$ \emph{UnitRef} \\
&$|$& \emph{TypeRef} \EXP{/} \emph{UnitRef}\\
&$|$& \emph{TypeRef} \TYP{per} \emph{UnitRef}\\
&$|$& \emph{TypeRef} \KWD{in} \emph{DimRef}\\

\emph{DimRef} &::=& \emph{StaticArg} \\

\emph{StaticArg} &::=& \TYP{Unity} \\
&$|$& \TYP{dimensionless}\\
&$|$& \emph{StaticArg} $\csdot$ \emph{StaticArg} \\
&$|$& \emph{StaticArg} \emph{StaticArg} \\
&$|$& \emph{StaticArg} \EXP{/} \emph{StaticArg}\\
&$|$& \EXP{1}\EXP{/}\emph{StaticArg}\\
&$|$& \emph{StaticArg} \texttt{\^} \emph{StaticArg}\\
&$|$& \emph{StaticArg} \TYP{per} \emph{StaticArg}\\
&$|$& \emph{DUPreOp} \emph{StaticArg}\\
&$|$& \emph{StaticArg} \emph{DUPostOp}\\
&$|$& \emph{TypeRef} \\
&$|$& \texttt{(}\emph{StaticArg}\texttt{)} \\

\emph{DUPreOp} &::=&\TYP{square} \\
&$|$& \TYP{cubic} \\
&$|$& \TYP{inverse}\\

\emph{DUPostOp}&::=&\TYP{squared} \\
&$|$& \TYP{cubed}\\

\end{Grammar}

\begin{GrammarTwo}
\emph{Expr} &::=& \emph{UnitExpr}\\

\emph{UnitExpr} &::=& \emph{UnitRef}\\
&$|$& \emph{Expr} \emph{UnitRef}\\
&$|$& \emph{Expr} $\csdot$ \emph{UnitRef}\\
&$|$& \emph{Expr} \EXP{/} \emph{UnitRef}\\
&$|$& \emph{Expr} \TYP{per} \emph{UnitRef}\\
&$|$& \emph{Expr} \KWD{in} \emph{UnitRef}\\

\emph{UnitRef} &::=& \emph{StaticArg} \\

\end{GrammarTwo}

There are special type-like constructs called \emph{dimensions} that are
separate from other types (described in \chapref{types}).
There are also special constructs that modify types and values called
\emph{units} that are instances of dimensions.
These are used to describe physical quantities.

\Library\ define dimensions and units for the standard SI system of measurement
based on meters, kilograms, seconds, amperes, and so on (as described in
\secref{lib:siunits}).   \Library\ also provide
supplemental units of measurement, such as feet and miles (as described in
\secref{lib:englishunits}).  For example,
\library\ provide a dimension named \TYP{Length} whose default unit is named
\TYP{meter} and abbreviated \verb$m_$.
By rendering convention, this abbreviation is rendered in roman type
without the underscore: \EXP{\mathrm{m}}. In contrast, the variable \txt{m} is rendered as in standard
mathematical notation: \VAR{m}.
See \secref{rendering} for a discussion of formatting conventions for tokens.

For readability, plural forms of the unit names are defined as equivalent to the
corresponding singular forms; thus one can write \EXP{\TYP{meters} \mathbin{\TYP{per}} \TYP{second}},
for example.
Standard SI prefixes may be used on both the name and the symbol, so that \TYP{nanometer}
and \EXP{\mathrm{nm}} are also units of the dimension named \TYP{Length}, related to \TYP{meter}
and \EXP{\mathrm{m}} by a conversion factor of $10^{-9}$.

Every dimension may have a default unit that is used for representing
values of quantities of that dimension if no unit is specified explicitly.
\Library\ define these default dimensions and units:

\begin{tabular}{ll@{\hspace{2\tabcolsep}}ll}
\TYP{Length} & \TYP{meter} & \TYP{MagneticFlux} & \TYP{weber} \\
\TYP{Mass} & \TYP{kilogram} & \TYP{MagneticFluxDensity} & \TYP{tesla} \\
\TYP{Time} & \TYP{second} & \TYP{Inductance} & \TYP{henry} \\
\TYP{Frequency} & \TYP{hertz} & \TYP{Velocity} & \EXP{\TYP{meters} \mathbin{\TYP{per}} \TYP{second}} \\
\TYP{Force} & \TYP{newton} & \TYP{Acceleration} & \EXP{\TYP{meters} \mathbin{\TYP{per}} \TYP{second}\mskip 4mu plus 4mu\TYP{squared}} \\
\TYP{Pressure} & \TYP{pascal} & \TYP{Angle} & \TYP{radian} \\
\TYP{Energy} & \TYP{joule} & \TYP{SolidAngle} & \TYP{steradian} \\
\TYP{Power} & \TYP{watt} & \TYP{LuminousIntensity} & \TYP{candela} \\
\TYP{Temperature} & \TYP{kelvin} & \TYP{LuminousFlux} & \TYP{lumen} \\
\TYP{Area} & \EXP{\mathbin{\TYP{square}}\mskip 4mu plus 4mu\TYP{meter}} & \TYP{Illuminance} & \TYP{lux} \\
\TYP{Volume} & \EXP{\mathbin{\TYP{cubic}}\mskip 4mu plus 4mu\TYP{meter}} & \TYP{RadionuclideActivity} & \TYP{becquerel} \\
\TYP{ElectricCurrent} & \TYP{ampere} & \TYP{AbsorbedDose} & \TYP{gray} \\
\TYP{ElectricCharge} & \TYP{coulomb} & \TYP{DoseEquivalent} & \VAR{sievert} \\
\TYP{ElectricPotential} & \TYP{volt} & \TYP{AmountOfSubstance} & \TYP{mole} \\
\TYP{Capacitance} & \TYP{farad} & \TYP{CatalyticActivity} & \TYP{katal} \\
\TYP{Resistance} & \TYP{ohm} & \TYP{MassDensity} & \EXP{\TYP{kilograms} \mathbin{\TYP{per}} \mathbin{\TYP{cubic}}\mskip 4mu plus 4mu\TYP{meter}} \\
\TYP{Conductance} & \TYP{siemens} \\
\end{tabular}


In addition, \library\ define the dimension \TYP{Information} with
units \TYP{bit} and \TYP{byte}
(and the plurals \TYP{bits} and \TYP{bytes}), a \TYP{byte} being equal to 8
\TYP{bits}.
To avoid confusion, SI prefixes are \emph{not} provided for these units;
instead, programmers must
use appropriate powers of 2 or 10, for example \EXP{10^{6}
  \TYP{bits}} or \EXP{2^{20} \TYP{bits}}.

Here are some examples of the use of dimensions and units:
%x: RR64 Length = 1.3 m_
%t: RR64 Time = 5 s_
%v: RR64 Velocity = x / t
%w: RR64 Velocity in nm_/s_ = 17 nm_/s_
%x: RR64 Velocity in furlongs per fortnight = v in furlongs per fortnight
\begin{Fortress}
\(x\COLON \mathbb{R}64\,\TYP{Length} = 1.3\,\mathrm{m}\)\\
\(t\COLON \mathbb{R}64\,\TYP{Time} = 5\,\mathrm{s}\)\\
\(v\COLON \mathbb{R}64\,\TYP{Velocity} = x / t\)\\
\(w\COLON \mathbb{R}64\,\TYP{Velocity}\, \KWD{in}\, \mathrm{nm}/\mathrm{s} = 17\,\mathrm{nm}/\mathrm{s}\)\\
\(x\COLON \mathbb{R}64\,\TYP{Velocity}\, \KWD{in}\, \TYP{furlongs} \mathbin{\TYP{per}} \TYP{fortnight} = v\, \KWD{in}\, \TYP{furlongs} \mathbin{\TYP{per}} \TYP{fortnight}\)
\end{Fortress}



Dimensions and units can be multiplied, divided, and raised to rational powers to produce new dimensions and units. Both the numerator and the denominator of a rational power of a dimension or a unit must be a valid \emph{\KWD{nat} parameter} instantiation
(as described in \secref{natparams}).
Multiplication can be expressed using juxtaposition or \EXP{\cdot};
division can be expressed using \EXP{/} or \TYP{per}.
The syntactic operators \TYP{square} and \TYP{cubic} may be applied to a dimension or unit in order to raise
it to the second power, third power, respectively; the special postfix syntactic operators \TYP{squared}
and \TYP{cubed} may be used for the same purpose.
The syntactic operator \TYP{inverse} may be applied to a dimension or unit to divide it into 1.
All of these syntactic operators are merely syntactic sugar, expanded before type checking.

\note{I manually added a space between inverse and ohms below.}
%grams per cubic centimeter
%meter per second squared
%inverse ohms
\begin{Fortress}
\(\TYP{grams} \mathbin{\TYP{per}} \mathbin{\TYP{cubic}}\mskip 4mu plus 4mu\TYP{centimeter} \)\\
\(\TYP{meter} \mathbin{\TYP{per}} \TYP{second}\mskip 4mu plus 4mu\TYP{squared} \)\\
\(\mathbin{\TYP{inverse}}\ \TYP{ohms}\)
\end{Fortress}

One can also write \EXP{1/X} as a synonym for \EXP{X^{-1}} if
\VAR{X} is either a dimension or a unit.

Most numerical values in Fortress are dimensionless quantity values.
Multiplying or dividing a dimensionless value by a unit produces a dimensioned value;
thus \EXP{5\,\mathrm{s}} is the dimensioned value equal to five seconds, which has
numerical value \EXP{5} and \TYP{second} for its unit.
A dimensioned value may also be
multiplied or divided by a unit, and the result is to combine the units;
the expression
%% \EXP{17\,\mathrm{nm}/\mathrm{s}}
\EXP{(17\,\mathrm{nm})/\mathrm{s}}
first multiplies \EXP{17} by
\EXP{\mathrm{nm}} to produce the dimensioned value seventeen nanometers,
which is then divided by the unit \EXP{\mathrm{s}}
to produce seventeen nanometers per second.

The unit of a dimensioned value may be changed to another unit of the same dimension
by using the \KWD{in} operator, which takes a quantity to its left and a unit to its right.
The \KWD{in} operator changes the unit and multiplies or divides the numerical value
by an appropriate conversion constant so as to preserve the overall dimensioned value.
Thus \EXP{1.3\,\mathrm{m}\, \KWD{in}\, \mathrm{nm}} produces \EXP{1 300 000 000\,\mathrm{nm}}.

Multiplying or dividing a dimensionless numerical type by a unit produces an equivalent
dimensioned numerical type with that unit associated; thus \EXP{\mathbb{R}64\,\TYP{meter}} is a
type that is just like \EXP{\mathbb{R}64} but whose values are values of dimension
\TYP{Length} measured in \TYP{meters}.  A dimensioned numerical type may be
further multiplied or divided by a unit.
As a convenience,
if a dimension has a default unit, a numerical type may
also be multiplied or divided by a dimension, in which case the result is as if
the default unit for that dimension had been used.
The \KWD{in} operator may also
be used to change the unit associated with a dimensioned type; in this situation
the effect is merely to alter the unit associated with the type; no numerical
operation is performed at run time.

Certain aggregate types, such as \TYP{Vector}, may also have associated units.

There are two reasons for using dimensions and units.  One is that the \KWD{in}
operator can supply conversion factors automatically.  The other is that
certain programming errors may be detected at compile time.
When dimensioned values are added, subtracted, or compared,
it is a static error if the units do not match.
When dimensioned values are multiplied or divided, their units
are multiplied or divided.  When taking the square root of a dimensioned
value, the unit of the result is the square root of the argument's unit.
Other numerical functions, such as \EXP{\sin} and \EXP{\log}, require
dimensionless arguments.

A variable whose declared
type includes a dimension without an accompanying unit
is understood to have the default unit for that dimension.
Thus, in most cases, the runtime unit of an expression can be statically
inferred. However, there are exceptions. For example,
consider the following declaration:
\note{I manually edited the array type.}
% a : Object[3] = [5 mg_, 3 m_, 4 s_]
\begin{Fortress}
{\tt~}\pushtabs\=\+\( a \mathrel{\mathtt{:}} \TYP{Object}[3] = [5\,\mathrm{mg}, 3\,\mathrm{m}, 4\,\mathrm{s}]\)\-\\\poptabs
\end{Fortress}

Now suppose we declare a function that takes an array of objects and returns
one of its elements:
\note{I manually edited the array type.}
% f(xs:Object[3]) = xs[1]
\begin{Fortress}
{\tt~}\pushtabs\=\+\( f(\VAR{xs}\COLONOP\TYP{Object}[3]) = \VAR{xs}_{1}\)\-\\\poptabs
\end{Fortress}

The value of the call \EXP{f(a)} is \EXP{3\,\mathrm{m}}. However, the static
type of  \EXP{f(a)} is simply \TYP{Object}. When the value \EXP{3\,\mathrm{m}}
is placed in an array of objects, the value is boxed, and the unit associated
with the value must be included as part of the boxed value. However, unboxed
values need not include unit information at runtime, as this information is
statically evident in such cases.

\note{Guy's explanation ---
The thing to understand is that a measurement is a value object,
but storing it in an \EXP{\TYP{Array}\llbracket\TYP{Object}\rrbracket} will force the value object
to be boxed.  Boxing causes type information to be associated
with the value.  Unit information is not really ``erased'' at compile
time; rather, unit information is ``static'', that is, part of the type,
and so the unit information can be omitted from a run-time
representation exactly when type information can be omitted from
a run-time representation.  The situation is exactly the same as for
dimensionless numerical types such as ints and floats.}


There are also special static parameters for units and dimensions; see
\secref{dimunitparams}.
