%%%%%%%%%%%%%%%%%%%%%%%%%%%%%%%%%%%%%%%%%%%%%%%%%%%%%%%%%%%%%%%%%%%%%%%%%%%%%%%%
%   Copyright 2009, Oracle and/or its affiliates.
%   All rights reserved.
%
%
%   Use is subject to license terms.
%
%   This distribution may include materials developed by third parties.
%
%%%%%%%%%%%%%%%%%%%%%%%%%%%%%%%%%%%%%%%%%%%%%%%%%%%%%%%%%%%%%%%%%%%%%%%%%%%%%%%%

\section{Type Inference for Components}
\seclabel{type-inference-components}

Type inference for Fortress has been described as a procedure performed
over a whole Fortress program in \chapref{type-inference}. In this
section, we explain how this procedure can 
be adapted to perform type inference over a simple program component. 
For a compound component,
concatenate all declarations of all constituents in
the order specified by the constructing \shellcommand{link}
operation.  Constituent compound component declarations are
recursively concatenated.


Type inference over a simple component $C$ is performed by first
expanding $C$ into a self-contained Fortress program, as follows:
\begin{enumerate}
\item All program constructs corresponding to declarations in APIs exported by
$C$ are expanded so that they include all types and static parameters
included in the exported APIs. 
\item All types provided by all declarations in the APIs imported by $C$
are prepended to $C$. Note that these declarations must include
types for all variables, functions, fields, and methods;
otherwise the APIs that declare them are not well-formed.
\item In order for the resulting expanded program to be well-formed, 
we assume that all declarations in these APIs are
expanded into special definitions that include \emph{empty bodies}.
The empty body of such a definition is a conceptual body which
cannot be expressed directly in Fortress programs.
Because declarations in APIs do not have any elided type,
type inference ignores empty bodies.
\end{enumerate}
Once $C$ is expanded, type inference 
is performed over all program constructs that still include elided types.
Empty bodies are ignored.
