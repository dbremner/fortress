%%%%%%%%%%%%%%%%%%%%%%%%%%%%%%%%%%%%%%%%%%%%%%%%%%%%%%%%%%%%%%%%%%%%%%%%%%%%%%%%
%   Copyright 2009, Oracle and/or its affiliates.
%   All rights reserved.
%
%
%   Use is subject to license terms.
%
%   This distribution may include materials developed by third parties.
%
%%%%%%%%%%%%%%%%%%%%%%%%%%%%%%%%%%%%%%%%%%%%%%%%%%%%%%%%%%%%%%%%%%%%%%%%%%%%%%%%

 \section{Advanced Features of Fortress Operations}
\seclabel{advancedops}

The system we have described thus far
provides much of the desired functionality of a component system.
However it has a few significant weaknesses:
\begin{enumerate}

\item
It exposes to everyone all the \apisN\ used
in the development of a project.

\item
By allowing access to these \apisN,
it inhibits significant cross-component optimization.

\item
It prevents components that use two different implementations of the same
\apiN\ from being linked,
even if they never actually pass references to that \apiN\
between each other.

\item
It restricts the upgradability of compound components,
as described earlier.

\end{enumerate}

We can mitigate all these shortcomings
by providing two simple operations,
\shellcommand{hide} and \shellcommand{constrain}.
Informally,
\shellcommand{hide} makes \apisN\ no longer visible from outside the component
so that they cannot be upgraded,
and \shellcommand{constrain} merely prevents them from being exported.
An \apiN\ that is constrained but not hidden
can still be upgraded.
There are other subtle consequences of this distinction,
which we discuss as they arise.

Some of the properties about the \apisN\ exported by a component
discussed in \secref{basicops}
are actually properties of \apisN\ that are visible or provided by a component.
For example,
\apisN\ visible in a component cannot be imported by that component,
even if they are not exported.
Other properties are really properties only of the exported \apisN.
Most importantly,
components that do not export any common \apisN\ can be linked,
as can components that share only visible \apisN.

\paragraph{Constrain}

A \shellcommand{constrain} operation takes
a component name of an installed component,
a new component name, and a set of \apisN,
and produces a new component that does not export any of the \apisN\ specified.
Syntactically, we write:

%constrain(source:String, destination = source, apis:Set[\String\]):()
{\small\verb+constrain(source:String, destination = source, apis:Set[\String\]):()+}

If no \shellcommand{destination} name is provided, the name of the \shellcommand{source}
is used.

The set of \apisN\ provided must be a subset of the \apisN\ exported
by the component.
Also, recall that every \apiN\ used by an \apiN\ exported by a component
must be imported or exported by that component.
Thus, if we constrain an \apiN\ that is used by any other \apiN\ exported
by the component,
then we must also constrain that other \apiN.

If the component is a simple component,
we first link it by itself,
and then apply \shellcommand{constrain} to the result.

\paragraph{Hide}

A \shellcommand{hide} operation is like a \shellcommand{constrain} operation,
except that the given set of \apisN\ is subtracted from
the visible and provided \apisN, along with the exported \apisN,
in the resulting component.

%hide(source:String, destination = source, apis:Set[\String\]):()
{\small\verb+hide(source:String, destination = source, apis:Set[\String\]):()+}

The requirement of \apisN\ being imported or exported whenever
an \apiN\ using them is exported
also applies to visible \apisN.
Thus, if we hide an \apiN\ used by another exported \apiN,
we must hide that other \apiN\ as well.

\paragraph{Link}

With constrained \apisN,
there is a new restriction on link:
Any \apiN\ visible in one constituent and imported by another
must be exported by some constituent.
This restriction is necessary
because an \apiN\ visible in a component cannot be imported by that component.
Thus, if one of the component's constituents imports that \apiN,
then the \apiN\ must be provided by some other constituent.
Other than that,
the \shellcommand{link} operation is largely unchanged:
the visible \apisN\ are just all the \apisN\ visible in any constituent,
and the provided \apisN\ are just those exported by any constituent.
There is a subtle additional restriction
on how linked components can be upgraded,
which we discuss below.

Rather than requiring users and developers
to call \shellcommand{constrain} and \shellcommand{hide} directly,
we provide optional parameters to the \shellcommand{link} operation
to do these operations immediately.
The \shellcommand{link} operation has the following type:

%link(result:String, constituents:String..., exports = {}, hide = {}):()
{\small\verb+link(result:String, constituents:String..., exports = {}, hide = {}):()+}

If the \shellcommand{exports} clause is present,
only those \apisN\ listed in the set following \shellcommand{exports}
are exported;
the others are constrained.
If the \shellcommand{hide} clause is present,
those \apisN\ listed in the set following \shellcommand{hide} are hidden.
An exception is thrown if the \shellcommand{exports} clause
contains any \apiN\ not exported by any constituent,
or if the \shellcommand{hide} clause contains any \apiN\ not visible in any
constituent. 

Hiding enables us to handle the rare case
in which programmers want to link multiple components
that implement the same \apiN\
without upgrading them to use the same implementation.
Before linking,
the programmer simply hides (or constrains) the \apiN\
in every component that exports it
except the one that should provide the implementation
for the new compound component.

For example, suppose we wish to link the following two components:
\begin{itemize}
\item A component \TYP{NetApp} that imports
\TYP{Fortress.IO} and exports the
\TYP{Fortress.Net} \apiN.

\item A component \TYP{EditApp} that imports
\TYP{Fortress.IO} and exports the
\TYP{Fortress.Swing.Textrf} \apiN.
\end{itemize}
We want to link these two components to use in building
an application for editing messages
and sending them over a network. But
we want to use different implementations of \TYP{Fortress.IO}
(e.g., \TYP{IOApp1} and \TYP{IOApp2} for the two components).
We simply perform the following operations:

%link(temp1, NetApp, IOApp1, exports = { Fortress.Net },          hide = { Fortress.IO })
%link(temp2, EditApp,IOApp2, exports = { Fortress.Swing.Textrf }, hide = { Fortress.IO })
%link(NetEdit, temp1, temp2)
\shellcommand{link(temp1, NetApp, IOApp1, exports = \{Fortress.Net\},          hide = \{Fortress.IO\})}\\
\shellcommand{link(temp2, EditApp,IOApp2, exports = \{Fortress.Swing.Textrf\}, hide = \{Fortress.IO\})}\\
\shellcommand{link(NetEdit, temp1, temp2)}

In this case,
the \TYP{NetEdit} component does not export,
or even make visible,
\TYP{Fortress.IO} at all.


\paragraph{Upgrade}

\begin{figure*}
\begin{center}
\input{\home/basic/components/newcrypto.eepic}
\end{center}
\caption{Upgrading with hidden \apisN: Crossed out \apisN\ are hidden.}
\figlabel{newcrypto}
\end{figure*}

For the \shellcommand{upgrade} operation,
there is no change at all in the semantics.
However,
because hiding and constraining \apisN\ allow us
to change the \apisN\ exported by a component,
it is possible to do some upgrades
that are not possible without these operations.

For example,
suppose we have a component \secio\
that exports \TYP{Fortress.IO} and \TYP{Fortress.Security},
and we want to upgrade \coolcrypto\ with \secio.
As discussed above, we cannot use \secio\ directly
because \ironlink\ exports \TYP{Fortress.IO} but not \TYP{Fortress.Security}.
We can get around this restriction by doing two upgrades,
one with \TYP{Fortress.Security}\ hidden
and the other with \TYP{Fortress.IO}\ hidden.

%hide(IOSecurity, NewIO,       {Fortress.Security})
%hide(IOSecurity, NewSecurity, {Fortress.IO})
%upgrade(CoolCryptoApp,      NewSecurity, temp1)
%upgrade(CoolCryptoApp.3.0, temp1,       NewIO)
\shellcommand{hide(IOSecurity, NewIO,       {Fortress.Security})}\\
\shellcommand{hide(IOSecurity, NewSecurity, {Fortress.IO})}\\
\shellcommand{upgrade(CoolCryptoApp,      NewSecurity, temp1)}\\
\shellcommand{upgrade(CoolCryptoApp.3.0, temp1,       NewIO)}

The resulting component is shown in \figref{newcrypto}.

The interplay between imported, exported, visible and provided \apisN\
introduces subtleties that not present in our discussion above.
In particular,
the last of the three conditions imposed for well-formedness of upgrades
is modified to state
that for any constituent that is not subsumed by a replacement component,
either it can be upgraded with the replacement,
or its \emph{visible} \apisN\ are disjoint
from the \apisN\ exported by the replacement
(i.e., it is unaffected by the upgrade).
To maintain the invariant that no two constituents export
the same \apiN,
we need another condition,
which was implied by the previous condition
when no \apisN\ were constrained or hidden:
if the replacement subsumes any constituents of the target,
then its exported \apisN\ must exactly match
the exported \apisN\ of some subset of the constituents of the target.
In practice, this restriction is rarely a problem;
in most cases, a user wishes to upgrade a target
with a new version of a single constituent component,
where the \apisN\ exported by the old and new versions
are either an exact match,
or there are new \apisN\ introduced by the new component
that have no implementation in the target.
