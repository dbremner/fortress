%%%%%%%%%%%%%%%%%%%%%%%%%%%%%%%%%%%%%%%%%%%%%%%%%%%%%%%%%%%%%%%%%%%%%%%%%%%%%%%%
%   Copyright 2009, Oracle and/or its affiliates.
%   All rights reserved.
%
%
%   Use is subject to license terms.
%
%   This distribution may include materials developed by third parties.
%
%%%%%%%%%%%%%%%%%%%%%%%%%%%%%%%%%%%%%%%%%%%%%%%%%%%%%%%%%%%%%%%%%%%%%%%%%%%%%%%%

\renewcommand{\api}{A}
\section{Basic Fortress Operations}
\seclabel{basicops}

We now describe the operations that can be performed on a fortress
by developers and end-users
for developing, installing, testing, and maintaining components.
We can think of these operations
as commands to an interactive shell provided by the fortress.

In this section,
we discuss operations on a fortress in their most basic form,
postponing the discussion of more advanced options,
including additional optional parameters,
to \secref{advancedops}.
Although these more advanced options are critical to performing
some real-world tasks with components,
it is easier to describe their behavior
after the basic forms of operations have been discussed.

\paragraph{Compile}
This operation takes the source code
for a simple component (or \apiN) definition
and produces a new component object (or \apiN\ object)
that is installed on the fortress.
Its type is as follows:

%compile(file:String):()
\shellcommand{compile(file:String):()}

For example,
suppose \shellcommand{IronCrypto.fss} contains
the source code for the aforementioned
\ironcryp\ application,
which imports \TYP{Fortress.IO}
and \TYP{Fortress.Security},
and exports
\TYP{Fortress.Crypto}.
Suppose we also have source code, \shellcommand{IronIO.fss}, for another
application, \ironio, which imports nothing and exports \TYP{Fortress.IO}.
We generate these components by compiling the source files:

%compile("IronIO.fss")
%compile("IronCrypto.fss")
\shellcommand{compile("IronIO.fss")}\\
\shellcommand{compile("IronCrypto.fss")}

The results are depicted diagrammatically
in \figref{ironcrypto-simple}.
\begin{figure}
\begin{center}
\input{\home/basic/components/ironcrypto.eepic}
\end{center}
\caption{Simple components in box notation:
A component is represented by a box,
with the name of the component at the top of the box.
The arrow protruding from the upper right corner of a box
is labeled with the \apisN\ exported by the component.
The arrow pointing into the bottom of a box
is labeled with \apisN\ imported by the component.
If no \apisN\ are imported, we elide the arrow.}
\figlabel{ironcrypto-simple}
\end{figure}

\paragraph{Link}

A collection of one or more components exporting different \apisN\
may be combined to form a new, compound, component
by calling the \shellcommand{link} operation,
passing the names of the components to link
along with the name of the resulting compound component.
Syntactically, a \shellcommand{link} operation is written as follows:\footnote{We
present only the basic form of \shellcommand{link} here.
\shellcommand{link} has additional optional arguments
that we discuss in the \secref{advancedops}.}

%link(result:String, constituents:String...):()
\shellcommand{link(result:String, constituents:String...):()}

The components being linked are called \emph{constituents}
of the resulting component,
which exports all the \apisN\ exported by any of its constituents,
and imports the \apisN\ imported by at least one of its constituents
but not exported by any of them.

For example,
we can link the \ironio\ and \ironcryp\ libraries
compiled above:

%link(IronLink, IronIO, IronCrypto)
\shellcommand{link(IronLink, IronIO, IronCrypto)}

The resulting component,
illustrated in \figref{simplelink},
imports \fortsec\ and exports \fortio\ and \fortcryp.
\begin{figure}
\begin{center}
\input{\home/basic/components/ironlink.eepic}
\end{center}
\caption{A compound component:
A component inside another component
is a constituent of the component that immediately encloses it.}
\figlabel{simplelink}
\end{figure}

\shellcommand{link} does not distinguish between simple and compound
components, so we can get arbitrarily nested components.  For example,
we can construct an application \coolcrypto\
by compiling another source code, \shellcommand{IronSecurity.fss},
for the library \ironsec\
that imports \fortio\ and exports \fortsec,
and then linking the result with \ironlink.

%compile(IronSecurity.fss)
%link(CoolCryptoApp, IronSecurity, IronLink)
\shellcommand{compile(IronSecurity.fss)}\\
\shellcommand{link(CoolCryptoApp, IronSecurity, IronLink)}

The resulting components are illustrated in \figref{coolcrypto}.

Two components cannot be linked if they export the same
\apiN.\!\footnote{There is one exception to this rule:
the special \apiN\ \upgapi, which is used during upgrades discussed below.}
This restriction is made for the sake of simplicity;
it allows programmers to link a set of components
without having to specify explicitly
which constituent exporting an \apiN\ $A$
provides the implementation exported by the linked component,
and which constituent connects to the constituents that import $A$:
only one component exports $A$, so there is only one choice.
Although we lose expressiveness with this design,
the user interface to link is vastly simplified,
and it is rare that including multiple components that export a given \apiN\
in a set of linked components is even desirable.
We discuss how even such rare cases can be supported in
\secref{advancedops}.
\note{
THERE IS ANOTHER RESTRICTION ON LINKING,
but it is trivially met if you can't hide \apisN.}

For a compound component,
in addition to the exported and imported \apisN,
we want to know what its constituents are.
It is an invariant of the system
that for any compound component $C$,
any \apiN\ imported by any of its constituents
is either imported by $C$ or exported by one of its constituents.
This property is crucial for executing components, as we discuss below.
A simple component (i.e., one produced directly by compilation)
has no constituents.

\begin{figure}
\begin{center}
\input{\home/basic/components/coolcrypto.eepic}
\end{center}
\caption{Repeated linking}
\figlabel{coolcrypto}
\end{figure}

\paragraph{Execute}

Components provide implementations of the \apisN\ they export.
A component is \emph{executable} if it imports no \apisN\
and it exports the special \apiN\ \execapi,
defined as follows:

%api Executable
%run(args:String...):()
%end
\begin{Fortress}
\(\KWD{api} \TYP{Executable}\)\\
\(\VAR{run}(\VAR{args}\COLONOP\TYP{String}\ldots)\COLONOP()\)\\
\(\KWD{end}\)
\end{Fortress}

An executable component may be \emph{executed} by calling the
\shellcommand{execute} operation,
resulting in a call to the component's implementation of the
\VAR{run} function in a new process.
Arguments to the \VAR{run} function are passed to the shell:

%execute(componentName:String, args:String...):()
\shellcommand{execute(componentName:String, args:String...):()}

We say that a component is being executed when
\shellcommand{execute} has been called on that component
and has not yet returned,
or if it is the constituent component of a component being executed.
During an execution,
references may be made to \apisN\ exported by a component being executed,
which may in turn make references to \apisN\ that it imports.

For references to an \apiN\ $\api$ exported by the component,
if the component is simple,
then it contains the code necessary to evaluate
any reference to an \apiN\ it exports,
possibly making references to \apisN\ that it imports to do so.
If the component is compound,
then it contains a unique constituent
that exports $\api$;
the reference is resolved to that constituent component.

For external references within a constituent component,
recall that all such references in a component
must be to \apisN\ that the component imports.
A component being executed either does not import any \apiN\
(and thus there are no external references to resolve),
or else is a constituent of another component
that is being executed.
In the latter case, the constituent defers the reference
to its enclosing component.

For example,
suppose \coolcrypto\ above is the constituent
of some executable component,
and when that component is executed,
it generates a reference to \TYP{SecretKey} in \fortcryp,
which it resolves to \coolcrypto.
\coolcrypto\ resolves this reference to \ironlink,
which resolves it to \ironcryp,
which is a simple component.
Suppose that in evaluating this reference,
\ironcryp\ generates
a reference to \TYP{PublicKey} in \fortsec.
Because \ironcryp\ imports \fortsec,
it resolves this reference to its enclosing component, \ironlink,
which in turn resolves it to \coolcrypto,
which resolves it to \ironsec,
which is a simple component.


Not all projects are compiled to components
that export \execapi.
For example, a library component does not usually export \execapi.


\paragraph{Upgrade}

Compound components may be upgraded
with new constituent components by calling an \shellcommand{upgrade} operation,
passing the name of the component to upgrade (the \emph{target}),
the name of a component to upgrade with (the \emph{replacement}),
and a name for the resulting component (which we call the \emph{result}).
The type of the \shellcommand{upgrade} operation is as follows:

%upgrade(target:String, replacement:String, result = target):()
\shellcommand{upgrade(target:String, replacement:String, result = target):()}

If no result name is provided,
the result is bound to the name of the target,
and the target is uninstalled (see below).

For example,
we can upgrade \coolcrypto\
with a component \coolsec,
which exports \fortsec\
and imports nothing to \TYP{CoolCryptoApp.2.0}.

%upgrade(CoolCryptoApp, CoolSecurity, CoolCryptoApp.2.0)
\shellcommand{upgrade(CoolCryptoApp, CoolSecurity, CoolCryptoApp.2.0)}

The resulting component is illustrated in \figref{upgraded}.
Notice that the constituent, \ironsec, exporting \\
\fortsec\
has been replaced.

A component can be upgraded
only if it exports the special \apiN\ \upgapi,
defined as follows:

%api Upgradable
%import Component from Components
%
%isValidUpgrade(that:Component):Boolean
%upgrade(that:Component):Component requires { isValidUpgrade(that) }
%
%end
\begin{Fortress}
\(\KWD{api} \TYP{Upgradable}\)\\
\(\KWD{import} \TYP{Component} \KWD{from} \TYP{Components}\)\\[4pt]
\(\VAR{isValidUpgrade}(\VAR{that}\COLONOP\TYP{Component})\COLONOP\TYP{Boolean}\)\\
\(\VAR{upgrade}(\VAR{that}\COLONOP\TYP{Component})\COLONOP\TYP{Component} \KWD{requires} \{\,\VAR{isValidUpgrade}(\VAR{that})\,\}\)\\[4pt]
\(\KWD{end}\)
\end{Fortress}


The \upgapi\ \apiN\ imports a special \apiN\ \TYP{Components} that
provides handles on \TYP{Component} and \TYP{Api} objects.
The \TYP{Components} \apiN\ is described
in \chapref{lib:api-components}.

An \shellcommand{upgrade} operation on a component
invokes the \VAR{isValidUpgrade} function,
as declared in the \apiN\ \upgapi.
This function must take a component and return \VAR{true}
if and only if it is legal to upgrade with respect to that component.
Developers can define their own versions of this component
to restrict how their components can be upgraded.
For example,
they can prevent upgrades with older versions of a component,
or with a matching component from an untrusted vendor.

\begin{figure*}
\begin{center}
\input{\home/basic/components/coolcrypto2.eepic}
\end{center}
\caption{An upgraded component}
\figlabel{upgraded}
\end{figure*}

The \upgapi\ \apiN\ presents a problem for our model. Its implementation
by the various constituent components in a compound
component must be accessed during an \shellcommand{upgrade} operation.
However, because the exported \apisN\ of the constituent components must
be disjoint, they cannot all export \upgapi\ after linking.

We solve this problem by introducing an additional step during linking.
In a \shellcommand{link} operation,
a special component, called a \emph{restriction component},
is constructed automatically, based on the provided constituents.
This component exports the \upgapi\ \apiN;
its implementation is a function
of all the constituents provided to the \shellcommand{link} operation.
The provided constituents are then used to construct a new set
of constituents that are identical to the provided constituents
except that they do not export \upgapi.
These new constituents are then combined,
along with the restriction component,
to form the constituents of a new compound component.

In addition to the constraints
imposed by a component's \VAR{isValidUpgrade} function,
there are several other conditions
that must be met in order for an upgrade to be valid.
These conditions are necessary to ensure
that the resulting component is well-formed
and imports and exports the same \apisN\ as the target:
\footnote{
  These
  conditions are sufficient
  provided there are no hidden or constrained \apisN,
  which are discussed in \secref{advancedops}.
}
\begin{enumerate}

\item
Every \apiN\ imported by the replacement must be either
imported or exported by the target.

\item
The \apisN\ exported by the replacement
must be a subset of those exported by the target.

\item
If the replacement does not export all the APIs that a constituent exports
then either the replacement and constituent do not export any \apisN\ in common
or the constituent can be upgraded with the replacement.

\end{enumerate}

The rationale for the first two conditions is straightforward:
If an \apiN\ is imported by the replacement
but not imported or exported by the target,
then references to that \apiN\ cannot be resolved in the result
(unless we also import that \apiN\ in the result).
If an \apiN\ is exported by the replacement but not the target,
then the result will export an \apiN\ not exported by the target.

The third condition says that the constituents of the target
can be partitioned into three sets:
those that are subsumed by the replacement,
those that are unaffected by the upgrade,
and all the rest, which can be upgraded with the replacement.
This condition enables recursive propagation of upgrades.
That is, an upgrade not only replaces constituents
at the top level of the component,
but is also propagated into any constituents
with which it exports some \apisN\ in common.
Thus, in the example above,
we could have upgraded \coolcrypto\
with a component that exports \fortio.
However,
we could not have upgraded \coolcrypto\
with a component that exports both \fortsec\ and \fortio\
because \ironlink\ exports \fortio\ but not \fortsec.
In \secref{advancedops},
we show how hiding and constraining \apisN\ can help us
get around many of the limitations that this condition imposes.

Recall that in our system,
unlike with dynamic linking,
components are encapsulated
so that an upgrade to one component
does not affect any other component on the system.
We can imagine that all operations on components
copy the components that they operate on
rather than share them.
Because components are immutable,
these two interpretations are semantically indistinguishable.
Convenience operations that support mass upgrades
are provided on fortresses
(e.g., an \shellcommand{upgradeAll} operation
that takes a component and upgrades all components in the fortress
that can be upgraded with its argument).


\newcommand{\prereqs}{\shellcommand{prereqs}}
\paragraph{Extract and Install}

A component installed on a fortress may be \emph{extracted} by calling an
\shellcommand{extract} operation on the fortress,
passing the name of the component as an argument,
along with an argument \prereqs,
denoting the names of all \apisN\ that must be installed on any fortress
before this component can be installed.

%extract(componentName:String, prereqs:Set[\String\] = {}):()
{\small\verb+extract(componentName:String, prereqs:Set[\String\] = {}):()+}

Furthermore,
the destination fortress must have a component that exports these \apisN\
and is a valid upgrade of the extracted component.
Intuitively, a \prereqs\ argument allows a component to be
serialized without having to include all of its libraries; new
libraries can be provided
when the component is installed at a destination fortress.

The \prereqs\ argument is optional;
if omitted, the extracted component can be installed on any fortress.
Any component can be extracted; however
only compound components can be extracted
with a \prereqs\ argument:
because extracted
components must be upgradable with respect to a component
exporting their \prereqs,
no \prereqs\ argument makes sense for a simple component.

The \apisN\ included in a \prereqs\ argument
must be the \apisN\ exported by some subset
of the extracted component's constituents
(or a subset of the constituents of one of its constituents,
and so on, due to recursive updating).

The extracted component is serialized to a file,
including all the \apisN\ it refers to
(and, transitively, all \apisN\ they refer to)
and all constituent components,
except those that export the \prereqs.
This operation does not remove the extracted component from the fortress;
there is a separate \shellcommand{uninstall} operation for that.

When the component is extracted,
if no \prereqs\ were passed to the \shellcommand{extract} operation,
then the contents of the file can be deserialized by any fortress
into the extracted component,
which can be installed on the fortress.
However, if \prereqs\ were passed to \shellcommand{extract},
then the file must be deserialized into a component
that exports only the \instapi\ \apiN:

%api Installable
%import Component from Components
%reconstitute(candidate:Component):Component
%end
\begin{Fortress}
\(\KWD{api} \TYP{Installable}\)\\
\(\KWD{import} \TYP{Component} \KWD{from} \TYP{Components}\)\\
\(\VAR{reconstitute}(\VAR{candidate}\COLONOP\TYP{Component})\COLONOP\TYP{Component}\)\\
\(\KWD{end}\)
\end{Fortress}

The deserialized component is immediately linked with preferred
implementations of all of its imported \apisN.
(Preferred implementations of \apisN\ are maintained in a table by
a fortress, which maps each \apiN\ to a list of components that implements
it, in order of preference).
Because the deserialized and linked  component exports the \instapi\ \apiN,
it has a \VAR{reconstitute} function
that takes a \VAR{candidate} component,
which exports the \prereqs\ \apisN,
and checks whether the given component
satisfies the \VAR{isValidUpgrade} condition
of the extracted component.
If so, it returns the extracted component
upgraded with the given component.
The \VAR{reconstitute} function is called by the fortress
with a new component, formed by linking the preferred components
for each \apiN\ in the extracted components' \prereqs\ argument.

Note that an extracted component with \prereqs\ \apisN\
is \emph{not} the same as an
extracted component that imports the same \apisN\
but has no \prereqs\ \apisN.
The latter can always be installed on a fortress,
and then can be subsequently linked with any component
that exports the imported \apisN.
In contrast,
the fortress has no access to an extracted component
with \prereqs\ \apisN\
unless it has a component that exports these \apisN\
and satisfies the \VAR{isValidUpgrade} function of the extracted component.
This difference provides a means
for controlling access to the extracted component,
for security, legal, or other reasons.

Syntactically, an \shellcommand{install} operation takes the name of a file
constraining an extracted component.
The \shellcommand{install} operation is overloaded with another
operation that takes the name of a component
to match \prereqs.
If this optional argument is provided,
and the deserialized component exports the \instapi\ \apiN,
then the \VAR{reconstitute} function is called
with
the component denoted by the optional argument of \shellcommand{install},
rather than the fortress' preferred implementation of the \prereqs\ \apisN.
Install operations are written as follows:

%install(file:String):()
%install(file:String, prereqs:Set[\String\]):()
{\small\verb+install(file:String):()+\\
\verb+install(file:String, prereqs:Set[\String\]):()+}


By default,
a fortress adds a newly installed component to the head of the
``preferred'' list for every \apiN\ it exports.
However, this default may be overridden by the end-user;
an end-user may modify the table
or even map some \apisN\ differently during a particular installation.
If one or more of the \apisN\ required by an extracted component
is not mapped to an \apiN\ on the destination fortress,
an exception is thrown.

There is a corresponding operation for \apisN,
\shellcommand{installAPI}, that takes a serialization of a set
of \apisN\ and installs them into a fortress.

%installAPI(file:String):()
\shellcommand{installAPI(file:String):()}

This set of \apisN\ must be closed under imports. If an \apiN\
that is installed in this way is already installed on the fortress,
the definitions must match exactly,
or an exception is thrown.

\paragraph{Uninstall}
An \shellcommand{uninstall} operation takes the name of a
component as an argument and
removes the top-level binding of that component from a fortress.
Note that the uninstalled component
may have been linked to other components,
or used as a replacement in an upgrade,
and the result may still be installed;
an \shellcommand{uninstall} operation will not affect these other components.

%uninstall(file:String):()
\shellcommand{uninstall(file:String):()}

There is a corresponding operation for \apisN,
\shellcommand{uninstallAPI}, that removes an \apiN\ from a fortress.

%uninstallAPI(file:String):()
\shellcommand{uninstallAPI(file:String):()}

Typically, this operation is used only to remove \apisN\
that have been corrupted in some fashion.

\paragraph{Testing}
A component can be tested by calling the method
\VAR{runTests} on it: 
%runTests(inclusive = true): ()
\begin{Fortress}
\(\VAR{runTests}(\VAR{inclusive} = \VAR{true})\COLON ()\)
\end{Fortress}

This method runs all test functions defined in the component.
All test functions are run in parallel;
each test function is run for each
combination of test cases (bound in its generator list as described in
\secref{test-decls}) in parallel. 
In the case of a compound component,
the set of defined test functions consists of
all test functions defined by all constituent components and by all
exported \apisN.  The set of test functions run can be limited by
first hiding the tested component in a more restrictive \apiN. 
The set of test functions can also be expanded by linking with a component
defining additional test functions. 

The \VAR{runTests} method includes a keyword parameter \VAR{inclusive} that
defaults to \VAR{true}.  If this parameter is set to \VAR{false}, only test
functions defined in the \apisN\ exported by the component are run.
