%%%%%%%%%%%%%%%%%%%%%%%%%%%%%%%%%%%%%%%%%%%%%%%%%%%%%%%%%%%%%%%%%%%%%%%%%%%%%%%%
%   Copyright 2009, Oracle and/or its affiliates.
%   All rights reserved.
%
%
%   Use is subject to license terms.
%
%   This distribution may include materials developed by third parties.
%
%%%%%%%%%%%%%%%%%%%%%%%%%%%%%%%%%%%%%%%%%%%%%%%%%%%%%%%%%%%%%%%%%%%%%%%%%%%%%%%%

\chapter{Dimension and Unit Declarations}
\chaplabel{declaring-dimunits}

\note{Dimensions and units are not yet supported.}

\begin{Grammar}
\emph{DimUnitDecl}
&::=&
\KWD{dim} \emph{Id} \options{\EXP{=} \emph{Type}}
(\KWD{unit} $|$ \KWD{SI\_unit}) \emph{Id}$^+$
\options{\EXP{=} \emph{Expr}}\\
&$|$&
\KWD{dim} \emph{Id}
\options{\EXP{=} \emph{Type}} \options{\KWD{default} \emph{Id}}\\
&$|$&
(\KWD{unit} $|$ \KWD{SI\_unit})
\emph{Id}$^+$ \options{\EXP{\mathrel{\mathtt{:}}} \emph{Type}}
 \options{\EXP{=} \emph{Expr}}\\
\end{Grammar}

\section{Dimension Declarations}
Dimensions may be explicitly declared;
every declared dimension must be declared at the top level
of a program component,
not within a block expression or trait.
Other dimensions may be constructed by multiplying and dividing other dimensions,
as described in \chapref{dimunits}.
An explicitly declared dimension may be a \emph{base dimension} (with no definition specified)
or a \emph{derived dimension} (with a definition specified in the form of an initialization expression).

The set of all dimensions has the algebraic structure of a free abelian group.
The identity element of this group is the dimension \TYP{Unity}, which represents dimensionlessness.

For every two dimensions \VAR{D} and \VAR{E}, there is a
dimension \EXP{D E} (which may also be written \EXP{D \cdot E}),
corresponding to the product of the dimensions \VAR{D} and \VAR{E} and a
dimension \EXP{D/E}, corresponding to the quotient of the
dimensions \VAR{D} and \VAR{E}. The syntactic sugar \EXP{1/D} is equivalent
to \EXP{\TYP{Unity}/D} for all dimensions \VAR{D}.
A dimension can be raised to a rational power where both
the numerator and the denominator of the rational power
must be a valid \emph{\KWD{nat} parameter} instantiation;
\EXP{D^{0}} is the same as \TYP{Unity},
\EXP{D^{1}} is the same as \VAR{D}, and
\EXP{D^{m+n}} is the same as \EXP{D^{m} D^{n}}.
The syntactic sugar \EXP{D^{-n}} is the same as \EXP{\TYP{Unity}/D^{n}}.

Here are some examples of base dimension declarations:
%% dim Length
%% dim Mass
%% dim Time
%% dim ElectricCurrent
\begin{Fortress}
\(\KWD{dim} \TYP{Length}\)\\
\(\KWD{dim} \TYP{Mass}\)\\
\(\KWD{dim} \TYP{Time}\)\\
\(\KWD{dim} \TYP{ElectricCurrent}\)
\end{Fortress}
Here are some examples of computed dimensions:
%% Length / Time
%% Velocity / Time
%% Length \cdot Mass / Time^2
%% Length Mass Time^(-2)
%% ElectricCurrent / Length^2
\begin{Fortress}
\(\TYP{Length} / \TYP{Time}\)\\
\(\TYP{Velocity} / \TYP{Time}\)\\
\(\TYP{Length} \cdot \TYP{Mass} / \TYP{Time}^{2}\)\\
\(\TYP{Length}\mskip 4mu plus 4mu\TYP{Mass}\mskip 4mu plus 4mu\TYP{Time}^{-2}\)\\
\(\TYP{ElectricCurrent} / \TYP{Length}^{2}\)
\end{Fortress}
and here some of these computed dimensions are given names through the use of
derived dimension declarations:
%% dim Velocity = Length / Time
%% dim Acceleration = Velocity / Time
%% dim CurrentDensity = ElectricCurrent / Length^2
\begin{Fortress}
\(\KWD{dim} \TYP{Velocity} = \TYP{Length} / \TYP{Time}\)\\
\(\KWD{dim} \TYP{Acceleration} = \TYP{Velocity} / \TYP{Time}\)\\
\(\KWD{dim} \TYP{CurrentDensity} = \TYP{ElectricCurrent} / \TYP{Length}^{2}\)
\end{Fortress}

\section{Unit Declarations}
Every unit belongs to exactly one dimension, which is the type of the unit.
A dimension may have more than one unit,
but one of these units may be singled out as the \emph{default unit} for that dimension
by adding a \KWD{default} clause:
%% dim Length default meter
%% dim Mass default kilogram
%% dim Time default second
\begin{Fortress}
\(\KWD{dim} \TYP{Length} \KWD{default} \TYP{meter}\)\\
\(\KWD{dim} \TYP{Mass} \KWD{default} \TYP{kilogram}\)\\
\(\KWD{dim} \TYP{Time} \KWD{default} \TYP{second}\)
\end{Fortress}
The default unit is used when a numerical type is multiplied by a
dimension to produce a new type (see \chapref{dimunits}).
If no default clause is specified for a base dimension, then it has
no default unit.  If no default clause is specified for a derived
dimension, then it has a default unit if and only if all the
dimensions mentioned in its initialization expression have defaults,
in which case its default unit is calculated using the initialization
expression with each dimension replaced by its default unit.

Some units are explicitly declared;
every declared unit must be declared at the top level of
a program component, not
within a block expression or trait.
Other units may be constructed by multiplying and dividing other units.
An explicitly declared unit may be a \emph{base unit} (with no definition specified)
or a \emph{derived unit} (with a definition specified in the form of an initialization expression).

The set of all units, like the set of all dimensions, has the algebraic structure of a free abelian group.
The identity element of this group is the unit \TYP{dimensionless}, of
dimension \TYP{Unity}.
Note that there may be other units of dimension \TYP{Unity}, such as
\TYP{radian} and \TYP{steradian}, but only \TYP{dimensionless} is the
identity of the group of all units.
(Note that there is a straightforward homomorphism of units onto
dimensions, wherein every unit is mapped to its dimension.)

Here are some examples of base unit declarations:
%% unit meter : Length
%% unit kilogram : Mass
%% unit second : Time
%% unit ampere : ElectricCurrent
\begin{Fortress}
\(\KWD{unit} \TYP{meter} \mathrel{\mathtt{:}} \TYP{Length}\)\\
\(\KWD{unit} \TYP{kilogram} \mathrel{\mathtt{:}} \TYP{Mass}\)\\
\(\KWD{unit} \TYP{second} \mathrel{\mathtt{:}} \TYP{Time}\)\\
\(\KWD{unit} \TYP{ampere} \mathrel{\mathtt{:}} \TYP{ElectricCurrent}\)
\end{Fortress}
Here are some examples of computed units:
%% meter / second
%% (meter / second) / second
%% meter \cdot kilogram / second^2
%% meter kilogram second^(-2)
%% ampere / meter^2
\begin{Fortress}
\(\TYP{meter} / \TYP{second}\)\\
\((\TYP{meter} / \TYP{second}) / \TYP{second}\)\\
\(\TYP{meter} \cdot \TYP{kilogram} / \TYP{second}^{2}\)\\
\(\TYP{meter}\mskip 4mu plus 4mu\TYP{kilogram}\mskip 4mu plus 4mu\TYP{second}^{-2}\)\\
\(\TYP{ampere} / \TYP{meter}^{2}\)
\end{Fortress}
and here some computed units are given names through the use of
derived unit declarations:
%% unit newton: Force = meter \cdot kilogram / second^2
%% unit joule: Energy = newton meter
%% unit pascal: Pressure = newton / meter^2
\begin{Fortress}
\(\KWD{unit} \TYP{newton}\COLON \TYP{Force} = \TYP{meter} \cdot \TYP{kilogram} / \TYP{second}^{2}\)\\
\(\KWD{unit} \TYP{joule}\COLON \TYP{Energy} = \TYP{newton}\mskip 4mu plus 4mu\TYP{meter}\)\\
\(\KWD{unit} \TYP{pascal}\COLON \TYP{Pressure} = \TYP{newton} / \TYP{meter}^{2}\)
\end{Fortress}
In the preceding examples, the initialization expression for each unit
is itself a unit.  It is also permitted for the initialization
expression to be a dimensioned numerical value, in which case
the unit being declared is related to the unit of the dimensioned
numerical value by a numerical conversion factor.

As with an ordinary variable declaration, one may omit the dimension for a unit
if there is an initialization expression; the dimension of the declared unit is
the dimension of the unit of the expression.

\note{COMMENTED TEXT; I like the explanation below better. -- Eric

Every unit can be reduced to a canonical unit as follows.  A base unit is
canonical; a \KWD{unit} parameter is canonical; a defined unit is replaced
by its initialization expression,
but only if the value of the expression
is a unit and not a dimensioned value, where every unit in that expression
is replaced by its canonical unit;
and finally all arithmetic is performed
so as to reduce the expression to a single unit. Two units are considered
equivalent if their canonical units are identical.

Now here is a slightly different process.
}

Every unit can be reduced to a canonical value as follows.  A base unit is
multiplied by the value \EXP{1}; a \KWD{unit} parameter is multiplied by
the value \EXP{1}; a defined unit is replaced by its initialization
expression and then every unit in that expression is replaced by its
canonical form;
and finally all arithmetic is performed so as to reduce the
units to a single unit and the
numerical values to a single numerical value.  A dimensioned value with unit
\VAR{U} is convertible by the \KWD{in} operator to a value with unit
\VAR{V} if the canonical values for \VAR{U} and \VAR{V} have the same
unit; the conversion involves multiplying the numerical value by the
ratio of the numerical value of the canonical form
of \VAR{V} to the numerical value of the canonical form of \VAR{U}.

For example, given the declarations:
%% dim Length
%% unit meter: Length; unit meters = meter
%% unit kilometer: Length = 10^3 meter; unit kilometers = kilometer
%% unit inch: Length = 2.54 \times 10^(-2) meter; unit inches = inch
%% unit foot: Length = 12 inch; unit feet = foot
%% unit mile: Length = 5280 foot; unit miles = mile
\begin{Fortress}
\(\KWD{dim} \TYP{Length}\)\\
\(\KWD{unit} \TYP{meter}\COLON \TYP{Length}; \KWD{unit} \TYP{meters} = \TYP{meter}\)\\
\(\KWD{unit} \TYP{kilometer}\COLON \TYP{Length} = 10^{3} \TYP{meter}; \KWD{unit} \TYP{kilometers} = \TYP{kilometer}\)\\
\(\KWD{unit} \TYP{inch}\COLON \TYP{Length} = 2.54 \times 10^{-2} \TYP{meter}; \KWD{unit} \TYP{inches} = \TYP{inch} \)\\
\(\KWD{unit} \TYP{foot}\COLON \TYP{Length} = 12\,\TYP{inch}; \KWD{unit} \TYP{feet} = \TYP{foot}\)\\
\(\KWD{unit} \TYP{mile}\COLON \TYP{Length} = 5280\,\TYP{foot}; \KWD{unit} \TYP{miles} = \TYP{mile}\)
\end{Fortress}
then one can say \EXP{3\,\TYP{miles}\, \KWD{in}\, \TYP{kilometers}} and
the \KWD{in} operator will multiply the numerical value \EXP{3} by
the amount of
\EXP{((2.54 \times 10^{-2})(12)(5280)/10^{3})},
or \EXP{25146/15625}.

Notice the subtle difference between these two declarations:
%% unit radian = meter/meter
%% unit radian = 1 meter/meter
\begin{Fortress}
\(\KWD{unit} \TYP{radian} = \TYP{meter}/\TYP{meter}\)\\
\(\KWD{unit} \TYP{radian} = 1\,\TYP{meter}/\TYP{meter}\)
\end{Fortress}
The first declaration defines \TYP{radian} to be equivalent to
\TYP{dimensionless}, and so a value with unit \TYP{radian} can be used
anywhere a dimensionless value can be used, and vice versa.
The second declaration defines \TYP{radian} to be convertible to
\TYP{dimensionless} but not equivalent, and so it is necessary to use
the \KWD{in} operator (or multiplication and division by \TYP{radian})
to convert between values in radians and truly dimensionless values.


\section{Abbreviating Dimension and Unit Declarations}
\seclabel{abbrev-dimunits}

For convenience, three forms of syntactic sugar are provided when declaring
dimensions and units.
First, in a \KWD{unit} declaration one may mention more than one name
before the colon, and the extra names are defined to be synonyms for the
first name; thus
%% unit foot feet ft_: Length
\begin{Fortress}
\(\KWD{unit} \TYP{foot}\mskip 4mu plus 4mu\TYP{feet}\mskip 4mu plus 4mu\mathrm{ft}\COLON \TYP{Length}\)
\end{Fortress}
means exactly the same thing as
%% unit foot: Length
%% unit feet: Length = foot
%% unit ft_: Length = foot
\begin{Fortress}
\(\KWD{unit} \TYP{foot}\COLON \TYP{Length}\)\\
\(\KWD{unit} \TYP{feet}\COLON \TYP{Length} = \TYP{foot}\)\\
\(\KWD{unit} \mathrm{ft}\COLON \TYP{Length} = \TYP{foot}\)
\end{Fortress}
Second, instead of the reserved word \KWD{unit} one may use the reserved
word \KWD{SI\_unit}, which has the effect of defining not only the
specified names but also names with the various SI prefixes attached.
If more than one name is specified, then the last name is assumed
to be a symbol and has symbol prefixes (such as \TYP{M} and \TYP{n})
attached; all other names have the full prefixes (such as
\TYP{mega} and \TYP{nano}) attached.
Thus
%% SI_unit name1 name2 name3: ...
\begin{Fortress}
\(\KWD{SI{\char'137}unit} {name}_{1}\;\;{name}_{2}\;\;{name}_{3}\COLON \ldots\)
\end{Fortress}
may be regarded as an abbreviation for
%% unit~name1~name2~name3: ...
%% unit yotta~name1~yotta name2~Y name3 = 10^24~name1
%% unit zetta~name1~zetta name2~Z name3 = 10^21 ~name1
%% unit exa~name1~exa name2~E name3 = 10^18 ~name1
%% unit peta~name1~peta name2~P name3 = 10^15 ~name1
%% unit tera~name1~tera name2~T name3 = 10^12 ~name1
%% unit giga~name1~giga name2~G name3 = 10^9 ~name1
%% unit mega~name1~mega name2~M name3 = 10^6 ~name1
%% unit kilo~name1~kilo name2~k name3 = 10^3 ~name1
%% unit hecto~name1~hecto name2~h name3 = 10^2 ~name1
%% unit deka~name1~deka name2~da name3 = 10 ~name1
%% unit deci~name1~deci name2~d name3 = 10^(-1) ~name1
%% unit centi~name1~centi name2~c name3 = 10^(-2) ~name1
%% unit milli~name1~milli name2~m name3 = 10^(-3) ~name1
%% unit micro~name1~micro name2~micro name3 = 10^(-6) ~name1
%% unit nano~name1~nano name2~n name3 = 10^(-9) ~name1
%% unit pico~name1~pico name2~p name3 = 10^(-12) ~name1
%% unit femto~name1~femto name2~f name3 = 10^(-15) ~name1
%% unit atto~name1~atto name2~a name3 = 10^(-18) ~name1
%% unit zepto~name1~zepto name2~z name3 = 10^(-21) ~name1
%% unit yocto~name1~yocto name2~y name3 = 10^(-24) ~name1
\begin{Fortress}
\(\KWD{unit} {name}_{1}~{name}_{2}~{name}_{3}\COLON \ldots\)\\
\(\KWD{unit} \TYP{yotta} {name}_{1}~\TYP{yotta} {name}_{2}~ \TYP{Y} {name}_{3} = 10^{24} {name}_{1}\)\\
\(\KWD{unit} \TYP{zetta} {name}_{1}~\TYP{zetta} {name}_{2}~ \TYP{Z} {name}_{3} = 10^{21}  {name}_{1}\)\\
\(\KWD{unit} \TYP{exa} {name}_{1}~\TYP{exa} {name}_{2}~ \TYP{E} {name}_{3} = 10^{18}  {name}_{1}\)\\
\(\KWD{unit} \TYP{peta} {name}_{1}~\TYP{peta} {name}_{2}~ \TYP{P} {name}_{3} = 10^{15}  {name}_{1}\)\\
\(\KWD{unit} \TYP{tera} {name}_{1}~\TYP{tera} {name}_{2}~ \TYP{T} {name}_{3} = 10^{12}  {name}_{1}\)\\
\(\KWD{unit} \TYP{giga} {name}_{1}~\TYP{giga} {name}_{2}~ \TYP{G} {name}_{3} = 10^{9}  {name}_{1}\)\\
\(\KWD{unit} \TYP{mega} {name}_{1}~\TYP{mega} {name}_{2}~ \TYP{M} {name}_{3} = 10^{6}  {name}_{1}\)\\
\(\KWD{unit} \TYP{kilo} {name}_{1}~\TYP{kilo} {name}_{2}~ \TYP{k} {name}_{3} = 10^{3}  {name}_{1}\)\\
\(\KWD{unit} \TYP{hecto} {name}_{1}~\TYP{hecto} {name}_{2}~ \TYP{h} {name}_{3} = 10^{2}  {name}_{1}\)\\
\(\KWD{unit} \TYP{deka} {name}_{1}~\TYP{deka} {name}_{2}~ \TYP{da} {name}_{3} = 10  {name}_{1}\)\\
\(\KWD{unit} \TYP{deci} {name}_{1}~\TYP{deci} {name}_{2}~ \TYP{d} {name}_{3} = 10^{-1}  {name}_{1}\)\\
\(\KWD{unit} \TYP{centi} {name}_{1}~\TYP{centi} {name}_{2}~ \TYP{c} {name}_{3} = 10^{-2}  {name}_{1}\)\\
\(\KWD{unit} \TYP{milli} {name}_{1}~\TYP{milli} {name}_{2}~ \TYP{m} {name}_{3} = 10^{-3}  {name}_{1}\)\\
\(\KWD{unit} \TYP{micro} {name}_{1}~\TYP{micro} {name}_{2}~ \mu {name}_{3} = 10^{-6}  {name}_{1}\)\\
\(\KWD{unit} \TYP{nano} {name}_{1}~\TYP{nano} {name}_{2}~ \TYP{n} {name}_{3} = 10^{-9}  {name}_{1}\)\\
\(\KWD{unit} \TYP{pico} {name}_{1}~\TYP{pico} {name}_{2}~ \TYP{p} {name}_{3} = 10^{-12}  {name}_{1}\)\\
\(\KWD{unit} \TYP{femto} {name}_{1}~\TYP{femto} {name}_{2}~ \TYP{f} {name}_{3} = 10^{-15}  {name}_{1}\)\\
\(\KWD{unit} \TYP{atto} {name}_{1}~\TYP{atto} {name}_{2}~ \TYP{a} {name}_{3} = 10^{-18}  {name}_{1}\)\\
\(\KWD{unit} \TYP{zepto} {name}_{1}~\TYP{zepto} {name}_{2}~ \TYP{z} {name}_{3} = 10^{-21}  {name}_{1}\)\\
\(\KWD{unit} \TYP{yocto} {name}_{1}~\TYP{yocto} {name}_{2}~ \TYP{y} {name}_{3} = 10^{-24}  {name}_{1}\)
\end{Fortress}
where $\mu$ is the Unicode character U+00B5 MICRO SIGN.
Third, a \KWD{dim} declaration and a \KWD{unit} or
\EXP{\KWD{SI{\char'137}unit}} declaration
may be collapsed into a single declaration by writing the \KWD{unit}
or \EXP{\KWD{SI{\char'137}unit}}
declaration in place of the \KWD{default} clause in the \KWD{dim} declaration
and omitting the colon and dimension from the \KWD{unit} declaration.
Thus
%% dim Length  SI_unit meter meters m_
%% dim Power = Energy/Time  SI_unit watt watts W_ = joule/second
\begin{Fortress}
\(\KWD{dim} \TYP{Length}  \KWD{SI{\char'137}unit} \TYP{meter}\mskip 4mu plus 4mu\TYP{meters}\mskip 4mu plus 4mu\mathrm{m}\)\\
\(\KWD{dim} \TYP{Power} = \TYP{Energy}/\TYP{Time}  \KWD{SI{\char'137}unit} \TYP{watt}\mskip 4mu plus 4mu\TYP{watts}\mskip 4mu plus 4mu\mathrm{W} = \TYP{joule}/\TYP{second}\)
\end{Fortress}
is understood to abbreviate
%% dim Length default meter; SI_unit meter meters m_: Length
%% dim Power = Energy/Time default watt; SI_unit watt watts W_: Power = joule/second
\begin{Fortress}
\(\KWD{dim} \TYP{Length} \KWD{default} \TYP{meter}; \KWD{SI{\char'137}unit} \TYP{meter}\mskip 4mu plus 4mu\TYP{meters}\mskip 4mu plus 4mu\mathrm{m}\COLON \TYP{Length}\)\\
\(\KWD{dim} \TYP{Power} = \TYP{Energy}/\TYP{Time} \KWD{default} \TYP{watt}; \KWD{SI{\char'137}unit} \TYP{watt}\mskip 4mu plus 4mu\TYP{watts}\mskip 4mu plus 4mu\mathrm{W}\COLON \TYP{Power} = \TYP{joule}/\TYP{second}\)
\end{Fortress}
In this way the names of the seven SI base units, along with all
possible plural and prefixed forms, may be concisely defined as follows:
%% dim Length  SI_unit meter meters m_
%% dim Mass default kilogram; SI_unit gram grams g_: Mass
%% dim Time  SI_unit second seconds s_
%% dim ElectricCurrent  SI_unit ampere amperes A_
%% dim Temperature  SI_unit kelvin kelvins K_
%% dim AmountOfSubstance  SI_unit mole moles mol_
%% dim LuminousIntensity  SI_unit candela candelas cd_
\begin{Fortress}
\(\KWD{dim} \TYP{Length}  \KWD{SI{\char'137}unit} \TYP{meter}\mskip 4mu plus 4mu\TYP{meters}\mskip 4mu plus 4mu\mathrm{m} \)\\
\(\KWD{dim} \TYP{Mass} \KWD{default} \TYP{kilogram}; \KWD{SI{\char'137}unit} \TYP{gram}\mskip 4mu plus 4mu\TYP{grams}\mskip 4mu plus 4mu\mathrm{g}\COLON \TYP{Mass}\)\\
\(\KWD{dim} \TYP{Time}  \KWD{SI{\char'137}unit} \TYP{second}\mskip 4mu plus 4mu\TYP{seconds}\mskip 4mu plus 4mu\mathrm{s}\)\\
\(\KWD{dim} \TYP{ElectricCurrent}  \KWD{SI{\char'137}unit} \TYP{ampere}\mskip 4mu plus 4mu\TYP{amperes}\mskip 4mu plus 4mu\mathrm{A}\)\\
\(\KWD{dim} \TYP{Temperature}  \KWD{SI{\char'137}unit} \TYP{kelvin}\mskip 4mu plus 4mu\TYP{kelvins}\mskip 4mu plus 4mu\mathrm{K}\)\\
\(\KWD{dim} \TYP{AmountOfSubstance}  \KWD{SI{\char'137}unit} \TYP{mole}\mskip 4mu plus 4mu\TYP{moles}\mskip 4mu plus 4mu\mathrm{mol}\)\\
\(\KWD{dim} \TYP{LuminousIntensity}  \KWD{SI{\char'137}unit} \TYP{candela}\mskip 4mu plus 4mu\TYP{candelas}\mskip 4mu plus 4mu\mathrm{cd}\)
\end{Fortress}
Note the subtle difference in the declaration of \TYP{Mass} that
allows the default unit to be \TYP{kilogram} rather than \TYP{gram}.


\section{Absorbing Units}
\seclabel{absorbing-units}

\begin{Grammar}
\emph{StaticParam}
&::=& \emph{Id} \option{\emph{Extends}} \options{\KWD{absorbs} \KWD{unit}} \\
&$|$& \KWD{unit} \emph{Id} \options{\EXP{\mathrel{\mathtt{:}}} \emph{Type}}
 \options{\KWD{absorbs} \KWD{unit}}\\
\end{Grammar}

The declaration of a type parameter or a \KWD{unit} parameter for
a parameterized trait may contain a clause
``\EXP{\KWD{absorbs}\;\;\KWD{unit}}''; at
most one static parameter of a trait may have this clause.  An
instance of a parameterized trait
with a static parameter that ``\EXP{\KWD{absorbs}\;\;\KWD{unit}}'' may
be multiplied
or divided by a unit, the result being another instance of that
parameterized trait in which the static argument corresponding to
the unit-absorbing parameter has been multiplied or divided by the unit.

A few examples should make this clear.  Given the declaration
%% trait Vector[\EltType extends Number absorbs unit, nat len\] \ldots end
\begin{Fortress}
\(\KWD{trait} \TYP{Vector}\llbracket\TYP{EltType} \KWD{extends} \TYP{Number} \KWD{absorbs}\;\;\KWD{unit}, \KWD{nat} \VAR{len}\rrbracket \ldots \KWD{end}\)
\end{Fortress}
then \EXP{\TYP{Vector}\llbracket\TYP{Float}, 3\rrbracket \TYP{meter}}
means the same as
\EXP{\TYP{Vector}\llbracket\TYP{Float}\mskip 4mu plus 4mu\TYP{meter}, 3\rrbracket},
and \EXP{\TYP{Vector}\llbracket\TYP{Float}, 3\rrbracket /
  \TYP{second}} means the same as
\EXP{\TYP{Vector}\llbracket\TYP{Float} / \TYP{second}, 3\rrbracket}.
Therefore,
%[(3 m_) (2 m_) (5 m_)]
\EXP{[(3\,\mathrm{m}) (2\,\mathrm{m}) (5\,\mathrm{m})]}
means the same as
%[3 2 5] m_
\EXP{[3\;2\;5] \mathrm{m}}.
%
Similarly, given the declaration
%% trait Float[\unit U absorbs unit, nat e, nat s\] \ldots end
\begin{Fortress}
\(\KWD{trait} \TYP{Float}\llbracket\KWD{unit} U \KWD{absorbs}\;\;\KWD{unit}, \KWD{nat} e, \KWD{nat} s\rrbracket \ldots \KWD{end}\)
\end{Fortress}
then \EXP{\TYP{Float}\llbracket\TYP{meter}, 11, 53\rrbracket /
  \TYP{second}} means the same as
\EXP{\TYP{Float}\llbracket\TYP{meter} / \TYP{second}, 11, 53\rrbracket}, and
\begin{Fortress}
%% Float[\dimensionless, 8, 24\] meter kilogram / second^2
\EXP{\TYP{Float}\llbracket\mathbin{\TYP{dimensionless}}, 8, 24\rrbracket \TYP{meter}\mskip 4mu plus 4mu\TYP{kilogram} / \TYP{second}^{2}}
\end{Fortress}
means the same as
\begin{Fortress}
%% Float[\dimensionless meter kilogram / second^2, 8, 24\]
\EXP{\TYP{Float}\llbracket\mathbin{\TYP{dimensionless}}\, \TYP{meter}\mskip 4mu plus 4mu\TYP{kilogram} / \TYP{second}^{2}, 8, 24\rrbracket},
\end{Fortress}
which is the same as
%% Float[\meter kilogram / second^2, 8, 24\]
\begin{Fortress}
\EXP{\TYP{Float}\llbracket\TYP{meter}\mskip 4mu plus 4mu\TYP{kilogram} / \TYP{second}^{2}, 8, 24\rrbracket}.
\end{Fortress}
This is the mechanism by which meaning is given to the multiplication
and division of library-defined types by units.
