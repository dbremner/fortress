%%%%%%%%%%%%%%%%%%%%%%%%%%%%%%%%%%%%%%%%%%%%%%%%%%%%%%%%%%%%%%%%%%%%%%%%%%%%%%%%
%   Copyright 2009, Oracle and/or its affiliates.
%   All rights reserved.
%
%
%   Use is subject to license terms.
%
%   This distribution may include materials developed by third parties.
%
%%%%%%%%%%%%%%%%%%%%%%%%%%%%%%%%%%%%%%%%%%%%%%%%%%%%%%%%%%%%%%%%%%%%%%%%%%%%%%%%

\chapter{Internal Document}

%%%%%%%%%%%%%%%%%%%%%%%%%%%%%%%%%%%%%%%%%%%%%%%%%%%%%%%%%%%%%%%%%%%%%%%%%%%%%%%%
%   Copyright 2009 Sun Microsystems, Inc.,
%   4150 Network Circle, Santa Clara, California 95054, U.S.A.
%   All rights reserved.
%
%   U.S. Government Rights - Commercial software.
%   Government users are subject to the Sun Microsystems, Inc. standard
%   license agreement and applicable provisions of the FAR and its supplements.
%
%   Use is subject to license terms.
%
%   This distribution may include materials developed by third parties.
%
%   Sun, Sun Microsystems, the Sun logo and Java are trademarks or registered
%   trademarks of Sun Microsystems, Inc. in the U.S. and other countries.
%%%%%%%%%%%%%%%%%%%%%%%%%%%%%%%%%%%%%%%%%%%%%%%%%%%%%%%%%%%%%%%%%%%%%%%%%%%%%%%%

\section{Frequently Asked Questions}

\begin{itemize}
\item Are spaces by tabulation allowed?

No.  It is a static error for a Fortress program to contain CHARACTER TABULATION (U+0009) or LINE TABULATION (U+000B) outside a comment.  See \secref{characters}.

\item Cyclic definitions of top-level variables

\item \KWD{par} and \KWD{seq} and \KWD{for}

\item ``immediate supertype'' vs ``explicit supertype'' and
``immediately extend'' vs ``explicitly extend''

\item The term ``structural type'' historically means a type that corresponds to
the structure of its values.  Tuple types are structural, as are record types
and arrow types (depending on your view of the structure of a function).
Union, intersection, and fixed point types can be either structural or
nominal depending on the types they're composed of.

I suggest we use the term ``structured type'' to refer to types composed of
other types, like those you mention above, as well as generic types.
A purely nominal type system can include structured types, but not structural
types.  If it includes structured types, it's possible for two syntactically
distinct types to have the same extension (e.g., A U B and B U A).  In fact,
if there are type aliases, even two unstructured types can have the same
extension.

Note, BTW, that all structural types are (necessarily) structured.  A type
is structural by virtue of the fact that its structure corresponds to the
structure of its values.

Also, perhaps arrow types are better described as nominal.  It's really not
the structure of the function that they represent; it's the behavior of the
function.  We can think of the ``name'' of the type as ``->''; it's a generic
type parameterized by its input and output type.  They're often characterized
as structural, but this may be more of a historical accident; they first
appeared in languages with mostly structural types.

\item Spawn is considered an IO operation.  Why is that?

We did this because we don't want you to use spawn within an atomic expression.  A lot of tricky issues arise when you do this, and we just want to avoid them for now.  Eventually, we will probably relax this restriction (as we may also want to do for IO operations more generally).

\item How can a programmer create variable-sized arrays in Fortress, and how to take advantage of the static bounds checking offered by the language at the same time?

\url{http://projectfortress.sun.com/Projects/Community/wiki/AnySizeArrays}

\item Why does Fortress have a multiple dispatch instead of a single dispatch?

We want to disallow ambiguous calls such as the following:
% trait T
%     f(ZZ32) = 1
% end
% object S extends T
%     f(Number) = 2
% end
% z: ZZ32 = 3
% S.(z)
\begin{Fortress}
{\tt~}\pushtabs\=\+\( \KWD{trait}\:T\)\\
{\tt~~~~}\pushtabs\=\+\(     f(\mathbb{Z}32) = 1\)\-\\\poptabs
\( \KWD{end}\)\\
\( \KWD{object}\:S \KWD{extends}\:T\)\\
{\tt~~~~}\pushtabs\=\+\(     f(\TYP{Number}) = 2\)\-\\\poptabs
\( \KWD{end}\)\\
\( z\COLON \mathbb{Z}32 = 3\)\\
\( S.(z)\)\-\\\poptabs
\end{Fortress}


\item A function type may be an intersection of several arrow types
instead of an arrow type.

\item Rendering issues vs ASCII conversion issues

\item Functional methods rewriting rules are for scoping
not for overloading checking.

\item Operator: \KWD{juxtaposition}
  \begin{itemize}
  \item It is a reserved word because it cannot be used as an identifier.
  \item It is the only operator name that is not composed of ALLCAPS.
  \end{itemize}

\item Hacks: arrow types in mathematical notation
  \begin{itemize}
  \item Functions can be declared with arrow types unless they have
  generic types.
  \item Arrow types can use $\times$ instead of tuple types unless they are
  generic types.
  \item Method declarations cannot use these hacks.
  \item Field declarations can use arrow types unless they have generic
  types.
  \end{itemize}

\item Terminology
  \begin{itemize}
  \item A declaration is either an abstract declaration or a concrete
  declarations which is a definition.
  \item Special reserved words are ``keywords''.  Special words that are
  not reserved include context-sensitive keywords such as \KWD{name}.  We
  want to use \VAR{name} as an identifier in value contexts and use
  \KWD{name} as a keyword in type contexts.  However, we don't support this
  yet.  Currently, \KWD{name} is a special reserved word.  Finally,
  reserved words that are not special include not-yet special reserved
  words such as \KWD{private} and forbidden identifiers such as
  \VAR{alpha}.
  \end{itemize}

\item The Fortress Core Library and the Fortress Standard Library
  \begin{itemize}
  \item While the Fortress Standard Library can be written in Fortress, the
  Fortress Core Library cannot be written in real Fortress.  We provide the
  ``fictitious'' Fortress Core Library in Fortress to help programmers
  understand the Fortress Core Library.
  \item While the Fortress Standard Library can be redefined in Fortress, the
  Fortress Core Library cannot be redefined in real Fortress.
  For example, the top-level variables \VAR{true} and \VAR{false} declared in
  the Fortress Core Library cannot be redefined.
  \item While unit operators such as \TYP{per}, \TYP{cubic}, and
  \TYP{squared} are defined in the Fortress Standard Library and can be
  redefined, they are operators only in the unit contexts.
  \end{itemize}

\item Why not single inheritance between objects?
  \begin{itemize}
  \item We can do that but...
  \item we can mimic the functionality by using wrapped fields;
  \item we can enjoy nice property of objects being leaves of type
  hierarchy such as excluding each other.
  \end{itemize}

\item Meaning of extension: how is it different from subtyping?

Extension means there exists an \KWD{extends} clause.

\item \TYP{BottomType}
  \begin{itemize}
\item Can a programmer write \TYP{BottomType}?  No.
\item \EXP{(\TYP{BottomType}\ldots)} is not \TYP{BottomType}.
  \end{itemize}

\item Spaces

We allow spaces between enclosing operators and static arguments: \verb+<| [\ZZ32\] 3 |>+.
Also for the functional names and static arguments.

\item Abstract methods in a trait and implementing concrete methods in an object

An abstract method can be implemented by concrete methods whose return types are
subtypes of the abstract method.

\item \KWD{label} / \KWD{exit} across object / function expressions (07/28/08)
  \begin{itemize}
\item one-shot continuation semantics
\item How to do the closure conversion?
% label foo
%   object extends Bar
%     m() = exit foo with 5
%   end
% end
\begin{Fortress}
{\tt~}\pushtabs\=\+\( \KWD{label}\:\VAR{foo}\)\\
{\tt~~}\pushtabs\=\+\(   \KWD{object}\;\;\KWD{extends}\:\TYP{Bar}\)\\
{\tt~~}\pushtabs\=\+\(     m() = \;\KWD{exit}\:\VAR{foo} \KWD{with} 5\)\-\\\poptabs
\(   \KWD{end}\)\-\\\poptabs
\( \KWD{end}\)\-\\\poptabs
\end{Fortress}
is desugared to
% object Bar'(xit: ZZ32 -> BottomType) extends Bar
%   m() = xit(5)
% end
% label foo
%   Bar'(fn (e:ZZ32): BottomType => exit foo with e)
% end
\begin{Fortress}
{\tt~}\pushtabs\=\+\( \KWD{object}\:\TYP{Bar}'(\VAR{xit}\COLON \mathbb{Z}32 \rightarrow \TYP{BottomType}) \KWD{extends}\:\TYP{Bar}\)\\
{\tt~~}\pushtabs\=\+\(   m() = \VAR{xit}(5)\)\-\\\poptabs
\( \KWD{end}\)\\
\( \KWD{label}\:\VAR{foo}\)\\
{\tt~~}\pushtabs\=\+\(   \TYP{Bar}'(\KWD{fn} (e\COLONOP\mathbb{Z}32)\COLON \TYP{BottomType} \Rightarrow\;\KWD{exit}\:\VAR{foo} \KWD{with}\:e)\)\-\\\poptabs
\( \KWD{end}\)\-\\\poptabs
\end{Fortress}
Even though Fortress programmers are not allowed to write \TYP{BottomType},
the desugarer can use it internally.
  \end{itemize}

\item How can I re-export stuff I
imported without having to maintain cut-and-pasted declarations in my api?

%% api A
%%   trait T
%%   f(T)
%% end

%% component C
%%   import A.{...}
%%   object O extends T
%%   f(O)
%% end

%% component D
%%   import A.{T}
%%   object O extends T
%%   f(O)
%% end
\begin{Fortress}
\(\KWD{api}\:A\)\\
{\tt~~}\pushtabs\=\+\(  \KWD{trait}\:T\)\\
\(  f(T)\)\-\\\poptabs
\(\KWD{end}\)\\[4pt]
\(\KWD{component}\:C\)\\
{\tt~~}\pushtabs\=\+\(  \KWD{import}\:A.\{\ldots\}\)\\
\(  \KWD{object}\:O \KWD{extends}\:T\)\\
\(  f(O)\)\-\\\poptabs
\(\KWD{end}\)\\[4pt]
\(\KWD{component}\:D\)\\
{\tt~~}\pushtabs\=\+\(  \KWD{import}\:A.\{T\}\)\\
\(  \KWD{object}\:O \KWD{extends}\:T\)\\
\(  f(O)\)\-\\\poptabs
\(\KWD{end}\)
\end{Fortress}

\end{itemize}

%%%%%%%%%%%%%%%%%%%%%%%%%%%%%%%%%%%%%%%%%%%%%%%%%%%%%%%%%%%%%%%%%%%%%%%%%%%%%%%%
%   Copyright 2009 Sun Microsystems, Inc.,
%   4150 Network Circle, Santa Clara, California 95054, U.S.A.
%   All rights reserved.
%
%   U.S. Government Rights - Commercial software.
%   Government users are subject to the Sun Microsystems, Inc. standard
%   license agreement and applicable provisions of the FAR and its supplements.
%
%   Use is subject to license terms.
%
%   This distribution may include materials developed by third parties.
%
%   Sun, Sun Microsystems, the Sun logo and Java are trademarks or registered
%   trademarks of Sun Microsystems, Inc. in the U.S. and other countries.
%%%%%%%%%%%%%%%%%%%%%%%%%%%%%%%%%%%%%%%%%%%%%%%%%%%%%%%%%%%%%%%%%%%%%%%%%%%%%%%%

\chapter{Proposed Features (for Internal Uses)}
\applabel{future}

\note{This chapter is not up to date.}

\section{Not Yet Implemented Features for the Next Release}

\begin{itemize}
\item Incorporate Stix fonts

\item Unicode canonicalization
% :
%  \pageref{pgmUnicodeSyn},
%  \secref{characters},
%  \secref{preprocessing-character-literals},
%  \secref{preprocessing-string-literals},
%  \appref{app-rendering},
%  \appref{operators}

\item ASCII conversion
% :
%  ``ASCII Conversion'',
%  \pageref{nutshellAscii}, \pageref{overviewAscii}, \pageref{pgmAscii},
%  \secref{characters}, \secref{position}, \secref{scanning},
%  \secref{whitespace},
% \secref{preprocessing-character-literals},
% \secref{preprocessing-string-literals},
%  \secref{multichar-enclosers},
% \pageref{components-ascii-conversion},
%  \appref{operators}, \pageref{compOthers}

\item \secref{position} Lines, Pages and Position
  \begin{itemize}
  \item
A character is on page $n$
(respectively line $n$) of a program
if exactly $n-1$ page terminators (respectively line terminators)
precede that character in the program.
Thus, for example,
the $n$th line terminator of a program
is the last character on line $n$.
A character is on line $k$ of page $n$
if it is on page $n$
and is preceded by exactly $k-1$ line terminators on page $n$.
A character is at line position $k$ on line $n$
if it is preceded by exactly $k-1$ characters on line $n$
other than page terminators.
Note that a page terminator does \emph{not} terminate a line;
thus, the character immediately following a page terminator
need not be at line position 1.

  \item
Before any other processing,
a Fortress program undergoes \emph{ASCII conversion}
(see \secref{ascii-conversion}),
which may replace chunks of ASCII characters
with single (typically non-ASCII) characters
and remove some other characters.
We expect that IDEs will typically display a program
by rendering the converted sequence of characters
rather than the actual input sequence.
Thus, a program may appear to have fewer (and different) characters
than it actually does.
Nonetheless,
the page, line and position of a character
is based on the program before conversion.\footnote{Of course,
an IDE may actually do the conversion
and manipulate the converted program,
in which case,
the page, line and position of characters will reflect the conversion.
}
  \end{itemize}

\item \secref{preprocessing-character-literals} Character Literals

The sequence of enclosed characters may be
a sequence of four or more hexadecimal digits
identifying the code point of a Unicode character
(e.g., `\txt{001C}', \txt{`FBAB'}, `\txt{1D11E}'),
one of \txt{TAB}, \txt{NEWLINE} or \txt{RETURN},
a sequence of ASCII characters
that would be converted by the second step of ASCII conversion
to a single Unicode character or
the official \unicode\ name or an alternative name
of a Unicode character
with spaces and hyphens intact
(e.g., `\txt{PLUS-MINUS SIGN}').

\item Bug fixes with the rendering tool

\secref{numerals}

\end{itemize}


\section{Not Yet Implemented Features}
\begin{itemize}
\item Syntax abstraction:
\secref{scanning}

\item Fix citations:

Victor's email titled ``Re: Fortress 1.0 spec (revision 4793)'' sent on
31 Mar 2008

\item \txt{a''}, \txt{a'\_prime}, \txt{a\_prime'}, \txt{a\_prime\_prime} are rendered identically and they are meant to denote the same identifier.
Victor proposed to treat them as we treat synonym operators: a single definition for one is for all the other names.

\item Jan's short note \#82: ``Unboxed Types: the design space'' (07/24/07)
 \begin{itemize}
\item Only heap objects have a notion of object identity, and for heap objects
\TYP{SEQUIV} is object identity. All types have a notion of strict equivalence, which can be based upon content.
\item What do we mean by the \KWD{value} modifier? Do we mean an ``unboxed type'' with the \KWD{value} modifier? Do we mean an ``immutable object'' with the \KWD{value} modifier? We mean an ``unboxed type'' with the \KWD{value} modifier.
\item Flattened representation
\item Copying semantics
\item A fixed-size representation
\item Header-freedom: If we have a field or array of unboxed object type, the representation should not need to carry around header or type information as part of the representation.
\item Immutability: It is not necessary for unboxed objects to be immutable.
\item Object identity: No object identity. Strict equivalence is defined recursively for value objects, based upon content.
\item Unboxed traits: What happens when we have a field or an array element whose type is a trait type, one of whose instances is an unboxed type? Box the object or similar techniques.
\item Size of an unboxed object which is recursively defined with unboxed traits. Forbid recursive definitions in value objects.
\item Do we want to permit arrays of fixed bounds to be unboxed? No. Unboxed arrays are a different other thing. Among other things, an array of fixed bounds may be a segment of another array.
\item What happens if we mutate a LinearSequence? Only initializing writes are permitted and the sequence is then frozen.
 \end{itemize}

\item Mutual recursion in value objects (08/02)
 \begin{itemize}
\item Eric: There's a technical problem with our decision to forbid recursive definitions in value objects. What if there is mutual recursion spanning component boundaries, and the recursion isn't evident via the APIs? If there's an error for recursive value objects in components but no error for recursive value objects spanning component boundaries, it doesn't look like consistent.
\item Jan: Module recursion is a bad idea.
\item David: Would it make sense, after all, to mandate some ordering of APIs?
\item Dan: How about just forcing a value object to declare its contents explicitly in the API?
\item Eric: Value objects should be layed out in APIs.
\item David: Link-time error.
\item If there is no cycle in value objects spanning component boundaries, the programmers can get precise size information. If there is a cycle, linker will have a way to break the cycle but, in this case, there is no size guarantee.
Contingent upon Eric's agreement.
 \end{itemize}

\item Thread fairness guarantee (Jan's email titled ``Re: Clarifying fairness guarantees'' sent on 14 Mar 2007)

\item \secref{scope} Reach and Scope of a Declaration

It is a static error for a reference to a name
to occur at any point in a program
at which the name is not in scope in the appropriate namespace
(as defined below),
except immediately after the `.'
of a dotted field access or dotted method invocation
when the name is the name of the field or dotted method
of the static type of the receiver expression
(see Sections~\ref{sec:field-access}
and~\ref{sec:dotted-method-calls}). (???)

It is a static error
for two (explicit) declarations with overlapping reaches
to declare the same name,
\marginnote{other than \KWD{self},}
\emph{even if the name is declared in different namespaces},
unless one of the following (disjoint) conditions holds:
... (???)

\item Getters and setters (with help from the static front end)
% :
%  ``Abstract Field Declarations'',
%  \pageref{overviewGetter},
%  \pageref{overviewGetterTwo},
%  \pageref{overviewGetterThree},
%  \pageref{memGetter}, ``Providing the Appearance of a Single Object'',
%  \pageref{compGetters}, \chapref{declarations}, \secref{trait-decls},
% \secref{value-traits},
%  \secref{methods}, \secref{field-access}, \secref{assignment},
%  \secref{generators}

 A getter is required to have a type consistent with a setter.

Scott: 11.2 in the spec (Field Declarations) says ``Fields are variables local to an object. They must not be referred to outside their enclosing object declarations.'' I think this should be phrased differently. When I originally read that, I took it to mean that fields were always ``private.''

\item Modifiers
 \begin{itemize}
 \item \KWD{abstract}%: \secref{methods}, \pageref{compMod}
 \item \KWD{atomic}%: \secref{atomic}, \chapref{functions}
 \item \KWD{getter}%: \secref{methods}
 \item \KWD{hidden}%: \pageref{overviewMod}
 \item \KWD{io}%: \pageref{decl-io}, \pageref{evalModTwo}, \chapref{functions},
%Because IO clauses are statically checked, they should be part of an arrow type.
 \item \KWD{override}
 \item \KWD{private}%: \pageref{compModThree}
 \item \KWD{settable}%: \pageref{overviewMod}, \pageref{evalMod},
% \secref{value-traits}
 \item \KWD{setter}%: \secref{methods}
 \item \KWD{test}
 \item \KWD{widens}
 \item \KWD{wrapped}%: \secref{type-param}
 \end{itemize}

\item \TYP{Object}:
 \begin{itemize}
\item %\secref{overview-objects-traits}
 Objects and Traits

a multiple inheritance hierarchy rooted at trait \TYP{Object} (???)

\item \secref{trait-decls} Trait Declarations

In addition, every trait implicitly extends the trait \TYP{Object}
if it does not do so explicitly.
We define the extension relation to be the transitive closure
of implicit and explicit extension.
That is, trait \VAR{T} \emph{extends} trait \VAR{U}
if and only if \VAR{T} explicitly or implicitly
extends \VAR{U} or if there is some trait \VAR{S}
that \VAR{T} explicitly extends and that extends \VAR{U}.
The extension relation induced by a program is the smallest
relation satisfying these conditions.
This relation must form an acyclic hierarchy
rooted at trait \TYP{Object}.

\item %\secref{object-decls}
 Object Declarations

If an object declaration has no \KWD{extends} clause,
the object implicitly extends only trait \TYP{Object}.

\item \secref{type-param} Type Parameters

If a type parameter does not have an \KWD{extends} clause, it has an
implicit ``\EXP{\KWD{extends} \TYP{Object}}'' clause.

 \end{itemize}

\item Fortress libraries:
 \TYP{LinearSequence} and \TYP{HeapSequence}
% :
%   \pageref{evalHeapSequence}, \secref{arrays}, \chapref{functions},
%   \pageref{tuple-expr}

 \item Acyclic type hierarchy: \chapref{types},
 ``Acyclic Core Fortress with Field Definitions''


\item Type aliases
% :
%  \pageref{compOthers}, ``Type Aliases'', \chapref{declarations}

\item \TYP{Stopped} exception in early termination of threads:
 \secref{threads-parallelism}

\item Deferred exception thrown from a spawned thread
% :
% \pageref{deferred-exception}

\item Value types%: \secref{value-objects}, \secref{immutability}

\item Unpasting
% : ``Matrix Unpasting'', \pageref{unpasting},
% \secref{extremum-expr}, \secref{aggregate-expr}, \appref{concrete-syntax}

\item ``Types in the Fortress Standard Libraries'' and ``Literals'':
\TYP{Numeral}, \EXP{\mathbb{Z}}, \EXP{\mathbb{N}}, \EXP{\mathbb{Q}},
\EXP{\mathbb{N}32} (\secref{natparams}),
\TYP{Interval}, \TYP{Polar},
   \TYP{BooleanLiteral}, \TYP{NaturalNumeral}, \TYP{Literal},
 \TYP{NaturalNumeralWithExplicitRadix},
   \TYP{RadixPointNumeral}, \TYP{RadixPointNumeralWithExplicitRadix},
   \TYP{IntegerConstant}, \TYP{RationalValueTimesPi}, \TYP{NaturalStatic}

\item Covariant list with \KWD{comprises} clause (Victor's email titled
``Problem with covariant List?'' on 10/27/07)

\item \secref{local-var-decls} Local Variable Declarations

It is a static error if such a variable is referred to before it has been
given a value; an immutable variable is initialized by another variable
declaration and a mutable variable is initialized by assignment.
It is also a static error if an immutable variable is initialized
more than once.

\item \secref{label-expr} Label and Exit

It is a static error if the target is not in scope
    in the label namespace at the \KWD{exit} expression.
    That is, the target must be the name of a statically enclosing \KWD{label}
    expression, and the \KWD{exit} expression must not be within a \KWD{spawn}
    expression that is contained in the \KWD{label} expression.
    If no target is specified, the target is implicitly the name
    of the smallest statically enclosing \KWD{label} expression;
    it is a static error if there is no such expression.

\item \secref{typecase} Typecase Expressions

If no matched clause is found, a \TYP{MatchFailure} exception is thrown
(\TYP{MatchFailure} is an unchecked exception).
{\bf We want to make a nonexhaustive typecase an error.}

For a \KWD{typecase} expression with an identifier rather than a binding,
the identifier must be an immutable variable in scope at that point.
...
the first matching typecase clause is evaluated,
and within that clause,
the static type of the variable
is the intersection of its original type and the guarding type.
Otherwise, it is a static error.

\item \secref{case-expr} Case Expressions

It is a static error if the specified operator is not defined for these types
or if the operator's return type is not \TYP{Boolean}.

 \item Types of branch expressions

The current spec says that the type of a branch expression is the least upper bound of the types of all the branches. This may lose too much precision. We agreed to revise the story as follows: If there exists a branch with type \VAR{T} that the type of every other branch is either subtype of \VAR{T} or is coerced to \VAR{T}, then the type of the branch expression is \VAR{T}; otherwise, the type of the branch expression is the least upper bound of the types of all the branches.

\item Ranges: ``Ranges''
   \begin{itemize}
   \item strided ranges
   \item \TYP{RangeOfStaticSize} and \TYP{StaticRange}
   \item operators: \EXP{\subset}, \EXP{\subseteq}, \EXP{=}, \EXP{\supseteq},
\EXP{\supset}, \EXP{\cap}, \EXP{\left|\ldots\right|}
   \end{itemize}

\item \EXP{\subseteq} in \TYP{Set}: \secref{aggregate-expr}

\item Array expressions:
Section 13.27.4 in Fortress 1.0 beta says that ``The element expressions in an array expression may be either scalars or array expressions themselves. If an element is an array expression, it  (pasted) into the enclosing expression. This pasting works because arrays never contain other arrays as elements.''
 \begin{itemize}
 \item Do we agree?
 \item Consider how it is related to matrix unpasting.
 \item Consider Guy's tensor proposal.
 \item We want both matrix pasting and nested arrays.
 \end{itemize}

\item Array Comprehensions%: \pageref{overviewComp}, \secref{arrays}
 \begin{itemize}
 \item Array comprehensions do not have to be comprehensive -- may not define a value for every cell. What is the value of an unitialized array cell?
Uncaught exception.
 \end{itemize}

\item Reduction variables with \TYP{Monoid}
% :
%  \pageref{evalRedTwo}, \pageref{redexpr-redvar},
%  \pageref{evalMonoid}, \secref{also-block}, \secref{for-expr}, \pageref{caseRed}

\item Keyword parameters and keyword expressions
% :
%  \pageref{nutshellKwd},
%  \pageref{overviewKwd},
%  \pageref{overloadKwd}, \pageref{overloadKwdTwo},
%  \secref{applicability-with-special-params},
%  \pageref{adv-over-kwd}, \secref{subscripting-ops},
%  \secref{subscripted-assignment}, \chapref{types}, \chapref{declarations},
%  \chapref{functions}, \pageref{tuple-expr}

\item \TYP{ReadOnlyHeapSequence} to implement varargs parameters.

\item Varargs expressions%: \pageref{tuple-expr}

\item Case expressions%: \pageref{caseRed}

 \item Abstract function declarations
% :
%  ``Abstract Function Declarations'',
%  \pageref{compAnalysis}, \pageref{compAnalysisTwo}, \chapref{declarations},
%  \secref{methods}

\item Overloaded function contracts: \secref{contracts}

\item Method contracts: \secref{method-contracts}

\item Total coverage of types
 \begin{itemize}
 \item Meet rule
 \item \KWD{typecase} expressions
 \item abstract function declarations
 \item \KWD{throws} clauses
 \item Victor's email titled ``Type relationships: subtyping, exclusion, equivalence, covering, coextension'' on 10/25/07
 \item Victor's email titled ``Revised write-up, plus related issues'' on 10/30/07
 \item Type equivalence
%% trait T comprises{ U, V } end
%% trait U extends T
%%   g():()
%%   h(self):()
%% end
%% trait V extends T
%%   g():()
%%   h(self):()
%% end
%% f(u:U):()
%% f(v:V):()

%% f(t)
%% t.g()
%% h(t)
\begin{Fortress}
\(\KWD{trait}\:T \KWD{comprises}\{\,U, V\,\} \KWD{end}\)\\
\(\KWD{trait}\:U \KWD{extends}\:T\)\\
{\tt~~}\pushtabs\=\+\(  g()\COLONOP()\)\\
\(  h(\KWD{self})\COLONOP()\)\-\\\poptabs
\(\KWD{end}\)\\
\(\KWD{trait}\:V \KWD{extends}\:T\)\\
{\tt~~}\pushtabs\=\+\(  g()\COLONOP()\)\\
\(  h(\KWD{self})\COLONOP()\)\-\\\poptabs
\(\KWD{end}\)\\
\(f(u\COLONOP{}U)\COLONOP()\)\\
\(f(v\COLONOP{}V)\COLONOP()\)\\[4pt]
\(f(t)\)\\
\(t.g()\)\\
\(h(t)\)
\end{Fortress}
All rejected.
 \end{itemize}

\item Type relations:
 \begin{itemize}
 \item Victor's email titled ``Type relationships: subtyping, exclusion, equivalence, covering, coextension'' on 10/25/07
 \item Victor's email titled ``Revised write-up, plus related issues'' on 10/30/07
 \item
The concept of type equivalence (``$S \& T = T \& S$'', for example, or
``$V = S \& T$'') ought to be made more explicit and upfront.
When it's stated that subtypingis reflexive, for example, it's implied that
``reflexive'' is defined in terms of syntactic equality.
 But now we're defining equivalences between different syntactic types,
and ``reflexive'' is actually with respect to these equivalences.  I think
there are also holes in the definition of equivalence -- do we have anywhere,
for example, that ``$S = T \Rightarrow
\EXP{\TYP{Foo}\llbracket{}S\rrbracket} = \EXP{\TYP{Foo}\llbracket{}T\rrbracket}$''?
 \item
If we have a notion of type equivalence, there's no reason to use the earlier
language like ``the intersection of A and B is ...'', implying that
intersection is an operation that produces some canonical representation.
We might as well just state the equivalence ``$A \& B = \dots$'',
making the two freely interchangeable.
 \item
I don't see any reason not to cover fixed-point types here, given that unions
and intersections are similarly unexpressible but are included anyway.
  \item We want the property that type equivalence means the same thing as bidirectional subtyping.
 \end{itemize}

\item Exceptions
\begin{itemize}
\item Chained exceptions: \secref{try-expr}
\end{itemize}

\item Overloading
\begin{itemize}
\item Overloading with static parameters
\begin{itemize}
\item David's ``nice'' overloading proposal (His email titled ``Comparison of generic overloading rules'' sent on 05/18/07)
\item Victor's email titled ``Re: Overloading and the open world assumption''
sent on 11/08/07
\item What is the exclusion rule for a pair of overloaded declarations with different static parameters?
\item Do we want to consider only the exclusion between ground types or also the exclusion between the bounds of static parameters?

\end{itemize}

\item %\secref{overview-function-overloading}
 Function Overloading

Of course, function applications are
statically checked to ensure that \emph{some} definition will be
applicable at run time, and that the definition to apply will
be unambiguous.

  \item A functional which takes a single parameter of type \TYP{Any}
 (or parametric type of bound \TYP{Any}) cannot be overloaded.
% (\pageref{adv-over-any})
\end{itemize}

\item Coercions
% :
%  ``Conversions and Coercions'',
%  ``Overloaded Functional Declarations'' (Appendix),
%  \secref{resolving-overloading}, \chapref{overloaded-declarations},
%  \chapref{types}, \chapref{declarations}, \secref{opr-ident},
%  \secref{trait-decls}, \secref{aggregate-expr}, \secref{principles-overloading}
\begin{itemize}
 \item ``has a type'' vs ``is a subtype of the type or could be coerced to the type''
\item Coercion between tuple types (Victor's email titled ``Coercion between tuple types'' on 07/16/07) and arrow types

\item Vectors and matrices: \secref{aggregate-expr}

\item Do we allow coercion declarations in object expressions?

\item \secref{literals}

The Fortress Standard Library defines coercions
from numerals to integers (for simple numerals)
and rational numbers (for compound numerals).

\item \emph{Future versions of this specification will include a table of
  coercions supported by the standard libraries.}

\item \emph{Future versions of this specification will include a table of
  widening coercions supported by the standard libraries.}

\item If tests and coercion

Suppose a test in an expression has a type that is not \TYP{Boolean}
(or a subtype of \TYP{Boolean}) but is coercible to \TYP{Boolean}.
Presumably, that is not a static error? But note that F1.0 beta
makes no mention of this case; all tests in an if expression must be of
type \TYP{Boolean}. There are similar problems with other expression types;
we need to revisit them all in light of the coercion facility...

\end{itemize}

\item Where clauses and conditional extension
% :
%  ``Where Clauses'', \chapref{declarations}, \chapref{functions},
%  \secref{trait-decls}, \secref{methods}, \secref{func-expr}


\item Intersection and union types: ``Intersection and Union Types'',

  \item Multifix Operators
% : ``Chained and Multifix Operators'',
%   \pageref{opr-intro}, \chapref{operatordefs}, \pageref{opr-multifix}

  \item Synonym Operators: Victor's email titled ``Synonyms [Fwd: Re: What the heck are 0-width spaces for?]'' on 09/12/07

\item Enclosing operator parameters

\item Static Expressions
% : ``Static Expressions'', \pageref{staticParamExpr}

\item Type inference
% :
%  ``Type Inference'',
%  \pageref{arrayAInf},
%  \pageref{arrayBInf},
%  \pageref{arrayCInf},
%  \pageref{setInf}, \pageref{var-inf}, \secref{function-decls}, \secref{idn-ref},
%  \secref{dotted-method-calls}
 \begin{itemize}
 \item
 There seems to be a circular dependency between inference and juxtaposition
 disambiguation -- you have to know if the subexpressions have arrow types
 to disambiguate, but you can't do that without using inference, which
 requires knowing how the variable of unknown type is used.
\item Type Inference (Dan's email titled ``Comments on a first reading of the spec'' on 06/05/07)
\item Do we want to forbid such cases where type inference infers
\TYP{BottomType}s for static parameters?
 \end{itemize}

\item Tests and properties
% :
%  ``Tests and Properties'',
%  \pageref{overviewTest},
%  \pageref{compTest}, \pageref{compOthers}, \chapref{declarations},
%  \secref{trait-decls}

\item Components and APIs
  \begin{itemize}

\item Component system
% :
%  ``Components and APIs'', ``Components and APIs'' (Appendix),
%  ``Imported Declarations and Qualified Names'', \chapref{names},
%  \pageref{philosophyComponent},
%  \pageref{nutshellComponent},
%  \pageref{overviewComponent},
%  \pageref{pgmComponent},
%  \pageref{evalComponet}

\item %\secref{components}
 Components
 \begin{itemize}
\item
It is a static error if
a name specified by an import %%-unqualified
statement
is also declared by a top-level or functional method declaration
in the importing component or API,
or if it is also specified by another import %%-unqualified
statement,
unless all top-level or functional declarations for that name
in the importing component or any of APIs
imported by an import %%-unqualified
statement that specifies that name
form a valid set of overloaded declarations.
It is also a static error if any specified name
is not declared by a declaration imported from the API.
\item
In addition, it is a static error if in a set of overloaded declarations,
any of the following are true:
\begin{enumerate}
\item Any declaration imported on demand is more specific than any explicit declaration.
\item Any top-level function declaration is more specific than any functional method declaration.
(This rule actually applies to declarations in a single component as well.)
\item
Any top-level function declaration imported on demand is more specific than any other declaration not in the same API as the top-level function declaration.
\end{enumerate}
It is also a static error to explicitly import a name from an API
that does not declare it, or to have multiple import-on-demand statements
with the same API.

 \end{itemize}

\item Import statements and qualified names%: \pageref{juxt-qualified}

\item Resolve ambiguity in favor of longest API name.

 \item ``\EXP{\KWD{import}\:A.f}'' does not imply
 ``\EXP{\KWD{import}\;\;\KWD{api}\:A}''.

\item Component/API names with the URL of the development team and the time stamp

 \item API integration
  \begin{itemize}
\item
We agreed to extend Fortress to allow APIs to extend other APIs.  If an API A extends another API B, then every declaration of B is also a declaration of A.  It is a static error if a name declared in B is also declared in A unless both declarations are of top-level functions or functional methods, in which case, the declarations are overloaded.

The extension relation among APIs must be acyclic.
\item Selective extension of specific types
%% api FortressLibrary
%%   extends Whatever.{Boolean}
%%   extends Whatever2.{Thread}
%%   ...
%% end
\begin{Fortress}
\(\KWD{api}\:\TYP{FortressLibrary}\)\\
{\tt~~}\pushtabs\=\+\(  \KWD{extends}\:\TYP{Whatever}.\{\TYP{Boolean}\}\)\\
\(  \KWD{extends}\:\TYP{Whatever2}.\{\TYP{Thread}\}\)\\
\(  \ldots\)\-\\\poptabs
\(\KWD{end}\)
\end{Fortress}
  \end{itemize}

 \item ``\EXP{\KWD{api}\:A; \KWD{import}\;\;\KWD{api}\:B; \KWD{end}}''
 does not export \VAR{B}.
\item How can I re-export stuff I
imported without having to maintain cut-and-pasted declarations in my api?

%% api A
%%   trait T
%%   f(T)
%% end

%% component C
%%   import A.{...}
%%   object O extends T
%%   f(O)
%% end

%% component D
%%   import A.{T}
%%   object O extends T
%%   f(O)
%% end
\begin{Fortress}
\(\KWD{api}\:A\)\\
{\tt~~}\pushtabs\=\+\(  \KWD{trait}\:T\)\\
\(  f(T)\)\-\\\poptabs
\(\KWD{end}\)\\[4pt]
\(\KWD{component}\:C\)\\
{\tt~~}\pushtabs\=\+\(  \KWD{import}\:A.\{\ldots\}\)\\
\(  \KWD{object}\:O \KWD{extends}\:T\)\\
\(  f(O)\)\-\\\poptabs
\(\KWD{end}\)\\[4pt]
\(\KWD{component}\:D\)\\
{\tt~~}\pushtabs\=\+\(  \KWD{import}\:A.\{T\}\)\\
\(  \KWD{object}\:O \KWD{extends}\:T\)\\
\(  f(O)\)\-\\\poptabs
\(\KWD{end}\)
\end{Fortress}

  \end{itemize}

\item Domain specific languages
% :
%  ``Support for Domain-Specific Languages'',
%  \pageref{nutshellDsl},
%  \pageref{overviewDsl},
%  \secref{scanning},
%  \pageref{otherRendering}

\item Automatic Generation of APIs
% :
%  \pageref{overviewAutoAPI}

\item \VAR{copy}: \secref{threadlocal}

\item Distributions
% :
%  \pageref{nutshellDist},
%  \pageref{overviewDist},
%  \pageref{parallel-intro},
%  \secref{regions}, \secref{arrays}, \secref{generators}

\item Extra supports from the parser:
  \begin{itemize}
    \item Lines, pages, and position
% :
%       \secref{position}

    \item Unicode characters for character and string quotation marks
% :
%       \secref{preprocessing-character-literals},
%       \secref{preprocessing-string-literals}

    \item Digit-group separator
% :
%       \pageref{overviewDigit},
%       \secref{characters}, \secref{whitespace}, \secref{numerals},
%       \secref{lexical-operators}

    \item Radix specifier
% :
%       \secref{numerals}, \pageref{numeralsRadix}
  \end{itemize}

\item Tail-call Optimization: \secref{function-app}


\item Static checking: \chapref{types}
 \begin{itemize}
 \item Subtyping and coercion relationships between tuple types and arrow types:
 \chapref{types}

 \item \KWD{comprises}
% :  \pageref{adv-over-comprises}, \chapref{types}
%  \pageref{compComprises}, \secref{trait-decls}
 \begin{itemize}
 \item Are we willing to have type analysis to derive exclusion relationships from object trait types and comprises clauses? For example,
%% trait List[\T\] comprises {Empty, Cons[\T\]} end
%% object Empty extends List[\T\] where {T extends Object} end
%% object Cons[\T\] extends List[\T\] end
%% trait Set[\T\] end
\begin{Fortress}
\(\KWD{trait}\:\TYP{List}\llbracket{}T\rrbracket \KWD{comprises} \{\TYP{Empty}, \TYP{Cons}\llbracket{}T\rrbracket\} \KWD{end}\)\\
\(\KWD{object}\:\TYP{Empty} \KWD{extends}\:\TYP{List}\llbracket{}T\rrbracket \KWD{where} \{T \KWD{extends}\:\TYP{Object}\} \KWD{end}\)\\
\(\KWD{object}\:\TYP{Cons}\llbracket{}T\rrbracket \KWD{extends}\:\TYP{List}\llbracket{}T\rrbracket \KWD{end}\)\\
\(\KWD{trait}\:\TYP{Set}\llbracket{}T\rrbracket \KWD{end}\)
\end{Fortress}
can we derive that \EXP{\TYP{List}\llbracket{}T\rrbracket}
and \EXP{\TYP{Set}\llbracket{}T\rrbracket} are exclusive from the above code?
 {\bf YES}
\item Or do we require excludes clauses for them to be exclusive? {\bf NO}
\item If we are willing to have such an analysis, Victor argued that the semantics that comprises clauses imply exclusion is less compelling. What do others think?
{\bf We infer exclusion relationships based on comprises clauses and objet trait types. comprises clauses do not imply exclusion.}
 \end{itemize}

 \item Definite assignment: ``Definite Assignment''


\item %\secref{overview-aggregate}
 Aggregate Expressions

Whether an array aggregate expression evaluates to an array, a vector, or a
matrix is inferred from context
(e.g., the static type of a variable that such an expression is bound to).

\item %\secref{memory-ops}
 Memory and Memory Operations

The type system and memory model in Fortress guarantee that an
\emph{initializing write} is performed to every location if it is ever read,
and that this write occurs before any read of the location.

\item %\secref{overloading-terms}
 Terminology and Notation

Although there may be multiple declarations with the same functional name,
it is a static error for their static parameters to differ
(up to $\alpha$-equivalence),
or for one declaration to have static parameters and another to not have them.

\item %\secref{functional-applicability}
 Applicability to Named Functional Calls

Note that the static
type of the self parameter is the trait or object trait type in which
the declaration $\f(\Ps):\Us$ occurs.

\item \chapref{tests} Tests and Properties

Tests can refer to non-tests but
it is a static error for a non-test to refer to any test.


\item Dimensions and units
% :
%  ``Dimensions and Units'', ``Dimension and Unit Declarations'',
%  \pageref{nutshellExample},
%  \pageref{overviewDim},
%  \pageref{movingDim},
%  \pageref{fastDim},
%  \pageref{particleDim}, \pageref{compOthers}, \chapref{declarations}
 \begin{itemize}
 \item Dimension and unit parameters
% :
%   ``Dimension and Unit Parameters'', \pageref{overviewDimParam},
%   \secref{type-param}
 \item Dimension and unit operators
% :
%   \secref{reserved-words}, \secref{multichar-enclosers},
%   \pageref{otherRendering}, \pageref{opr-intro}
\item Dimensions and units (Dan's email titled ``Clarifying dimension/unit semantics'' on 08/24/07)
 \end{itemize}

\item %\secref{components}
 Components
 \begin{itemize}
\item
It is a static error if
a name specified by an import %%-unqualified
statement
is also declared by a top-level or functional method declaration
in the importing component or API,
or if it is also specified by another import %%-unqualified
statement,
unless all top-level or functional declarations for that name
in the importing component or any of APIs
imported by an import %%-unqualified
statement that specifies that name
form a valid set of overloaded declarations.
It is also a static error if any specified name
is not declared by a declaration imported from the API.
\item
In addition, it is a static error if in a set of overloaded declarations,
any of the following are true:
\begin{enumerate}
\item Any declaration imported on demand is more specific than any explicit declaration.
\item Any top-level function declaration is more specific than any functional method declaration.
(This rule actually applies to declarations in a single component as well.)
\item
Any top-level function declaration imported on demand is more specific than any other declaration not in the same API as the top-level function declaration.
\end{enumerate}
It is also a static error to explicitly import a name from an API
that does not declare it, or to have multiple import-on-demand statements
with the same API.

 \end{itemize}

\item \chapref{overloaded-declarations} Overloaded Functional Declarations
 \begin{itemize}
\item Furthermore, these rules are checked statically.
\item As a result, these two declarations are statically rejected.
\item These declarations would be statically rejected as ...
 \end{itemize}

 \end{itemize}

\item Library refactoring
 \begin{itemize}
 \item
There are two implicitly imported APIs:
  \begin{itemize}
  \item
\TYP{FortressBuiltin}:
functionality that must not be overridden by the programmer;
referred by unqualified names; \EXP{()}, \EXP{\rightarrow}, tuples,
\TYP{Object}, \TYP{IntLiteral}, \TYP{FloatLiteral}, \TYP{StringLiteral},
\TYP{Boolean}, \EXP{\mathbb{Z}32}, \EXP{\mathbb{Z}64}, \EXP{\mathbb{R}64},
\TYP{Integral}, \TYP{Number}, \TYP{String}, \TYP{Exception},
\TYP{CheckedException}, \TYP{UncheckedException}, \EXP{\sequiv},
\EXP{\TYP{BinaryWord}\llbracket{}n\rrbracket}, ...
  \item
\TYP{FortressLibrary}:
may be hidden or overridden using an explicit import statement
If there's no explicit import statement,
referred by both unqualified and qualifed names; used names are imported
If there are explicit import statements,
it overrides implicitly imported APIs.
  \end{itemize}
 \item
In particular, \TYP{FortressBuiltin} is implicitly imported in \TYP{FortressLibrary}.
 \item
The compiler knows these names even though actual definitions are in libraries. We're using massive recursive imports.
 \item
Types defined in \TYP{FortressBuiltin}, such as \TYP{Any} and \TYP{Object}, are not reserved but they are in the namespace by default.
 \item
Does \TYP{FortressBuiltin} depend only on entities in \TYP{FortressBuiltin}? We hope so.
 \end{itemize}

\item FAQ

\end{itemize}

\section{Syntax}
\begin{itemize}
\item Mathematical notation for alternatives

Victor's email from 03/27/07 15:34
titled ``More mathematical syntax''

\end{itemize}

\section{Types}
\begin{itemize}
\item Guy's proposal for making tensors the principal user-level indexed aggregate type in Fortress
(Guy's email titled ``Arrays, matrices, and tensors (short)'' on 12/15/06)
We discussed the ``story \#1: Arrays''.  Among others,
we discussed how we would require that ``it is forbidden for a subscript on the left-hand-side of an assignment to generate the same integer twice.''
There were three options: 1) require extending a \TYP{NoDuplicate} trait;
2) give an undefined semantics; and 3) give some operational semantics.
Jan and Guy were for the option 3).
We decided to continue the discussion of the ``story \#1: Arrays'' after Guy's detailed write up.
Meanwhile Jan may work on a semantics for the option 3) and all may discuss syntax issues via emails.  Other stories will be discussed after story \#1.


\item Dependent types
  \begin{itemize}
  \item static expressions that are ranges
  \end{itemize}
\item Type inference
\item Conditional extension
\item Selftypes (Eric's email titled ``self types (was Re: isLeftZero: duh)'' sent on 14 Mar 2007)
\item Arrow types with contracts

Function contracts consist of three parts: a \KWD{requires} part,
an \KWD{ensures} part, and an \KWD{invariant} part.
All three parts are evaluated in the scope of the function body
extended with a special variable \emph{arg}, bound to an immutable
array of all function arguments. (This array is useful for describing
contracts in the presence of higher-order functions;
see \secref{arrow-types}).

Is the special variable \emph{arg} reserved?
What if one of the function's parameters is named \VAR{arg}?

At runtime, when a function is bound to a variable or parameter
$f$ whose type includes a contract,
the contract is evaluated when the function is called through a reference to $f$.
(Note that this contract is distinct from the contract attached to the function
bound to $f$, which is also evaluated upon a call to $f$).
This contract is evaluated in the enclosing scope of the arrow type,
extended with any keyword argument names provided in the arrow type,
and the special variable $arg$
bound to an immutable array containing the arguments provided at the call site
(in the order provided at the call site).
If the \KWD{requires} clauses stipulated in the type of $f$ are not satisfied,
a \TYP{CallerViolation} exception is thrown.
Otherwise, if the \KWD{requires} clauses attached to the function bound to $f$ is not satisfied,
a \TYP{ContractBinding} exception is thrown.
If any other part of the contract of the function bound to $f$ is not satisfied,
a \TYP{CalleeViolation} exception is thrown.
Otherwise, if any part of the contract stipulated in the type of $f$ is not satisfied,
a \TYP{ContractBinding} exception is thrown.

  \begin{itemize}
\item Function expressions with contracts

We want to allow function expressions to carry contracts; we need to work
out how these contracts are evaluated, and what the implications are on
variable scoping. -- Eric
  \end{itemize}

\item Functions as objects
  \begin{itemize}
  \item A supertype, say Function, of all arrow types.

  \item Arrow types are traits that can be extended by other traits.
    Subtraits of arrow types should define or declare a method with the
    name ``apply'' (or ``call'', as you wish).  More details from Sam's
    proposal here.

Is the method called ``opr juxtaposition''??  Or would that just mess  up
operator precedence?
-- Jan

  \end{itemize}

\item \ignore

We're considering supporting \ignore\ (pronounced ``whatever'') not only for
variables but also for type variables.

We have several use cases for this:
 \begin{enumerate}
 \item Eric doesn't need to define extra trait \TYP{List} in addition to
        \EXP{\TYP{List}\llbracket{}T\rrbracket}.
 \item Jan can use \EXP{\TYP{List}\llbracket\,\_\,\rrbracket} in a
     \KWD{typecase} expression.
 \item Jan can use \ignore\ for redundant static parameters (required by
     overloading restrictions) for overloaded methods.
 \item We have several proposals for supporting some forms of pattern
        matching including David and Sam's and Eric's.
 \end{enumerate}

   This requires extending the left-hand-sides of \KWD{typecase} clauses to
   be binding places of type variables.

   There are several problems:
 \begin{enumerate}
     \item It's unclear whether
     \EXP{\TYP{List}\llbracket\TYP{String}\rrbracket} is a binding place or
     a     reference        of \TYP{String} in a \KWD{typecase} expression.
     \item Using \ignore in any place where a type variable can occur would
        have
        the same power with ``wildcards'' in Java.
     \item Restricting occurences of \ignore\ seems to be ad hoc.
 \end{enumerate}

Jan's new example:

% g(x:A[\T\]) where { T extends Object } =
%     typecase x of
%       A[\U\] where { U extends Object } => ...
%       ...
%     end
\begin{Fortress}
{\tt~}\pushtabs\=\+\( g(x\COLONOP{}A\llbracket{}T\rrbracket) \KWD{where} \{\,T \KWD{extends}\:\TYP{Object}\,\} =\)\\
{\tt~~~~}\pushtabs\=\+\(     \KWD{typecase}\:x \KWD{of}\)\\
{\tt~~}\pushtabs\=\+\(       A\llbracket{}U\rrbracket \KWD{where} \{\,U \KWD{extends}\:\TYP{Object}\,\} \Rightarrow \ldots\)\\
\(       \ldots\)\-\\\poptabs
\(     \KWD{end}\)\-\-\\\poptabs\poptabs
\end{Fortress}
And the interesting question is to what extent we're allowed to use \VAR{U}
(or not) given the existence of stuff like \TYP{Empty}.
My hope had been that we could do stuff like type-equality tests this way,
but that hope may have been a vain one.  And this syntax is just a pipe dream
which I'm faking using overloading (and the dreaded \TYP{\_\_Proxy} type)
for the moment.

  \item \secref{where-clauses} Where Clauses
    \begin{itemize}
    \item
It must be possible to infer which method is referred to at the call site.
If there is not enough information to infer which method is called,
type checking rejects the
program and requires more type information from the programmer.
Programmers always can provide more type information by using type
ascription as described in \secref{type-ascription}.
    \item
Where can a where-clauses type variable occur?
    \end{itemize}

\end{itemize}

\section{Declarations}
\begin{itemize}
\item Tuples
\begin{itemize}
\item Variable declarations with ``...''

%(x: ZZ, rest: ZZ...) = f(z)
\begin{Fortress}
\((x\COLON \mathbb{Z}, \VAR{rest}\COLON \mathbb{Z}\ldots) = f(z)\)
\end{Fortress}
where \VAR{f} has type %ZZ -> (ZZ, ZZ...)
\EXP{\mathbb{Z} \rightarrow (\mathbb{Z}, \mathbb{Z}\ldots)}.

\item I don't think the current syntax permits me to write:

%f : ((A,B),C) -> D
%f((a : A, b : B), c : C) = ...
\begin{Fortress}
\(f \mathrel{\mathtt{:}} ((A,B),C) \rightarrow D\)\\
\(f((a \mathrel{\mathtt{:}} A, b \mathrel{\mathtt{:}} B), c \mathrel{\mathtt{:}} C) = \ldots\)
\end{Fortress}

Is this desirable? Right now there's a bit of ruler-across-the-knuckles
going on with complex desugarings of generator lists. Minor tweaks to
\emph{ValParams} might accomplish this.
\end{itemize}

\item More powerful type aliases

For floating-point numbers, Fortress supports types {$\mathbb{R}$32} and
{$\mathbb{R}$64}
to be 32 and 64-bit IEEE 754 floating-point numbers respectively,
and defines two functions on types:
\EXP{\TYP{Double}\bsTP{F}} is a floating-point type twice the size of the
floating-point type \EXP{F}, and \EXP{\TYP{Extended}\bsTP{F}} is a
floating-point
type sufficiently larger than the floating-point type \EXP{F} to
perform summations of ``reasonable'' size.\footnote{
This formulation of floating-point types follows a proposal under
consideration by the IEEE 754 committee.}

% type Double[\F extends FloatNumber[\e, s, ...\]\] where { nat e, nat s, ... }
%    = FloatNumber[\e+3, 2 s - 6, ... \]
\begin{Fortress}
{\tt~}\pushtabs\=\+\( \KWD{type}\:\TYP{Double}\llbracket{}F \KWD{extends}\:\TYP{FloatNumber}\llbracket{}e, s, \ldots\rrbracket\rrbracket \KWD{where} \{\,\KWD{nat}\:e, \KWD{nat}\:s, \ldots\,\}\)\\
{\tt~~~}\pushtabs\=\+\(    = \TYP{FloatNumber}\llbracket{}e+3, 2 s - 6, \ldots\,\rrbracket\)\-\-\\\poptabs\poptabs
\end{Fortress}

\end{itemize}

\section{Traits}
\begin{itemize}
\item Do we want to allow object trait types to appear in \KWD{extends} clauses?
(Jan's email titled ``Re: Question about not\_passing\_yet/extendsParam.fss'' sent
on 14 Nov 2007)
\item Type variables in \KWD{excludes} clauses

%% trait Indexed [\I extends Equality\] excludes Indexed[\U\] where [\U\]{I excludes U}
%%   ...
%% end
\begin{Fortress}
\(\KWD{trait}\:\TYP{Indexed} \llbracket{}I \KWD{extends}\:\TYP{Equality}\rrbracket \KWD{excludes}\:\TYP{Indexed}\llbracket{}U\rrbracket \KWD{where} \llbracket{}U\rrbracket\{I \KWD{excludes}\:U\}\)\\
{\tt~~}\pushtabs\=\+\(  \ldots\)\-\\\poptabs
\(\KWD{end}\)
\end{Fortress}
\end{itemize}

\section{Functions}
\begin{itemize}
\item %% \secref{functional-applicability}
Applicability to Named Functional Calls

whether a ``named functional call'' is rewritten to a ``dotted method call''
only if the function name is the name of a method provided by the enclosing
trait/object declaration or object expression (this requires a slight rewriting
of the relevant paragraph).
%% in \secref{functional-applicability}).

\item Relaxing restrictions on static parameters of overloaded functionals
 or replace static parameters with where clauses as much as possible
  \begin{itemize}
  \item Jan's email titled ``Re: Issues related to ``spec'' vs ``nice'' overloading of generics'' on 05/21/07
  \item Jan's email titled ``Another overloading conundrum'' on 04/06/07
  \item Jan's example
%% array1[\T extends Object, nat s0\]():Array1[\T,0,s0\] = __DefaultArray1[\T,0,s0\]()
%% array1[\T extends Number, nat s0\]():Vector[\T,s0\] = vector[\T,s0\]()
\begin{Fortress}
\({array}_{1}\llbracket{}T \KWD{extends}\:\TYP{Object}, \KWD{nat}\:s_{0}\rrbracket()\COLONOP\TYP{Array1}\llbracket{}T,0,s_{0}\rrbracket = \_\_DefaultArray1\llbracket{}T,0,s_{0}\rrbracket()\)\\
\({array}_{1}\llbracket{}T \KWD{extends}\:\TYP{Number}, \KWD{nat}\:s_{0}\rrbracket()\COLONOP\TYP{Vector}\llbracket{}T,s_{0}\rrbracket = \VAR{vector}\llbracket{}T,s_{0}\rrbracket()\)
\end{Fortress}

  \item Marc's example
%% conjugate[\T extends Number\](x:Complex[\T\]):Complex[\T\]
%% conjugate[\T extends RealNumber\](x:T):T
\begin{Fortress}
\(\VAR{conjugate}\llbracket{}T \KWD{extends}\:\TYP{Number}\rrbracket(x\COLONOP\TYP{Complex}\llbracket{}T\rrbracket)\COLONOP\TYP{Complex}\llbracket{}T\rrbracket\)\\
\(\VAR{conjugate}\llbracket{}T \KWD{extends}\:\TYP{RealNumber}\rrbracket(x\COLONOP{}T)\COLONOP{}T\)
\end{Fortress}
  \end{itemize}
\item Overloaded functionals with different \KWD{throws} clauses
\end{itemize}

\section{Expressions}
\begin{itemize}
\item Exiting from a function expression

Jan's email from 06/05/08 12:01
titled ``Re: cleaning up exceptions hierarchy''

Eric proposed to statically disallow it.

\item Types of branch expressions:
We discussed propagation of the type information around the expression to the subexpressions of a branch expression. We may want to include this propagation later.
%% z: ZZ32
%% if p then z else 0 end
%% if p then z else Identity[\+\] end
%% if p then 0 else Identity[\+\] end

%% a: ZZ32 = if p then z else FFFF0000FFFF_16 end
\begin{Fortress}
\(z\COLON \mathbb{Z}32\)\\
\(\KWD{if}\:p \KWD{then}\:z \KWD{else} 0 \KWD{end}\)\\
\(\KWD{if}\:p \KWD{then}\:z \KWD{else}\:\TYP{Identity}\llbracket+\rrbracket \KWD{end}\)\\
\(\KWD{if}\:p \KWD{then} 0 \KWD{else}\:\TYP{Identity}\llbracket+\rrbracket \KWD{end}\)\\[4pt]
\(a\COLON \mathbb{Z}32 = \;\KWD{if}\:p \KWD{then}\:z \KWD{else} \mathtt{FFFF0000FFFF}_{16} \KWD{end}\)
\end{Fortress}

\item IO within \KWD{atomic} expressions

\secref{spawn},
Jan's email from 07/31/06 15:51
titled ``Trial balloon: I/O for Fortress?''

\item Definite assignments (Victor's email titled ``Committed revision 4582, variable initialization questions'' on 08/02/07)
\item Static checking for MatchFailure (Dan's email titled ``Re: Typecase questions'' on 08/07/07)
\item Variation of for loop:

% for generators do
%     body1
% else
%     body2
% end
\begin{Fortress}
{\tt~}\pushtabs\=\+\( \KWD{for}  \null\)\pushtabs\=\+\(\VAR{generators} \KWD{do}\)\\
\(     {body}_{1}\)\-\\\poptabs
\( \KWD{else}\)\\
{\tt~~~~}\pushtabs\=\+\(     {body}_{2}\)\-\\\poptabs
\( \KWD{end}\)\-\\\poptabs
\end{Fortress}
The \emph{body}$_1$ is executed once for each iteration generated by the \emph{generators}, but if the \emph{generators} generate no iterations, then \emph{body}$_2$ is executed once.
\end{itemize}

\section{Exceptions}
\begin{itemize}
\item Exception behavior of atomic expressions in Fortress
  \begin{itemize}
  \item Checked exceptions must be caught before they escape from the atomic expression, or it is a static error.  If necessary, programmers can be wrap a checked exception in an unchecked exception to get it out of the transaction.
  \item There is some syntax which says which checked exceptions are allowed to escape from an atomic expression.
  \item They propagate to the context of the atomic, as above.  This is the de facto behavior in the absence of a story about checked exceptions, since it's the same as the behavior for unchecked exceptions. (the status quo)
  \end{itemize}
\item Naked type variables in a throws clause or in a catch clause.

Since a \KWD{throws} clause is a set, for full generality, we would
probably need expressible union types at the call site,
or a variable representing a ``set'' of types.

\item When we have our \TYP{StackTraceElement} trait,
add the following methods to the \TYP{Exception} trait:
\begin{itemize}
\item \EXP{\VAR{fillInStackTrace}()\COLON \TYP{Exception}}:
Called to initialize the stack trace data in a newly created exception.
This method can be called more than once.
\item \EXP{\VAR{getStackTrace}()\COLON \TYP{StackTraceElement}[]}
\item \EXP{\VAR{setStackTrace}(\VAR{stackTrace}\COLON \TYP{StackTraceElement}[])\COLON ()}:
\end{itemize}

\item light-weight exceptions
\item interaction between exceptions and transactions
\item interaction between exceptions and multithreads

\item syntactic sugar for expand/wrap/catch/unwrap idiom

It occurs to me that we could have a very concise form of syntactic  sugar
for this expand/wrap/catch/unwrap idiom via a syntax expander  (or maybe
just primitive syntactic sugar if necessary), which I'll  call ``handle''.
This expander eta expands every function argument  (promoting every checked
exception of the function that is not  anticipated by the callee), and
wraps the whole function call in a  \KWD{try}/\KWD{catch} block that
unwraps the exceptions. For example, with the  above definition of \VAR{f},
and the following function \VAR{g}:

% g(ZZ):ZZ throws {IOFailure, SQLError} = ...
\begin{Fortress}
{\tt~}\pushtabs\=\+\( g(\mathbb{Z})\COLONOP\mathbb{Z} \KWD{throws} \{\TYP{IOFailure}, \TYP{SQLError}\} = \ldots\)\-\\\poptabs
\end{Fortress}

the expression:

% handle[ f[\ZZ,ZZ,3\](g) ]
\begin{Fortress}
{\tt~}\pushtabs\=\+\( \VAR{handle}\llbracket,f\llbracket\mathbb{Z},\mathbb{Z},3\rrbracket(g)\,]\)\-\\\poptabs
\end{Fortress}

expands to:

% try
%   f[\ZZ,ZZ,3\]
%     (fn n =>
%       try g n
%       catch e
%         SQLError => throw WrappedException(e)
%       end)
% catch e
%   WrappedException[\SQLError\] => throw e.inner
% end
\begin{Fortress}
{\tt~}\pushtabs\=\+\( \KWD{try}\)\\
{\tt~~}\pushtabs\=\+\(   f\llbracket\VAR{null}\)\pushtabs\=\+\(\ZZ,\mathbb{Z},3\rrbracket\)\\
\(     (\KWD{fn} n \Rightarrow\)\\
{\tt~~}\pushtabs\=\+\(       \KWD{try} g n\)\\
\(       \KWD{catch} e\)\\
{\tt~~}\pushtabs\=\+\(         \TYP{SQLError} \Rightarrow \KWD{throw} \TYP{WrappedException}(e)\)\-\\\poptabs
\(       \KWD{end})\)\-\-\-\\\poptabs\poptabs\poptabs
\( \KWD{catch} e\)\\
{\tt~~}\pushtabs\=\+\(   \TYP{WrappedException}\llbracket\TYP{SQLError}\rrbracket \Rightarrow \KWD{throw} e.\VAR{inner}\)\-\\\poptabs
\( \KWD{end}\)\-\\\poptabs
\end{Fortress}

Now, the context of the call site must deal with a thrown
\TYP{SQLError}, or
a static error is signaled. Moreover, without  ``handle'' the call

% f[\ZZ,ZZ,3\](g)
\begin{Fortress}
{\tt~}\pushtabs\=\+\( f\llbracket\mathbb{Z},\mathbb{Z},3\rrbracket(g)\)\-\\\poptabs
\end{Fortress}

will result in a static error.

-- Eric

\item \KWD{throws} clauses with naked type variables

\item Inference for \KWD{throws} clauses

No, and we shouldn't. \KWD{throws} clauses are documentation that a
programmer should provide in a function's header.  -- Eric

Okay, but I thought Jan wanted to do this.  I thought Sukyoung did too,
actually.  Only for ``internal'' functions though.  Probably I
misunderstood what they were saying--it was related to that talk at OOPSLA
last year.  Anyway, I don't think we should worry about this too much until
we get type inference for function/method calls worked out. -- Victor

\end{itemize}

\section{Overloading}
\begin{itemize}
\item Identifying the intersection of any two types with \KWD{comprises} clauses with the union of their common subtypes

%% trait S comprises {U, V} end
%% trait T comprises {V, W} end
%% trait U extends S excludes W end
%% trait V extends {S, T} end
%% trait W extends T end
\begin{Fortress}
\(\KWD{trait}\:S \KWD{comprises} \{U, V\} \KWD{end}\)\\
\(\KWD{trait}\:T \KWD{comprises} \{V, W\} \KWD{end}\)\\
\(\KWD{trait}\:U \KWD{extends}\:S \KWD{excludes}\:W \KWD{end}\)\\
\(\KWD{trait}\:V \KWD{extends} \{S, T\} \KWD{end}\)\\
\(\KWD{trait}\:W \KWD{extends}\:T \KWD{end}\)
\end{Fortress}

\txt{S\&T} \verb+~~+ \txt{V}
We agreed to revise the Meet rule to address this with the coverage check.

  \item overloading in multiple scopes
  \item overloading with different static parameters
  \item exporting partial overloading

I'm hoping we export all of an overloading, or none of it. -- Jan

This a reasonable restriction that I'm happy to back, unless someone
has both a use case and a soundness proof for a less restrictive
rule. :-) -- Eric

The use case I have in mind is that within a component, you might
manipulate internal data types, and use overloading on those types
with methods that are also exported.  I don't have a specific example
in mind, but this seems like a useful and natural feature to have.  So
I'd like to revisit this issue sometime, but not before 1.0. :) -- Victor

\end{itemize}

\section{Operators}
\begin{itemize}
\item Restricted fixity operators (Victor's email titled ``Re: Magic factorial'' on 09/10/07)

\item \appref{operator-precedence}

(In the future,
we may allow tokens with mixtures of `\texttt{/}' and `\texttt{\char'134}',
in which case the left and right brackets should match from outside in.
But for now, such tokens are simply illegal.)

(As in case %\ref{multicharacter brackets, case 2} 
above,
we may allow tokens with mixtures of `\texttt{/}' and `\texttt{\char'134}'
in the future.)
\end{itemize}

\section{Tests and Properties}
\begin{itemize}
\item Properties (Eric's email titled ``Properties, concurrency'' on 07/30/07)
\end{itemize}

\section{Components and APIs}
\begin{itemize}
\item %\secref{apis} 
APIs (Victor)

Jan's email ``Re: Correct API syntax for object type without constructor''
July 9, 2007 8:14:09 AM

\item Namespace of APIs (Victor)

Because API names consist of one or more identifiers separated by dots, and ``field access'' (i.e., getter and setter invocations) may produce similar sequences, we need some rule here. One possibility is to forbid the declaration of any (simple) name in a component from being the same as the first identifier of any name of an imported API.

% api A.B
%  f():()
% end

% component C
%  export Executable
%  import A.B
%  object O
%    f() = ()
%  end
%  object A
%    B: T = O
%  end
%  run(args) = A.B.f()
% end
\begin{Fortress}
{\tt~}\pushtabs\=\+\( \KWD{api}\:A.B\)\\
{\tt~}\pushtabs\=\+\(  f()\COLONOP()\)\-\\\poptabs
\( \KWD{end}\)\\[4pt]
\( \KWD{component}\:C\)\\
{\tt~}\pushtabs\=\+\(  \KWD{export}\:\TYP{Executable}\)\\
\(  \KWD{import}\:A.B\)\\
\(  \KWD{object}\:O\)\\
{\tt~~}\pushtabs\=\+\(    f() = ()\)\-\\\poptabs
\(  \KWD{end}\)\\
\(  \KWD{object}\:A\)\\
{\tt~~}\pushtabs\=\+\(    B\COLON T = O\)\-\\\poptabs
\(  \KWD{end}\)\\
\(  \VAR{run}(\VAR{args}) = A.B.f()\)\-\\\poptabs
\( \KWD{end}\)\-\\\poptabs
\end{Fortress}
We agreed on the high-level idea but the details should be worked out more.
%Refer \secref{qualified-names}.

\item API evolution / subtyping
\item Automatically imported APIs (Victor's email titled
``Automatically imported APIs (was: Dan's unresolved issues)'' on 08/31/07)

  \item \secref{basicops} {\bf Extract and Install}

``The set of APIs must be closed under imports.'' What if standard
  lib APIs are imported?
  \item \secref{advancedops} {\bf Constrain}

 ``If we constrain an
  API that is used by any other API exported by the component, then we must
  also constrain that other API.'' This seems problematic. Suppose that two
  components both export a standard lib. Is there any way to link them?

(Possible solution: Perhaps we should provide a facility for ``flipping''
an API export to an import. A flipped API, when linked, could upgrade its
constituent.)

  \item Exact matching between components and APIs
    \begin{itemize}
    \item
% api A
%   trait T end
%   trait V extends T end
% end
%
% component C
% export A
%   trait T end
%   trait U extends T end
%   trait V extends U end
% end
\begin{Fortress}
{\tt~}\pushtabs\=\+\( \KWD{api} A\)\\
{\tt~~}\pushtabs\=\+\(   \KWD{trait} T \KWD{end}\)\\
\(   \KWD{trait} V \KWD{extends} T \KWD{end}\)\-\\\poptabs
\( \KWD{end}\)\\[4pt]
\( \KWD{component} C\)\\
\( \KWD{export} A\)\\
{\tt~~}\pushtabs\=\+\(   \KWD{trait} T \KWD{end}\)\\
\(   \KWD{trait} U \KWD{extends} T \KWD{end}\)\\
\(   \KWD{trait} V \KWD{extends} U \KWD{end}\)\-\\\poptabs
\( \KWD{end}\)\-\\\poptabs
\end{Fortress}
    \item
% trait A end
% private trait B end
% trait C extends {A,B} end
\begin{Fortress}
{\tt~}\pushtabs\=\+\( \KWD{trait} A \KWD{end}\)\\
\( \KWD{private}\;\;\KWD{trait} B \KWD{end}\)\\
\( \KWD{trait} C \KWD{extends} \{A,B\} \KWD{end}\)\-\\\poptabs
\end{Fortress}
    \end{itemize}

  \item Exporting multiple APIs with overlapping names
% api A
%   e():ZZ32
%   trait T
%     f(): T
%   end
% end
% api B
%   e():ZZ32
%   trait T
%     g(): ZZ32
%   end
% end
% component G
%   export {A, B}
%   e() = 42
%   trait A.T
%     f() = self
%   end
%   trait B.T
%     g() = 13
%   end
% end
\begin{Fortress}
{\tt~}\pushtabs\=\+\( \KWD{api} A\)\\
{\tt~~}\pushtabs\=\+\(   e()\COLONOP\mathbb{Z}32\)\\
\(   \KWD{trait} T\)\\
{\tt~~}\pushtabs\=\+\(     f()\COLON T\)\-\\\poptabs
\(   \KWD{end}\)\-\\\poptabs
\( \KWD{end}\)\\
\( \KWD{api} B\)\\
{\tt~~}\pushtabs\=\+\(   e()\COLONOP\mathbb{Z}32\)\\
\(   \KWD{trait} T\)\\
{\tt~~}\pushtabs\=\+\(     g()\COLON \mathbb{Z}32\)\-\\\poptabs
\(   \KWD{end}\)\-\\\poptabs
\( \KWD{end}\)\\
\( \KWD{component} G\)\\
{\tt~~}\pushtabs\=\+\(   \KWD{export} \{A, B\}\)\\
\(   e() = 42\)\\
\(   \KWD{trait} A.T\)\\
{\tt~~}\pushtabs\=\+\(     f() = \mathord{\KWD{self}}\)\-\\\poptabs
\(   \KWD{end}\)\\
\(   \KWD{trait} B.T\)\\
{\tt~~}\pushtabs\=\+\(     g() = 13\)\-\\\poptabs
\(   \KWD{end}\)\-\\\poptabs
\( \KWD{end}\)\-\\\poptabs
\end{Fortress}


  \item Why not allow components to import and export the same API, but
  prevent cyclic component linking (except for a special
  \shellcommand{linkCyclic} command)?

  \item Subtyping of APIs  (Is this just imports?  No.)
%component C extends A (* an API *)
\begin{Fortress}
\(\KWD{component} C \KWD{extends} A \mathtt{(*}\;\hbox{\rm  an API \unskip}\;\mathtt{*)}\)
\end{Fortress}
  \item Coercion of APIs?  Widening?
  \item Deprecation of API members in new versions?
\end{itemize}

\section{Parallelism}
\begin{itemize}
\item Generators with natural order

it seems like there's still a question about whether every generator really does have a ``natural order''.  I guess that if some generators don't have a natural order, then either you can't call sequential on them, or else different calls to sequential may get a different order.

\item \secref{distributions}

In future versions of this specification, we may
distinguish \VAR{local} and \VAR{sequential} distributions for this
purpose.
\end{itemize}

\section{Libraries}
\begin{itemize}
\item %\secref{opr-lattices} 
Lattices

If a lattice has both meet and join bounds, then the
identity (the bound) for either is a zero for the other, and therefore
THE zero for the other, because meet and join are commutative.
I'll add this.

\secref{advanced-rationals} The Trait Fortress.Standard.RationalQuantity

I found that the trait \TYP{RationalQuantity} declares a coercion from
\TYP{IntegerQuantity} but \EXP{\mathbb{Q}} does not.
\TYP{RationalQuantity}, \EXP{\mathbb{Q}}, and \EXP{\mathbb{Z}} don't
declare coercions from corresponding literals either.

\item \EXP{\mathbb{Q}} and \EXP{\mathbb{Z}}
%% \chapref{lib:numbers-basic}

\emph{Future versions of this specification will include trait declarations
  definitions for the other numeric quantities.}

  \item Define a top-level function `$:\sim:$' and make a refernce.
%% from \secref{conditional-operators-impl}.
\end{itemize}

\section{Others}
\begin{itemize}
\item Native stuff (Victor's email titled ``Re: Thinking about native stuff some more'' on 08/21/07)
  \begin{itemize}
  \item foreign function interface
  \item reflection mechanism
  \end{itemize}

\item Purity and ``sandbox''
  \begin{itemize}
  \item How about a ``sandbox'' function that executes a function for its
  value, throwing away all effects?  It is a call site annotation to call
  a non-pure function in a pure context.
  \item What do we want we purity?  What semantics can programmers exploit?
  \item Is purity a part of the type?
  \item purity for enabling CSE?
  \item purity for David's example?
  \end{itemize}

\item \KWD{idiom}

Guy's email titled ``Notes on today's meeting'' on 12/12/05

Idioms are like properties but they are hints to compilers.  Like
properties, idioms describe that the left-hand-side and right-hand-side of
$\Rightarrow$ are same.  Unlike properties, idioms instruct compilers to
replace any occurrence of the left-hand-side of $\Rightarrow$ to the
right-hand-side of $\Rightarrow$.  For example,
%idiom FORALL (x: ZZ, y : ZZ) floor(x/y) => floordiv(x,y)
%idiom x/N (log2 N=k) => x >> k
%idiom floor(x/(MAXINT-1)) => 0
%test data[] = { -MAXINT-1, -MAXINT, -MAXINT+1, -2, -1, 0, 1, 2, MAXINT-1, MAXINT, 43 }
\begin{Fortress}
\(\KWD{idiom} \forall (x\COLON \mathbb{Z}, y \mathrel{\mathtt{:}} \mathbb{Z})\; \VAR{floor}(x/y) \Rightarrow \VAR{floordiv}(x,y)\)\\
\(\KWD{idiom} x/N ({log}_{2} N=k) \Rightarrow x \gg k\)\\
\(\KWD{idiom} \VAR{floor}(x/(\OPR{MAXINT}-1)) \Rightarrow 0\)\\
\(\KWD{test} \VAR{data}[] = \{\,-\OPR{MAXINT}-1, -\OPR{MAXINT}, -\OPR{MAXINT}+1, -2, -1, 0, 1, 2, \OPR{MAXINT}-1, \OPR{MAXINT}, 43\,\} \)
\end{Fortress}
[The last two idioms were by David, intended as illustrations
of screw cases.]
a unit testing framework should check that
$floor(x/y)$ and $floordiv(x,y)$ are same and the compiler should replace
any occurrence of $floor(x/y)$ to $floordiv(x,y)$.

\item Pattern matching mechanism

\item Chapter Compliant Implementation

I suppose it might help clarify things if we include a new chapter  that
states the properties that must hold for a ``compliant  implementation''.
For example, we could say things like ``A compliant  implementation
provides a ``compile'' procedure that takes a sequence  of Unicode
characters as input.  If this input is not a valid Fortress  program, it is
a static error.  Otherwise...'' -- Eric

\item Rewriting inheritance in terms of implicit declarations so that they
  are not parts of the "inner theory core".

\item Equivalence and equality:
  Equivalence $\equiv$ ought to be built in (ie automagically overridden by every trait / object declaration).
  Equality $=$ is overridable and should be for apples-to-apples comparisons.  By default equality is equivalence.  It may be its own trait, with equivalence as the default definition.
  Right now equivalence on reference objects is pointer equality, and equivalence on value objects is equality of content.  Equality on functions (and thus on value objects with function-typed fields) throws an exception.  The first bone of contention is equivalence on immutable reference objects, with the following options:
  \begin{itemize}
     \item Always return False.  This makes the default definition of equality junk, and is strictly less expressive than equivalence on mutable reference objects.  It seems bad, but was considered.
     \item As any mutable reference object, using pointer equality (current semantics).  This explicitly rules out combining indistinguishable instances of immutable reference objects---they can still be distinguished by $\equiv$ and thus must be kept distinct.  Thus, for example, a string constant mentioned in the body of a loop must be copied at each loop iteration.  This permits us to give a deterministic semantics.
     \item If mutable reference objects would be considered equivalent, immutable reference objects would be considered equivalent.  If immutable reference objects are equivalent, they must be otherwise semantically indistinguishable.  This is \emph{ad hoc} and lies between the previous solution and the next one.
     \item By content, except that if any field has function type (or perhaps if equivalence checking of fields throws that exception, since subtyping may make this ambiguous) we treat it like a mutable reference object.  This would rule out pointer equality unless we did hash consing, so immutable reference objects would become more expensive than their mutable counterparts (either equivalence checks would get more expensive, or allocation would become more expensive).
     \item By content.  This causes equivalence on immutable reference objects to fail if they contain functions, even when an equivalent mutable reference object would not.  This may encourage programmers to throw in gratuitous unused mutable fields, and is thus probably a bad idea.
  \end{itemize}
  The second bone of contention is function equality.  Note that changing the definition of function equality may make some of the choices above more or less usable.
  \begin{itemize}
     \item Always return false.  Less bad than for objects, but still deceptive.  Would mean that equality on value objects with function-typed fields always returns false, too.
     \item Always throw an exception (current semantics).  Equality on value objects with function-typed fields always throws the same exception.
     \item When $f\equiv g$ the functions must be semantically interchangeable.  Otherwise the functions may or may not be semantically interchangeable.  Note that always returning false is a valid implementation of this semantics, and that there is non-determinism here.
  \end{itemize}
A final issue to settle is what happens when we compare functions to other objects.  Do we throw an exception, or just return ``false''?

\item Embedding figures and links into Fortress code?
 \begin{itemize}
 \item
How does one embed figures into Fortress code? E.g., GIF, JPG, ODF, etc. (Maybe this is an IDE question rather than a languages question... dunno.) I think it's sufficient to be able to embed figures into comments.
 \item
How does one embed links into Fortress code? E.g., to URLs, to other parts of the source code, etc. (Maybe this is an IDE question rather than a languages question... dunno.) It may be sufficient to be able to embed links into comments... dunno. I guess it's a matter of (sudokode here)
%    (* <a href=mylink> *) v_max := blahblahblah
\begin{Fortress}
{\tt~~~~}\pushtabs\=\+\(    \mathtt{(*}\;\hbox{\rm  <a href=mylink> \unskip}\;\mathtt{*)} v_{\mathrm{max}} \ASSIGN \VAR{blahblahblah}\)\-\\\poptabs
\end{Fortress}
versus
%    <a href=mylink> v_max := blahblahblah
\begin{Fortress}
{\tt~~~~}\pushtabs\=\+\(    <a\:\VAR{href}=\VAR{mylink}> v_{\mathrm{max}} \ASSIGN \VAR{blahblahblah}\)\-\\\poptabs
\end{Fortress}
or something. (I actually have no idea how I would want this done. I'm just not crazy about embedding a comment when its only purpose is to host a link.)
 \item
We agreed that it would be nice to have a such a thing, and only within comments. I don't think we decided on a particular marker to indicate such things though.
 \end{itemize}

\item What are the chances of being able to comment out the rest of a line simply by adding a single character?
 \begin{itemize}
 \item
One thing cool about Fortran is the ability to add comments with a single character.
 \item
Having ``left-hand'' comments would be nice as well.
 \item
We agreed to leave this alone until it actually hurts.
 \end{itemize}

\end{itemize}


\section{Examples}
\subsection{Scope of inherited method declarations}
% f(x: ZZ) = x+1
% trait T
%   f(x: ZZ): ZZ = x+2
% end
% object O extends T
%   g(x: ZZ) = f(x) + 3
% end
\begin{Fortress}
{\tt~}\pushtabs\=\+\( f(x\COLON \mathbb{Z}) = x+1\)\\
\( \KWD{trait}\:T\)\\
{\tt~~}\pushtabs\=\+\(   f(x\COLON \mathbb{Z})\COLON \mathbb{Z} = x+2\)\-\\\poptabs
\( \KWD{end}\)\\
\( \KWD{object}\:O \KWD{extends}\:T\)\\
{\tt~~}\pushtabs\=\+\(   g(x\COLON \mathbb{Z}) = f(x) + 3\)\-\\\poptabs
\( \KWD{end}\)\-\\\poptabs
\end{Fortress}

Allow this and \EXP{O.g(4) = 9}.

\subsection{Scope of \KWD{self}}
% trait T
%   getter y(): ZZ
% end
% object O(x: ZZ)
%   n() = object extends T
%           y: ZZ = self.x
%         end
% end
\begin{Fortress}
{\tt~}\pushtabs\=\+\( \KWD{trait}\:T\)\\
{\tt~~}\pushtabs\=\+\(   \KWD{getter}\:y()\COLON \mathbb{Z}\)\-\\\poptabs
\( \KWD{end}\)\\
\( \KWD{object}\:O(x\COLON \mathbb{Z})\)\\
{\tt~~}\pushtabs\=\+\(   n() =  \null\)\pushtabs\=\+\(\KWD{object}\;\;\KWD{extends}\:T\)\\
{\tt~~~~~~~~}\pushtabs\=\+\(           y\COLON \mathbb{Z} = \mathord{\KWD{self}}.x\)\-\\\poptabs
\(         \KWD{end}\)\-\-\\\poptabs\poptabs
\( \KWD{end}\)\-\\\poptabs
\end{Fortress}

Note that the reference to \EXP{\mathord{\KWD{self}}} in the initialization expression for
\VAR{y} cannot refer to the object being constructed, so if this is allowed, then that
\EXP{\mathord{\KWD{self}}} must refer to the object \VAR{O}.

\subsection{Object name shadowed by methods}
% object O(x: ZZ)
%   O(y: ZZ) = y+3
%   m(z: ZZ) = O(z)
% end
\begin{Fortress}
{\tt~}\pushtabs\=\+\( \KWD{object}\:O(x\COLON \mathbb{Z})\)\\
{\tt~~}\pushtabs\=\+\(   O(y\COLON \mathbb{Z}) = y+3\)\\
\(   m(z\COLON \mathbb{Z}) = O(z)\)\-\\\poptabs
\( \KWD{end}\)\-\\\poptabs
\end{Fortress}

Allow this and \EXP{O(1).m(2) = 5}.

\subsection{Function name shadowed by keyword parameters}
% f(f: ZZ = 0) = f+1
% g(g: ZZ->ZZ = fn x => x) = g(2)       (* this is not a recursive function *)
\begin{Fortress}
{\tt~}\pushtabs\=\+\( f(f\COLON \mathbb{Z} = 0) = f+1\)\\
\( g(g\COLON \mathbb{Z}\rightarrow\mathbb{Z} = \;\KWD{fn}\:x \Rightarrow x) = g(2)       \mathtt{(*}\;\hbox{\rm  this is not a recursive function \unskip}\;\mathtt{*)}\)\-\\\poptabs
\end{Fortress}

Allowed because the names of keyword parameters, unlike the names of other parameters,
are part of the ``interface'' of the functional.

\subsection{Type names referred by imported entities in a component implementing the types}

A fully qualified type, \EXP{\TYP{APIName}.\VAR{typeName}}, ``available'' in a component,
\TYP{ComponentName}, that implements the API, \TYP{APIName},
denotes the same type \VAR{typeName}, which should be defined in the component,
\TYP{ComponentName}.  For example:

% api A
%   trait T
%     m(): T
%   end
% end

% api B
%   import A.T
%   f(): T
% end

% component C
%   import B.f
%   export A
%   trait T
%     m(): T
%     private n(): ()
%   end
%   g() = f().n()
% end
\begin{Fortress}
{\tt~}\pushtabs\=\+\( \KWD{api}\:A\)\\
{\tt~~}\pushtabs\=\+\(   \KWD{trait}\:T\)\\
{\tt~~}\pushtabs\=\+\(     m()\COLON T\)\-\\\poptabs
\(   \KWD{end}\)\-\\\poptabs
\( \KWD{end}\)\\[4pt]
\( \KWD{api}\:B\)\\
{\tt~~}\pushtabs\=\+\(   \KWD{import}\:A.T\)\\
\(   f()\COLON T\)\-\\\poptabs
\( \KWD{end}\)\\[4pt]
\( \KWD{component}\:C\)\\
{\tt~~}\pushtabs\=\+\(   \KWD{import}\:B.f\)\\
\(   \KWD{export}\:A\)\\
\(   \KWD{trait}\:T\)\\
{\tt~~}\pushtabs\=\+\(     m()\COLON T\)\\
\(     \KWD{private}\:n()\COLON ()\)\-\\\poptabs
\(   \KWD{end}\)\\
\(   g() = f().n()\)\-\\\poptabs
\( \KWD{end}\)\-\\\poptabs
\end{Fortress}

the method call \EXP{f().n()} is valid.

However, a fully qualified type, \EXP{\TYP{APIName}.\VAR{typeName}},
``occurring'' in a component, \TYP{ComponentName}, that implements the API, \TYP{APIName}
is not allowed because \TYP{APIName} is not imported to the component.

\subsection{Overloaded operators}

\begin{itemize}
\item Are we allowed to say:
% trait Foo[\opr OP\]
%   f(x: T[\OP\], y: ZZ32) = ...
%   f(x: T[\OTIMES\], y: ZZ32) = ...
%   opr OP(self, other: FOO[\OP\]) = ...
%   opr OTIMES(self, other: FOO[\OP\]) = ...
% end
\begin{Fortress}
{\tt~}\pushtabs\=\+\( \KWD{trait}\:\TYP{Foo}\llbracket\KWD{opr} \mathord{\OPR{OP}}\rrbracket\)\\
{\tt~~}\pushtabs\=\+\(   f(x\COLON T\llbracket\OPR{OP}\rrbracket, y\COLON \mathbb{Z}32) = \ldots\)\\
\(   f(x\COLON T\llbracket\otimes\rrbracket, y\COLON \mathbb{Z}32) = \ldots\)\\
\(   \KWD{opr} \mathord{\OPR{OP}}(\KWD{self}, \VAR{other}\COLON \OPR{FOO}\llbracket\OPR{OP}\rrbracket) = \ldots\)\\
\(   \KWD{opr} \mathord{\otimes}(\KWD{self}, \VAR{other}\COLON \OPR{FOO}\llbracket\OPR{OP}\rrbracket) = \ldots\)\-\\\poptabs
\( \KWD{end}\)\-\\\poptabs
\end{Fortress}
now that \OPR{OP} is forbidden to be \EXP{\otimes},
so maybe it's okay???

If \VAR{T} is a trait, it is not allowed because \EXP{T\llbracket\OPR{OP}\rrbracket}
may not exclude \EXP{T\llbracket\otimes\rrbracket}.

No. We don't want to allow this. The static checker does not take the implicit constraint from the body of the trait.

\item Are we allowed to say:
% trait Foo[\nat n\]
%   f(x: T[\n+1\], y: ZZ32) = ...
%   f(x: T[\0\], y: ZZ32) = ...
% end
% object T[\nat n\]( ... ) end
\begin{Fortress}
{\tt~}\pushtabs\=\+\( \KWD{trait}\:\TYP{Foo}\llbracket\KWD{nat}\:n\rrbracket\)\\
{\tt~~}\pushtabs\=\+\(   f(x\COLON T\llbracket{}n+1\rrbracket, y\COLON \mathbb{Z}32) = \ldots\)\\
\(   f(x\COLON T\llbracket{}0\rrbracket, y\COLON \mathbb{Z}32) = \ldots\)\-\\\poptabs
\( \KWD{end}\)\\
\( \KWD{object}\:T\llbracket\KWD{nat}\:n\rrbracket( \ldots ) \KWD{end}\)\-\\\poptabs
\end{Fortress}
now that \EXP{n+1} is necessarily nonzero, so maybe it's okay???

Not allowed.
\item Are we allowed to say:
% trait Foo[\nat n\]
%   f(x: T[\1\], y: ZZ32) = ...
%   f(x: T[\0\], y: ZZ32) = ...
% end
% object T[\nat n\]( ... ) end
\begin{Fortress}
{\tt~}\pushtabs\=\+\( \KWD{trait}\:\TYP{Foo}\llbracket\KWD{nat}\:n\rrbracket\)\\
{\tt~~}\pushtabs\=\+\(   f(x\COLON T\llbracket{}1\rrbracket, y\COLON \mathbb{Z}32) = \ldots\)\\
\(   f(x\COLON T\llbracket{}0\rrbracket, y\COLON \mathbb{Z}32) = \ldots\)\-\\\poptabs
\( \KWD{end}\)\\
\( \KWD{object}\:T\llbracket\KWD{nat}\:n\rrbracket( \ldots ) \KWD{end}\)\-\\\poptabs
\end{Fortress}
now that \KWD{nat} parameters are all instantiated, so maybe it's okay???

Allowed.
\item Are we even allowed to say this simpler case:
% f(x: ZZ32, y: ZZ32) = ...
% f(x: T[\0\], y: ZZ32) = ...
\begin{Fortress}
{\tt~}\pushtabs\=\+\( f(x\COLON \mathbb{Z}32, y\COLON \mathbb{Z}32) = \ldots\)\\
\( f(x\COLON T\llbracket{}0\rrbracket, y\COLON \mathbb{Z}32) = \ldots\)\-\\\poptabs
\end{Fortress}
if our implementation strategy for \KWD{nat} parameters is ``don't instantiate''?

Allowed.

\item Do we allow this:
% trait F[\nat n\] extends G[\n\] end
% trait G[\nat n\] extends F[\n+1\] end
\begin{Fortress}
{\tt~}\pushtabs\=\+\( \KWD{trait}\:F\llbracket\KWD{nat}\:n\rrbracket \KWD{extends}\:G\llbracket{}n\rrbracket \KWD{end}\)\\
\( \KWD{trait}\:G\llbracket\KWD{nat}\:n\rrbracket \KWD{extends}\:F\llbracket{}n+1\rrbracket \KWD{end}\)\-\\\poptabs
\end{Fortress}
Not allowed by the type acyclicity restriction.

\end{itemize}

\subsection{Implicitly Imported Types}
When a component imports a functional \VAR{f}
(either a top-level function or a functional method) by an import statement,
the imported \VAR{f} may be overloaded with a functional \VAR{f} declared by the component.
When a component imports a top-level declaration \VAR{f} from an API \VAR{A},
all the relevant types to type check the uses of \VAR{f} are implicitly imported from \VAR{A}.
However, these implicitly imported types for type checking are not expressible by programmers;
programmers must import the types by import statements to use them.
For example, the two functional calls in the following component \VAR{C} are valid:
% api A
%   trait T
%     m() : ()
%   end
% end

% api B
%   import A.{...}
%   f(): T
%   g(T): ()
% end

% component C
%   import B.{f, g}
%   export Executable
%   run (args) = do f().m(); g(f()) end
% end
\begin{Fortress}
{\tt~}\pushtabs\=\+\( \KWD{api}\:A\)\\
{\tt~~}\pushtabs\=\+\(   \KWD{trait}\:T\)\\
{\tt~~}\pushtabs\=\+\(     m() \mathrel{\mathtt{:}} ()\)\-\\\poptabs
\(   \KWD{end}\)\-\\\poptabs
\( \KWD{end}\)\\[4pt]
\( \KWD{api}\:B\)\\
{\tt~~}\pushtabs\=\+\(   \KWD{import}\:A.\{\ldots\}\)\\
\(   f()\COLON T\)\\
\(   g(T)\COLON ()\)\-\\\poptabs
\( \KWD{end}\)\\[4pt]
\( \KWD{component}\:C\)\\
{\tt~~}\pushtabs\=\+\(   \KWD{import}\:B.\{f, g\}\)\\
\(   \KWD{export}\:\TYP{Executable}\)\\
\(   \VAR{run} (\VAR{args}) = \;\KWD{do}\:f().m(); g(f()) \KWD{end}\)\-\\\poptabs
\( \KWD{end}\)\-\\\poptabs
\end{Fortress}
because \VAR{T} is implicitly imported from \VAR{B} to type check the functional calls.
However, the programmers cannot write \VAR{T} in \VAR{C}
because \VAR{T} is not imported by an import statement.

\subsection{Component and Value Namespace}
% component Surprise
% import api Foo
% export Executable

% object Foo
%   foo():String = "foo"
%   bar():String = "foo"
%   baz():String = "foo"
% end

% run(args:String...) = do
%   Foo.foo()
%   Foo.bar()
%   Foo.baz()
% end

% end
\begin{Fortress}
{\tt~}\pushtabs\=\+\( \KWD{component}\:\TYP{Surprise}\)\\
\( \KWD{import}\;\;\KWD{api}\:\TYP{Foo}\)\\
\( \KWD{export}\:\TYP{Executable}\)\\[4pt]
\( \KWD{object}\:\TYP{Foo}\)\\
{\tt~~}\pushtabs\=\+\(   \VAR{foo}()\COLONOP\TYP{String} =\;\;\hbox{\rm``\STR{foo}''}\)\\
\(   \VAR{bar}()\COLONOP\TYP{String} =\;\;\hbox{\rm``\STR{foo}''}\)\\
\(   \VAR{baz}()\COLONOP\TYP{String} =\;\;\hbox{\rm``\STR{foo}''}\)\-\\\poptabs
\( \KWD{end}\)\\[4pt]
\( \VAR{run}(\VAR{args}\COLONOP\TYP{String}\ldots) = \;\KWD{do}\)\\
{\tt~~}\pushtabs\=\+\(   \TYP{Foo}.\VAR{foo}()\)\\
\(   \TYP{Foo}.\VAR{bar}()\)\\
\(   \TYP{Foo}.\VAR{baz}()\)\-\\\poptabs
\( \KWD{end}\)\\[4pt]
\( \KWD{end}\)\-\\\poptabs
\end{Fortress}

Static error for the declaration of the object \TYP{Foo} in the above example.

\subsection{Disjunctive Constraints}

Here is a program that will only typecheck correctly if the fortress typechecker supports these disjunctions:
%% component NeedsDisjunction
%%   export Executable

%%   object Right end
%%   object Wrong end

%%   bar[\T\](p_1:Plussable[\T\], p_2:Plussable[\T\]):Plussable[\T\] = p_1

%%   trait Plussable[\T\]
%%   end

%%   object Parent extends {Plussable[\Right\], Plussable[\Wrong\]}
%%   end

%%   foo(p:Plussable[\Wrong\]):() = ()

%%   run(args:String...):() = do
%%     a:Parent = Parent
%%     b:Parent = Parent

%%     c = bar(a,b)
%%     foo(c)
%%   end
%% end
\begin{Fortress}
\(\KWD{component}\:\TYP{NeedsDisjunction}\)\\
{\tt~~}\pushtabs\=\+\(  \KWD{export}\:\TYP{Executable}\)\\[4pt]
\(  \KWD{object}\:\TYP{Right} \KWD{end}\)\\
\(  \KWD{object}\:\TYP{Wrong} \KWD{end}\)\\[4pt]
\(  \VAR{bar}\llbracket{}T\rrbracket(p_{\mathrm{1}}\COLONOP\TYP{Plussable}\llbracket{}T\rrbracket, p_{\mathrm{2}}\COLONOP\TYP{Plussable}\llbracket{}T\rrbracket)\COLONOP\TYP{Plussable}\llbracket{}T\rrbracket = p_{\mathrm{1}}\)\\[4pt]
\(  \KWD{trait}\:\TYP{Plussable}\llbracket{}T\rrbracket\)\\
\(  \KWD{end}\)\\[4pt]
\(  \KWD{object}\:\TYP{Parent} \KWD{extends} \{\TYP{Plussable}\llbracket\TYP{Right}\rrbracket, \TYP{Plussable}\llbracket\TYP{Wrong}\rrbracket\}\)\\
\(  \KWD{end}\)\\[4pt]
\(  \VAR{foo}(p\COLONOP\TYP{Plussable}\llbracket\TYP{Wrong}\rrbracket)\COLONOP() = ()\)\\[4pt]
\(  \VAR{run}(\VAR{args}\COLONOP\TYP{String}\ldots)\COLONOP() = \;\KWD{do}\)\\
{\tt~~}\pushtabs\=\+\(    a\COLONOP\TYP{Parent} = \TYP{Parent}\)\\
\(    b\COLONOP\TYP{Parent} = \TYP{Parent}\)\\[4pt]
\(    c = \VAR{bar}(a,b)\)\\
\(    \VAR{foo}(c)\)\-\\\poptabs
\(  \KWD{end}\)\-\\\poptabs
\(\KWD{end}\)
\end{Fortress}
When \EXP{\VAR{bar}()} is called without explicit type parameters,
we make its signature into:
%bar[\$i\](p_1:Plussable[\$i\], p_2:Plussable[\$i\]):Plussable[\$i\]
\begin{Fortress}
\(\VAR{bar}\llbracket\$i\rrbracket(p_{\mathrm{1}}\COLONOP\TYP{Plussable}\llbracket\$i\rrbracket, p_{\mathrm{2}}\COLONOP\TYP{Plussable}\llbracket\$i\rrbracket)\COLONOP\TYP{Plussable}\llbracket\$i\rrbracket.\)
\end{Fortress}
\EXP{\VAR{bar}()} then becomes a legal overloading if,
``\EXP{\$i} = Right OR \EXP{\$i} = Wrong''.
Later, the call to \EXP{\VAR{foo}()} forces the issue:
\EXP{\$i} must be ``Right'', but unless we kept both possibilities alive, our program might not typecheck correctly.


\subsection{Weirdness with Multiple Inheritance and Polymorphism}

We noticed that you can get some fairly odd behavior if you allow inheriting from the same trait multiple times with different static arguments.  For example in the following code there is no good way to decide what the value of \EXP{\VAR{foo}()} will be.
%% component Bad
%% export Executable

%% trait Parent[\T\]
%% end

%% object goodChild[\T\] extends Parent[\T\]
%% end

%% object badChild extends { Parent[\ZZ32\], Parent[\String\] }
%% end

%% spank[\T\](x: Parent[\T\]): Parent[\T\] =
%%   typecase a = "UnOh" of
%%     T => x
%%     else => goodCHild[\T\]
%%   end

%% foo(): ZZ32 =
%%   typecase x = spank(badChild) of
%%     badChild => 1
%%     else => 0
%%   end

%% run(args: String):() = ignore(foo())

%% end
\begin{Fortress}
\(\KWD{component}\:\TYP{Bad}\)\\
\(\KWD{export}\:\TYP{Executable}\)\\[4pt]
\(\KWD{trait}\:\TYP{Parent}\llbracket{}T\rrbracket\)\\
\(\KWD{end}\)\\[4pt]
\(\KWD{object}\:\VAR{goodChild}\llbracket{}T\rrbracket \KWD{extends}\:\TYP{Parent}\llbracket{}T\rrbracket\)\\
\(\KWD{end}\)\\[4pt]
\(\KWD{object}\:\VAR{badChild} \KWD{extends} \{\,\TYP{Parent}\llbracket\mathbb{Z}32\rrbracket, \TYP{Parent}\llbracket\TYP{String}\rrbracket\,\}\)\\
\(\KWD{end}\)\\[4pt]
\(\VAR{spank}\llbracket{}T\rrbracket(x\COLON \TYP{Parent}\llbracket{}T\rrbracket)\COLON \TYP{Parent}\llbracket{}T\rrbracket =\)\\
{\tt~~}\pushtabs\=\+\(  \KWD{typecase}\:a =\;\;\hbox{\rm``\STR{UnOh}''}\;\;\KWD{of}\)\\
{\tt~~}\pushtabs\=\+\(    T \Rightarrow x\)\\
\(    \KWD{else} \Rightarrow \VAR{goodCHild}\llbracket{}T\rrbracket\)\-\\\poptabs
\(  \KWD{end}\)\-\\[4pt]\poptabs
\(\VAR{foo}()\COLON \mathbb{Z}32 =\)\\
{\tt~~}\pushtabs\=\+\(  \KWD{typecase}\:x = \VAR{spank}(\VAR{badChild}) \KWD{of}\)\\
{\tt~~}\pushtabs\=\+\(    \VAR{badChild} \Rightarrow 1\)\\
\(    \KWD{else} \Rightarrow 0\)\-\\\poptabs
\(  \KWD{end}\)\-\\[4pt]\poptabs
\(\VAR{run}(\VAR{args}\COLON \TYP{String})\COLONOP() = \VAR{ignore}(\VAR{foo}())\)\\[4pt]
\(\KWD{end}\)
\end{Fortress}



\subsection{\KWD{where} Clauses on Methods}
%value trait LinearSequence[\T extends Object, nat n\]
%   coercion[\nat b\](x:BinaryLinearSequence[\b,n\]) where {T extends  BinaryWord[\b\]}
%   opr [n: IndexInt] : T throws { IndexOutOfBounds }
%   opr [\nat k\] [r: StaticWidthRange[\k\]] : LinearSequence[\T,k\]
%         throws { IndexOutOfBounds } where { k <= n }
%   opr [n: IndexInt] := (v:T) : LinearSequence[\T,n\] throws  { IndexOutOfBounds }
%   opr [\nat k\] [r: StaticWidthRange[\k\]] := (v:LinearSequence[\ T,k\]) : LinearSequence[\T,n\]
%         throws { IndexOutOfBounds } where { k <= n }
%   opr || [\nat m\](self, LinearSequence[\T,m\]) : LinearSequence[\T,n +m\]
%end
\begin{Fortress}
\(\KWD{value}\;\;\KWD{trait} \TYP{LinearSequence}\llbracket{}T \KWD{extends} \TYP{Object}, \KWD{nat} n\rrbracket\)\\
{\tt~~~}\pushtabs\=\+\(   \KWD{coercion}\llbracket\KWD{nat} b\rrbracket(x\COLONOP\TYP{BinaryLinearSequence}\llbracket{}b,n\rrbracket) \KWD{where} \{T \KWD{extends}  \TYP{BinaryWord}\llbracket{}b\rrbracket\}\)\\
\(   \KWD{opr} [n\COLON \TYP{IndexInt}] \mathrel{\mathtt{:}} T \KWD{throws} \{\,\TYP{IndexOutOfBounds}\,\}\)\\
\(   \KWD{opr} \mathord{\llbracket}\null\)\pushtabs\=\+\(\KWD{nat} k\rrbracket [r\COLON \TYP{StaticWidthRange}\llbracket{}k\rrbracket] \mathrel{\mathtt{:}} \TYP{LinearSequence}\llbracket{}T,k\rrbracket\)\\
\(         \KWD{throws} \{\,\TYP{IndexOutOfBounds}\,\} \KWD{where} \{\,k \leq n\,\}\)\-\\\poptabs
\(   \KWD{opr} [n\COLON \TYP{IndexInt}] \ASSIGN (v\COLONOP{}T) \mathrel{\mathtt{:}} \TYP{LinearSequence}\llbracket{}T,n\rrbracket \KWD{throws}  \{\,\TYP{IndexOutOfBounds}\,\}\)\\
\(   \KWD{opr} \mathord{\llbracket}\null\)\pushtabs\=\+\(\KWD{nat} k\rrbracket [r\COLON \TYP{StaticWidthRange}\llbracket{}k\rrbracket] \ASSIGN (v\COLONOP\TYP{LinearSequence}\llbracket\,T,k\rrbracket) \mathrel{\mathtt{:}} \TYP{LinearSequence}\llbracket{}T,n\rrbracket\)\\
\(         \KWD{throws} \{\,\TYP{IndexOutOfBounds}\,\} \KWD{where} \{\,k \leq n\,\}\)\-\\\poptabs
\(   \KWD{opr} \| \llbracket\KWD{nat} m\rrbracket(\KWD{self}, \TYP{LinearSequence}\llbracket{}T,m\rrbracket) \mathrel{\mathtt{:}} \TYP{LinearSequence}\llbracket{}T,n +m\rrbracket\)\-\\\poptabs
\(\KWD{end} \)
\end{Fortress}


%value trait IntegerValue[\bool negative, nat magnitude\]
%      extends { IntegerConstant, ExtendedIntegerValue[\negative,magnitude,false\],
%                RationalValue[\negative, magnitude, 1\] }
%   opr /(self, other: IntegerValue[\neg,mag\] : RationalValue [\negative XOR neg, magnitude, mag\] where { mag =/= 0 }
%   opr /(self, other: IntegerValue[\neg,mag\] : ExtendedRationalValue [\negative XOR neg, magnitude, mag\]
%   opr SQRT(self) : RationalValueTimesSqrt[\magnitude\] where { NOT  negative }
%   ...
%end
%value trait ExtendedIntegerValue[\bool negative, nat magnitude, bool infinite\]
%      extends { ExtendedIntegerConstant,
%                ExtendedRationalValue[\negative,
%                                       if NOT infinite then magnitude elif negative then -1 else 1 end,
%                                       if infinite then 0 else 1 end\] }
%   opr /(self, other: ExtendedIntegerValue[\neg,mag,inf\]) : RationalValue[\ ... \] where { ... }
%   opr /(self, other: ExtendedIntegerValue[\neg,mag,inf\]) : ExtendedRationalValue[\ ... \]
%   opr SQRT(self) : RationalValueTimesSqrt[\false,1,1,magnitude\] where { NOT negative AND NOT infinite }
%   ...
%end
\begin{Fortress}
\(\KWD{value} \null\)\pushtabs\=\+\(\KWD{trait} \TYP{IntegerValue}\llbracket\KWD{bool} \VAR{negative}, \KWD{nat} \VAR{magnitude}\rrbracket\)\\
\(      \KWD{extends} \{\,\null\)\pushtabs\=\+\(\TYP{IntegerConstant}, \TYP{ExtendedIntegerValue}\llbracket\VAR{negative},\VAR{magnitude},\VAR{false}\rrbracket, \)\\
\(                \TYP{RationalValue}\llbracket\VAR{negative}, \VAR{magnitude}, 1\rrbracket\,\}\)\-\-\\\poptabs\poptabs
{\tt~~~}\pushtabs\=\+\(   \KWD{opr} /(\KWD{self}, \VAR{other}\COLON \TYP{IntegerValue}\llbracket\VAR{neg},\VAR{mag}\rrbracket \mathrel{\mathtt{:}} \TYP{RationalValue} \llbracket\VAR{negative} \xor \VAR{neg}, \VAR{magnitude}, \VAR{mag}\rrbracket \KWD{where} \{\,\VAR{mag} \neq 0\,\}\)\\
\(   \KWD{opr} /(\KWD{self}, \VAR{other}\COLON \TYP{IntegerValue}\llbracket\VAR{neg},\VAR{mag}\rrbracket \mathrel{\mathtt{:}} \TYP{ExtendedRationalValue} \llbracket\VAR{negative} \xor \VAR{neg}, \VAR{magnitude}, \VAR{mag}\rrbracket\)\\
\(   \KWD{opr} \mathord{\surd}(\KWD{self}) \mathrel{\mathtt{:}} \TYP{RationalValueTimesSqrt}\llbracket\VAR{magnitude}\rrbracket \KWD{where} \{\,\neg  \VAR{negative}\,\}\)\\
\(   \ldots\)\-\\\poptabs
\(\KWD{end} \)\\
\(\KWD{value} \null\)\pushtabs\=\+\(\KWD{trait} \TYP{ExtendedIntegerValue}\llbracket\KWD{bool} \VAR{negative}, \KWD{nat} \VAR{magnitude}, \KWD{bool} \VAR{infinite}\rrbracket\)\\
\(      \KWD{extends} \{\,\null\)\pushtabs\=\+\(\TYP{ExtendedIntegerConstant},\)\\
\(                \TYP{ExtendedRationalValue}\llbracket\null\)\pushtabs\=\+\(\VAR{negative},\)\\
\(                                       \KWD{if} \neg \VAR{infinite} \KWD{then} \VAR{magnitude} \KWD{elif} \VAR{negative} \KWD{then} -1 \KWD{else} 1 \KWD{end},\)\\
\(                                       \KWD{if} \VAR{infinite} \KWD{then} 0 \KWD{else} 1 \KWD{end}\rrbracket\,\}\)\-\-\-\\\poptabs\poptabs\poptabs
{\tt~~~}\pushtabs\=\+\(   \KWD{opr} /(\KWD{self}, \VAR{other}\COLON \TYP{ExtendedIntegerValue}\llbracket\VAR{neg},\VAR{mag},\inf\rrbracket) \mathrel{\mathtt{:}} \TYP{RationalValue}\llbracket\,\ldots\,\rrbracket \KWD{where} \{\,\ldots\,\}\)\\
\(   \KWD{opr} /(\KWD{self}, \VAR{other}\COLON \TYP{ExtendedIntegerValue}\llbracket\VAR{neg},\VAR{mag},\inf\rrbracket) \mathrel{\mathtt{:}} \TYP{ExtendedRationalValue}\llbracket\,\ldots\,\rrbracket\)\\
\(   \KWD{opr} \mathord{\surd}(\KWD{self}) \mathrel{\mathtt{:}} \TYP{RationalValueTimesSqrt}\llbracket\VAR{false},1,1,\VAR{magnitude}\rrbracket \KWD{where} \{\,\neg \VAR{negative} \wedge \neg \VAR{infinite}\,\}\)\\
\(   \ldots\)\-\\\poptabs
\(\KWD{end}\)
\end{Fortress}


\subsection{Conditional Subtyping}
%trait RationalQuantity[\unit U absorbs unit, bool ninf, bool lt, bool eq, bool gt, bool pinf, bool nan\]
%    extends { RationalQuantity[\U, ninf', lt', eq', gt', pinf', nan'\]
%              where { bool ninf', bool lt', bool eq', bool gt', bool pinf', bool nan',
%                      ninf ->  ninf', lt -> lt', eq -> eq', gt -> gt', pinf -> pinf', nan -> nan' },
%              Field[\RationalQuantity[\U, ninf, lt, eq, gt, pinf, nan\],
%                     RationalQuantity[\U, ninf, lt, false, gt, pinf, nan\],+,-,DOT,/,zero,one\]
%              where { lt AND eq AND gt AND NOT ninf AND NOT pinf AND NOT nan, U = dimensionless },
%              CommutativeMonoid[\RationalQuantity[\U, ninf, lt, eq, gt, pinf, nan\],+,-,zero\]
%              TotalOrderOperators[\RationalQuantity[\U, ninf, lt, eq, gt, inf, nan\],<,<=,>=,>,CMP\]
%              where { NOT nan } }
%    where { ninf OR lt OR eq OR gt OR pinf OR nan }
%end
\begin{Fortress}
\(\KWD{trait} \TYP{RationalQuantity}\llbracket\KWD{unit} U \KWD{absorbs}\;\;\KWD{unit}, \KWD{bool} \VAR{ninf}, \KWD{bool} \VAR{lt}, \KWD{bool} \VAR{eq}, \KWD{bool} \VAR{gt}, \KWD{bool} \VAR{pinf}, \KWD{bool} \VAR{nan}\rrbracket\)\\
{\tt~~~~}\pushtabs\=\+\(    \KWD{extends} \{\,\null\)\pushtabs\=\+\(\TYP{RationalQuantity}\llbracket{}U, \VAR{ninf}', \VAR{lt}', \VAR{eq}', \VAR{gt}', \VAR{pinf}', \VAR{nan}'\rrbracket\)\\
\(              \KWD{where} \{\,\null\)\pushtabs\=\+\(\KWD{bool} \VAR{ninf}', \KWD{bool} \VAR{lt}', \KWD{bool} \VAR{eq}', \KWD{bool} \VAR{gt}', \KWD{bool} \VAR{pinf}', \KWD{bool} \VAR{nan}',\)\\
\(                      \VAR{ninf} \rightarrow  \VAR{ninf}', \VAR{lt} \rightarrow \VAR{lt}', \VAR{eq} \rightarrow \VAR{eq}', \VAR{gt} \rightarrow \VAR{gt}', \VAR{pinf} \rightarrow \VAR{pinf}', \VAR{nan} \rightarrow \VAR{nan}'\,\},\)\-\\\poptabs
\(              \TYP{Field}\llbracket\null\)\pushtabs\=\+\(\TYP{RationalQuantity}\llbracket{}U, \VAR{ninf}, \VAR{lt}, \VAR{eq}, \VAR{gt}, \VAR{pinf}, \VAR{nan}\rrbracket,\)\\
\(                     \TYP{RationalQuantity}\llbracket{}U, \VAR{ninf}, \VAR{lt}, \VAR{false}, \VAR{gt}, \VAR{pinf}, \VAR{nan}\rrbracket,+,-,\cdot,/,\VAR{zero},\VAR{one}\rrbracket\)\-\\\poptabs
\(              \KWD{where} \{\,\VAR{lt} \wedge \VAR{eq} \wedge \VAR{gt} \wedge \neg \VAR{ninf} \wedge \neg \VAR{pinf} \wedge \neg \VAR{nan}, U = \mathbin{\TYP{dimensionless}}\,\},\)\\
\(              \TYP{CommutativeMonoid}\llbracket\TYP{RationalQuantity}\llbracket{}U, \VAR{ninf}, \VAR{lt}, \VAR{eq}, \VAR{gt}, \VAR{pinf}, \VAR{nan}\rrbracket,+,-,\VAR{zero}\rrbracket\)\\
\(              \TYP{TotalOrderOperators}\llbracket\TYP{RationalQuantity}\llbracket{}U, \VAR{ninf}, \VAR{lt}, \VAR{eq}, \VAR{gt}, \inf, \VAR{nan}\rrbracket,<,\leq,\geq,>,\OPR{CMP}\rrbracket\)\\
\(              \KWD{where} \{\,\neg \VAR{nan}\,\}\,\}\)\-\\\poptabs
\(    \KWD{where} \{\,\VAR{ninf} \vee \VAR{lt} \vee \VAR{eq} \vee \VAR{gt} \vee \VAR{pinf} \vee \VAR{nan}\,\}\)\-\\\poptabs
\(\KWD{end}\)
\end{Fortress}



\subsection{No Wrapped Fields Whose Types Are Naked Type Variables}

We don't want to allow wrapped fields to have types
that are naked type variables
because then we cannot tell what methods an object has,
and therefore, what names may be shadowed within the object definition:
%object F[\T\](x:ZZ, wrapped y:T)
%  f()  = object extends Object
%                getX() = x
%         end
%end
%
%trait T1 (* has getter for x *)
%   x:ZZ = 2
%end
\begin{Fortress}
\(\KWD{object} F\llbracket{}T\rrbracket(x\COLONOP\mathbb{Z}, \KWD{wrapped} y\COLONOP{}T)\)\\
{\tt~~}\pushtabs\=\+\(  f()  = \null\)\pushtabs\=\+\(\KWD{object} \null\)\pushtabs\=\+\(\KWD{extends} \TYP{Object}\)\\
\(                \VAR{getX}() = x\)\-\\\poptabs
\(         \KWD{end}\)\-\-\\\poptabs\poptabs
\(\KWD{end}\)\\[4pt]
\(\KWD{trait} T_{1} \mathtt{(*}\;\hbox{\rm  has getter for x \unskip}\;\mathtt{*)}\)\\
{\tt~~~}\pushtabs\=\+\(   x\COLONOP\mathbb{Z} = 2\)\-\\\poptabs
\(\KWD{end}\)
\end{Fortress}
With conditional subtyping (or conditional methods),
we may also not be able to tell
whether a trait has a particular method,
but we can at least bound the names of the methods it may have.
Still, we should watch out for this issue
in the formulation of conditional subtyping.


\subsection{Tricky Example for Explaining Shadowing}

%trait S
%  f() = 1  (* Declaration 1 *)
%  g() = 5
%end
%
%trait T
%  f() = 2  (* Declaration 2 *)
%  m() = 6
%  n(): S
%end
%
%object O extends S
%  h():T = object extends T
%            m() = f()
%            n() = object extends S
%                    g() = f()
%                  end
%          end
%end
\begin{Fortress}
\(\KWD{trait} S\)\\
{\tt~~}\pushtabs\=\+\(  f() = 1  \mathtt{(*}\;\hbox{\rm  Declaration 1 \unskip}\;\mathtt{*)}\)\\
\(  g() = 5\)\-\\\poptabs
\(\KWD{end}\)\\[4pt]
\(\KWD{trait} T\)\\
{\tt~~}\pushtabs\=\+\(  f() = 2  \mathtt{(*}\;\hbox{\rm  Declaration 2 \unskip}\;\mathtt{*)}\)\\
\(  m() = 6\)\\
\(  n()\COLON S\)\-\\\poptabs
\(\KWD{end}\)\\[4pt]
\(\KWD{object} O \KWD{extends} S\)\\
{\tt~~}\pushtabs\=\+\(  h()\COLONOP{}T = \null\)\pushtabs\=\+\(\KWD{object}\;\;\KWD{extends} T\)\\
{\tt~~~~~~~~~~}\pushtabs\=\+\(            m() = f()\)\\
\(            n() = \null\)\pushtabs\=\+\(\KWD{object}\;\;\KWD{extends} S\)\\
{\tt~~~~~~~~}\pushtabs\=\+\(                    g() = f()\)\-\\\poptabs
\(                  \KWD{end}\)\-\-\\\poptabs\poptabs
\(          \KWD{end}\)\-\-\\\poptabs\poptabs
\(\KWD{end}\)
\end{Fortress}

The reach of declaration 1
includes the entire declaration of O,
but it is shadowed by declaration 2,
within the object expression that extends \VAR{T},
which in turn is shadowed by declaration 1
in the object expression that extends \VAR{S}.

\subsection{A Pattern for Shadowing}

% var count: ZZ = 0
% getGlobalCount() = count
% setGlobalCount(i:ZZ) = count := i

% object LocalCounter(var count: ZZ)
%   localIncrement() = count += 1
%   getter totalCount = count + getGlobalCount()
%   atomic transfer() = setGlobalCount(totalCount)
% end
\begin{Fortress}
{\tt~}\pushtabs\=\+\( \KWD{var} \VAR{count}\COLON \mathbb{Z} = 0\)\\
\( \VAR{getGlobalCount}() = \VAR{count}\)\\
\( \VAR{setGlobalCount}(i\COLONOP\mathbb{Z}) = \VAR{count} \ASSIGN i\)\\[4pt]
\( \KWD{object} \TYP{LocalCounter}(\KWD{var} \VAR{count}\COLON \mathbb{Z})\)\\
{\tt~~}\pushtabs\=\+\(   \VAR{localIncrement}() = \VAR{count} \mathrel{+}= 1\)\\
\(   \KWD{getter} \VAR{totalCount} = \VAR{count} + \VAR{getGlobalCount}()\)\\
\(   \KWD{atomic} \VAR{transfer}() = \VAR{setGlobalCount}(\VAR{totalCount})\)\-\\\poptabs
\( \KWD{end}\)\-\\\poptabs
\end{Fortress}

Note the need for \VAR{getGlobalCount} and \VAR{setGlobalCount} just to get
access to the top-level variable \VAR{count} within the body of
\TYP{LocalCounter} (in which the top-level \VAR{count} variable declaration
is shadowed).  Presumably the exported API would not include them.

I think we were toying aroud with a direct means to acess a shadowed name
using a new reserved word "\KWD{outer}" (or Guy proposed "periscope").  I
don't like this, but it would eliminate the need for the pattern.

\subsection{Implicitly Parameterized Object Expressions}

What are the values of the last two expressions?

% value trait A
%   x:ZZ
% end
% f[\T\](y:ZZ) = object extends A x = y end
% f[\RR\](1) = f[\ZZ\](1)
% f[\RR\](1) = f[\RR\](1)
\begin{Fortress}
{\tt~}\pushtabs\=\+\( \KWD{value}\;\;\KWD{trait}\:A\)\\
{\tt~~}\pushtabs\=\+\(   x\COLONOP\mathbb{Z}\)\-\\\poptabs
\( \KWD{end}\)\\
\( f\llbracket{}T\rrbracket(y\COLONOP\mathbb{Z}) = \;\KWD{object}\;\;\KWD{extends}\:A\:x = y \KWD{end}\)\\
\( f\llbracket\mathbb{R}\rrbracket(1) = f\llbracket\mathbb{Z}\rrbracket(1)\)\\
\( f\llbracket\mathbb{R}\rrbracket(1) = f\llbracket\mathbb{R}\rrbracket(1)\)\-\\\poptabs
\end{Fortress}


\section{Things to Check for Consistency}

\begin{itemize}
\item For each section in the Expressions chapter, describe the following:
  \begin{itemize}
  \item syntax
  \item how it is reduced / how we evaluate it
  \item value
  \item type
  \end{itemize}

\item Avoid the following words:
 must > should > may / might
 avoid "may not"
 use "might not" or "need not"
  \item AVOID global, top-level (except top-level within components/APIs)
  \item AVOID statement (except import/export statements)
  \item AVOID compliant compilers (We may have a meta circular API for
  compliant compilers.)
  \item Three levels of Fortress libraries
    \begin{itemize}
    \item the Fortress core language
    \item the Fortress core APIs (implicitly imported libraries)
    \item the Fortress standard libraries
    \end{itemize}
  \item execute: programs, components, Fortress applications, functions,
  threads, tests
  \item evaluate: expressions,
  \item should $\Rightarrow$ shall, must
  \item may not $\Rightarrow$ might not, need not
  \item . $\Rightarrow$ `.'
  \item ``...,'' $\Rightarrow$ ``...'',
  \item api $\Rightarrow$ API
  \item iff $\Rightarrow$ if and only if
  \item reduction $\Rightarrow$ evaluation
  \item compile-time $\Rightarrow$ static
  \item object type, object trait  $\Rightarrow$ \objecttype
  \item ``scope over'' $\Rightarrow$ ``in scope''
  \item ``a static error is signaled'' $\Rightarrow$ ``it is a static error''
  \item ``a runtime error is signaled'' $\Rightarrow$ ``an exception is thrown''
  \item Boolean (as an adjective) $\Rightarrow$ boolean
  \item greatest lower bound $\Rightarrow$ intersection types
  \item least upper bound $\Rightarrow$ union types
  \item built-in types $\Rightarrow$ simple standard types
  \item topmost, uppermost $\Rightarrow$ leftmost (define leftmost early on)
  \item atomic block $\Rightarrow$ atomic expression
  \item function and/or method $\Rightarrow$ functional
  \item visible $\Rightarrow$ in scope (cf. scope defined in the Names chapter)
  \item \OPR{in} $\Rightarrow$ \KWD{in}
  \item \EXP{\VAR{indices}()} $\Rightarrow$ \VAR{indices}
  \item \EXP{\mathord{\KWD{self}}} vs \VAR{self}
  \item What errors do we want to catch?
  \item What codes would be rejected that we'd like to reject?
\end{itemize}

