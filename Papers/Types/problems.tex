% Outline for this section

% Many programs that users would like to write are unsafe
% Undefinable meets
%  ex1: String, ZZ example
%  ex2: [\X <: Any\]ArrayList[\X\], [\Y<: Foo\]List[\Y\] ([\X <: Foo\]ArrayList[\X\] doesn't work)
% Don't meet the subtype rule
%  ex3: ArrayList[\T\], List[\T\]

% Solution
% 1) Exclusion Relation
%  symmetric, irreflexive
%  change subtyping to ensure that the intersection of two excluding types is bottom
%  to lift to existential type schema: all instantiations exclude
%  we want same behavior when type checking programs with type variables
% 2) Outlaw multiple instance inheritance
%  lets us add a new kind of exclusion
%  can use that to reason that ex2 is safe
%  need to build this reasoning into subtyping for existential types
%  same machinery will help us fix ex3 in the next section

% What we will do in this section 
%  need to know constraints under which a type is not bottom => need to know when two types don't exclude
%  can get regular exclusion from not exclude

% Helpful notation
% Not Bottom \[\Delta \vdash T \not\equiv \TYP{BottomType}|\C\]
% Don't exclude under constraints \[\Delta \vdash T  \nexc S | \C\]

% How to get all instantiations exclude (written \exc)
% \infrule
%  {\Delta \vdash S \nexc T|False}
%  {\Delta \vdash S \exc T}

While the rules presented in Section~\ref{sec:rules} allow programmers to
write valid sets of overloaded generic function declarations,
they sometimes reject ``seemingly'' valid overloadings.
In fact, these are not false negatives;
many declarations that programmers would like to write are actually unsafe due to multiple inheritance.

For example, even though the following function declarations look like
a valid overloading, they are not because the Meet Rule is not satisfied:
\begin{FortressCode}
{\tt ~~}\+\VAR{simple}\, \TYP{String}\COLON\TYP{String} \\
  \VAR{simple}\, \mathbb{Z}\COLON\mathbb{Z}\-
\end{FortressCode}
Moreover, in Fortress it is impossible to disambiguate the declarations by providing the meet
because intersection types are not allowed in the Fortress syntax.
In a language with single inheritance, we might infer that these
overloadings were safe because a class can only have a single superclass.
However, due to multiple inheritance, we cannot be sure that these types
do not have a common subtype, no matter what the programmer intends.

Now consider this less trivial set of overloaded functions:
\begin{FortressCode}
{\tt ~~}\+\VAR{foo}\llbracket{}X \SHORTCUT{<} \TYP{Any}\rrbracket\TYP{ArrayList}\llbracket{}X\rrbracket\COLON\mathbb{Z} \\
  \VAR{foo}\llbracket{}Y\mathord{\SHORTCUT{<}}\: \mathbb{Z}\rrbracket\TYP{List}\llbracket{}Y\rrbracket\COLON\mathbb{Z} \\
  \VAR{foo}\llbracket{}W\mathord{\SHORTCUT{<}}\: \mathbb{Z}\rrbracket\TYP{ArrayList}\llbracket{}W\rrbracket\COLON\mathbb{Z}\-
\end{FortressCode}
where \EXP{\TYP{ArrayList}\llbracket{}T\rrbracket \SHORTCUT{<} \TYP{List}\llbracket{}T\rrbracket} for all types $T$.\footnote{We use this standard declaration for \TYP{ArrayList} throughout.}
The first two declarations are incomparable under specificity---%
the first declaration applies to all instantiations of type constructor \TYP{ArrayList},
whereas the second declaration applies only to instantiations of
type constructor \TYP{List} with subtypes of type \EXP{\mathbb{Z}}. The third definition,
which is the ``obvious" candidate to disambiguate the two, is not actually the meet;
the domain of this meet candidate is the existential type:
\begin{FortressCode}
{\tt ~~}\+\exists\llbracket{}W \SHORTCUT{<} \mathbb{Z}\rrbracket\TYP{ArrayList}\llbracket{}W\rrbracket\-
\end{FortressCode}
and needs to be proven equivalent to the domain of the computed meet:
\begin{FortressCode}
{\tt ~~}\+\exists\llbracket{}X \SHORTCUT{<} \TYP{Any}, Y \SHORTCUT{<} \mathbb{Z}\rrbracket\bigl(\TYP{ArrayList}\llbracket{}X\rrbracket \cap \TYP{List}\llbracket{}Y\rrbracket\bigr)\-
\end{FortressCode}
which requires that the latter be a subtype of the former.
However, there is no type \EXP{W \SHORTCUT{<} \mathbb{Z}} such that:
\\[.5em]
\hspace*{.5em}
%\jgsub[\EXP{X \SHORTCUT{<} \TYP{Any}, Y \SHORTCUT{<} \mathbb{Z}}]{\EXP{\TYP{ArrayList}\llbracket{}X\rrbracket \cap \TYP{List}\llbracket{}Y\rrbracket}}{\EXP{\TYP{ArrayList}\llbracket{}W\rrbracket}}
\ensuremath{
\begin{array}{l}
{\EXP{X \SHORTCUT{<} \TYP{Any}, Y \SHORTCUT{<} \mathbb{Z}}}\,\vdash
\\
\quad\quad\quad\quad
{\EXP{\TYP{ArrayList}\llbracket{}X\rrbracket \cap \TYP{List}\llbracket{}Y\rrbracket}}\;{\subtypeof}\;{\EXP{\TYP{ArrayList}\llbracket{}W\rrbracket}}
\end{array}
}
\\[.5em]
Our definition of the meet is not faulty: these declarations actually are unsafe.
Consider the (user-defined) constructed type:
\begin{FortressCode}
{\tt ~~}\+\TYP{BadList} \SHORTCUT{<} \bigl\lbrace\TYP{ArrayList}\llbracket\TYP{String}\rrbracket, \TYP{List}\llbracket\mathbb{Z}\rrbracket\bigr\rbrace\-
\end{FortressCode}
Our meet candidate is not applicable to \TYP{BadList}, but the two other definitions of \VAR{foo} are.
Since neither of those is more specific than the other,
this set of overloaded declarations must be rejected.



The following overloading example,
in which the second declaration is more specific than the first,
is also ill-formed:
\begin{FortressCode}
{\tt ~~}\+\VAR{tail}\llbracket{}X \SHORTCUT{<} \TYP{Any}\rrbracket\TYP{List}\llbracket{}X\rrbracket\COLON\TYP{List}\llbracket{}X\rrbracket \\
  \VAR{tail}\llbracket{}Y \SHORTCUT{<} \TYP{Any}\rrbracket\TYP{ArrayList}\llbracket{}Y\rrbracket\COLON\TYP{ArrayList}\llbracket{}Y\rrbracket\-
\end{FortressCode}
The declarations do not satisfy the Return Type Rule
because we cannot find a specific type \EXP{V \SHORTCUT{<} \TYP{Any}} such that:
\\[.8em]
\begin{tabular}{lr}
%% \jgsub[\EXP{X \SHORTCUT{<} \TYP{Any}, Y \SHORTCUT{<} \TYP{Any}}]{\EXP{\TYP{ArrayList}\llbracket{}V\rrbracket \rightarrow \TYP{ArrayList}\llbracket{}V\rrbracket}}{\EXP{\TYP{ArrayList}\llbracket{}Y\rrbracket \cap \TYP{List}\llbracket{}X\rrbracket \rightarrow \TYP{List}\llbracket{}X\rrbracket}}
\multicolumn{2}{l}{\EXP{X \SHORTCUT{<} \TYP{Any}, Y \SHORTCUT{<} \TYP{Any}} $\vdash$} \\
\quad\quad\quad\quad\quad\quad
&\EXP{\TYP{ArrayList}\llbracket{}V\rrbracket \rightarrow \TYP{ArrayList}\llbracket{}V\rrbracket} \\
&$\subtypeof$ \EXP{\TYP{ArrayList}\llbracket{}Y\rrbracket \cap \TYP{List}\llbracket{}X\rrbracket \rightarrow \TYP{List}\llbracket{}X\rrbracket}
\end{tabular}
\\[.8em]
%% \EXP{X \SHORTCUT{<} \TYP{Any}, Y \SHORTCUT{<} \TYP{Any}} \; &\vdash \EXP{\TYP{ArrayList}\llbracket{}V\rrbracket \rightarrow \TYP{ArrayList}\llbracket{}V\rrbracket} \\
%% \;&\subtypeof\; \EXP{\TYP{ArrayList}\llbracket{}Y\rrbracket \cap \TYP{List}\llbracket{}X\rrbracket \rightarrow \TYP{List}\llbracket{}X\rrbracket}
Once again, the type \TYP{BadList} proves that this set of overloaded declarations must be rejected. Consider the instance:
\begin{FortressCode}
{\tt ~~}\+\VAR{tail}\, \TYP{List}\llbracket\mathbb{Z}\rrbracket\COLON\TYP{List}\llbracket\mathbb{Z}\rrbracket\-
\end{FortressCode}
of the first declaration. We need to find an instance of the second declaration that is applicable to \TYP{BadList}
and has a return type that is a subtype of \EXP{\TYP{List}\llbracket\mathbb{Z}\rrbracket},
but the only instance of the second declaration applicable to \TYP{BadList} is:
\begin{FortressCode}
{\tt ~~}\+\VAR{tail}\, \TYP{ArrayList}\llbracket\TYP{String}\rrbracket\COLON\TYP{ArrayList}\llbracket\TYP{String}\rrbracket\-
\end{FortressCode}
whose return type is not a subtype of \EXP{\TYP{List}\llbracket\mathbb{Z}\rrbracket}.
Therefore, this set of overloaded declarations must be rejected.
