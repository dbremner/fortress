\subsection{Type Inference}

Before describing a system for ensuring progress and preservation,
it is important to discuss some implications of these conditions on
type inference in a programming language.
The application of a function declaration to a type requires
instantiation of the type parameters of the declaration.
In most programming languages with parametric polymorphism,
a type inference mechanism automatically instantiates type parameters
based on the types of the arguments and the enclosing context.
But note that our progress and preservation conditions do not require
that the type parameters of the function declaration 
that is (dynamically) most specific 
of those applicable to the ilks of the argument values 
are the same as the type parameters
of the function declaration that is (statically) most specific 
of those applicable to the static types of the argument expressions.
Thus, the results of static type inference do not tell us how to
instantiate the type parameters of a most specific function
declaration at run time.  
In the Fortress programming language,
type inference is performed statically, 
and the results of that inference 
are passed to the run-time system 
to ensure that run-time type inference at a function call is sound.
The rules for overloaded function declarations
introduced in Section~\ref{sec:rules} ensure that
the declaration of the dynamically most applicable function declaration,
when instantiated with whatever we infer at run time,
is more specific than the declaration of the statically most applicable
function declaration, instantiated with what was inferred at compile time.
%% For the dynamically most applicable function declaration,
%% the instantiated declaration with whatever instantiation we infer
%% must be more specific than the instantiated declaration of the
%% statically most applicable function declaration with what was inferred
%% at compile time.


%% For the dynamically most applicable function declaration,
%% the instantiated declaration with whatever instantiation we infer
%% must be more specific than the instantiated declaration of the
%% statically most applicable function declaration with what was inferred
%% at compile time.
%% The rules on the overloaded function declarations
%% introduced in Section~\ref{sec:rules} ensure that
%% dynamically inferred types satisfy this requirement.



Aside from this caveat, our system for checking overloaded declarations 
is largely independent of how a specific type inference engine would choose 
instantiations\footnote{Type inference manifests itself as the choice of instantiation of type variables in existential and universal subtyping; specifically, $\bar{V}$ in the inference rules for subtyping in Figure~\ref{fig:sub}. Mitchell \cite{mitchell88} first showed how type inference interacts with polymorphic subtyping.}. Thus we do not discuss the specific features of a type inference
system further in this paper.

% Section 2.1 last para: We should comment that this definition of
% well-formedness agrees with that of the old system. If we view
% monomorphic function declarations as polymorphic declarations with no
% type parameters, then for each monomorphic declaration there is
% exactly one instantiation of it (with an empty vector of type
% arguments).  Then Preservation specifically says that if (monomorphic)
% declaration f U: V is applicable to T, then for the most specific
% declaration that is applicable to T, f' U': V', we have that V' <: V.
% Thus if the polymorphic well-formedness is satisfied, so is the
% monomorphic well-formedness.

\subsection{Modularity}

To demonstrate the modularity of our design,
we present a lightweight modeling of program modules,
and show how applying our rules to each module separately
suffices to guarantee the soundness of the entire program.
In our model, 
a program is a \emph{module}, which may be either \emph{simple} or \emph{compound}.
A \emph{simple module} consists of 
(i) a class table and 
(ii) a collection of function declarations.
That is, a simple module is just a program as described in the rest of this paper.
It is well-formed if it satisfies the well-formedness conditions of a whole program,
as described in previous sections.

A \emph{compound module} combines multiple modules, 
possibly renaming members (i.e., classes and functions) of its constituents.
More precisely, a compound module is a collection of \emph{filters},
where a filter consists of a module 
and a complete mapping from names of members of the module to names.
The name of a member that is not renamed is simply mapped to itself.

The semantics of a compound module is the semantics of the simple module
that results from recursively \emph{flattening} the compound module as follows:
\begin{itemize}

\item
Flattening a simple module simply yields the same module.

\item Flattening a compound module \VAR{C}
consisting of filters (module/mapping pairs) $(c_1, m_1), \ldots, (c_N, m_N)$
yields a simple module whose class table and collection of function declarations
is the concatenation of the class tables and collections of function declarations
of $s_1, \ldots, s_N$, where $s_i$ is the simple module resulting from
first flattening $c_i$ and then renaming all members
of the resulting simple module according to the mapping $m_i$.

\end{itemize}
A compound module is well-formed if its flattened version is well-formed.
This requirement implies that the type hierarchies in each constituent component 
are consistent with the type hierarchy in the flattened version.

We can now use this model of modularity to see 
that we can separately compile and combine modules.

First consider the case of a collection of modules with no overlapping function names
such that each module has been checked separately 
to ensure that the overloaded functions within them satisfy the overloading rules.
Because the type hierarchies of each constituent of a compound module 
must be consistent with that of the compound module, 
all overloaded functions in the resulting compound module
also obey the overloading rules.

Now consider the case of a collection of separately checked modules 
with some overlapping function names.
When overloaded functions from separate modules are combined, 
there are three rules that might be violated
by the resulting overloaded definitions: 
(1) the Meet Rule, (2) the No Duplicates Rule, (3) the Return Type Rule.
If the Meet Rule is violated, 
the programmer need only define yet another module to combine 
that defines the missing meets of the various overloaded definitions.
If the No Duplicates Rule or the Return Type Rule is violated, 
the programmer can fix the problem by renaming functions 
from one or more combined components to avoid the clash; 
the programmer can then define another component 
with more overloadings of the same function name 
that dispatch to the various renamed functions in the manner the programmer intends.

Consider the following example:\footnote{Suggested by 
an anonymous reviewer of a previous version of this paper.}
Suppose we have a type Number in module \VAR{A}, with a
function:
\begin{FortressCode}
{\tt ~~}\+\VAR{add} \COLONOP (\TYP{Number}, \TYP{Number}) \rightarrow \TYP{Number}\-
\end{FortressCode}
Suppose we have the type and function:
\begin{FortressCode}
{\tt ~~}\+\TYP{BigNum} \SHORTCUT{<} \TYP{Number} \\
  \VAR{add} \COLONOP (\TYP{BigNum}, \TYP{BigNum}) \rightarrow \TYP{BigNum}\-
\end{FortressCode}
in module \VAR{B} and the type and function:
\begin{FortressCode}
{\tt ~~}\+\TYP{Rational} \SHORTCUT{<} \TYP{Number} \\
  \VAR{add} \COLONOP (\TYP{Rational}, \TYP{Rational}) \rightarrow \TYP{Rational}\-
\end{FortressCode}
in module \VAR{C}.

Each of modules \VAR{B} and \VAR{C} satisfy the No Duplicates and Meet rules.
Now, suppose we define two compound modules \VAR{D} and \VAR{E}, 
each of which combines modules \VAR{B} and \VAR{C} without renaming \VAR{add}.
In each of \VAR{D} and \VAR{E}, 
we have an ambiguity 
in dispatching calls to \VAR{add} with types \EXP{(\TYP{BigNum}, \TYP{Rational})} or \EXP{(\TYP{Rational}, \TYP{BigNum})}.
Our rules require adding two declarations in each of \VAR{D} and \VAR{E}
to resolve these ambiguities.

Now let us suppose we wish to combine \VAR{D} and \VAR{E} into a compound component \VAR{F}.
Without renaming, this combination would violate the No Duplicates Rule;
each of \VAR{D} and \VAR{E} has an implementation of \EXP{\VAR{add}(\TYP{Bignum}, \TYP{Rational})}. 
To resolve this conflict, 
the program can rename \VAR{add} from both \VAR{D} and \VAR{E}, 
and define a new \VAR{add} in \VAR{F}. 
This new definition could dispatch to either of the renamed functions from \VAR{D} or \VAR{E}, 
or it could do something entirely different, 
depending on the programmer's intent.
