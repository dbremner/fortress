%%%%%%%%%%%%%%% BEGIN REDUCTION FIGURE %%%%%%%%%%%%%%%%
\begin{figure}[t]
\begin{tabular}{c}
\begin{minipage}{.45\textwidth}
\newjudge{Existential reduction}{\jtreds{\delta}{\delta}{\phi}}
\infrule
  {\jcequiv{T}{\BottomType}{\TRUE}}
  {\jtreds{\exttype{T}}{\BottomType}{[]}}

% \infrule
%   {\jcnonequiv{T}{\BottomType}{\C} \\
%    \jcsolve{\C}{\NONE}}
%   {\jtred{\exttype{T}}{\exttype{T}}}
% 
% \infrule
%   {\jcnonequiv{T}{\BottomType}{\C} \\
%    \jcsolve{\C}{\phi} \andalso \phi[\Delta] = \NONE}
%   {\jtred{\exttype{T}}{\exttype{T}}}

\infrule
  {\jcnonequiv{T}{\BottomType}{\C} \\
   \jcsolve{\C}{\phi} \andalso \phi[\Delta] = \Delta'}
  {\jtreds{\exttype{T}}{\exttype[\Delta']{\phi(T)}}{\phi}}


%\vspace*{-.5em}
\newjudge{Bounds substitution}{\phi[\Delta] = \Delta}
%\vspace*{-.8em}
\infrule
  {\Delta = \bds{X}{M}
      \andalso \phi(\bar{X}) = \bar{Y}, \bar{T}
      \andalso \bar{\bar{N}} = \bar{\phi^{-1}[Y,\Delta]} \\[.2em]
    \forall i. \quad \jcsub[\bds{Y}{N}]{\phi(X_i)}{\bar{\phi(M_i)}}{\TRUE}}
  {\phi[\Delta] = \bds{Y}{N}}

% \infrule
%   {\Delta = \bds{X}{M}
%       \andalso \phi(\bar{X}) = \bar{Y} \sqcup \bar{T}
%       \andalso \bar{N} = \phi^{-1}[\bar{Y},\Delta] \\
%     \exists i. \quad \jcsub[\bds{Y}{N}]{\phi(X_i)}{\bd{\phi(M_i)}}{\C_i}
%       \andalso \C_i \neq \FALSE}
%   {\phi[\Delta] = \NONE}

%\vspace*{-.5em}
\newjudge{Bounds transfer}{\phi^{-1}[X,\Delta] = \bar{T}}
%\vspace*{-.8em}
\infrule
  {\{\bar{T}\} = \{\phi(\Delta(X)) \mid X \in \textit{dom}(\Delta),\; \phi(X) = Y\}}
  {\phi^{-1}[Y,\Delta] = \textit{conjuncts}(\bigcap \bar{T})}

\vspace*{-1em}
%% CONJUNCTS DEF
\[
\begin{array}{l}
\textit{conjuncts}(T)\\
\syndef
  \begin{cases}
    \textit{conjuncts}(T_1), \textit{conjuncts}(T_2)\quad & \text{if } T = T_1 \cap T_2 \\
    T & \text{otherwise}
  \end{cases}
\end{array}
\]
\end{minipage}
\end{tabular}
  \caption{Reduction of existential types}
  \label{fig:exred}
\end{figure}
%%%%%%%%%%%%%%% END REDUCTION FIGURE %%%%%%%%%%%%%%%%

We can recover the usual subtyping judgment and define the exclusion judgment as follows:\\[1em]
\begin{tabular}{cc}
\begin{minipage}{0.23\textwidth}
\infrule
{\jcsub{S}{T}{\TRUE}}
{\jgsub{S}{T}}
\end{minipage}
&
\begin{minipage}{0.23\textwidth}
\infrule
{\jexc{S}{T}{\TRUE}}
{\jgtemplate{S}{\exc}{T}}
\end{minipage}
\\[1.5em]
\end{tabular}
\noindent which state that, under assumptions $\Delta$, if the predicate generates the constraint $\TRUE$ (or some constraint that simplifies to $\TRUE$), then that predicate is proved.

% We now use the non-equivalence judgment to make rigorous our intuition about
% how polymorphic exclusion affects the mechanical verification of the overloading rules.
To allow more overloaded functions such as \VAR{tail} and \VAR{foo} as valid overloadings,
we adjust the subtype relation to take into account the relationships between type variables:\\[1em]
%described by the constraints defined in Section~\ref{sec:constraints}:
\begin{tabular}{c}
\begin{minipage}{0.45\textwidth}
\infrule
  {\jtred{\delta}{\delta'} \andalso \jqsub{\delta'}{T}}
  {\jqsub{\delta}{T}}
\end{minipage}
\\[1.5em]
\begin{minipage}{0.45\textwidth}
\infrule
  {\jtreds{\exttype[\Delta_1]{T}}{\delta'}{\phi} \\[.3em]
   \jqsub{S}{\unitype[{\phi[\Delta_1]}]{\left(\arrowtype{\phi(T)}{\phi(U)}\right)}}}
  {\jqsub{S}{\unitype[\Delta_1]{\arrowtype{T}{U}}}}
\end{minipage}
\\[2em]
\end{tabular}
A reduction judgment on existential types $\vdash\delta\eqred\delta',\phi$
defined in Figure~\ref{fig:exred} reduces $\delta = \exttype{T}$ to
$\delta' = \exttype[\Delta']{T'}$ with the substitution $\phi$
under the assumption that $T$ is not equivalent to \BottomType; or,
if $T$ is equivalent to \BottomType, the reduced existential $\delta'$ is \BottomType.
When the substitution is unnecessary we omit it.
The $solve$ operation is like that defined in Smith and Cartwright~\cite{smith08}.
In fact, an existential type $\delta$ reduces to $\delta'$
such that $T \ni \delta$ if and only if $T \ni \delta'$; therefore,
we have $\delta' \le \delta$.
Reducing an existential type in this fashion involves the same kind of
type analysis required for type checking generalized algebraic data types
\cite{simonet07,jones09}.



As an example, consider the following reduction:\\[.8em]
\begin{tabular}{l}
$\vdash$ \EXP{\exists\llbracket{}X \SHORTCUT{<} \TYP{Any}, Y \SHORTCUT{<} \mathbb{Z}\rrbracket\bigl(\TYP{ArrayList}\llbracket{}X\rrbracket \cap \TYP{List}\llbracket{}Y\rrbracket\bigr)}\\[.3em]
\andalso~~~~~~$\eqred$ \EXP{\exists\llbracket{}W \SHORTCUT{<} \mathbb{Z}\rrbracket\TYP{ArrayList}\llbracket{}W\rrbracket}
\end{tabular}
\\[.8em]
with substitution $[W/X,\, W/Y]$. We first check under what constraints $\C$
the intersection \EXP{\TYP{ArrayList}\llbracket{}X\rrbracket \cap \TYP{List}\llbracket{}Y\rrbracket} is not equivalent to \BottomType:
with polymorphic exclusion (specifically, the absence of multiple instantiation inheritance)
we know that \EXP{\TYP{ArrayList}\llbracket{}X\rrbracket} excludes \EXP{\TYP{List}\llbracket{}Y\rrbracket},
which makes their intersection equivalent
to \BottomType, unless $X \equiv Y$ is true. Solving the constraint $X
\equiv Y$ yields a type substitution like $\phi = [W/X,\, W/Y]$. The
judgment $\phi[\Delta] = \Delta'$ lets us construct reduced bounds
from $\phi$ and the original bounds $\Delta$. To do this, we first
partition $\phi(\bar{X})$ into a list of type variables $\bar{Y}$ and a
list of other types $\bar{T}$. In our example, $\phi(X, Y) = W$ gets
partitioned into $W$ and $\emptyset$. Then we need to construct a new
bound $\phi^{-1}[Y_i, \Delta]$ for each $Y_i$ in $\bar{Y}$ by
conjoining the bounds for every type variable in $\phi^{-1}(Y)$. In
our example, $X$ and $Y$ map to $W$, so the bounds for $W$ are \EXP{\lbrace\TYP{Any}, \mathbb{Z}\rbrace},
which we take as the new bounds environment $\Delta'$. We must
ensure that the substitution does not produce invalid bounds, so we
check $\bar{\Delta' \vdash \phi(X) <: \bar{\phi(M)}}$. In our example,
\EXP{W \SHORTCUT{<} \lbrace\mathbb{Z}, \TYP{Any}\rbrace} easily proves that \EXP{W \SHORTCUT{<} \mathbb{Z}} and \EXP{W \SHORTCUT{<} \TYP{Any}}. With
$\Delta'$ the reduced existential type is simply
$\exists\ob{\Delta'}\phi(T)$, where $T$ is the constituent type of the
original existential. In our example, the final, reduced existential
type is \EXP{\exists\llbracket{}W \SHORTCUT{<} \mathbb{Z}\rrbracket\TYP{ArrayList}\llbracket{}W\rrbracket}.
%
Once we have augmented the subtyping relation with existential
reduction, we can finally check that the declarations $\D(\VAR{foo})$ from
Section~\ref{sec:problems} satisfy the Meet Rule.


Similiarly, to check that the declarations $\D(\VAR{tail})$ from Section~\ref{sec:problems}
satisfy the Return Type Rule, we need to show the following:\\[.8em]
\begin{tabular}{l}
$\vdash$
\EXP{\forall\llbracket{}X\SHORTCUT{<}\TYP{Any}\rrbracket\TYP{ArrayList}\llbracket{}X\rrbracket \rightarrow \TYP{ArrayList}\llbracket{}X\rrbracket}\\
$\le$
\EXP{\forall\llbracket{}X\SHORTCUT{<}\TYP{Any},Y\SHORTCUT{<}\TYP{Any}\rrbracket\bigl(\TYP{ArrayList}\llbracket{}X\rrbracket \cap \TYP{List}\llbracket{}Y\rrbracket\bigr)}\\
\hspace*{11.2em}
\EXP{\rightarrow \TYP{List}\llbracket{}Y\rrbracket}
\end{tabular}
\\[.8em]
%EXISTS[\X <: Any, Y <: ZZ\](ArrayList[\X\] CAP List[\Y\])
%\andalso~~~$\eqred$ \EXP{\exists\llbracket{}W \SHORTCUT{<} \mathbb{Z}\rrbracket\TYP{ArrayList}\llbracket{}W\rrbracket}
By the adjusted subtype relation in this section and the rules in Figure~\ref{fig:exred},
we can show the following:\\[.8em]
\begin{tabular}{l}
$\vdash$
\EXP{\exists\llbracket{}X\SHORTCUT{<}\TYP{Any}, Y\SHORTCUT{<}\TYP{Any}\rrbracket\TYP{ArrayList}\llbracket{}X\rrbracket \cap \TYP{List}\llbracket{}Y\rrbracket}\\[.3em]
\andalso~~~$\eqred$ \EXP{\exists\llbracket{}W \SHORTCUT{<} \TYP{Any}\rrbracket\TYP{ArrayList}\llbracket{}W\rrbracket}
\end{tabular}
\\[.8em]
with substitution $[W/X, W/Y]$.  Because the substitution satisfies the following:\\[.8em]
\begin{tabular}{l}
$\vdash$
\EXP{\forall\llbracket{}X\SHORTCUT{<}\TYP{Any}\rrbracket\TYP{ArrayList}\llbracket{}X\rrbracket \rightarrow \TYP{ArrayList}\llbracket{}X\rrbracket}\\
$\le$
\EXP{\forall\llbracket{}X\SHORTCUT{<}\TYP{Any}\rrbracket\TYP{ArrayList}\llbracket{}X\rrbracket \rightarrow \TYP{List}\llbracket{}X\rrbracket}\\
\end{tabular}
\\[.8em]
we can verify that the declarations $\D(\VAR{tail})$ satisfy the Return Type Rule.


% A similar analysis shows that if an instance of a universal arrow has
% the domain \TYP{BottomType}, then it is irrelevant for the purposes of
% guaranteeing Progress. Therefore we can use our reduction rule for
% existential types to aid in the verification of the Return Type Rule
% by augmenting the subtype rules for universal arrows:
% \\[-1.5em]
% \infrule
%   {\jtreds{\exttype[\Delta_1]{T}}{\delta'}{\phi} \andalso
%    \jqsub{S}{\unitype[{\phi[\Delta_1]}]{\left(\arrowtype{\phi(T)}{\phi(U)}\right)}}}
%   {\jqsub{S}{\unitype[\Delta_1]{\arrowtype{T}{U}}}}
%  % {\Delta \vdash S <: reduce(\EXP{\forall\llbracket\bar{X} \SHORTCUT{<} \bar{M}\rrbracket{}S \hbox{\char'134}\VAR{to}\, T})}
%  % {\Delta \vdash S <: \EXP{\forall\llbracket\bar{X} \SHORTCUT{<} \bar{M}\rrbracket{}S \hbox{\char'134}\VAR{to}\, T}}
% It is easy to see from the previous existential reduction instance that this subtype judgment can be proven,
% thus allowing us to verify the Return Type Rule for the function $\D(tail)$ from Section~4.

%% Because this subtype judgment holds for the previous existential
%% reduction instance, we can verify the Return Type Rule for the
%% function $\D(tail)$ from Section~4.
%% %% It is easy to see from the previous existential reduction instance
%% %% that this subtype judgment holds, thus allowing us to verify
%% %% the Return Type Rule for the function $\D(tail)$ from Section~4. 
