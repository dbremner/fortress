\subsection{Types}

Following Kennedy and Pierce \cite{kennedy07},
we define a world of types ranged over by metavariables $S$, $T$, $U$, $V$, and $W$. 
Types are of four forms: 
\emph{type variables} 
(ranged over by metavariables $X$, $Y$, and $Z$);
\emph{constructed types} 
(ranged over by metavariables $K$, $L$, $M$ and $N$), 
written \EXP{C\llbracket\bar{T}\rrbracket} 
where \VAR{C} is a type constructor 
and \EXP{\bar{T}} is a list of types; 
\emph{structural types},
consisting of arrow and tuple types;
and \emph{compound types},
consisting of intersection and union types. 
In addition, 
there are two special constructed types, \Any\ and \BottomType, explained below.
\TODO{Make \BottomType\ special, not a constructed type.}
% A type is of one of five forms:
% a \emph{type variable} 
% (represented by metavariable $X$, $Y$, or $Z$);
% \emph{constructed types} 
% (represented by metavariable $K$, $L$, $M$ or $N$), 
% which is either a type constructor application \EXP{C\llbracket\bar{T}\rrbracket},  
% where \VAR{C} is a type constructor 
% and \EXP{\bar{T}} is a list of types, 
% or the special constructed type \Any;
% a \emph{structural type},
% which is either an arrow type or a tuple type;
% a \emph{compound type},
% which is either an intersection or union type; 
% or the special type \BottomType, 
% which represents the uninhabited type 
% (i.e., no value belongs to \BottomType).
The abstract syntax of types is defined in BNF as follows
(where $\bar{\emph{A}}$ indicates 
a possibly empty comma-separated sequence of syntactic elements $\emph{A}$):
%and we often abbreviate sequences
%of larger syntactic constructs such as $T \extends U$ 
%by writing bars over each variable:
%$\bar{T \extends U}$.)
\[
\begin{array}{@{}l@{\;}l@{\;}l@{\;\;\;\;\;\;}l@{}}
\emph{T} &::=& \emph{X} & \hbox{\rm type variable}\\
&\mid& \emph{C}\llbracket\bar{\emph{T}}\rrbracket & \hbox{\rm type constructor application}\\
&\mid& \emph{T} \rightarrow \emph{T} & \hbox{\rm arrow type}\\
&\mid& ( \bar{\emph{T}} ) & \hbox{\rm tuple type}\\
&\mid& \emph{T} \cap \emph{T} & \hbox{\rm intersection type}\\
&\mid& \emph{T} \cup \emph{T} & \hbox{\rm union type}\\
&\mid& \Any \\
&\mid& \BottomType \\
\end{array}
\]
% \[
% \begin{array}{@{}l@{\;}l@{\;}l@{\;\;\;\;\;\;}l@{}}
% \emph{Type} &::=& \emph{Id} & \hbox{\rm type variable}\\
% &\mid& \emph{Id}\llbracket\bar{\emph{Type}}\rrbracket & \hbox{\rm constructed type}\\
% &\mid& \emph{Type} \rightarrow \emph{Type} & \hbox{\rm arrow type}\\
% &\mid& ( \bar{\emph{Type}} ) & \hbox{\rm tuple type}\\
% &\mid& \emph{Type} \cap \emph{Type} & \hbox{\rm intersection type}\\
% &\mid& \emph{Type} \cup \emph{Type} & \hbox{\rm union type}\\
% %&\mid& \Any \\
% %&\mid& \BottomType \\
% \end{array}
% \]

A tuple type of length one is synonymous with its element type. 
A tuple type with any \BottomType\ element 
is synonymous with \BottomType.
As in Fortress, 
compound types---intersection and union types---and \BottomType\ 
are \emph{not} first-class:
they cannot be written in a program; 
rather, they are used by the type analyzer during type checking.
For example, type variables may have multiple bounds, 
so that any valid instantiation of such a variable
must be a subtype of the intersection of its bounds.

% All type checking \TODO{Do we prefer ``type-checking'',
% ``type checking'', or ``typechecking''?)
% is done within the context of a \emph{class table} $\T$, 
% which is a set of type constructor declarations 
% (at most one declaration for each type constructor) 
% of the following form:
% \[
% C\tplist{X}{M} \extends \{\bar{N}\}
% \]

\TODO{Only considering ground types at first.
(Define ground types.)}

Constructed types (other than \Any\ and \BottomType)
are applications of \emph{type constructors}.
A \emph{class table} $\T$
is a set of type constructor declarations 
(at most one declaration for each type constructor) 
of the following form:
\[
C\tplist{X}{M} \extends \{\bar{N}\}
\]
This declaration indicates that an application $C\obb{U}$ 
(\emph{i}) is \emph{well-formed} (with respect to $\T$)
if and only if $|\bar{U}| = |\bar{X}|$ and
for all $i$ and $j$, 
$U_i \subtypeof \substb{U}{X}M_{ij}$ for each bound $M_{ij}$
(where $\subtypeof$ is the subtyping relation defined below, 
and $\substb{U}{X}M_{ij}$ is $M_{ij}$ 
with $U_k$ substituted for each occurrence of $X_k$ in $M_{ij}$ 
for $1 \leq k \leq |\bar{U}|$);
and 
(\emph{ii}) is a subtype of $\substb{U}{X}N_l$ for $1 \leq l \leq |\bar{N}|$.
Thus, a class table induces 
a (nominal) \emph{subtyping relation} over the constructed types 
by taking the reflexive and transitive closure 
of the subtyping relation derived from the declarations in the class table.
In addition, 
every type is a subtype of \Any\ and a supertype of \BottomType.
For any type $T$ (of any form),
we write $T \in \T$ to mean that 
any constructed type occurring in type $T$
is well-formed with respect to $\T$.

A class table $\T$ is \emph{well-formed} if the resulting subtyping relation 
on its constructed types is a partial order.
As usual for languages with nominal subtyping, 
we allow recursive and mutually recursive references in $\T$.
\TODO{What does ``recursive and mutually recursive references'' mean?}
A class table $\T'$ is an \emph{extension} of $\T$ (written $\T' \ctext \T$)
if every constructor declaration in $\T$ is also in $\T'$ 
and the subtype relation on $\T'$ agrees with that of $\T$. 
Consequently, if $\T' \ctext \T$ then, for all types $T$, $T \in \T$ implies $T \in \T'$.
We typically omit explicit reference to the class table when it is understood, 
and we assume that the class table is well-formed.

Given a well-formed class table 
containing the type constructor declaration for $C$ above 
and a well-formed application $C\obb{T}$,
we denote the set of its explicitly declared supertypes by
\[
\myextends{C\obb{T}} = \{ \bar{\substb{T}{X}N} \}
\]
and the set of ancestors of $C\obb{T}$ 
(defined recursively) by
\[
\ancestors(C\obb{T}) 
   = \{C\obb{T}\} \cup 
     \hspace*{-4ex} \bigcup_{M \in \myextends{C\obb{T}}} \hspace{-4ex} \ancestors(M).
\]
(To reduce clutter, 
nullary applications are written without brackets; 
for example, \EXP{C\llbracket\,\rrbracket} is written \VAR{C}.) 
\TODO{Other for \Any\ and \BottomType, 
do we actually do this?}

Structural and compound types are \emph{well-formed} 
(with respect to a class table) 
if their constituent types are well-formed. 
We extend the subtyping relation to
structural and compound types in the usual way:
Arrow types are contravariant in their domain types 
and covariant in their range types 
(i.e., $\arrowtype{S}{T} \subtypeof \arrowtype{U}{V}$
if and only if $U \subtypeof S$ and $T \subtypeof V$).
One tuple type is a subtype of another 
if and only if they have the same number of elements, 
and each element of the first is a subtype of the corresponding element of the other 
(i.e., $( \tuple{\bar{S}} ) \subtypeof ( \tuple{\bar{T}} )$
if and only if $|\bar{S}| = |\bar{T}|$
and $S_i \subtypeof T_i$ for all $1 \leq i \leq |\bar{S}|$).
An intersection type is by definition the most general type that is a subtype
of each of its element types: $(A \cap B) <: A$, $(A \cap B) <: B$, and for all types $T$,
if $T <: A$ and $T <: B$ then $T <: (A \cap B)$.
Similarly, a union type is by definition the most specific type that is a supertype
of each of its element types: $A <: (A \cup B)$, $B <: (A \cup B)$, and for all types $T$,
if $A <: T$ and $B <: T$ then $(A \cup B) <: T$.

To extend the subtyping relation to type variables,
we require  a \emph{type environment}, 
which maps type variables to bounds:
\[
\Delta = \bds{X}{M}
\]
In the context of $\Delta$, 
each type variable $X_i$ is a subtype of each of its bounds $M_{ij}$.
Note that the type variables $X_i$ may appear within the bounds $M_{ij}$.
We write $\jgsub{S}{T}$ 
to indicate the judgment that $S$ is a subtype of $T$ 
in the context of $\Delta$. When $\Delta$ is understood to be empty, 
we write this judgment as simply $S \subtypeof T$. The subtype judgment
can only be made on types $S, T \in \T$, so the judgment takes a class table $\T$ as an implicit parameter.
The types $S$ and $T$ are said to be \emph{equivalent}, written $S \equiv T$, when $S \subtypeof T$ and $T \subtypeof S$.

To allow separate compilation of program components, 
we do not assume that the class table is complete;
there might be declarations yet unknown.
Specifically, 
we cannot infer that two constructed types 
have no common constructed subtype (other than \BottomType) 
from the lack of any such type in the class table.
However, we do assume that each declaration is complete, 
and furthermore, 
that any type constructor used in the class table 
(e.g., in a bound or a supertype of another declaration)
is declared in the table, 
so that all the supertypes of a constructed type 
are known.


\subsection{Values and Ilks}

Types are intended to describe the values that might be produced by
an expression or passed into a function.
In Fortress, for example, there are three kinds of values: 
objects, functions, and tuples;
every object belongs to at least one constructed type,
every function belongs to at least one arrow type,
and every tuple belongs to at least one tuple type.
When we say that two types \VAR{T} and \VAR{U} have \emph{the same extent},
we mean that for every value \VAR{v}, 
\VAR{v} belongs to \VAR{T} if and only if \VAR{v} belongs to \VAR{U}.

We place a requirement on values and on the type system that describes them: 
Although a value may belong to more than one type, 
for every value \VAR{v} there is a unique type \EXP{\VAR{ilk}(v)} 
(the \emph{ilk} of the value) 
that is \emph{representable in the type system}\footnote{The
type system presented here satisfies this requirement 
simply by providing intersection types.  
The Fortress type system happens to satisfy it in another way as well, 
which is typical of object-oriented language designs: 
every object is created as an instance of a single nominal constructed type, 
and this type is its ilk.} 
and has the property that for every type \VAR{T}, 
if \VAR{v} belongs to \VAR{T} then \EXP{\VAR{ilk}(v) \SHORTCUT{<} T};
moreover, \EXP{\VAR{ilk}(v) \neq \TYP{BottomType}}.  
(This notion of \VAR{ilk} corresponds to what is sometimes called the
``class'' or ``run-time type'' of the value.\footnote{%
We prefer the term ``ilk'' to ``run-time type'' 
because the notion---and usefulness---of 
the most specific type to which a value belongs 
is not confined to run time.
We prefer it to the term ``class,'' 
which is used in {\it The Java Language Specification}~\cite{JavaSpec}, 
because not every language uses the term ``class'' 
or requires that every value belong to a class.  
For those who like acronyms, 
we offer the mnemonic retronyms 
``implementation-level kind'' 
and ``intrinsically least kind.''})

The implementation significance of ilks is that it is possible to
select the dynamically most specific applicable function
from an overload set using only the ilks of the argument values; no
other information about the arguments is needed.

In a sound type system,
if an expression is determined by the type system to have type \VAR{T}, 
then every value computed by the expression at run time
will belong to type \VAR{T}; 
moreover, 
whenever a function whose ilk is \EXP{U\rightarrow{}V} is applied to an argument value,
then the argument value will belong to type \VAR{U}.


\subsection{Overloaded Functions}
\label{terms}

A function declaration consists of 
a name, 
a sequence of type parameter declarations 
(enclosed in white square brackets), 
a type indicating the domain of the function, 
and a type indicating the range of the function.  
A type parameter declaration consists of
a type parameter name and its bounds.
We omit the white square brackets of a declaration 
when the sequence of type parameter declarations is empty.
The abstract syntax of function declarations is as follows:
\[
\begin{array}{lll}
\emph{Decl} &::=& 
%\emph{Id}\llbracket\bar{\emph{Id}}\SHORTCUT{<}\bar{\emph{Type}}\rrbracket \emph{Type} \COLON \emph{Type}\\
\declg{\textit{Id}}{\textit{Id}}{\textit{Type}}{\textit{Type}}{\textit{Type}}  \\
&\mid& 
%\emph{Id}\ \emph{Type} \COLON \emph{Type}\\
\decl{\textit{Id}}{\textit{Type}}{\textit{Type}}
\end{array}
\]

For example, in the following function declaration:

\begin{FortressCode}
{\tt ~~}\+f\llbracket{}X \SHORTCUT{<} M, Y \SHORTCUT{<} N\rrbracket\bigl(\TYP{List}\llbracket{}X\rrbracket, \TYP{Tree}\llbracket{}Y\rrbracket\bigr)\COLON \TYP{Map}\llbracket{}X, Y\rrbracket\-
\end{FortressCode}
the name of the function is \VAR{f}, 
the type parameter declarations are \EXP{X \SHORTCUT{<} M} and \EXP{Y \SHORTCUT{<} N}, 
the domain type is \EXP{\bigl(\TYP{List}\llbracket{}X\rrbracket, \TYP{Tree}\llbracket{}Y\rrbracket\bigr)}, which is a tuple type,
and the range type is \EXP{\TYP{Map}\llbracket{}X, Y\rrbracket}. We will often abbreviate a function
as \hdeclg{f}{\Delta}{S}{T} when we do not want to emphasize the bounds (we are abusing notation by letting
$\Delta$ range over both type environments and bounds definitions).

A function declaration \declg{f}{X}{N}{S}{T}
may be \emph{instantiated} with type arguments $\bar{W}$ 
if $|\bar{W}| = |\bar{X}|$ and $W_i \subtypeof \substb{W}{X} N_{ij}$ for all $i$ and $j$;
we call $\substb{W}{X} \decl{f}{S}{T}$
the \emph{instantiation} of $f$ with $\bar{W}$. 
When we do not care about $\bar{W}$, 
we just say that $\decl{f}{U}{V}$
is an \emph{instance} of $f$ (and it is understood that $U=\substb{W}{X}S$
and $V=\substb{W}{X}T$ for some (unstated) $\bar{W}$).
%
We use the metavariable $\D$ for a finite collection of sets of
function declarations and $\D(f)$ for the set in
$\D$ that contains all declarations named $f$.
%
An instance \decl{f}{U}{V} of a declaration $f$ 
is \emph{applicable} to a type $T$ 
if and only if $T \subtypeof U$.
A function declaration is \emph{applicable} to a type 
if and only if at least one of its instances is.
%
For any two function declarations $f_1, f_2 \in \D(f)$, 
$f_1$ is \emph{more specific} than $f_2$ 
(written $f_1 \ms f_2$)
if and only if for every type $T$ 
such that $f_1$ is applicable to $T$, 
$f_2$ is also applicable to $T$.


%% \begin{figure}
%%   \begin{minipage}{.462\textwidth}
    
%%   \fbox{\textbf{Type equivalence reduction:} \quad \jtred{\Delta}{T}{T}}
%%   \TODO{rules for flattening/distributing $\cap, \cup$}
%%   \TODO{is subtype/exclusion judgment in premises ok?}
  
%%   % INTERSECTION
%%   \infrule
%%     {\jgsub{\Delta}{T}{U}}
%%     {\jtred{\Delta}{T \cap U}{T}}
%%   \infrule
%%     {\jgsub{\Delta}{U}{T}}
%%     {\jtred{\Delta}{T \cap U}{U}}
%%   \infrule
%%     {\jexc{\Delta}{T}{U}}
%%     {\jtred{\Delta}{T \cap U}{\TYP{BottomType}}}
  
%%   % UNION
%%   \infrule
%%     {\jgsub{\Delta}{T}{U}}
%%     {\jtred{\Delta}{T \cup U}{U}}
%%   \infrule
%%     {\jgsub{\Delta}{U}{T}}
%%     {\jtred{\Delta}{T \cup U}{T}}
  
%%   % ARROW
%%   % \infrule
%%   %   {\jtred{\Delta}{T}{T'}}
%%   %   {\jtred{\Delta}{T \rightarrow U}{T' \rightarrow U}}
%%   % \infrule
%%   %   {\jtred{\Delta}{U}{U'}}
%%   %   {\jtred{\Delta}{T \rightarrow U}{T \rightarrow U'}}
  
%%   % TUPLE
%%   % \infrule
%%   %   {\jtred{\Delta}{T_i}{T_i'}}
%%   %   {\jtred{\Delta}{(T_1, \ldots, T_i, \ldots, T_n)}{(T_1, \ldots, T_i', \ldots, T_n)}}
%%   \infrule
%%     {T_i = \TYP{BottomType}}
%%     {\jtred{\Delta}{(T_1, \ldots, T_i, \ldots, T_n)}{\TYP{BottomType}}}
  
%%   % CONSTRUCTED
%%   % \infrule
%%   %   {\jtred{\Delta}{T_i}{T_i'}}
%%   %   {\jtred{\Delta}{C\ob{T_1, \ldots, T_i, \ldots, T_n}}{C\ob{T_1, \ldots, T_i', \ldots, T_n}}}
  
%%   % VARIABLE
%%   \TODO{should be a subtype judgment instead?}
%%   \infrule
%%     {\EXP{X \SHORTCUT{<} \TYP{BottomType} \in \Delta}}
%%     {\jtred{\Delta}{X}{\TYP{BottomType}}}
  
%%   % REDUCTION CONTEXT GRAMMAR
%%   \newcommand{\OR}{\;|\;}
%%   \[ E \; ::= \; [] \OR E \rightarrow U \OR T \rightarrow E \]
%%   \[ \OR (T_1, \ldots, E, \ldots, T_n) \]
%%   \[ \OR C\ob{T_1, \ldots, E, \ldots, T_n} \]
  
%%   % CONGRUENCE
%%   \infrule
%%     {\jtred{\Delta}{T}{T'}}
%%     {\jtred{\Delta}{E[T]}{E[T']}}
  
%%   \end{minipage}
%%   \caption{Type equivalence reduction}
%%   \label{fig:tred}
%% \end{figure}
