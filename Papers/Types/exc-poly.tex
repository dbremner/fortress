Checking the declarations $\D(\VAR{tail})$ and $\D(\VAR{foo})$ is trickier.
Certainly neither \TYP{List} nor \TYP{ArrayList} declares an exclusion of the other,
so seemingly the exclusion relation does not help us check these declarations.
A fundamental problem is that a type such as \TYP{BadList} might be a subtype
of multiple instantiations of the type constructor \TYP{List} (in this case,
a subtype of \EXP{\TYP{List}\llbracket\mathbb{Z}\rrbracket} directly and of \EXP{\TYP{List}\llbracket\TYP{String}\rrbracket} via \EXP{\TYP{ArrayList}\llbracket\TYP{String}\rrbracket}).
A type hierarchy in which
some type extends multiple instantiations of the same type constructor is said
to exhibit \emph{multiple instantiation inheritance} \cite{kennedy07}.

We address this problem by imposing an additional restriction on well-formed class tables,
which, in effect, adds additional power to the exclusion relation:
\begin{description}
\item[Polymorphic Exclusion Rule] 
A type (other than \BottomType) 
must not be a subtype of multiple distinct instantiations of a type constructor. 
\end{description}
Thus,
distinct well-formed applications $C\obb{T}$ and $C\obb{U}$ 
of a type constructor $C$ exclude each other; 
we use $\bexcp$ for this relation.
Furthermore, 
if $M$ is a subtype of $C\obb{T}$ 
and $N$ is a subtype of $C\obb{U}$, 
then $M$ and $N$ also exclude each other;
we use $\excp$ for this relation, 
and we add it to the other exclusion sub-relations 
whose union forms the exclusion relation over constructed types.

Polymorphic exclusion is easy to statically enforce, 
and practice suggests that it is not onerous: 
it is already in the Java\texttrademark\ programming language.
Furthermore, there is good reason to enforce it:
Kennedy and Pierce have shown that multiple instantiation inheritance
is one of three conditions that lead to the undecidability of
nominal subtyping with variance \cite{kennedy07}.

With polymorphic exclusion in place, 
the type \TYP{BadList} is no longer well-formed in its class table.
However, our subtype judgment still cannot find a type \EXP{W \SHORTCUT{<} \mathbb{Z}} to prove that:
\\[.5em]
\hspace*{.5em}
%\jgsub[\EXP{X \SHORTCUT{<} \TYP{Any}, Y \SHORTCUT{<} \mathbb{Z}}]{\EXP{\TYP{ArrayList}\llbracket{}X\rrbracket \cap \TYP{List}\llbracket{}Y\rrbracket}}{\EXP{\TYP{ArrayList}\llbracket{}W\rrbracket}}
\ensuremath{
\begin{array}{l}
{\EXP{X \SHORTCUT{<} \TYP{Any}, Y \SHORTCUT{<} \mathbb{Z}}}\,\vdash
\\
\quad\quad\quad\quad
{\EXP{\TYP{ArrayList}\llbracket{}X\rrbracket \cap \TYP{List}\llbracket{}Y\rrbracket}}\;{\subtypeof}\;{\EXP{\TYP{ArrayList}\llbracket{}W\rrbracket}}
\end{array}
}
\\[.5em]
% \begin{align*}
% \jgsub[\EXP{X \SHORTCUT{<} \TYP{Any}, Y \SHORTCUT{<} \mathbb{Z}}]{\EXP{\TYP{ArrayList}\llbracket{}X\rrbracket \cap \TYP{List}\llbracket{}Y\rrbracket}}{\EXP{\TYP{ArrayList}\llbracket{}W\rrbracket}}
% \end{align*}
which is necessary to satisfy the Meet Rule.
All that is known about $X$ and $Y$ for this subtype check is that
they are subtypes of \TYP{Any} and \EXP{\mathbb{Z}}, respectively.
If we could somehow equate $X$ and $Y$, transferring $Y$'s tighter
bound \EXP{\mathbb{Z}} to \VAR{X}, then we could use $W = X$ to prove the assertion.
Syntactically, the types $X$ and $Y$ do not provide enough information
to prove the subtype assertion, but with additional assumptions about
their structure (namely, that their extents are equal) we can indeed
prove the assertion.  Incorporating this kind of deduction into our
subtyping judgment as described in Section~\ref{sec:constraints}
allows many more sets of overloaded function declarations
such as \VAR{tail} and \VAR{foo} as valid overloadings.



\begin{comment}
%%%%%%%%%%%%%%%%%%%%%%%%%%%%%%%%%%%%%%%%%%%%%%%%%%%%%%%%%%%%%%%%
% \noindent \textbf{--- scott has not gotten to this line ---}
% 
% Polymorphic exclusion lets us determine whether one existential type
% has the same extent (has the same monomorphic subtypes) as another.
% For example, using polymorphic exclusion, we know that
% the following existential type:
% \begin{FortressCode}
\% (1) \exists\llbracket{}X \SHORTCUT{<} \TYP{Any}, Y \SHORTCUT{<} \TYP{Foo}\,\rrbracket\, \bigl(\TYP{ArrayList}\llbracket{}X\rrbracket \cap \TYP{List}\llbracket{}Y\rrbracket\bigr) \\
\% 
\end{FortressCode}
has the same extent as:
% \begin{FortressCode}
\% (2) \exists\llbracket{}Z \SHORTCUT{<} \TYP{Foo}\rrbracket\, \TYP{ArrayList}\llbracket{}Z\rrbracket \\
\% 
\end{FortressCode}
because, for all ground types \VAR{S} and \VAR{T}, the intersection type
% \EXP{\TYP{List}\llbracket{}S\rrbracket \cap \TYP{ArrayList}\llbracket{}T\rrbracket} is equivalent to \TYP{BottomType}
% unless \VAR{S} is equivalent to \VAR{T}.
% This lets us deduce that \VAR{S} (equivalent to\VAR{T})
% is an appropriate instantiation for \VAR{Z}.
% Therefore, now the following declarations:
% \begin{FortressCode}
%  \\
% \2\2\+h\llbracket{}X \SHORTCUT{<} \TYP{Any}\rrbracket\TYP{ArrayList}\llbracket{}X\rrbracket\COLON\mathbb{Z} \\
% h\llbracket{}Y\mathord{\SHORTCUT{<}}\: \TYP{Foo}\rrbracket\TYP{List}\llbracket{}Y\rrbracket\COLON\mathbb{Z}\-
% \end{FortressCode}
% satisfy the Meet Rule thanks to the following disambiguating declaration:
% \begin{FortressCode}
%  \\
% \2\2\+h\llbracket{}Z \SHORTCUT{<} \TYP{Foo}\rrbracket\TYP{ArrayList}\llbracket{}Z\rrbracket\COLON\mathbb{Z}\-
% \end{FortressCode}
% If we incorporate this form of derivation into the subtyping rules for
% existential and universal types, we can prove that many more sets of
% overloaded declarations satisfy the overloading rules.
% 
% In this section,
% we extend the exclusion relation to existential and universal types and
% refine the subtyping rules for them.
% %
% %  need to know constraints under which a type is not bottom => need to know when two types don't exclude
% %  can get regular exclusion from not exclude
% %
% We also augment our judgments to include \emph{constraints},
% which state subtyping among types that we have inferred to hold.
% For example, because of polymorphic exclusion,
% if we infer that a given type $S$ is a subtype of two instantiations \EXP{C\llbracket\bar{T}\rrbracket} and
% \EXP{C\llbracket\bar{U}\rrbracket} of the same type constructor \VAR{C}, we can infer constraints that
% \EXP{T_{i} \equiv U_{i}} for each \EXP{T_{i} \in \bar{T}} and \EXP{U_{i} \in \bar{U}}.
% We also carry constraints that hold under the supposition
% that an intersection type is not \TYP{BottomType}.
% For example, when we decide whether two existential types (1) and (2) above have the same extent,
% if we explore the ramifications of the possibility that
% \EXP{\TYP{List}\llbracket{}S\rrbracket \cap \TYP{ArrayList}\llbracket{}T\rrbracket} is not equivalent to \TYP{BottomType},
% we need to infer and carry the constraint that \EXP{S \equiv T}.
% 
% To augment the rules in this section, we introduce the following notation:
% \begin{enumerate}
% 
% \item 
% We write $\jcnonequiv{T}{\BottomType}{\C}$
% to indicate that a type $T$ is not equivalent to \BottomType\ 
% under the environment $\Delta$ and
% the assumption that constraints $\C$ hold.
% 
% \item 
% We write  $\jnexc{T}{S}{\C}$ to indicate that
% $T$ does not exclude $S$ under the environment $\Delta$ and
% the assumption that constraints $\C$ hold.
% \end{enumerate}
% 
% % How to get all instantiations exclude (written \exc)
% % \infrule
% %  {\Delta \vdash S \nexc T|False}
% %  {\Delta \vdash S \exc T}
\end{comment}
