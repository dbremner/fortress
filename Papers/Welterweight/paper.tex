\documentclass[10pt]{sigplanconf}
\usepackage{amsmath,graphicx,url,color,alltt,fortify,verbatim,bcprules,tabularx,theorem}
\advance \textheight by 4pt

% make a big red TODO label
\newcommand{\TODO}[1]{\textbf{\emph{\textcolor{red}{TODO}}}: \textsf{\footnotesize #1}}
% \newcommand{\TODO}[1]{}

\newcommand{\kwd}[1]{\mathtt{#1}}

\newcommand{\emptyseq}{\bullet}
\newcommand{\whennonempty}[2]{{\def\nonemptytempA{#1}\def\nonemptytempB{}\ifx\nonemptytempA\nonemptytempB\else#2\fi}}
\newcommand{\turnstile}{\vdash}
%newcommand
\newcommand{\ms}{\preceq}
\renewcommand{\bar}{\overline}
\newcommand{\meet}{\wedge}
\newcommand{\C}{\mathcal{C}}
\newcommand{\quoted}[1]{\begin{quote}#1\end{quote}}
\newcommand{\exc}{\mathrel{\lozenge}}
\newcommand{\nexc}{\mathrel{\hbox to 0pt{$\mskip -1.4mu\not$\hss}\lozenge}}
\newcommand{\smalllozenge}{\vcenter{\hbox{\scalebox{.5}{$\lozenge$}}}}
\newcommand{\normallozenge}{\vcenter{\hbox{$\lozenge$}}}
% \newcommand{\altlozenge}{\ooalign{\hfil$\normallozenge$\hfil\cr\hfil$\smalllozenge$\hfil}}
\newcommand{\altlozenge}{\ooalign{\hfil$\vcenter{\hbox{$\lozenge$}}$\hfil\cr\hfil$\cdot$\hfil}}
\newcommand{\bexc}{\mathrel{\altlozenge}}
\newcommand{\bexcp}{\mathrel{\altlozenge}_\textrm{m}}
\newcommand{\bnexc}{\mathrel{\hbox to 0pt{$\mskip -1.4mu\not$\hss}\altlozenge}}

\newcommand{\countof}{\hbox{\tt\#}}
\newcommand{\apply}{\hbox{\tt\char'100}}
\newcommand{\comprises}{\mathrel{\mathord{\equiv}\mathord{\bigcup}}}

\newcommand{\fresh}[1]{\textit{fresh}({#1})}
\newcommand{\freeVar}[1]{\textit{freeVars({#1})}}

\newcommand{\dontcare}{\_\!\_}
\newcommand{\excr}{\triangleright}
\newcommand{\excl}{\triangleleft}
\newcommand{\excre}{\excr_\textrm{x}}
\newcommand{\excle}{\excl_\textrm{x}}
\newcommand{\excrc}{\excr_\textrm{c}}
\newcommand{\exclc}{\excl_\textrm{c}}
\newcommand{\excro}{\excr_\textrm{o}}
\newcommand{\exclo}{\excl_\textrm{o}}
% \newcommand{\excrp}{\excr_\textrm{m}}
% \newcommand{\exclp}{\excl_\textrm{m}}
\newcommand{\excrx}{\excr_*}
\newcommand{\exclx}{\excl_*}
\newcommand{\excx}{\exc_*}

\newcommand{\exce}{\exc_\textrm{x}}
\newcommand{\excc}{\exc_\textrm{c}}
\newcommand{\exco}{\exc_\textrm{o}}
\newcommand{\excp}{\exc_\textrm{m}}

\newcommand{\propop}{\ensuremath{\mathrel{\ast}}}

\newcommand{\ancestors}{\textit{ancestors}}
\newcommand{\ancexcludes}{\textit{excludes}^*}
\newcommand{\myexcludes}[1]{{#1}.\textit{excludes}}
\newcommand{\mycomprises}[1]{{#1}.\textit{comprises}}
\newcommand{\myextends}[1]{{#1}.\textit{extends}}

\newcommand{\extends}{\ensuremath{<:}}
\newcommand{\subtypeof}{\ensuremath{<:}}
\newcommand{\nsubtypeof}{\not \subtypeof}
\newcommand{\supertypeof}{\ensuremath{:>}}
\newcommand{\leinner}{\ensuremath{\lesssim}}

\newcommand{\alphaequiv}{\ensuremath{\stackrel{\alpha}{\sim}}}
\newcommand{\cequiv}{\ensuremath{\sim}}

\newcommand{\arrowtype}[2]{\mbox{\ensuremath{({#1} \rightarrow {#2})}}}
\newcommand{\tuple}[1]{\ensuremath{(#1)}}
\newcommand{\tupleb}[1]{\ensuremath{(\bar{#1})}}
\newcommand{\bigtuple}[1]{\ensuremath{\big(#1\big)}}
\newcommand{\bigtupleb}[1]{\ensuremath{\big(\bar{#1}\big)}}

\newcommand{\uniontype}[2]{\mbox{\ensuremath{({#1} \cup {#2})}}}
\newcommand{\intersectiontype}[2]{\mbox{\ensuremath{({#1} \cap {#2})}}}

\newcommand{\dom}{\ensuremath{\mathit{dom}}}
\newcommand{\arrow}{\ensuremath{\mathit{arrow}}}
\newcommand{\FV}{\ensuremath{\mathit{FV}}}

% indented code block
\newenvironment{ttquote}%
{\begin{quote}\begin{alltt}}
{\end{alltt}\end{quote}}

\newcommand{\verythinmathspace}{\mskip0.5\thinmuskip}

\newcommand{\set}[1]{\ensuremath{\{{#1}\}}}
\newcommand{\setb}[1]{\ensuremath{\{\bar{#1}\}}}
\newcommand{\bigset}[1]{\ensuremath{\big\{{#1}\big\}}}
\newcommand{\bigsetb}[1]{\ensuremath{\big\{\bar{#1}\big\}}}

% put in oxford brackets
\newcommand{\ob}[1]{\ensuremath{\llbracket\verythinmathspace {#1} \verythinmathspace\rrbracket}}
\newcommand{\bigob}[1]{\ensuremath{\big\llbracket\verythinmathspace {#1} \verythinmathspace\big\rrbracket}}
% put in oxford brackets and an overbar
\newcommand{\obb}[1]{\ensuremath{\llbracket\verythinmathspace \bar{#1} \verythinmathspace\rrbracket}}
\newcommand{\bigobb}[1]{\ensuremath{\big\llbracket\verythinmathspace \bar{#1} \verythinmathspace\big\rrbracket}}
% make a type param bound with the given name
\newcommand{\bd}[1]{\ensuremath{\{{#1}\}}}
\newcommand{\bigbd}[1]{\ensuremath{\big\{{#1}\big\}}}
\newcommand{\bdb}[1]{\ensuremath{\{\bar{#1}\}}}
\newcommand{\bigbdb}[1]{\ensuremath{\big\{\bar{#1}\big\}}}
% syntactic definition
\newcommand{\syndef}{\ensuremath{\overset{\mathrm{def}}{=}}}
% make a substitution
\newcommand{\subst}[2]{\ensuremath{[#1/#2]}}
% make a substitution with bars
\newcommand{\substb}[2]{\ensuremath{[\bar{#1}/\bar{#2}]}}
% type parameter list with bounds and oxford brackets
\newcommand{\tplist}[2]{\ensuremath{\ob{\bds{#1}{#2}}}}
% monomorphic fn decl
\newcommand{\decl}[3]{\mbox{\ensuremath{{#1}\,{#2}\!:\!{#3}}}}
% a generic function declaration 
\newcommand{\declg}[5]{\mbox{\ensuremath{#1 \tplist{#2}{#3}\, #4\!:\!#5}}}
\newcommand{\hdeclg}[4]{\mbox{\ensuremath{#1 \ob{#2}\, #3\!:\!#4}}}
% a class table T
\newcommand{\T}{\ensuremath{\mathcal{T}}}
% class table extension
\newcommand{\ctext}{\ensuremath{\supseteq}}
% a declaration set D
\newcommand{\D}{\ensuremath{\mathcal{D}}}
% a declaration set restricted to a function name
\newcommand{\Df}[1][f]{\D_{\!#1}}
% existential type
\newcommand{\existstype}[2]{\ensuremath{\exists\bigob{#1}{#2}}}
\newcommand{\existstypeb}[2]{\ensuremath{\exists\bigobb{#1}{#2}}}
% universal type
\newcommand{\foralltype}[2]{\ensuremath{\forall\bigob{#1}{#2}}}
\newcommand{\foralltypeb}[2]{\ensuremath{\forall\bigobb{#1}{#2}}}
% reduced existential type
\newcommand{\reduce}[1]{\ensuremath{{#1}_r}}

%%%%% Any and Bottom %%%%

\newcommand{\Any}{\TYP{Any}}
\newcommand{\Bottom}{\TYP{Bottom}}

\newcommand{\FALSE}{\textrm{false}}
\newcommand{\TRUE}{\textrm{true}}

\newcommand{\NONE}{\bullet}

\newcommand{\eqred}{\overset{\equiv}{\longrightarrow}}

%%%%%%% JUDGMENTS %%%%%%%%

%%% NEW SYNTACTIC JUDGMENT
%\newcommand{\newjudge}[2]{\fbox{\textbf{#1:} \quad \ensuremath{#2}}}
\newcommand{\newjudge}[2]{\hbox{{#1:} \quad \fbox{\ensuremath{#2}}}}

% non constrained judgements
\newcommand{\jgtemplate}[4][\Delta]{\ensuremath{\whennonempty{#1}{{#1}\,}\turnstile\,{#2}\;{#3}\;{#4}}}
\newcommand{\jgbtemplate}[4][\Delta]{\ensuremath{\whennonempty{#1}{{#1}\,}\turnstile\,\bar{{#2}\;{#3}\;{#4}}}}
\newcommand{\jgTHREEtemplate}[8]{\ensuremath{\whennonempty{#1}{{#1}\,}\turnstile\,{#2}\;{#3}\;{#4}, {#5}\;{#3}\;{#6}, {#7}\;{#3}\;{#8}}}

\newcommand{\jgshorttemplate}[3][\Delta]{\ensuremath{\whennonempty{#1}{{#1}\,}\turnstile\,{#2}\;{#3}}}
\newcommand{\jgbshorttemplate}[3][\Delta]{\ensuremath{\whennonempty{#1}{{#1}\,}\turnstile\,\bar{{#2}\;{#3}}}}

% ground subtyping
\newcommand{\jgsub}[3][\Delta]{\jgtemplate[#1]{#2}{\subtypeof}{#3}}
\newcommand{\jgnequiv}[3][\Delta]{\jgtemplate[#1]{#2}{\not \equiv}{#3}}

% evaluation
\newcommand{\jevalstep}[3][\Delta]{\jgtemplate[#1]{#2}{\longrightarrow}{#3}}
\newcommand{\jevalstepTWO}[5][\Delta]{\ensuremath{\whennonempty{#1}{{#1}\,}\turnstile\,{#2}\;{\longrightarrow}\;{#3}, {#4}\;{\longrightarrow}\;{#5}}}
\newcommand{\jbevalstep}[3][\Delta]{\jgbtemplate[#1]{#2}{\longrightarrow}{#3}}
\newcommand{\jevalstar}[3][\Delta]{\jgtemplate[#1]{#2}{\longrightarrow^*}{#3}}

% typing
\newcommand{\jtype}[3][\Delta;\Gamma]{\jgtemplate[#1]{#2}{\mathrel{:}}{#3}}
\newcommand{\jbtype}[3][\Delta;\Gamma]{\jgbtemplate[#1]{#2}{\mathrel{:}}{#3}}

%subtyping
\newcommand{\jsubtype}[3][\Delta]{\jgtemplate[#1]{#2}{\subtypeof}{#3}}
\newcommand{\jbsubtype}[3][\Delta]{\jgbtemplate[#1]{#2}{\subtypeof}{#3}}
\newcommand{\jnotsubtype}[3][\Delta]{\jgtemplate[#1]{#2}{\not\subtypeof}{#3}}
\newcommand{\jequivtype}[3][\Delta]{\jgtemplate[#1]{#2}{\equiv}{#3}}
\newcommand{\jbequivtype}[3][\Delta]{\jgbtemplate[#1]{#2}{\equiv}{#3}}
\newcommand{\jnotequivtype}[3][\Delta]{\jgtemplate[#1]{#2}{\not\equiv}{#3}}

% well-formed types
\newcommand{\jwftype}[2][\Delta]{\jgshorttemplate[#1]{#2}{\mathsf{ok}}}
\newcommand{\jbwftype}[2][\Delta]{\jgbshorttemplate[#1]{#2}{\mathsf{ok}}}

\newcommand{\jwfdecl}[2][\Delta]{\jgshorttemplate[#1]{#2}{\mathsf{ok}}}
\newcommand{\jbwfdecl}[2][\Delta]{\jgbshorttemplate[#1]{#2}{\mathsf{ok}}}

\newcommand{\jwfmeth}[2][\Delta;\Gamma]{\jgshorttemplate[#1]{#2}{\mathsf{ok}}}
\newcommand{\jbwfmeth}[2][\Delta;\Gamma]{\jgbshorttemplate[#1]{#2}{\mathsf{ok}}}


% subtyping on quantified types
\newcommand{\jle}[3][\Delta]{\jgtemplate[#1]{#2}{\le}{#3}}
\newcommand{\jleinner}[3][\Delta]{\jgtemplate[#1]{#2}{\leinner}{#3}}

% constrained judgments
\newcommand{\jgconstrtemplate}[5][\Delta]{\ensuremath{\whennonempty{#1}{{#1}\,}\turnstile\,{#2}\;{#3}\;{#4}\,\Leftarrow\,{#5}}}
% ground subtyping with constraints
\newcommand{\jsub}[4][\Delta]{\jgconstrtemplate[#1]{#2}{\subtypeof}{#3}{#4}}
% not subtype
\newcommand{\jnsub}[4][\Delta]{\jgconstrtemplate[#1]{#2}{\not \subtypeof}{#3}{#4}}
% type exclusion
\newcommand{\jexc}[4][\Delta]{\jgconstrtemplate[#1]{#2}{\exc}{#3}{#4}}
% type non-exclusion
\newcommand{\jnexc}[4][\Delta]{\jgconstrtemplate[#1]{#2}{\nexc}{#3}{#4}}
% equivalence
\newcommand{\jequiv}[4][\Delta]{\jgconstrtemplate[#1]{#2}{\equiv}{#3}{#4}}
% nonequivalence
\newcommand{\jnequiv}[4][\Delta]{\jgconstrtemplate[#1]{#2}{\not\equiv}{#3}{#4}}


% contrapositive judgements
\newcommand{\jgcontratemplate}[5][\Delta]{\ensuremath{\whennonempty{#1}{{#1}\,}\turnstile\,{#2}\;{#3}\;{#4}\,\Rightarrow\,{#5}}}
% ground subtyping with constraints
\newcommand{\jcsub}[4][\Delta]{\jgcontratemplate[#1]{#2}{\subtypeof}{#3}{#4}}
% not subtype
\newcommand{\jcnsub}[4][\Delta]{\jgcontratemplate[#1]{#2}{\not \subtypeof}{#3}{#4}}
% type exclusion
\newcommand{\jcexc}[4][\Delta]{\jgcontratemplate[#1]{#2}{\exc}{#3}{#4}}
% type non-exclusion
\newcommand{\jcnexc}[4][\Delta]{\jgcontratemplate[#1]{#2}{\nexc}{#3}{#4}}
% equivalence
\newcommand{\jcequiv}[4][\Delta]{\jgcontratemplate[#1]{#2}{\equiv}{#3}{#4}}
% nonequivalence
\newcommand{\jcnequiv}[4][\Delta]{\jgcontratemplate[#1]{#2}{\not\equiv}{#3}{#4}}


% applicability of a domain or fndecl to a type
\newcommand{\japp}[3][\Delta]{\jgtemplate[#1]{#2}{\ni}{#3}}
% specificity between fndecls
\newcommand{\jms}[3][\Delta]{\jgtemplate[#1]{#2}{\ms}{#3}}

% constraints
% convert a bound environment into a constraint
\newcommand{\toConstraint}[2]{\ensuremath{\textit{toConstraint}({#1})\,=\,{#2}}}
% convert a constraint into a bound environment
\newcommand{\toBounds}[2]{\ensuremath{\textit{toBounds}({#1})\,=\,{#2}}}

% apply substitution to constraint
\newcommand{\japply}[4][\Delta]{\ensuremath{\whennonempty{#1}{{#1}\,}\turnstile\,\textit{apply}({#2}, {#3})\,=\,{#4}}}
% solve constraint to get a substitution and the residual constraints
\newcommand{\jsolve}[4][\Delta]{\ensuremath{\whennonempty{#1}{{#1}\,}\turnstile\,\textit{unify}({#2})\,=\,{#3}\,,\;{#4}}}



% type reduction
\newcommand{\jtred}[2]{\ensuremath{\Delta \turnstile\,{#1} \eqred {#2}}}
\newcommand{\jtreds}[3]{\ensuremath{\Delta \turnstile\,{#1} \eqred {#2}\,,\;{#3}}}


% for tabularx environments to have a right-aligned, stretched col
\newcolumntype{R}{>{\raggedleft\arraybackslash}X}%

\theorembodyfont{\rm}
\newtheorem{lemma}{Lemma}
\newtheorem{theorem}{Theorem}
% Our proofs are more like proof sketches!! EricAllen 7/15/2011
\newenvironment{proof}{\noindent \textbf{Proof:} }{\hfill $\Box$}
\newenvironment{psketch}{\noindent \textbf{Proof sketch:} }{\hfill $\Box$}

\begin{document}

\conferenceinfo{OOPSLA '11}{October 22--27, 2011, Portland, Oregon, USA.}
\CopyrightYear{2011}
\copyrightdata{978-1-4503-0940-0/11/10}

\titlebanner{draft}        % These are ignored unless
\preprintfooter{draft}     % 'preprint' option specified.

\title{Welterweight Fortress DRAFT}
\subtitle{}
\authorinfo{David Chase}{Oracle Labs}{david.r.chase@oracle.com}
\authorinfo{Justin Hilburn}{Oracle Labs}{justin.hilburn@oracle.com}
\authorinfo{Victor Luchangco}{Oracle Labs}{victor.luchangco@oracle.com}
\authorinfo{Karl Naden}{Oracle Labs}{karl.naden@oracle.com}
\authorinfo{Sukyoung Ryu}{KAIST}{sryu.cs@kaist.ac.kr}
\authorinfo{Guy L. Steele Jr.}{Oracle Labs}{guy.steele@oracle.com}
\authorinfo{John Tristan}{Oracle Labs}{jean.baptiste.tristan@oracle.com}

\makeatletter
\def \@maketitle {%
 \begin{center}
 \@settitlebanner
 \let \thanks = \titlenote
 \noindent \LARGE \bfseries \@titletext \par
 %\vskip 6pt
 %\noindent \Large \@subtitletext \par
 \vskip 6pt
   \noindent \@setauthor{9pc}{i}{\@false}\hspace{1.5pc}%
             \@setauthor{9pc}{ii}{\@false}\hspace{1.5pc}%
             \@setauthor{10pc}{iii}{\@false}\hspace{1.5pc}%
             \@setauthor{9pc}{iv}{\@true}\par
\vspace{12pt plus 2pt}
 \noindent \@setauthor{9pc}{v}{\@false}\hspace{1.5pc}%
           \@setauthor{9pc}{vi}{\@false}\hspace{1.5pc}%
           \@setauthor{11pc}{vii}{\@false}\par
\vspace{10pt plus 2pt}
 \end{center}}
\makeatother
\maketitle


\begin{abstract}
%
\begin{abstract}

The Fortress programming language integrates traditional mathematical
notation into an object-oriented framework based on traits with
multiple inheritance, overloading (of both methods and functions)
resolved by symmetric dynamic dispatch, static types, and separately
compiled modules.  One innovation is
\emph{functional methods}, which (like conventional ``dotted methods'')
are declared within traits and may be inherited, but are invoked by
ordinary function calls (or mathematical operator syntax) rather
than conventional ``dotted method calls,'' and therefore compete
in overloading resolution with ordinary function declarations.
A component/API system governs visibility of traits, objects, and
functions, and allows separate compilation of components.

A longstanding problem with multiple inheritance is what to do when
methods inherited from several parents conflict.  Many approaches have
been explored in the literature; most fail to obey the
intuitively desirable requirement that the function or method invoked
be the uniquely most specific one that is both accessible and
applicable.  Fortress requires that the signatures in every overload
set form a meet-bounded lattice; therefore it is impossible for any
function or method call to be ambiguous.  This idea goes back nearly
two decades, but Fortress appears to be the first programming language
to adopt and statically enforce it.  Because this rule guarantees
confluence, it enables a distributed implementation of dispatching
that allows selective export and selective optimization.

We exhibit a source-to-source rewrite from a source language
(a stripped-down version of Fortress) to a related target language
that is simpler than the Java\texttrademark\ programming language and is readily
supported by the Java Virtual Machine.  The demonstrated rewriting is
a practical basis for separate compilation and is easily extended to
explicitly type-parameterized methods and functions.

\end{abstract}






Fortress~\cite{fortress}

\end{abstract}

\category{D.3.3}{Programming Languages}{Language Constructs and Features---classes and objects, inheritance, modules, packages, polymorphism}

\terms{Languages}

\keywords{object-oriented programming, multiple dispatch, 
symmetric dispatch, multiple inheritance, overloading, ilks, run-time types, static types,
components, modularity, meet rule,
methods, multimethods, separate compilation, Fortress}

\section{Introduction}
\label{sec:introduction}
%%%%%%%%%%%%%%%%%%%%%%%%%%%%%%%%%%%%%%%%%%%%%%%%%%%%%%%%%%%%%%%%%%%%%%%%%%%%%%%%%
%   Copyright 2012, Oracle and/or its affiliates.
%   All rights reserved.
%
%
%   Use is subject to license terms.
%
%   This distribution may include materials developed by third parties.
%
%%%%%%%%%%%%%%%%%%%%%%%%%%%%%%%%%%%%%%%%%%%%%%%%%%%%%%%%%%%%%%%%%%%%%%%%%%%%%%%%

\newchap{Introduction}{introduction}

%     - what is Fortress: purpose, target audience, etc.
%     - what is this document: target audience
%         - beginner's reference manual (?)
%     - how to read this document
%     - other relevant documents
%     - people/acknowledgments



\newpage

\section{Notation}
\label{sec:notation}
We use the term \emph{monogram} to refer to a single letter (Latin or Greek)  that, rather than being used for decorative purposes, is itself possibly ``decorated'' with one or more prime marks and/or a sequence of one or more subscripts.  Examples of monograms are $x$, $\beta$, $e'$, $\alpha_2$, and $\tau'_{15\,27}$.

We write $\bar{x}$ as shorthand for a possibly empty comma-separated sequence $x_1, x_2, \ldots, x_n$ for some freely chosen nonnegative integer $n$;
thus $\bar{x}$ may expand to `` '' or ``$x_1$'' or ``$x_1, x_2$'' or ``$x_1, x_2, x_3$'' or ``$x_1, x_2, x_3, x_4$'' and so on.
More generally, for any expression, that same expression with an overbar is shorthand for a possibly empty comma-separated sequence
of copies of that expression with two transformations applied to each copy: (a) any subexpression that is underlined one or more times
is replaced by a copy of that subexpression with one underline removed, and (b) any subexpression that is a monogram that is not underlined
is replaced by a copy of that monogram with an additional subscript $i$ appended, where $i$ is the number of the copy (starting from the left with 1).
Thus $\bar{\underline{[\tau/P]}\tau'}$ means $[\tau/P]\tau'_1, [\tau/P]\tau'_2, \dots, [\tau/P]\tau'_n$,
so one possible concrete expansion is $[\tau/P]\tau'_1, [\tau/P]\tau'_2, [\tau/P]\tau'_3, [\tau/P]\tau'_4, [\tau/P]\tau'_5$.
If overbar constructions are nested, they are expanded outermost first.
Therefore the shorthand
$f\bigobb{P \extends \bdb{\tau}}$ means $f\bigob{P_1 \extends \bdb{\tau_1}, P_2 \extends \bdb{\tau_2}, \ldots, P_n \extends \bdb{\tau_n}}$, which in turn means:
\[ \begin{array}{l@{}l}
    f \big\llbracket \mskip0.5\thinmuskip & P_1 \extends \bd{\tau_{1\,1}, \tau_{1\,2}, \ldots, \tau_{1\,{n_1}}}, \\
                                          & P_2 \extends \bd{\tau_{2\,1}, \tau_{2\,2}, \ldots, \tau_{2\,{n_2}}}, \\
                                          & \ldots, \\
                                          & P_n \extends \bd{\tau_{m\,1}, \tau_{m\,2}, \ldots, \tau_{m\,{n_m}}} \mskip 0.5\thinmuskip \big\rrbracket
\end{array} \]
so one possible concrete expansion is:
\[ \begin{array}{l@{}l}
    f \big\llbracket \mskip0.5\thinmuskip & P_1 \extends \bd{\tau_{1\,1}, \tau_{1\,2}, \tau_{1\,3}}, \\
                                          & P_2 \extends \bd{\tau_{2\,1}}, \\
                                          & P_3 \extends \bd{\,}, \\
                                          & P_4 \extends \bd{\tau_{4\,1}, \tau_{4\,2}, \tau_{4\,3}, \tau_{4\,4}, \tau_{4\,5}, \tau_{4\,6}} \mskip 0.5\thinmuskip \big\rrbracket
\end{array} \]
Note the use of whitespace between subscripts so that $\tau_{4\,12}$ is clearly different from $\tau_{41\,2}$.

The function $\countof$ returns an integer saying how many arguments it was given; thus $\countof(\bar{x})$ tells the length of the sequence into which $\bar{x}$ has expanded.

After all occurrences of the overbar construction have been expanded, three other shorthand substitutions take place:
\begin{itemize}
\item Every monogram whose base letter has been described as \emph{ranging over} a {\sc bnf} nonterminal of a specified grammar is replaced by a token sequence generated by the grammar from that nonterminal.
\item Each occurrence of the symbol ``$\dontcare$'' is replaced by any sequence of tokens that is correctly balanced within respect to parentheses, braces, and brackets of all kinds,
such that every comma or semicolon in the sequence is contained within at least one matched pair of parentheses, braces, or brackets. (This is used as a ``don't care'' indication when asking whether any of a set of constructs matches a certain syntactic pattern.)
\item Each occurrence of the symbol ``$\emptyseq$'' is deleted.  (This symbol is used as an explicit indication that an empty sequence of symbols is intended.)
\end{itemize}

When any of these shorthands is used in an overall context (such as a {\sc bnf} rule, inference rule, axiom, or expository sentence or paragraph), it is as if there were an infinite number of instantiations of that context, one for each possible expansion of the shorthand.  Three consistency constraints must be obeyed in performing the substitutions for any single such context:
\begin{itemize}
\item
If the same monogram (with identical decorations) has an additional subscript attached to it
by more than one overbar construction, then all such overbar constructions are constrained to produce the same number of copies in any given instantiation of the rule;
otherwise the choices for the number of copies produced by each overbar construction is free and independent.
\item
If the base letter of a monogram ranges over a {\sc bnf} nonterminal, then multiple identical occurrences of the monogram must be replaced by identical copies of a single generated token sequence.
\item
If two distinct monograms each have base letters that range over a {\sc bnf} nonterminal that expands to simply ``identifier'', then they must be replaced with different identifiers.
\end{itemize}
The last two constraints rely on metavariable declarations such as those in Figure~\ref{fig:metavariables}.  A declaration such as
``$e$ ranges over expressions $e$'' means 
``monograms with base letter $e$ expand into expressions generated by {\sc bnf} nonterminal $e$'';
this may seem redundant, but only because by convention we frequently use a single-letter identifier as a {\sc bnf} nonterminal
and then go on to use that same single-letter identifier as a base letter for monograms.
A declaration such as
``$\alpha, \gamma, \rho, \chi, \eta$ range over lattice types $\alpha$'' means 
``monograms with base letter $\alpha$ or $\gamma$ or $\rho$ or $\chi$ or $\eta$ expand into expressions generated by {\sc bnf} nonterminal $\alpha$''
which is more clearly not a redundant statement.

As an additional convenience using these shorthands, we adopt these conventions:
\begin{itemize}
\item If a judgment has several comma-separated expressions to the right of the turnstile ``$\turnstile$'', it is
as if there were several distinct judgments, one containing each of the expressions to the right of the turnstile.
Thus the judgment $\jgTHREEtemplate{\Gamma}{\tau_1}{\extends}{\tau'_1}{\tau_2}{\tau'_2}{\tau_3}{\tau'_3}$
means the same as three separately written judgments:
\[\jgtemplate[\Gamma]{\tau_1}{\extends}{\tau'_1} \andalso \jgtemplate[\Gamma]{\tau_2}{\extends}{\tau'_2} \andalso \jgtemplate[\Gamma]{\tau_3}{\extends}{\tau'_3} \]
\item If a judgment has nothing to the right of the turnstile, it is
as if there were no judgment written at all.
\item If an inference rule has several comma-separated expressions or judgments as consequents, it is
as if there were several distinct inference rules, one containing each of the consequents.
\item If an inference rule has no consequents, it is
as if there were no inference rule written at all.
\end{itemize}

As an extreme (but useful) example of the application of these conventions, consider this axiom:

\infax{ \jbevalstep[\Delta]{\underline{E}\big[\pi(\underline{\bar{v}})\big]}{\underline{E}[v]} }

\noindent In order to apply this axiom to a particular case, we may freely choose to expand the largest overbar construction to produce, say, two copies:

\infax{ \jevalstepTWO[\Delta]{E\big[\pi_1(\bar{v})\big]}{E[v_1]}{E\big[\pi_2(\bar{v})\big]}{E[v_2]} }

\noindent Note that both the $\pi$ symbol and the second occurrence of $v$ receive subscripts in each copy, but the underlines (which are removed as part of the expansion process) prevent the first occurrence of $v$ (which happens to have a second overbar) and the two occurrences of $E$ from receiving subscripts.  Now we expand the remaining overbars, but because they will attach subscripts to the symbol $v$, and $v$ has already had subscripts attached by the larger overbar, we must choose the same number of copies (two) for each of these overbars:

\infax{ \jevalstepTWO[\Delta]{E\big[\pi_1(v_1,v_2)\big]}{E[v_1]}{E\big[\pi_2(v_1,v_2)\big]}{E[v_2]} }

\noindent Then this judgment with two comma-separated expressions to the right of the turnstile is understood to mean two distinct judgments:

\infax{ \jevalstep[\Delta]{E\big[\pi_1(v_1,v_2)]}{E[v_1]} \\[2pt]
        \jevalstep[\Delta]{E\big[\pi_2(v_1,v_2)]}{E[v_2]} }

A final note: we sometimes use parentheses or braces or brackets of different sizes within an expression purely to enhance readability;
the size of such a symbol does not affect its meaning in the formalism.


\section{Grammar}
\label{sec:grammar}
%%%%%%%%%%%%%%%%%%%%%%%%%%%%%%%%%%%%%%%%%%%%%%%%%%%%%%%%%%%%%%%%%%%%%%%%%%%%%%%%
%   Copyright 2012, Oracle and/or its affiliates.
%   All rights reserved.
%
%
%   Use is subject to license terms.
%
%   This distribution may include materials developed by third parties.
%
%%%%%%%%%%%%%%%%%%%%%%%%%%%%%%%%%%%%%%%%%%%%%%%%%%%%%%%%%%%%%%%%%%%%%%%%%%%%%%%%

\chapter{Simplified Grammar for Application Programmers and Library Writers}

\section{Components and API} 

 
\begin{longtable}[l]{p{3cm}rl}
$\mathsf{File}$ &  $\mathsf{::=}$  & $\mathsf{CompilationUnit}$ \\
 & $\big|$ &  $\mathsf{Imports}$$^?$ $\mathsf{Exports}$ $\mathsf{Decls}$$^?$ \\
 & $\big|$ &  $\mathsf{Imports}$$^?$ $\mathsf{AbsDecls}$ \\
 & $\big|$ &  $\mathsf{Imports}$ $\mathsf{AbsDecls}$$^?$ \\
$\mathsf{CompilationUnit}$ &  $\mathsf{::=}$  & $\mathsf{Component}$ \\
 & $\big|$ &  $\mathsf{Api}$ \\
$\mathsf{Component}$ &  $\mathsf{::=}$  & $\mathbf{component}$ $\mathsf{DottedId}$ $\mathsf{Imports}$$^?$ $\mathsf{Exports}$ $\mathsf{Decls}$$^?$ $\mathbf{end}$ \\
$\mathsf{Api}$ &  $\mathsf{::=}$  & $\mathbf{api}$ $\mathsf{DottedId}$ $\mathsf{Imports}$$^?$ $\mathsf{AbsDecls}$$^?$ $\mathbf{end}$ \\
$\mathsf{Imports}$ &  $\mathsf{::=}$  & $\mathsf{Import}$$^+$ \\
$\mathsf{Import}$ &  $\mathsf{::=}$  & $\mathbf{import}$ $\mathsf{ImportFrom}$ \\
 & $\big|$ &  $\mathbf{import}$ $\mathsf{AliasedDottedIds}$ \\
$\mathsf{ImportFrom}$ &  $\mathsf{::=}$  & $\mathbf{*}$ $\big($  $\mathbf{except}$ $\mathsf{Names}$ $\big)$$^?$ $\mathbf{from}$ $\mathsf{DottedId}$ \\
 & $\big|$ &  $\mathsf{AliasedNames}$ $\mathbf{from}$ $\mathsf{DottedId}$ \\
$\mathsf{Names}$ &  $\mathsf{::=}$  & $\mathsf{Name}$ \\
 & $\big|$ &  $\mathbf{\{}$ $\mathsf{NameList}$ $\mathbf{\}}$ \\
$\mathsf{NameList}$ &  $\mathsf{::=}$  & $\mathsf{Name}$ $\big($  $\mathbf{,}$ $\mathsf{Name}$ $\big)$$^*$ \\
$\mathsf{AliasedNames}$ &  $\mathsf{::=}$  & $\mathsf{AliasedName}$ \\
 & $\big|$ &  $\mathbf{\{}$ $\mathsf{AliasedNameList}$ $\mathbf{\}}$ \\
$\mathsf{AliasedName}$ &  $\mathsf{::=}$  & $\mathsf{Id}$ $\big($  $\mathbf{as}$ $\mathsf{DottedId}$ $\big)$$^?$ \\
 & $\big|$ &  $\mathbf{opr}$ $\mathsf{Op}$ $\big($  $\mathbf{as}$ $\mathsf{Op}$ $\big)$$^?$ \\
 & $\big|$ &  $\mathbf{opr}$ $\mathsf{LeftEncloser}$ $\mathsf{RightEncloser}$ $\big($  $\mathbf{as}$ $\mathsf{LeftEncloser}$ $\mathsf{RightEncloser}$ $\big)$$^?$ \\
$\mathsf{AliasedNameList}$ &  $\mathsf{::=}$  & $\mathsf{AliasedName}$ $\big($  $\mathbf{,}$ $\mathsf{AliasedName}$ $\big)$$^*$ \\
$\mathsf{AliasedDottedIds}$ &  $\mathsf{::=}$  & $\mathsf{AliasedDottedId}$ \\
 & $\big|$ &  $\mathbf{\{}$ $\mathsf{AliasedDottedIdList}$ $\mathbf{\}}$ \\
$\mathsf{AliasedDottedId}$ &  $\mathsf{::=}$  & $\mathsf{DottedId}$ $\big($  $\mathbf{as}$ $\mathsf{DottedId}$ $\big)$$^?$ \\
$\mathsf{AliasedDottedIdList}$ &  $\mathsf{::=}$  & $\mathsf{AliasedDottedId}$ $\big($  $\mathbf{,}$ $\mathsf{AliasedDottedId}$ $\big)$$^*$ \\
$\mathsf{Exports}$ &  $\mathsf{::=}$  & $\mathsf{Export}$$^+$ \\
$\mathsf{Export}$ &  $\mathsf{::=}$  & $\mathbf{export}$ $\mathsf{DottedIds}$ \\
$\mathsf{DottedIds}$ &  $\mathsf{::=}$  & $\mathsf{DottedId}$ \\
 & $\big|$ &  $\mathbf{\{}$ $\mathsf{DottedIdList}$ $\mathbf{\}}$ \\
$\mathsf{DottedIdList}$ &  $\mathsf{::=}$  & $\mathsf{DottedId}$ $\big($  $\mathbf{,}$ $\mathsf{DottedId}$ $\big)$$^*$ \\
\end{longtable} \hfill 

\section{Top-level Declarations} 

 
\begin{longtable}[l]{p{3cm}rl}
$\mathsf{Decls}$ &  $\mathsf{::=}$  & $\mathsf{Decl}$$^+$ \\
$\mathsf{Decl}$ &  $\mathsf{::=}$  & $\mathsf{TraitDecl}$ \\
 & $\big|$ &  $\mathsf{ObjectDecl}$ \\
 & $\big|$ &  $\mathsf{VarDecl}$ \\
 & $\big|$ &  $\mathsf{FnDecl}$ \\
$\mathsf{AbsDecls}$ &  $\mathsf{::=}$  & $\mathsf{AbsDecl}$$^+$ \\
$\mathsf{AbsDecl}$ &  $\mathsf{::=}$  & $\mathsf{AbsTraitDecl}$ \\
 & $\big|$ &  $\mathsf{AbsObjectDecl}$ \\
 & $\big|$ &  $\mathsf{AbsVarDecl}$ \\
 & $\big|$ &  $\mathsf{AbsFnDecl}$ \\
\end{longtable} \hfill 

\section{Trait Declaration} 

 
\begin{longtable}[l]{p{3cm}rl}
$\mathsf{TraitDecl}$ &  $\mathsf{::=}$  & $\mathsf{TraitHeader}$ $\mathsf{GoInATrait}$$^?$ $\mathbf{end}$ \\
$\mathsf{TraitHeader}$ &  $\mathsf{::=}$  & $\mathsf{TraitMods}$$^?$ $\mathbf{trait}$ $\mathsf{Id}$ $\mathsf{StaticParams}$$^?$ $\mathsf{Extends}$$^?$ $\mathsf{TraitClauses}$$^?$ \\
$\mathsf{TraitClauses}$ &  $\mathsf{::=}$  & $\mathsf{TraitClause}$$^+$ \\
$\mathsf{TraitClause}$ &  $\mathsf{::=}$  & $\mathsf{Excludes}$ \\
 & $\big|$ &  $\mathsf{Comprises}$ \\
$\mathsf{GoInATrait}$ &  $\mathsf{::=}$  & $\mathsf{GoFrontInATrait}$ $\mathsf{GoBackInATrait}$$^?$ \\
 & $\big|$ &  $\mathsf{GoBackInATrait}$ \\
$\mathsf{GoFrontInATrait}$ &  $\mathsf{::=}$  & $\mathsf{GoesFrontInATrait}$$^+$ \\
$\mathsf{GoesFrontInATrait}$ &  $\mathsf{::=}$  & $\mathsf{AbsFldDecl}$ \\
 & $\big|$ &  $\mathsf{GetterSetterDecl}$ \\
$\mathsf{GoBackInATrait}$ &  $\mathsf{::=}$  & $\mathsf{GoesBackInATrait}$$^+$ \\
$\mathsf{GoesBackInATrait}$ &  $\mathsf{::=}$  & $\mathsf{MdDecl}$ \\
$\mathsf{AbsTraitDecl}$ &  $\mathsf{::=}$  & $\mathsf{TraitHeader}$ $\mathsf{AbsGoInATrait}$$^?$ $\mathbf{end}$ \\
$\mathsf{AbsGoInATrait}$ &  $\mathsf{::=}$  & $\mathsf{AbsGoFrontInATrait}$ $\mathsf{AbsGoBackInATrait}$$^?$ \\
 & $\big|$ &  $\mathsf{AbsGoBackInATrait}$ \\
$\mathsf{AbsGoFrontInATrait}$ &  $\mathsf{::=}$  & $\mathsf{AbsGoesFrontInATrait}$$^+$ \\
$\mathsf{AbsGoesFrontInATrait}$ &  $\mathsf{::=}$  & $\mathsf{ApiFldDecl}$ \\
 & $\big|$ &  $\mathsf{AbsGetterSetterDecl}$ \\
$\mathsf{AbsGoBackInATrait}$ &  $\mathsf{::=}$  & $\mathsf{AbsGoesBackInATrait}$$^+$ \\
$\mathsf{AbsGoesBackInATrait}$ &  $\mathsf{::=}$  & $\mathsf{AbsMdDecl}$ \\
 & $\big|$ &  $\mathsf{AbsCoercion}$ \\
\end{longtable} \hfill 

\section{Object declaration} 

 
\begin{longtable}[l]{p{3cm}rl}
$\mathsf{ObjectDecl}$ &  $\mathsf{::=}$  & $\mathsf{ObjectHeader}$ $\mathsf{GoInAnObject}$$^?$ $\mathbf{end}$ \\
$\mathsf{ObjectHeader}$ &  $\mathsf{::=}$  & $\mathsf{ObjectMods}$$^?$ $\mathbf{object}$ $\mathsf{Id}$ $\mathsf{StaticParams}$$^?$ $\mathsf{ObjectValParam}$$^?$ $\mathsf{Extends}$$^?$ $\mathsf{FnClauses}$ \\
$\mathsf{ObjectValParam}$ &  $\mathsf{::=}$  & $\big($  $\mathsf{ObjectParams}$$^?$ $\big)$ \\
$\mathsf{ObjectParams}$ &  $\mathsf{::=}$  & $\big($  $\mathsf{ObjectParam}$ $\mathbf{,}$ $\big)$$^*$ $\mathsf{ObjectKeyword}$ $\big($  $\mathbf{,}$ $\mathsf{ObjectKeyword}$ $\big)$$^*$ \\
 & $\big|$ &  $\mathsf{ObjectParam}$ $\big($  $\mathbf{,}$ $\mathsf{ObjectParam}$ $\big)$$^*$ \\
$\mathsf{ObjectKeyword}$ &  $\mathsf{::=}$  & $\mathsf{ObjectParam}$ $\mathbf{=}$ $\mathsf{Expr}$ \\
$\mathsf{ObjectParam}$ &  $\mathsf{::=}$  & $\mathsf{FldMods}$$^?$ $\mathsf{Param}$ \\
 & $\big|$ &  $\mathbf{transient}$ $\mathsf{Param}$ \\
$\mathsf{GoInAnObject}$ &  $\mathsf{::=}$  & $\mathsf{GoFrontInAnObject}$ $\mathsf{GoBackInAnObject}$$^?$ \\
 & $\big|$ &  $\mathsf{GoBackInAnObject}$ \\
$\mathsf{GoFrontInAnObject}$ &  $\mathsf{::=}$  & $\mathsf{GoesFrontInAnObject}$$^+$ \\
$\mathsf{GoesFrontInAnObject}$ &  $\mathsf{::=}$  & $\mathsf{FldDecl}$ \\
 & $\big|$ &  $\mathsf{GetterSetterDef}$ \\
$\mathsf{GoBackInAnObject}$ &  $\mathsf{::=}$  & $\mathsf{GoesBackInAnObject}$$^+$ \\
$\mathsf{GoesBackInAnObject}$ &  $\mathsf{::=}$  & $\mathsf{MdDef}$ \\
$\mathsf{AbsObjectDecl}$ &  $\mathsf{::=}$  & $\mathsf{ObjectHeader}$ $\mathsf{AbsGoInAnObject}$$^?$ $\mathbf{end}$ \\
$\mathsf{AbsGoInAnObject}$ &  $\mathsf{::=}$  & $\mathsf{AbsGoFrontInAnObject}$ $\mathsf{AbsGoBackInAnObject}$$^?$ \\
 & $\big|$ &  $\mathsf{AbsGoBackInAnObject}$ \\
$\mathsf{AbsGoFrontInAnObject}$ &  $\mathsf{::=}$  & $\mathsf{AbsGoesFrontInAnObject}$$^+$ \\
$\mathsf{AbsGoesFrontInAnObject}$ &  $\mathsf{::=}$  & $\mathsf{ApiFldDecl}$ \\
 & $\big|$ &  $\mathsf{AbsGetterSetterDecl}$ \\
$\mathsf{AbsGoBackInAnObject}$ &  $\mathsf{::=}$  & $\mathsf{AbsGoesBackInAnObject}$$^+$ \\
$\mathsf{AbsGoesBackInAnObject}$ &  $\mathsf{::=}$  & $\mathsf{AbsMdDecl}$ \\
 & $\big|$ &  $\mathsf{AbsCoercion}$ \\
\end{longtable} \hfill 

\section{Variable Declaration} 

 
\begin{longtable}[l]{p{3cm}rl}
$\mathsf{VarDecl}$ &  $\mathsf{::=}$  & $\mathsf{VarWTypes}$ $\mathsf{InitVal}$ \\
 & $\big|$ &  $\mathsf{VarWoTypes}$ $\mathbf{=}$ $\mathsf{Expr}$ \\
 & $\big|$ &  $\mathsf{VarWoTypes}$ $\mathbf{:}$ $\mathsf{TypeRef}$ $\mathbf{...}$ $\mathsf{InitVal}$ \\
 & $\big|$ &  $\mathsf{VarWoTypes}$ $\mathbf{:}$ $\mathsf{SimpleTupleType}$ $\mathsf{InitVal}$ \\
$\mathsf{VarWTypes}$ &  $\mathsf{::=}$  & $\mathsf{VarWType}$ \\
 & $\big|$ &  $\big($  $\mathsf{VarWType}$ $\big($  $\mathbf{,}$ $\mathsf{VarWType}$ $\big)$$^+$ $\big)$ \\
$\mathsf{VarWType}$ &  $\mathsf{::=}$  & $\mathsf{VarMods}$$^?$ $\mathsf{Id}$ $\mathsf{IsType}$ \\
$\mathsf{VarWoTypes}$ &  $\mathsf{::=}$  & $\mathsf{VarWoType}$ \\
 & $\big|$ &  $\big($  $\mathsf{VarWoType}$ $\big($  $\mathbf{,}$ $\mathsf{VarWoType}$ $\big)$$^+$ $\big)$ \\
$\mathsf{VarWoType}$ &  $\mathsf{::=}$  & $\mathsf{VarMods}$$^?$ $\mathsf{Id}$ \\
$\mathsf{InitVal}$ &  $\mathsf{::=}$  & $\big($  $\mathbf{=}$ $\big|$ $\mathbf{:=}$ $\big)$ $\mathsf{Expr}$ \\
$\mathsf{AbsVarDecl}$ &  $\mathsf{::=}$  & $\mathsf{VarWTypes}$ \\
 & $\big|$ &  $\mathsf{VarWoTypes}$ $\mathbf{:}$ $\mathsf{TypeRef}$ $\mathbf{...}$ \\
 & $\big|$ &  $\mathsf{VarWoTypes}$ $\mathbf{:}$ $\mathsf{SimpleTupleType}$ \\
\end{longtable} \hfill 

\section{Function Declaration} 

 
\begin{longtable}[l]{p{3cm}rl}
$\mathsf{FnDecl}$ &  $\mathsf{::=}$  & $\mathsf{FnDef}$ \\
 & $\big|$ &  $\mathsf{AbsFnDecl}$ \\
$\mathsf{FnDef}$ &  $\mathsf{::=}$  & $\mathsf{FnMods}$$^?$ $\mathsf{FnHeaderFront}$ $\mathsf{FnHeaderClause}$ $\mathbf{=}$ $\mathsf{Expr}$ \\
$\mathsf{AbsFnDecl}$ &  $\mathsf{::=}$  & $\mathsf{FnMods}$$^?$ $\mathsf{FnHeaderFront}$ $\mathsf{FnHeaderClause}$ \\
 & $\big|$ &  $\mathsf{Name}$ $\mathbf{:}$ $\mathsf{ArrowType}$ \\
$\mathsf{FnHeaderFront}$ &  $\mathsf{::=}$  & $\mathsf{Id}$ $\mathsf{StaticParams}$$^?$ $\mathsf{ValParam}$ \\
 & $\big|$ &  $\mathsf{OpHeaderFront}$ \\
\end{longtable} \hfill 

\section{Headers} 

 
\begin{longtable}[l]{p{3cm}rl}
$\mathsf{Extends}$ &  $\mathsf{::=}$  & $\mathbf{extends}$ $\mathsf{TraitTypes}$ \\
$\mathsf{Excludes}$ &  $\mathsf{::=}$  & $\mathbf{excludes}$ $\mathsf{TraitTypes}$ \\
$\mathsf{Comprises}$ &  $\mathsf{::=}$  & $\mathbf{comprises}$ $\mathsf{ComprisingTypes}$ \\
$\mathsf{TraitTypes}$ &  $\mathsf{::=}$  & $\mathsf{TraitType}$ \\
 & $\big|$ &  $\mathbf{\{}$ $\mathsf{TraitTypeList}$ $\mathbf{\}}$ \\
$\mathsf{TraitTypeList}$ &  $\mathsf{::=}$  & $\mathsf{TraitType}$ $\big($  $\mathbf{,}$ $\mathsf{TraitType}$ $\big)$$^*$ \\
$\mathsf{ComprisingTypes}$ &  $\mathsf{::=}$  & $\mathsf{TraitType}$ \\
 & $\big|$ &  $\mathbf{\{}$ $\mathsf{ComprisingTypeList}$ $\mathbf{\}}$ \\
$\mathsf{ComprisingTypeList}$ &  $\mathsf{::=}$  & $\mathbf{.}$ $\mathbf{.}$ $\mathbf{.}$ \\
 & $\big|$ &  $\mathsf{TraitType}$ $\big($  $\mathbf{,}$ $\mathsf{TraitType}$ $\big)$$^*$ $\big($  $\mathbf{,}$ $\mathbf{.}$ $\mathbf{.}$ $\mathbf{.}$ $\big)$$^?$ \\
$\mathsf{FnHeaderClause}$ &  $\mathsf{::=}$  & $\mathsf{IsType}$$^?$ $\mathsf{FnClauses}$ \\
$\mathsf{FnClauses}$ &  $\mathsf{::=}$  & $\mathsf{Throws}$$^?$ \\
$\mathsf{Throws}$ &  $\mathsf{::=}$  & $\mathbf{throws}$ $\mathsf{MayTraitTypes}$ \\
$\mathsf{MayTraitTypes}$ &  $\mathsf{::=}$  & $\mathbf{\{}$ $\mathbf{\}}$ \\
 & $\big|$ &  $\mathsf{TraitTypes}$ \\
$\mathsf{CoercionClauses}$ &  $\mathsf{::=}$  & $\mathsf{Throws}$$^?$ \\
$\mathsf{UniversalMod}$ &  $\mathsf{::=}$  & $\mathbf{private}$ \\
$\mathsf{TraitMod}$ &  $\mathsf{::=}$  & $\mathbf{value}$ \\
 & $\big|$ &  $\mathsf{UniversalMod}$ \\
$\mathsf{TraitMods}$ &  $\mathsf{::=}$  & $\mathsf{TraitMod}$$^+$ \\
$\mathsf{ObjectMods}$ &  $\mathsf{::=}$  & $\mathsf{TraitMods}$ \\
$\mathsf{FnMod}$ &  $\mathsf{::=}$  & $\mathsf{LocalFnMod}$ \\
 & $\big|$ &  $\mathsf{UniversalMod}$ \\
$\mathsf{FnMods}$ &  $\mathsf{::=}$  & $\mathsf{FnMod}$$^+$ \\
$\mathsf{VarMod}$ &  $\mathsf{::=}$  & $\mathbf{var}$ \\
 & $\big|$ &  $\mathsf{UniversalMod}$ \\
$\mathsf{VarMods}$ &  $\mathsf{::=}$  & $\mathsf{VarMod}$$^+$ \\
$\mathsf{AbsFldMod}$ &  $\mathsf{::=}$  & $\mathbf{hidden}$ $\big|$ $\mathbf{settable}$ $\big|$ $\mathsf{UniversalMod}$ \\
$\mathsf{AbsFldMods}$ &  $\mathsf{::=}$  & $\mathsf{AbsFldMod}$$^+$ \\
$\mathsf{FldMod}$ &  $\mathsf{::=}$  & $\mathbf{var}$ \\
 & $\big|$ &  $\mathsf{AbsFldMod}$ \\
$\mathsf{FldMods}$ &  $\mathsf{::=}$  & $\mathsf{FldMod}$$^+$ \\
$\mathsf{ApiFldMod}$ &  $\mathsf{::=}$  & $\mathbf{hidden}$ $\big|$ $\mathbf{settable}$ $\big|$ $\mathsf{UniversalMod}$ \\
$\mathsf{ApiFldMods}$ &  $\mathsf{::=}$  & $\mathsf{ApiFldMod}$$^+$ \\
$\mathsf{LocalFnMod}$ &  $\mathsf{::=}$  & $\mathbf{atomic}$ \\
$\mathsf{LocalFnMods}$ &  $\mathsf{::=}$  & $\mathsf{LocalFnMod}$$^+$ \\
$\mathsf{StaticParams}$ &  $\mathsf{::=}$  & $\mathbf{\llbracket}$ $\mathsf{StaticParamList}$ $\mathbf{\rrbracket}$ \\
$\mathsf{StaticParamList}$ &  $\mathsf{::=}$  & $\mathsf{StaticParam}$ $\big($  $\mathbf{,}$ $\mathsf{StaticParam}$ $\big)$$^*$ \\
$\mathsf{StaticParam}$ &  $\mathsf{::=}$  & $\mathsf{Id}$ $\mathsf{Extends}$$^?$ \\
 & $\big|$ &  $\mathbf{nat}$ $\mathsf{Id}$ \\
 & $\big|$ &  $\mathbf{int}$ $\mathsf{Id}$ \\
 & $\big|$ &  $\mathbf{bool}$ $\mathsf{Id}$ \\
 & $\big|$ &  $\mathbf{opr}$ $\mathsf{Op}$ \\
 & $\big|$ &  $\mathbf{ident}$ $\mathsf{Id}$ \\
\end{longtable} \hfill 

\section{Parameters} 

 
\begin{longtable}[l]{p{3cm}rl}
$\mathsf{ValParam}$ &  $\mathsf{::=}$  & $\mathsf{BindId}$ \\
 & $\big|$ &  $\big($  $\mathsf{Params}$$^?$ $\big)$ \\
$\mathsf{Params}$ &  $\mathsf{::=}$  & $\big($  $\mathsf{Param}$ $\mathbf{,}$ $\big)$$^*$ $\mathsf{Keyword}$ $\big($  $\mathbf{,}$ $\mathsf{Keyword}$ $\big)$$^*$ \\
 & $\big|$ &  $\big($  $\mathsf{Param}$ $\mathbf{,}$ $\big)$$^*$ \\
 & $\big|$ &  $\mathsf{Param}$ $\big($  $\mathbf{,}$ $\mathsf{Param}$ $\big)$$^*$ \\
$\mathsf{Keyword}$ &  $\mathsf{::=}$  & $\mathsf{Param}$ $\mathbf{=}$ $\mathsf{Expr}$ \\
$\mathsf{PlainParam}$ &  $\mathsf{::=}$  & $\mathsf{BindId}$ $\mathsf{IsType}$$^?$ \\
 & $\big|$ &  $\mathsf{TypeRef}$ \\
$\mathsf{Param}$ &  $\mathsf{::=}$  & $\mathsf{PlainParam}$ \\
$\mathsf{OpHeaderFront}$ &  $\mathsf{::=}$  & $\mathbf{opr}$ $\mathsf{StaticParams}$$^?$ $\big($  $\mathsf{LeftEncloser}$ $\big|$ $\mathsf{Encloser}$ $\big)$ $\mathsf{Params}$ $\big($  $\mathsf{RightEncloser}$ $\big|$ $\mathsf{Encloser}$ $\big)$ $\big($  $\mathbf{:=}$ $\big($  $\mathsf{SubscriptAssignParam}$ $\big)$ $\big)$$^?$ \\
 & $\big|$ &  $\mathbf{opr}$ $\mathsf{StaticParams}$$^?$ $\mathsf{ValParam}$ $\mathsf{Op}$ \\
 & $\big|$ &  $\mathbf{opr}$ $\big($  $\mathsf{Op}$ $\big|$ $\mathsf{Encloser}$ $\big)$ $\mathsf{StaticParams}$$^?$ $\mathsf{ValParam}$ \\
$\mathsf{SubscriptAssignParam}$ &  $\mathsf{::=}$  & $\mathsf{Param}$ \\
\end{longtable} \hfill 

\section{Method Declaration} 

 
\begin{longtable}[l]{p{3cm}rl}
$\mathsf{MdDecl}$ &  $\mathsf{::=}$  & $\mathsf{MdDef}$ \\
 & $\big|$ &  $\mathsf{AbsMdDecl}$ \\
$\mathsf{MdDef}$ &  $\mathsf{::=}$  & $\mathsf{FnMods}$$^?$ $\mathsf{MdHeaderFront}$ $\mathsf{FnHeaderClause}$ $\mathbf{=}$ $\mathsf{Expr}$ \\
 & $\big|$ &  $\mathsf{Coercion}$ \\
$\mathsf{AbsMdDecl}$ &  $\mathsf{::=}$  & $\mathbf{abstract}$ $\mathbf{?}$ $\mathsf{FnMods}$$^?$ $\mathsf{MdHeaderFront}$ $\mathsf{FnHeaderClause}$ \\
$\mathsf{MdHeaderFront}$ &  $\mathsf{::=}$  & $\big($  $\mathsf{Receiver}$ $\mathbf{.}$ $\big)$$^?$ $\mathsf{Id}$ $\mathsf{StaticParams}$$^?$ $\mathsf{MdValParam}$ \\
 & $\big|$ &  $\mathsf{OpHeaderFront}$ \\
$\mathsf{Receiver}$ &  $\mathsf{::=}$  & $\mathsf{Id}$ \\
 & $\big|$ &  $\mathbf{self}$ \\
$\mathsf{GetterSetterDecl}$ &  $\mathsf{::=}$  & $\mathsf{GetterSetterDef}$ \\
 & $\big|$ &  $\mathsf{AbsGetterSetterDecl}$ \\
$\mathsf{GetterSetterDef}$ &  $\mathsf{::=}$  & $\mathsf{FnMods}$$^?$ $\mathsf{GetterSetterMod}$ $\mathsf{MdHeaderFront}$ $\mathsf{FnHeaderClause}$ $\mathbf{=}$ $\mathsf{Expr}$ \\
$\mathsf{GetterSetterMod}$ &  $\mathsf{::=}$  & $\mathbf{getter}$ $\big|$ $\mathbf{setter}$ \\
$\mathsf{AbsGetterSetterDecl}$ &  $\mathsf{::=}$  & $\mathbf{abstract}$ $\mathbf{?}$ $\mathsf{FnMods}$$^?$ $\mathsf{GetterSetterMod}$ $\mathsf{MdHeaderFront}$ $\mathsf{FnHeaderClause}$ \\
$\mathsf{Coercion}$ &  $\mathsf{::=}$  & $\mathbf{widening}$$^?$ $\mathbf{coercion}$ $\mathsf{StaticParams}$$^?$ $\mathbf{(}$ $\mathsf{Id}$ $\mathsf{IsType}$ $\mathbf{)}$ $\mathsf{CoercionClauses}$ $\mathbf{=}$ $\mathsf{Expr}$ \\
$\mathsf{AbsCoercion}$ &  $\mathsf{::=}$  & $\mathbf{widening}$$^?$ $\mathbf{coercion}$ $\mathsf{StaticParams}$$^?$ $\mathbf{(}$ $\mathsf{Id}$ $\mathsf{IsType}$ $\mathbf{)}$ $\mathsf{CoercionClauses}$ \\
\end{longtable} \hfill 

\section{Method Parameters} 

 
\begin{longtable}[l]{p{3cm}rl}
$\mathsf{MdValParam}$ &  $\mathsf{::=}$  & $\big($  $\mathsf{MdParams}$$^?$ $\big)$ \\
$\mathsf{MdParams}$ &  $\mathsf{::=}$  & $\big($  $\mathsf{MdParam}$ $\mathbf{,}$ $\big)$$^*$ $\mathsf{MdKeyword}$ $\big($  $\mathbf{,}$ $\mathsf{MdKeyword}$ $\big)$$^*$ \\
 & $\big|$ &  $\big($  $\mathsf{MdParam}$ $\mathbf{,}$ $\big)$$^*$ \\
 & $\big|$ &  $\mathsf{MdParam}$ $\big($  $\mathbf{,}$ $\mathsf{MdParam}$ $\big)$$^*$ \\
$\mathsf{MdKeyword}$ &  $\mathsf{::=}$  & $\mathsf{MdParam}$ $\mathbf{=}$ $\mathsf{Expr}$ \\
$\mathsf{MdParam}$ &  $\mathsf{::=}$  & $\mathsf{Param}$ \\
 & $\big|$ &  $\mathbf{self}$ \\
\end{longtable} \hfill 

\section{Field Declarations} 

 
\begin{longtable}[l]{p{3cm}rl}
$\mathsf{FldDecl}$ &  $\mathsf{::=}$  & $\mathsf{FldWTypes}$ $\mathsf{InitVal}$ \\
 & $\big|$ &  $\mathsf{FldWoTypes}$ $\mathbf{=}$ $\mathsf{Expr}$ \\
 & $\big|$ &  $\mathsf{FldWoTypes}$ $\mathbf{:}$ $\mathsf{TypeRef}$ $\mathbf{...}$ $\mathsf{InitVal}$ \\
 & $\big|$ &  $\mathsf{FldWoTypes}$ $\mathbf{:}$ $\mathsf{SimpleTupleType}$ $\mathsf{InitVal}$ \\
$\mathsf{FldWTypes}$ &  $\mathsf{::=}$  & $\mathsf{FldWType}$ \\
 & $\big|$ &  $\big($  $\mathsf{FldWType}$ $\big($  $\mathbf{,}$ $\mathsf{FldWType}$ $\big)$$^+$ $\big)$ \\
$\mathsf{FldWType}$ &  $\mathsf{::=}$  & $\mathsf{FldMods}$$^?$ $\mathsf{Id}$ $\mathsf{IsType}$ \\
$\mathsf{FldWoTypes}$ &  $\mathsf{::=}$  & $\mathsf{FldWoType}$ \\
 & $\big|$ &  $\big($  $\mathsf{FldWoType}$ $\big($  $\mathbf{,}$ $\mathsf{FldWoType}$ $\big)$$^+$ $\big)$ \\
$\mathsf{FldWoType}$ &  $\mathsf{::=}$  & $\mathsf{FldMods}$$^?$ $\mathsf{Id}$ \\
\end{longtable} \hfill 

\section{Abstract Filed Declaration} 

 
\begin{longtable}[l]{p{3cm}rl}
$\mathsf{AbsFldDecl}$ &  $\mathsf{::=}$  & $\mathsf{AbsFldWTypes}$ \\
 & $\big|$ &  $\mathsf{AbsFldWoTypes}$ $\mathbf{:}$ $\mathsf{TypeRef}$ $\mathbf{...}$ \\
 & $\big|$ &  $\mathsf{AbsFldWoTypes}$ $\mathbf{:}$ $\mathsf{SimpleTupleType}$ \\
$\mathsf{AbsFldWTypes}$ &  $\mathsf{::=}$  & $\mathsf{AbsFldWType}$ \\
 & $\big|$ &  $\big($  $\mathsf{AbsFldWType}$ $\big($  $\mathbf{,}$ $\mathsf{AbsFldWType}$ $\big)$$^+$ $\big)$ \\
$\mathsf{AbsFldWType}$ &  $\mathsf{::=}$  & $\mathsf{AbsFldMods}$$^?$ $\mathsf{Id}$ $\mathsf{IsType}$ \\
$\mathsf{AbsFldWoTypes}$ &  $\mathsf{::=}$  & $\mathsf{AbsFldWoType}$ \\
 & $\big|$ &  $\big($  $\mathsf{AbsFldWoType}$ $\big($  $\mathbf{,}$ $\mathsf{AbsFldWoType}$ $\big)$$^+$ $\big)$ \\
$\mathsf{AbsFldWoType}$ &  $\mathsf{::=}$  & $\mathsf{AbsFldMods}$$^?$ $\mathsf{Id}$ \\
$\mathsf{ApiFldDecl}$ &  $\mathsf{::=}$  & $\mathsf{ApiFldMods}$$^?$ $\mathsf{Id}$ $\mathsf{IsType}$ \\
\end{longtable} \hfill 

\section{Expressions} 

 
\begin{longtable}[l]{p{3cm}rl}
$\mathsf{Expr}$ &  $\mathsf{::=}$  & $\mathsf{AssignLefts}$ $\mathsf{AssignOp}$ $\mathsf{Expr}$ \\
 & $\big|$ &  $\mathsf{OpExpr}$ \\
 & $\big|$ &  $\mathsf{DelimitedExpr}$ \\
 & $\big|$ &  $\mathsf{FlowExpr}$ \\
 & $\big|$ &  $\mathbf{fn}$ $\mathsf{ValParam}$ $\mathsf{IsType}$$^?$ $\mathsf{Throws}$$^?$ $\mathbf{\Rightarrow}$ $\mathsf{Expr}$ \\
 & $\big|$ &  $\mathsf{Expr}$ $\mathbf{as}$ $\mathsf{TypeRef}$ \\
 & $\big|$ &  $\mathsf{Expr}$ $\mathbf{asif}$ $\mathsf{TypeRef}$ \\
$\mathsf{AssignLefts}$ &  $\mathsf{::=}$  & $\big($  $\mathsf{AssignLeft}$ $\big($  $\mathbf{,}$ $\mathsf{AssignLeft}$ $\big)$$^*$ $\big)$ \\
 & $\big|$ &  $\mathsf{AssignLeft}$ \\
$\mathsf{AssignLeft}$ &  $\mathsf{::=}$  & $\mathsf{SubscriptExpr}$ \\
 & $\big|$ &  $\mathsf{FieldSelection}$ \\
 & $\big|$ &  $\mathsf{BindId}$ \\
$\mathsf{SubscriptExpr}$ &  $\mathsf{::=}$  & $\mathsf{Primary}$ $\mathbf{[}$ $\mathsf{ExprList}$$^?$ $\mathbf{]}$ \\
$\mathsf{FieldSelection}$ &  $\mathsf{::=}$  & $\mathsf{Primary}$ $\mathbf{.}$ $\mathsf{Id}$ \\
$\mathsf{OpExpr}$ &  $\mathsf{::=}$  & $\mathsf{EncloserOp}$ $\mathsf{OpExpr}$$^?$ $\mathsf{EncloserOp}$$^?$ \\
 & $\big|$ &  $\mathsf{OpExpr}$ $\mathsf{EncloserOp}$ $\mathsf{OpExpr}$$^?$ \\
 & $\big|$ &  $\mathsf{Primary}$ \\
$\mathsf{EncloserOp}$ &  $\mathsf{::=}$  & $\mathsf{Encloser}$ \\
 & $\big|$ &  $\mathsf{Op}$ \\
$\mathsf{Primary}$ &  $\mathsf{::=}$  & $\mathsf{Comprehension}$ \\
 & $\big|$ &  $\mathsf{Id}$ $\mathbf{\llbracket}$ $\mathsf{StaticArgList}$ $\mathbf{\rrbracket}$ \\
 & $\big|$ &  $\mathsf{BaseExpr}$ \\
 & $\big|$ &  $\mathsf{LeftEncloser}$ $\mathsf{ExprList}$$^?$ $\mathsf{RightEncloser}$ \\
 & $\big|$ &  $\mathsf{Primary}$ $\mathbf{[}$ $\mathsf{ExprList}$$^?$ $\mathbf{]}$ \\
 & $\big|$ &  $\mathsf{Primary}$ $\mathbf{.}$ $\mathsf{Id}$ $\big($  $\mathbf{\llbracket}$ $\mathsf{StaticArgList}$ $\mathbf{\rrbracket}$ $\big)$$^?$ $\mathsf{TupleExpr}$ \\
 & $\big|$ &  $\mathsf{Primary}$ $\mathbf{.}$ $\mathsf{Id}$ $\big($  $\mathbf{\llbracket}$ $\mathsf{StaticArgList}$ $\mathbf{\rrbracket}$ $\big)$$^?$ $\mathbf{(}$ $\mathbf{)}$ \\
 & $\big|$ &  $\mathsf{Primary}$ $\mathbf{.}$ $\mathsf{Id}$ \\
 & $\big|$ &  $\mathsf{Primary}$ $\mathbf{\wedge}$ $\mathsf{BaseExpr}$ \\
 & $\big|$ &  $\mathsf{Primary}$ $\mathsf{ExponentOp}$ \\
 & $\big|$ &  $\mathsf{Primary}$ $\mathsf{TupleExpr}$ \\
 & $\big|$ &  $\mathsf{Primary}$ $\mathbf{(}$ $\mathbf{)}$ \\
 & $\big|$ &  $\mathsf{Primary}$ $\mathsf{Primary}$ \\
 & $\big|$ &  $\mathsf{TraitType}$ $\mathbf{.}$ $\mathbf{coercion}$ $\big($  $\mathbf{\llbracket}$ $\mathsf{StaticArgList}$ $\mathbf{\rrbracket}$ $\big)$$^?$ $\big($  $\mathsf{Expr}$ $\big)$ \\
$\mathsf{FlowExpr}$ &  $\mathsf{::=}$  & $\mathbf{exit}$ $\mathsf{Id}$$^?$ $\big($  $\mathbf{with}$ $\mathsf{Expr}$ $\big)$$^?$ \\
 & $\big|$ &  $\mathsf{Accumulator}$ $\big($  $\mathbf{[}$ $\mathsf{GeneratorList}$ $\mathbf{]}$ $\big)$$^?$ $\mathsf{Expr}$ \\
 & $\big|$ &  $\mathbf{atomic}$ $\mathsf{AtomicBack}$ \\
 & $\big|$ &  $\mathbf{tryatomic}$ $\mathsf{AtomicBack}$ \\
 & $\big|$ &  $\mathbf{throw}$ $\mathsf{Expr}$ \\
$\mathsf{AtomicBack}$ &  $\mathsf{::=}$  & $\mathsf{AssignLefts}$ $\mathsf{AssignOp}$ $\mathsf{Expr}$ \\
 & $\big|$ &  $\mathsf{OpExpr}$ \\
 & $\big|$ &  $\mathsf{DelimitedExpr}$ \\
$\mathsf{GeneratorList}$ &  $\mathsf{::=}$  & $\mathsf{Generator}$ $\big($  $\mathbf{,}$ $\mathsf{Generator}$ $\big)$$^*$ \\
$\mathsf{Generator}$ &  $\mathsf{::=}$  & $\mathsf{Id}$ $\mathbf{\leftarrow}$ $\mathsf{Expr}$ \\
 & $\big|$ &  $\mathbf{(}$ $\mathsf{Id}$ $\mathbf{,}$ $\mathsf{IdList}$ $\mathbf{)}$ $\mathbf{\leftarrow}$ $\mathsf{Expr}$ \\
 & $\big|$ &  $\mathsf{Expr}$ \\
\end{longtable} \hfill 

\section{Expressions Enclosed by Keywords} 

 
\begin{longtable}[l]{p{3cm}rl}
$\mathsf{DelimitedExpr}$ &  $\mathsf{::=}$  & $\mathsf{TupleExpr}$ \\
 & $\big|$ &  $\mathbf{object}$ $\mathsf{Extends}$$^?$ $\mathsf{GoInAnObject}$ $\mathbf{end}$ \\
 & $\big|$ &  $\mathsf{Do}$ \\
 & $\big|$ &  $\mathbf{label}$ $\mathsf{Id}$ $\mathsf{BlockElems}$ $\mathbf{end}$ $\mathsf{Id}$ \\
 & $\big|$ &  $\mathbf{while}$ $\mathsf{Expr}$ $\mathsf{Do}$ \\
 & $\big|$ &  $\mathbf{for}$ $\mathsf{GeneratorList}$ $\mathsf{DoFront}$ $\mathbf{end}$ \\
 & $\big|$ &  $\mathbf{if}$ $\mathsf{Expr}$ $\mathbf{then}$ $\mathsf{BlockElems}$ $\mathsf{Elifs}$$^?$ $\mathsf{Else}$$^?$ $\mathbf{end}$ \\
 & $\big|$ &  $\big($  $\mathbf{if}$ $\mathsf{Expr}$ $\mathbf{then}$ $\mathsf{BlockElems}$ $\mathsf{Elifs}$$^?$ $\mathsf{Else}$ $\mathbf{end}$ $\mathbf{?}$ $\big)$ \\
 & $\big|$ &  $\mathbf{case}$ $\mathsf{Expr}$ $\mathsf{Op}$$^?$ $\mathbf{of}$ $\mathsf{CaseClauses}$ $\mathsf{CaseElse}$$^?$ $\mathbf{end}$ \\
 & $\big|$ &  $\mathbf{typecase}$ $\mathsf{TypecaseBindings}$ $\mathbf{of}$ $\mathsf{TypecaseClauses}$ $\mathsf{CaseElse}$$^?$ $\mathbf{end}$ \\
 & $\big|$ &  $\mathbf{try}$ $\mathsf{BlockElems}$ $\mathsf{Catch}$$^?$ $\big($  $\mathbf{finally}$ $\mathsf{BlockElems}$ $\big)$$^?$ $\mathbf{end}$ \\
$\mathsf{Do}$ &  $\mathsf{::=}$  & $\big($  $\mathsf{DoFront}$ $\mathbf{also}$ $\big)$$^*$ $\mathsf{DoFront}$ $\mathbf{end}$ \\
$\mathsf{DoFront}$ &  $\mathsf{::=}$  & $\big($  $\mathbf{at}$ $\mathsf{Expr}$ $\big)$$^?$ $\mathbf{atomic}$ $\mathbf{?}$ $\mathbf{do}$ $\mathsf{BlockElems}$$^?$ \\
$\mathsf{TupleExpr}$ &  $\mathsf{::=}$  & $\big($  $\big($  $\mathsf{Expr}$ $\mathbf{,}$ $\big)$$^*$ $\big($  $\mathsf{Expr}$ $\mathbf{...}$ $\mathbf{,}$ $\big)$$^?$ $\mathsf{Binding}$ $\big($  $\mathbf{,}$ $\mathsf{Binding}$ $\big)$$^*$ $\big)$ \\
 & $\big|$ &  $\mathsf{NoKeyTuple}$ \\
$\mathsf{NoKeyTuple}$ &  $\mathsf{::=}$  & $\big($  $\big($  $\mathsf{Expr}$ $\mathbf{,}$ $\big)$$^*$ $\mathsf{Expr}$ $\mathbf{...}$ $\big)$ \\
 & $\big|$ &  $\big($  $\big($  $\mathsf{Expr}$ $\mathbf{,}$ $\big)$$^*$ $\mathsf{Expr}$ $\big)$ \\
$\mathsf{Elifs}$ &  $\mathsf{::=}$  & $\mathsf{Elif}$$^+$ \\
$\mathsf{Elif}$ &  $\mathsf{::=}$  & $\mathbf{elif}$ $\mathsf{Expr}$ $\mathbf{then}$ $\mathsf{BlockElems}$ \\
$\mathsf{Else}$ &  $\mathsf{::=}$  & $\mathbf{else}$ $\mathsf{BlockElems}$ \\
$\mathsf{CaseClauses}$ &  $\mathsf{::=}$  & $\mathsf{CaseClause}$$^+$ \\
$\mathsf{CaseClause}$ &  $\mathsf{::=}$  & $\mathsf{Expr}$ $\mathbf{\Rightarrow}$ $\mathsf{BlockElems}$ \\
$\mathsf{CaseElse}$ &  $\mathsf{::=}$  & $\mathbf{else}$ $\mathbf{\Rightarrow}$ $\mathsf{BlockElems}$ \\
$\mathsf{TypecaseBindings}$ &  $\mathsf{::=}$  & $\mathbf{(}$ $\mathsf{BindingList}$ $\mathbf{)}$ \\
 & $\big|$ &  $\mathsf{Binding}$ \\
 & $\big|$ &  $\mathsf{Id}$ \\
$\mathsf{BindingList}$ &  $\mathsf{::=}$  & $\mathsf{Binding}$ $\big($  $\mathbf{,}$ $\mathsf{Binding}$ $\big)$$^*$ \\
$\mathsf{Binding}$ &  $\mathsf{::=}$  & $\mathsf{BindId}$ $\mathbf{=}$ $\mathsf{Expr}$ \\
$\mathsf{TypecaseClauses}$ &  $\mathsf{::=}$  & $\mathsf{TypecaseClause}$$^+$ \\
$\mathsf{TypecaseClause}$ &  $\mathsf{::=}$  & $\mathsf{TypecaseTypeRefs}$ $\mathbf{\Rightarrow}$ $\mathsf{BlockElems}$ \\
$\mathsf{TypecaseTypeRefs}$ &  $\mathsf{::=}$  & $\mathbf{(}$ $\mathsf{TypeRefList}$ $\mathbf{)}$ \\
 & $\big|$ &  $\mathsf{TypeRef}$ \\
$\mathsf{Catch}$ &  $\mathsf{::=}$  & $\mathbf{catch}$ $\mathsf{Id}$ $\mathsf{CatchClauses}$ \\
$\mathsf{CatchClauses}$ &  $\mathsf{::=}$  & $\mathsf{CatchClause}$$^+$ \\
$\mathsf{CatchClause}$ &  $\mathsf{::=}$  & $\mathsf{TraitType}$ $\mathbf{\Rightarrow}$ $\mathsf{BlockElems}$ \\
$\mathsf{Comprehension}$ &  $\mathsf{::=}$  & $\mathbf{\{}$ $\mathsf{Expr}$ $\big|$ $\mathsf{GeneratorList}$ $\mathbf{\}}$ \\
 & $\big|$ &  $\mathbf{\{}$ $\mathsf{Entry}$ $\big|$ $\mathsf{GeneratorList}$ $\mathbf{\}}$ \\
 & $\big|$ &  $\mathbf{<}$ $\mathsf{Expr}$ $\big|$ $\mathsf{GeneratorList}$ $\mathbf{>}$ \\
 & $\big|$ &  $\mathbf{[}$ $\mathsf{ArrayComprehensionClause}$$^+$ $\mathbf{]}$ \\
$\mathsf{Entry}$ &  $\mathsf{::=}$  & $\mathsf{Expr}$ $\mathbf{\mapsto}$ $\mathsf{Expr}$ \\
$\mathsf{ArrayComprehensionLeft}$ &  $\mathsf{::=}$  & $\mathsf{IdOrInt}$ $\mathbf{\mapsto}$ $\mathsf{Expr}$ \\
 & $\big|$ &  $\big($  $\mathsf{IdOrInt}$ $\mathbf{,}$ $\mathsf{IdOrIntList}$ $\big)$ $\mathbf{\mapsto}$ $\mathsf{Expr}$ \\
$\mathsf{ArrayComprehensionClause}$ &  $\mathsf{::=}$  & $\mathsf{ArrayComprehensionLeft}$ \\
 & $\big|$ &  $\mathsf{GeneratorList}$ \\
$\mathsf{IdOrInt}$ &  $\mathsf{::=}$  & $\mathsf{Id}$ \\
 & $\big|$ &  $\mathsf{IntLiteral}$ \\
$\mathsf{IdOrIntList}$ &  $\mathsf{::=}$  & $\mathsf{IdOrInt}$ $\big($  $\mathbf{,}$ $\mathsf{IdOrInt}$ $\big)$$^*$ \\
$\mathsf{BaseExpr}$ &  $\mathsf{::=}$  & $\mathsf{NoKeyTuple}$ \\
 & $\big|$ &  $\mathsf{Literal}$ \\
 & $\big|$ &  $\mathsf{Id}$ \\
 & $\big|$ &  $\mathbf{self}$ \\
$\mathsf{ExprList}$ &  $\mathsf{::=}$  & $\mathsf{Expr}$ $\big($  $\mathbf{,}$ $\mathsf{Expr}$ $\big)$$^*$ \\
\end{longtable} \hfill 

\section{Local Declarations} 

 
\begin{longtable}[l]{p{3cm}rl}
$\mathsf{BlockElems}$ &  $\mathsf{::=}$  & $\mathsf{BlockElem}$$^+$ \\
$\mathsf{BlockElem}$ &  $\mathsf{::=}$  & $\mathsf{LocalVarFnDecl}$ \\
 & $\big|$ &  $\mathsf{Expr}$ $\big($  $\mathbf{,}$ $\mathsf{GeneratorList}$ $\big)$$^?$ \\
$\mathsf{LocalVarFnDecl}$ &  $\mathsf{::=}$  & $\mathsf{LocalFnDecl}$$^+$ \\
 & $\big|$ &  $\mathsf{LocalVarDecl}$ \\
$\mathsf{LocalFnDecl}$ &  $\mathsf{::=}$  & $\mathsf{LocalFnMods}$$^?$ $\mathsf{FnHeaderFront}$ $\mathsf{FnHeaderClause}$ $\mathbf{=}$ $\mathsf{Expr}$ \\
$\mathsf{LocalVarDecl}$ &  $\mathsf{::=}$  & $\mathsf{LocalVarWTypes}$ $\mathsf{InitVal}$ \\
 & $\big|$ &  $\mathsf{LocalVarWTypes}$ \\
 & $\big|$ &  $\mathsf{LocalVarWoTypes}$ $\mathbf{=}$ $\mathsf{Expr}$ \\
 & $\big|$ &  $\mathsf{LocalVarWoTypes}$ $\mathbf{:}$ $\mathsf{TypeRef}$ $\mathbf{...}$ $\mathsf{InitVal}$$^?$ \\
 & $\big|$ &  $\mathsf{LocalVarWoTypes}$ $\mathbf{:}$ $\mathsf{SimpleTupleType}$ $\mathsf{InitVal}$$^?$ \\
$\mathsf{LocalVarWTypes}$ &  $\mathsf{::=}$  & $\mathsf{LocalVarWType}$ \\
 & $\big|$ &  $\big($  $\mathsf{LocalVarWType}$ $\big($  $\mathbf{,}$ $\mathsf{LocalVarWType}$ $\big)$$^+$ $\big)$ \\
$\mathsf{LocalVarWType}$ &  $\mathsf{::=}$  & $\mathbf{var}$ $\mathbf{?}$ $\mathsf{Id}$ $\mathsf{IsType}$ \\
$\mathsf{LocalVarWoTypes}$ &  $\mathsf{::=}$  & $\mathsf{LocalVarWoType}$ \\
 & $\big|$ &  $\big($  $\mathsf{LocalVarWoType}$ $\big($  $\mathbf{,}$ $\mathsf{LocalVarWoType}$ $\big)$$^+$ $\big)$ \\
$\mathsf{LocalVarWoType}$ &  $\mathsf{::=}$  & $\mathbf{var}$ $\mathbf{?}$ $\mathsf{Id}$ \\
\end{longtable} \hfill 

\section{Literals} 

 
\begin{longtable}[l]{p{3cm}rl}
$\mathsf{Literal}$ &  $\mathsf{::=}$  & $\mathbf{(}$ $\mathbf{)}$ \\
 & $\big|$ &  $\mathsf{NumericLiteral}$ \\
 & $\big|$ &  $\mathsf{CharLiteral}$ \\
 & $\big|$ &  $\mathsf{StringLiteral}$ \\
 & $\big|$ &  $\mathbf{\{}$ $\mathsf{Entry}$ $\big($  $\mathbf{,}$ $\mathsf{Entry}$ $\big)$$^*$ $\mathbf{\}}$ \\
 & $\big|$ &  $\mathbf{[}$ $\mathsf{RectElements}$ $\mathbf{]}$ \\
$\mathsf{RectElements}$ &  $\mathsf{::=}$  & $\mathsf{Expr}$ $\mathsf{MultiDimCons}$$^*$ \\
$\mathsf{MultiDimCons}$ &  $\mathsf{::=}$  & $\mathsf{RectSeparator}$ $\mathsf{Expr}$ \\
$\mathsf{RectSeparator}$ &  $\mathsf{::=}$  & $\mathbf{;}$ $\mathbf{+}$ \\
 & $\big|$ &  $\mathsf{Whitespace}$ \\
\end{longtable} \hfill 

\section{Types} 

 
\begin{longtable}[l]{p{3cm}rl}
$\mathsf{IsType}$ &  $\mathsf{::=}$  & $\mathbf{:}$ $\mathsf{TypeRef}$ \\
$\mathsf{TypeRef}$ &  $\mathsf{::=}$  & $\mathsf{TraitType}$ \\
 & $\big|$ &  $\mathsf{ArrowType}$ \\
 & $\big|$ &  $\mathsf{TupleType}$ \\
 & $\big|$ &  $\big($  $\mathsf{TypeRef}$ $\mathbf{?}$ $\big)$ \\
$\mathsf{TraitType}$ &  $\mathsf{::=}$  & $\mathsf{DottedId}$ $\big($  $\mathbf{\llbracket}$ $\mathsf{StaticArgList}$ $\mathbf{\rrbracket}$ $\big)$$^?$ \\
 & $\big|$ &  $\mathbf{\{}$ $\mathsf{TypeRef}$ $\mathbf{\mapsto}$ $\mathsf{TypeRef}$ $\mathbf{\}}$ \\
 & $\big|$ &  $\mathbf{<}$ $\mathsf{TypeRef}$ $\mathbf{>}$ \\
 & $\big|$ &  $\mathsf{TypeRef}$ $\mathbf{[}$ $\mathsf{ArraySize}$$^?$ $\mathbf{]}$ \\
 & $\big|$ &  $\mathsf{TypeRef}$ $\mathbf{\wedge}$ $\mathsf{IntLiteral}$ \\
 & $\big|$ &  $\mathsf{TypeRef}$ $\mathbf{\wedge}$ $\big($  $\mathsf{ExtentRange}$ $\big($  $\mathbf{\times}$ $\mathsf{ExtentRange}$ $\big)$$^*$ $\big)$ \\
$\mathsf{ArrowType}$ &  $\mathsf{::=}$  & $\mathsf{TypeRef}$ $\mathbf{\rightarrow}$ $\mathsf{TypeRef}$ $\mathsf{Throws}$$^?$ \\
$\mathsf{TupleType}$ &  $\mathsf{::=}$  & $\big($  $\big($  $\mathsf{TypeRef}$ $\mathbf{,}$ $\big)$$^*$ $\big($  $\mathsf{TypeRef}$ $\mathbf{...}$ $\mathbf{,}$ $\big)$$^?$ $\mathsf{KeywordType}$ $\big($  $\mathbf{,}$ $\mathsf{KeywordType}$ $\big)$$^*$ $\big)$ \\
 & $\big|$ &  $\big($  $\big($  $\mathsf{TypeRef}$ $\mathbf{,}$ $\big)$$^*$ $\mathsf{TypeRef}$ $\mathbf{...}$ $\big)$ \\
 & $\big|$ &  $\mathsf{SimpleTupleType}$ \\
$\mathsf{KeywordType}$ &  $\mathsf{::=}$  & $\mathsf{Id}$ $\mathbf{=}$ $\mathsf{TypeRef}$ \\
$\mathsf{SimpleTupleType}$ &  $\mathsf{::=}$  & $\big($  $\mathsf{TypeRef}$ $\mathbf{,}$ $\mathsf{TypeRefList}$ $\big)$ \\
$\mathsf{TypeRefList}$ &  $\mathsf{::=}$  & $\mathsf{TypeRef}$ $\big($  $\mathbf{,}$ $\mathsf{TypeRef}$ $\big)$$^*$ \\
$\mathsf{StaticArgList}$ &  $\mathsf{::=}$  & $\mathsf{StaticArg}$ $\big($  $\mathbf{,}$ $\mathsf{StaticArg}$ $\big)$$^*$ \\
$\mathsf{StaticArg}$ &  $\mathsf{::=}$  & $\mathsf{Op}$ \\
 & $\big|$ &  $\mathsf{TypeRef}$ \\
 & $\big|$ &  $\mathbf{(}$ $\mathsf{StaticArg}$ $\mathbf{)}$ \\
$\mathsf{ArraySize}$ &  $\mathsf{::=}$  & $\mathsf{ExtentRange}$ $\big($  $\mathbf{,}$ $\mathsf{ExtentRange}$ $\big)$$^*$ \\
$\mathsf{ExtentRange}$ &  $\mathsf{::=}$  & $\mathsf{StaticArg}$$^?$ $\mathbf{\#}$ $\mathsf{StaticArg}$$^?$ \\
 & $\big|$ &  $\mathsf{StaticArg}$$^?$ $\mathbf{:}$ $\mathsf{StaticArg}$$^?$ \\
 & $\big|$ &  $\mathsf{StaticArg}$ \\
\end{longtable} \hfill 

\section{Symbols and Operators} 

 
\begin{longtable}[l]{p{3cm}rl}
$\mathsf{AssignOp}$ &  $\mathsf{::=}$  & $\mathbf{:=}$ \\
 & $\big|$ &  $\mathsf{Op}$ $\mathbf{=}$ \\
$\mathsf{Accumulator}$ &  $\mathsf{::=}$  & $\mathbf{\sum}$ $\big|$ $\mathbf{\prod}$ $\big|$ $\mathbf{BIG}$ $\mathsf{Op}$ \\
\end{longtable} \hfill 

\section{Identifiers} 

 
\begin{longtable}[l]{p{3cm}rl}
$\mathsf{IdList}$ &  $\mathsf{::=}$  & $\mathsf{Id}$ $\big($  $\mathbf{,}$ $\mathsf{Id}$ $\big)$$^*$ \\
$\mathsf{Name}$ &  $\mathsf{::=}$  & $\mathsf{Id}$ \\
 & $\big|$ &  $\mathbf{opr}$ $\mathsf{Op}$ \\
$\mathsf{DottedId}$ &  $\mathsf{::=}$  & $\mathsf{Id}$ $\big($  $\mathbf{.}$ $\mathsf{Id}$ $\big)$$^*$ \\
$\mathsf{BindId}$ &  $\mathsf{::=}$  & $\mathsf{Id}$ \\
 & $\big|$ &  $\mathbf{\_}$ \\
\end{longtable} \hfill 



\section{Wellformedness}
\label{sec:wellformedness}
%\input{wellformedness}

\section{Examples}
\label{sec:examples}
%\input{examples}

\section{Related Work}\label{sec:related}
%% * multiple dispatch
%   * fortress
%   * CLOS
%   * multijava
%   * cecil
% * type classes
%   * wadler 89
%   * qualified types (mpj)
%   * concepts (siek)
%   * inability to add ad-hoc overloaded functions
% * GADTs
%   * GADT inference (spj)
%   * HMG(X) (pottier)
%   * with OOP (russo)
%
\subsection{Overloading and dynamic dispatch.} 

\TODO{Add discussion of Castagna et al.\ here?}

Primarily, our system
strictly extends our previous effort \cite{allen07} with parametric polymorphism;
all previous properties and results remain intact. The inclusion of parametric
functions and types represents a shift in the research literature on overloading
and multiple dynamic dispatch.

Millstein and Chambers \cite{millstein02,millstein03} devised the language
Dubious to study overloaded functions with symmetric multiple dynamic dispatch
(\emph{multimethods}), and with Clifton and Leavens they developed MultiJava
\cite{multijava}, an extension of Java with Dubious' semantics for multimethods.
In \cite{feml}, Lee and Chambers presented F(E\textsc{ml}), a language with
classes, symmetric multiple dispatch, and parameterized modules, but without
parametricity at the level of functions or types. None of these systems support
polymorphic functions or types. F(E\textsc{ml})'s parameterized modules
(\emph{functors}) constitute a form of parametricity but they cannot be implicitly
applied; the functions defined therein cannot be \emph{overloaded} with those
defined in other functors. For a more detailed comparison of modularity and
dispatch between our system and these, we refer to the related work section of
our previous paper \cite{allen07}.

% I took out discussion of modularity here; it's charged and unnecessary
% in order to distinguish our work. EricAllen 7/15/2011
Overloading and multiple dispatch in the context of polymorphism 
has previously been studied by Bourdoncle and Merz \cite{bourdoncle97}. 
Their system, ML$_\le$, integrates parametric polymorphism, 
class-based object orientation, and multimethods,
but lacks multiple inheritance. 
Each multimethod (overloaded set) requires a unique specification (principal type), 
which prevents overloaded functions defined on disjoint domains; 
% the domains of the multimethod branches must partition the specification domain, 
% which eliminates subtype-based specialization;
and link-time checks are performed to ensure that multimethods are fully
implemented across modules. 
On the other hand, ML$_\le$ allows variance annotations on type constructors---% 
something we attribute to future work.

Litvinov~\cite{litvinov98} developed a type system for the Cecil language,
which supports bounded parametric polymorphism and multimethods.
Because Cecil has a type-erasure semantics, 
statically checked parametric polymorphism has no effect on run-time dispatch.

\subsection{Type classes.} Wadler and Blott \cite{wadler89} introduced
\emph{type class} as a means to specify and implement overloaded
functions like equality and arithmetic operators in Haskell. Other authors
have translated type classes to languages besides Haskell \cite{dreyer07,siek05,wehr07}.
Type classes encapsulate overloaded function declarations, with separate
\emph{instances} that define the behavior of those functions (called \emph{class methods})
for any particular type schema. Parametric polymorphism is then augmented to
express type class constraints, providing a way to quantify a type variable --- and
thus a function definition --- over all types that instantiate the type class. 

% In his thesis \cite{jonesbook} Jones generalized Haskell's underlying type
% system as \emph{qualified types}, in which the satisfaction of type predicates
% must be proved with constructed \emph{evidence}. Qualified type systems (such
% as Haskell) exhibit the \emph{principal types} property necessary for full
% Damas-Milner style type inference \cite{dm82,jonesbook}; our system conservatively
% assumes only \emph{local type inference} \cite{pierce00} --- implicit type
% instantiation for polymorphic function calls.

In systems with type classes, overloaded functions must be contained in some
type class, and their signatures must vary in exactly the same structural
position. This uniformity is necessary for an overloaded function call to
admit a principal type; with a principal type for some function call's context,
the type checker can determine the constraints under which a correct overloaded
definition will be found. Because of this requirement, type classes are ill-suited
for fixed, \emph{ad hoc} sets of overloaded functions like:
\begin{FortressCode}
{\tt ~~~~}\+\VAR{println}(\ultrathin)\COLON (\ultrathin) = \VAR{println}(\hbox{\rm\usefont{T1}{ptm}{m}{n}``\verythin''}) \\
    \VAR{println}(s\COLON \TYP{String})\COLON (\ultrathin) = \ldots\-
\end{FortressCode}
or functions lacking uniform variance in the domain and range\footnote{With the
\emph{multi-parameter type class} extension, one could define functions as these.
A reference to the method \mono{bar}, however, would require an explicit type
annotation like \mono{:: Int -> Bool} to apply to an \mono{Int}.} like:
\begin{FortressCode}
{\tt ~~~~}\+\VAR{bar}(x\COLON \mathbb{Z})\COLON \TYP{Boolean} = (x = 0) \\
    \VAR{bar}(x\COLON \TYP{Boolean})\COLON \mathbb{Z} =\; \KWD{if} x \KWD{then} 1 \KWD{else} 2 \KWD{end} \\
    \VAR{bar}(x\COLON \TYP{String})\COLON \TYP{String} = x\-
\end{FortressCode}
With type classes one can write overloaded functions with identical domain
types. Such behavior is consistent with the \emph{static}, \emph{type-based}
dispatch of Haskell, but it would lead to irreconcilable ambiguity in the
\emph{dynamic}, \emph{value-based} dispatch of our system.
%% In Appendix~\ref{app:haskell}, we present a further discussion of how our overloading resolution differs from that of Haskell and how our system might translate to that language, thereby addressing an existing inconsistency in modern type class extensions.

A broader interpretation of Wadler and Blott's \cite{wadler89} sees type
classes as program abstractions that quotient the space of ad-hoc polymorphism
into the much smaller space of class methods. Indeed, Wadler and Blott's title
suggests that the unrestricted space of ad-hoc polymorphism should be tamed,
whereas our work embraces the possible expressivity achieved from mixing ad-hoc
and parametric polymorphism by specifying the requisites for determinism and type safety.


\section{Conclusion and Discussion}\label{sec:conclusion}
%We have shown how to statically ensure safety of overloaded, polymorphic functions while imposing relatively minimal restrictions solely on function definition sites. We provide rules on definitions that can be checked modularly, irrespective of call sites, and we show how to mechanically verify that a program satisfies these rules. The type analysis required for implementing these checks involves subtyping on universal and existential types, which adds complexity not required for similar checks on monomorphic functions. We have defined an object-oriented language to explain our system of static checks, and we have implemented them as part of the open-source Fortress compiler \cite{Fortress}.

Further, we show that in order to check many ``natural'' overloaded functions with our system in the context of a generic, object-oriented language, richer type relations must be available to programmers---the subtyping relation prevalent among such languages does not afford enough type analysis alone. We have thus introduced an explicit, nominal exclusion relation to check safety of more interesting overloaded functions.

Variance annotations have proven to be a convenient and expressive addition to languages based on nominal subtyping \cite{bourdoncle97,kennedy07,scala}. They add additional complexity to polymorphic exclusion checking, so we leave them to future work.


\section*{Acknowledgments}
% This work is supported in part by the Engineering Research Center of Excellence Program of Korea Ministry of Education,
% Science and Technology(MEST) / National Research Foundation of Korea(NRF)
% (Grant 2011-0000974).

\bibliographystyle{plain}
% The bibliography should be embedded for final submission.
\bibliography{paper}
% \begin{thebibliography}{}
% \softraggedright
% 
% \input{biblio.tex}
% 
% \end{thebibliography}

% \appendix
\end{document}
