We write $\bar{x}$ as shorthand for a possibly empty comma-separated sequence $x_1, x_2, \ldots, x_n$ for some freely chosen nonnegative integer $n$;
thus $\bar{x}$ may expand to `` '' or ``$x_1$'' or ``$x_1, x_2$'' or ``$x_1, x_2, x_3$'' or ``$x_1, x_2, x_3, x_4$'' and so on.
More generally, for any expression, that same expression with an overbar is shorthand for a possibly empty comma-separated sequence
of copies of that expression with two transformations applied to each copy: (a) any subexpression that is underlined one or more times
is replaced by a copy of that subexpression with one underline removed, and (b) any subexpression that is a single letter, possibly with primes and/or subscripts attached, that is not underlined
is replaced by a copy of that possibly decorated identifier with an additional subscript $i$ appended, where $i$ is the number of the copy (starting from the left with 1). Thus $\bar{\underline{[\tau/P]}\tau'}$ means $[\tau/P]\tau'_1, [\tau/P]\tau'_2, \dots, [\tau/P]\tau'_n$,
so one possible concrete expansion is $[\tau/P]\tau'_1, [\tau/P]\tau'_2, [\tau/P]\tau'_3, [\tau/P]\tau'_4, [\tau/P]\tau'_5$.  As another example, the shorthand
$f\bigobb{P \extends \bdb{\tau}}$ means $f\bigob{P_1 \extends \bdb{\tau_1}, P_2 \extends \bdb{\tau_1}, \ldots, P_n \extends \bdb{\tau_n}}$ which in turn means:
\[ \begin{array}{l@{}l}
    f \big\llbracket \mskip0.5\thinmuskip & P_1 \extends \bd{\tau_{11}, \tau_{12}, \ldots, \tau_{1{n_1}}}, \\
                                          & P_2 \extends \bd{\tau_{21}, \tau_{22}, \ldots, \tau_{2{n_2}}}, \\
                                          & \ldots, \\
                                          & P_n \extends \bd{\tau_{m1}, \tau_{m2}, \ldots, \tau_{m{n_m}}} \mskip 0.5\thinmuskip \big\rrbracket
\end{array} \]
so one possible concrete expansion is:
\[ \begin{array}{l@{}l}
    f \big\llbracket \mskip0.5\thinmuskip & P_1 \extends \bd{\tau_{11}, \tau_{12}, \tau_{13}}, \\
                                          & P_2 \extends \bd{\tau_{21}}, \\
                                          & P_3 \extends \bd{\,}, \\
                                          & P_4 \extends \bd{\tau_{41}, \tau_{42}, \tau_{43}, \tau_{44}, \tau_{45}, \tau_{46}} \mskip 0.5\thinmuskip \big\rrbracket
\end{array} \]

The function $\#$ returns an integer sayig how many arguments it was given; thus $\#(\bar{x})$ tells the length of the sequence into which $\bar{x}$ has expanded.

We use ``$\dontcare$'' as a shorthand for any sequence of tokens that is correctly balanced within respect to parentheses, braces, and brackets,
such that every comma or semicolon in the sequence is contained within at least one matched pair of parentheses, braces, or brackets.

When either kind of shorthand is used in a BNF rule or inference rule, it is as if there were an infinite number of instantiations of the rule,
one for each possible expansion of the shorthand.  If such shorthands are used more than once in a rule, the effect is multiplicative (Cartesian product), except that
if multiple overbar shorthands attach subscripts to the same identifier (with identical decorations), the effect is to ``zip'' those shorthands, in which case they all
produce the same number of copies, using the same index $i$ in any given copy of the containing rule.

As an additional convenience using such shorthands, we adopt these conventions:
\begin{itemize}
\item If a judgment has several comma-separated expressions to the right of the turnstile, it is
as if there were several distinct judgments, one containing each of the expressions to the right of the turnstile.
Thus the judgment $\jgTHREEtemplate{\Gamma}{\tau_1}{\extends}{\tau'_1}{\tau_2}{\tau'_2}{\tau_3}{\tau'_3}$
means the same as three separately written judgments:
\[\jgtemplate[\Gamma]{\tau_1}{\extends}{\tau'_1} \andalso \jgtemplate[\Gamma]{\tau_2}{\extends}{\tau'_2} \andalso \jgtemplate[\Gamma]{\tau_3}{\extends}{\tau'_3} \]
\item If a judgment has nothing to the right of the turnstile, it is
as if there were no judgment written at all.
\item If an inference rule has several comma-separated expressions or judgments below the line as consequents, it is
as if there were several distinct inference rules, one containing each of the consequents below the line.
\item If an inference rule has nothing below the line, it is
as if there were no inference rule written at all.
\end{itemize}

Finally, we sometimes use parentheses or braces or brackets of different sizes within an expression purely to enhance readability;
the size of such a symbol does not affect its meaning in the formalism.
