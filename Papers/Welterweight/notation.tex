We use the term \emph{monogram} to refer to a single letter (Latin or Greek)  that, rather than being used for decorative purposes, is itself possibly ``decorated'' with one or more prime marks and/or a sequence of one or more subscripts.  Examples of monograms are $x$, $\beta$, $e'$, $\alpha_2$, and $\tau'_{15\,27}$.

We write $\bar{x}$ as shorthand for a possibly empty comma-separated sequence $x_1, x_2, \ldots, x_n$ for some freely chosen nonnegative integer $n$;
thus $\bar{x}$ may expand to `` '' or ``$x_1$'' or ``$x_1, x_2$'' or ``$x_1, x_2, x_3$'' or ``$x_1, x_2, x_3, x_4$'' and so on.
More generally, for any expression, that same expression with an overbar is shorthand for a possibly empty comma-separated sequence
of copies of that expression with two transformations applied to each copy: (a) any subexpression that is underlined one or more times
is replaced by a copy of that subexpression with one underline removed, and (b) any subexpression that is a monogram that is not underlined
is replaced by a copy of that monogram with an additional subscript $i$ appended, where $i$ is the number of the copy (starting from the left with 1).
Thus $\bar{\underline{[\tau/P]}\tau'}$ means $[\tau/P]\tau'_1, [\tau/P]\tau'_2, \dots, [\tau/P]\tau'_n$,
so one possible concrete expansion is $[\tau/P]\tau'_1, [\tau/P]\tau'_2, [\tau/P]\tau'_3, [\tau/P]\tau'_4, [\tau/P]\tau'_5$.
If overbar constructions are nested, they are expanded outermost first.
Therefore the shorthand
$f\bigobb{P \extends \bdb{\tau}}$ means $f\bigob{P_1 \extends \bdb{\tau_1}, P_2 \extends \bdb{\tau_2}, \ldots, P_n \extends \bdb{\tau_n}}$, which in turn means:
\[ \begin{array}{l@{}l}
    f \big\llbracket \mskip0.5\thinmuskip & P_1 \extends \bd{\tau_{1\,1}, \tau_{1\,2}, \ldots, \tau_{1\,{n_1}}}, \\
                                          & P_2 \extends \bd{\tau_{2\,1}, \tau_{2\,2}, \ldots, \tau_{2\,{n_2}}}, \\
                                          & \ldots, \\
                                          & P_n \extends \bd{\tau_{m\,1}, \tau_{m\,2}, \ldots, \tau_{m\,{n_m}}} \mskip 0.5\thinmuskip \big\rrbracket
\end{array} \]
so one possible concrete expansion is:
\[ \begin{array}{l@{}l}
    f \big\llbracket \mskip0.5\thinmuskip & P_1 \extends \bd{\tau_{1\,1}, \tau_{1\,2}, \tau_{1\,3}}, \\
                                          & P_2 \extends \bd{\tau_{2\,1}}, \\
                                          & P_3 \extends \bd{\,}, \\
                                          & P_4 \extends \bd{\tau_{4\,1}, \tau_{4\,2}, \tau_{4\,3}, \tau_{4\,4}, \tau_{4\,5}, \tau_{4\,6}} \mskip 0.5\thinmuskip \big\rrbracket
\end{array} \]
Note the use of whitespace between subscripts so that $\tau_{4\,12}$ is clearly different from $\tau_{41\,2}$.

The function $\countof$ returns an integer saying how many arguments it was given; thus $\countof(\bar{x})$ tells the length of the sequence into which $\bar{x}$ has expanded.

After all occurrences of the overbar construction have been expanded, three other shorthand substitutions take place:
\begin{itemize}
\item Every monogram whose base letter has been described as \emph{ranging over} a {\sc bnf} nonterminal of a specified grammar is replaced by a token sequence generated by the grammar from that nonterminal.
\item Each occurrence of the symbol ``$\dontcare$'' is replaced by any sequence of tokens that is correctly balanced within respect to parentheses, braces, and brackets of all kinds,
such that every comma or semicolon in the sequence is contained within at least one matched pair of parentheses, braces, or brackets. (This is used as a ``don't care'' indication when asking whether any of a set of constructs matches a certain syntactic pattern.)
\item Each occurrence of the symbol ``$\emptyseq$'' is deleted.  (This symbol is used as an explicit indication that an empty sequence of symbols is intended.)
\end{itemize}

When any of these shorthands is used in an overall context (such as a {\sc bnf} rule, inference rule, axiom, or expository sentence or paragraph), it is as if there were an infinite number of instantiations of that context, one for each possible expansion of the shorthand.  Three consistency constraints must be obeyed in performing the substitutions for any single such context:
\begin{itemize}
\item
If the same monogram (with identical decorations) has an additional subscript attached to it
by more than one overbar construction, then all such overbar constructions are constrained to produce the same number of copies in any given instantiation of the rule;
otherwise the choices for the number of copies produced by each overbar construction is free and independent.
\item
If the base letter of a monogram ranges over a {\sc bnf} nonterminal, then multiple identical occurrences of the monogram must be replaced by identical copies of a single generated token sequence.
\item
If two distinct monograms each have base letters that range over a {\sc bnf} nonterminal that expands to simply ``identifier'', then they must be replaced with different identifiers.
\end{itemize}
The last two constraints rely on metavariable declarations such as those in Figure~\ref{fig:metavariables}.  A declaration such as
``$e$ ranges over expressions $e$'' means 
``monograms with base letter $e$ expand into expressions generated by {\sc bnf} nonterminal $e$'';
this may seem redundant, but only because by convention we frequently use a single-letter identifier as a {\sc bnf} nonterminal
and then go on to use that same single-letter identifier as a base letter for monograms.
A declaration such as
``$\alpha, \gamma, \rho, \chi, \eta$ range over lattice types $\alpha$'' means 
``monograms with base letter $\alpha$ or $\gamma$ or $\rho$ or $\chi$ or $\eta$ expand into expressions generated by {\sc bnf} nonterminal $\alpha$''
which is more clearly not a redundant statement.

As an additional convenience using these shorthands, we adopt these conventions:
\begin{itemize}
\item If a judgment has several comma-separated expressions to the right of the turnstile ``$\turnstile$'', it is
as if there were several distinct judgments, one containing each of the expressions to the right of the turnstile.
Thus the judgment $\jgTHREEtemplate{\Gamma}{\tau_1}{\extends}{\tau'_1}{\tau_2}{\tau'_2}{\tau_3}{\tau'_3}$
means the same as three separately written judgments:
\[\jgtemplate[\Gamma]{\tau_1}{\extends}{\tau'_1} \andalso \jgtemplate[\Gamma]{\tau_2}{\extends}{\tau'_2} \andalso \jgtemplate[\Gamma]{\tau_3}{\extends}{\tau'_3} \]
\item If a judgment has nothing to the right of the turnstile, it is
as if there were no judgment written at all.
\item If an inference rule has several comma-separated expressions or judgments as consequents, it is
as if there were several distinct inference rules, one containing each of the consequents.
\item If an inference rule has no consequents, it is
as if there were no inference rule written at all.
\end{itemize}

As an extreme (but useful) example of the application of these conventions, consider this axiom:

\infax{ \jbevalstep[\Delta]{\underline{E}\big[\pi(\underline{\bar{v}})\big]}{\underline{E}[v]} }

\noindent In order to apply this axiom to a particular case, we may freely choose to expand the largest overbar construction to produce, say, two copies:

\infax{ \jevalstepTWO[\Delta]{E\big[\pi_1(\bar{v})\big]}{E[v_1]}{E\big[\pi_2(\bar{v})\big]}{E[v_2]} }

\noindent Note that both the $\pi$ symbol and the second occurrence of $v$ receive subscripts in each copy, but the underlines (which are removed as part of the expansion process) prevent the first occurrence of $v$ (which happens to have a second overbar) and the two occurrences of $E$ from receiving subscripts.  Now we expand the remaining overbars, but because they will attach subscripts to the symbol $v$, and $v$ has already had subscripts attached by the larger overbar, we must choose the same number of copies (two) for each of these overbars:

\infax{ \jevalstepTWO[\Delta]{E\big[\pi_1(v_1,v_2)\big]}{E[v_1]}{E\big[\pi_2(v_1,v_2)\big]}{E[v_2]} }

\noindent Then this judgment with two comma-separated expressions to the right of the turnstile is understood to mean two distinct judgments:

\infax{ \jevalstep[\Delta]{E\big[\pi_1(v_1,v_2)]}{E[v_1]} \\[2pt]
        \jevalstep[\Delta]{E\big[\pi_2(v_1,v_2)]}{E[v_2]} }

A final note: we sometimes use parentheses or braces or brackets of different sizes within an expression purely to enhance readability;
the size of such a symbol does not affect its meaning in the formalism.
