%%%%%%%%%%%%%%%%%%%%%%%%%%%%%%%%%%%%%%%%%%%%%%%%%%%%%%%%%%%%%%%%%%%%%%%%%%%%%%%%
%   Copyright 2009, Oracle and/or its affiliates.
%   All rights reserved.
%
%
%   Use is subject to license terms.
%
%   This distribution may include materials developed by third parties.
%
%%%%%%%%%%%%%%%%%%%%%%%%%%%%%%%%%%%%%%%%%%%%%%%%%%%%%%%%%%%%%%%%%%%%%%%%%%%%%%%%

\section{\Apis}
\seclabel{apis}

\note{All APIs should provide all types explicitly in all declarations.}

\begin{Grammar}
\emph{Api} &::=&
\KWD{api} \emph{APIName} \option{\emph{Imports}} \option{\emph{AbsDecls}} \KWD{end}
\options{\option{\KWD{api}} \emph{APIName}}\\

\emph{AbsDecls} &::=& \emph{AbsDecl}$^+$\\

\emph{AbsDecl} &::=& \emph{AbsTraitDecl}\\
&$|$& \emph{AbsObjectDecl}\\
&$|$& \emph{AbsVarDecl}\\
&$|$& \emph{AbsFnDecl}\\
&$|$& \emph{DimUnitDecl}\\
&$|$& \emph{TypeAlias}\\
&$|$& \emph{TestDecl}\\
&$|$& \emph{PropertyDecl}\\
&$|$& \emph{AbsExternalSyntax}\\

\end{Grammar}

\begin{GrammarTwo}
\emph{AbsTraitDecl}  &::=& \option{\emph{AbsTraitMods}}
\emph{TraitHeaderFront} \emph{AbsTraitClauses}
\option{\emph{AbsGoInATrait}} \KWD{end} \\
&&\options{\option{\KWD{trait}} \emph{Id}}\\

\emph{AbsTraitMods} &::=& \emph{AbsTraitMod}$^+$\\

\emph{AbsTraitMod} &::=& \KWD{value} $|$ \KWD{test}\\

\emph{AbsTraitClauses} &::=& \emph{AbsTraitClause}$^*$\\

\emph{AbsTraitClause} &::=& \emph{Excludes} \\
&$|$& \emph{AbsComprises} \\
&$|$& \emph{Where} \\

\emph{AbsComprises} &::=& \KWD{comprises} \emph{ComprisingTypes} \\

\emph{ComprisingTypes}
&::=& \emph{TraitType} \\
&$|$& \texttt{\{} \emph{ComprisingTypeList} \texttt{\}} \\

\emph{ComprisingTypeList}
&::=& \EXP{\ldots} \\
&$|$& \emph{TraitType}(\EXP{,} \emph{TraitType})$^*$ \options{\EXP{,} \ldots}\\

\end{GrammarTwo}

\begin{GrammarTwo}
\emph{AbsGoInATrait}
&::=& \option{\emph{AbsCoercions}}
\emph{AbsGoFrontInATrait} \option{\emph{AbsGoBackInATrait}}\\
&$|$& \option{\emph{AbsCoercions}}
\emph{AbsGoBackInATrait} \\
&$|$& \emph{AbsCoercions}\\

\emph{AbsCoercions} &::=& \emph{AbsCoercion}$^+$\\

\emph{AbsGoFrontInATrait} &::=& \emph{AbsGoesFrontInATrait}$^+$\\

\emph{AbsGoesFrontInATrait}
&::=& \emph{ApiFldDecl} \\
&$|$& \emph{AbsGetterSetterDecl} \\
&$|$& \emph{PropertyDecl} \\

\emph{AbsGoBackInATrait} &::=& \emph{AbsGoesBackInATrait}$^+$\\

\emph{AbsGoesBackInATrait}
&::=& \emph{AbsMdDecl} \\
&$|$& \emph{PropertyDecl} \\

\emph{AbsObjectDecl} &::=& \option{\emph{AbsObjectMods}}
 \emph{ObjectHeader} \option{\emph{AbsGoInAnObject}} \KWD{end}
\options{\option{\KWD{object}} \emph{Id}}\\

\emph{AbsObjectMods} &::=& \emph{AbsTraitMods}\\

\emph{AbsGoInAnObject}
&::=& \option{\emph{AbsCoercions}}
\emph{AbsGoFrontInAnObject} \option{\emph{AbsGoBackInAnObject}}\\
&$|$& \option{\emph{AbsCoercions}}
\emph{AbsGoBackInAnObject} \\
&$|$& \emph{AbsCoercions}\\

\emph{AbsGoFrontInAnObject} &::=& \emph{AbsGoesFrontInAnObject}$^+$\\

\emph{AbsGoesFrontInAnObject}
&::=& \emph{ApiFldDecl} \\
&$|$& \emph{AbsGetterSetterDecl} \\
&$|$& \emph{PropertyDecl} \\

\emph{AbsGoBackInAnObject} &::=& \emph{AbsGoesBackInAnObject}$^+$\\

\emph{AbsGoesBackInAnObject}
&::=& \emph{AbsMdDecl} \\
&$|$& \emph{PropertyDecl} \\

\emph{AbsCoercion} &::=&
\KWD{coerce}\option{\emph{StaticParams}}\texttt(\emph{BindId} \emph{IsType}\texttt)\emph{CoercionClauses} \option{\KWD{widens}}\\

\emph{ApiFldDecl} &::=& \option{\emph{ApiFldMods}}
\emph{BindId} \emph{IsType}\\

\emph{ApiFldMods} &::=& \emph{ApiFldMod}$^+$\\

\emph{ApiFldMod} &::=& \KWD{hidden} $|$ \KWD{settable} $|$ \KWD{test}\\

\emph{AbsVarDecl} &::=& \option{\emph{AbsVarMods}} \emph{VarWTypes}\\
&$|$& \option{\emph{AbsVarMods}} \emph{BindIdOrBindIdTuple} \EXP{\mathrel{\mathtt{:}}} \emph{Type}\EXP{...}\\
&$|$& \option{\emph{AbsVarMods}} \emph{BindIdOrBindIdTuple} \EXP{\mathrel{\mathtt{:}}} \emph{TupleType}\\

\emph{AbsVarMods} &::=& \emph{AbsVarMod}$^+$\\

\emph{AbsVarMod} &::=& \KWD{var} $|$ \KWD{test}\\

\end{GrammarTwo}


\Apis\ are compiled from special \apiN\ definitions.  These are source
files which declare the entities declared by the \apiN, the names of all
\apisN\ referred to by those declarations, and prose documentation.  In
short, the source code of an \apiN\ must specify all the information
that is traditionally provided for the published \apisN\ of libraries in
other languages.

The syntax of an \apiN\ definition is
identical to the syntax of a component definition, except that:
\begin{enumerate}

\item
An \apiN\ definition begins with \KWD{api}
rather than \KWD{component}.
As with components,
the identifiers associated with \apisN\ that are not included in \library\
are prefixed with the reverse of the URL of the development team.

\item
An \apiN\ does not include \KWD{export} statements.
(However, it does include \KWD{import} statements,
which name the other \apisN\ used in the \apiN\ definition.)

\item
Only declarations (not definitions!) are included in an \apiN\
definition except test and property declarations.
A method or field declaration may include the modifier \KWD{abstract}.
(Whether a declaration includes the modifier \KWD{abstract}
has a significant effect on its meaning, as discussed below).

\end{enumerate}


Import statements in APIs permit names declared in imported APIs to be used
in the importing API either qualified or unqualified,
just as in components.  Those names are \emph{not}, however,
part of the importing API,
and thus cannot be imported from that API by a component or another API.

For the sake of simplicity, every identifier reference in an \apiN\
definition must refer either to a declaration in a used \apiN\
(i.e., an \apiN\ named in an import statement,
or the Fortress core APIs, which are implicitly imported),
or to a declaration in the \apiN\ itself.
In this way, \apisN\ differ from signatures in most module systems:
they are not parametric in their external dependencies.

\section{Component and API Identity}

Formally, we introduce two functions on components,
$\imports$ and $\exports$,
that return the imported and exported \apisN\ of the component, respectively.
For any component $\comp$, %% \in \components$,
$\imports(\comp) \intersect \exports(\comp) = \emptyset$.
This restriction is required throughout to ground the semantics of
operations on components, as discussed in \secref{basicops}.

Every component
has a unique name, used for the purposes of component linking.
This name includes a user-provided identifier.
In the case of a simple component, the identifier is determined by
a component name given at
the top of the source file from which it is compiled.
A build script may keep a tally on version numbers and append them to the
first line of a component,
incrementing its tally on each compilation.
The name of a compound component is specified as an argument
to the \shellcommand{link} operation (described in \secref{basicops})
that defines it.
Component equivalence is determined nominally
to allow mutually recursive linking of components.
By programmer convention, identifiers associated with components
that are not included in \library\
begin with the reverse of the URL of the development team.
%Two components with the same name must be identical, even if they are
%contained in different fortresses.
A fortress does not allow the installation of
distinct components with the same name.
Component names are used during
\shellcommand{link} and \shellcommand{upgrade}
operations to ensure that the restrictions on
upgrades to a component are respected,
as explained in \secref{basicops}.

Every component also includes
a vendor name,
the name of the fortress it is compiled on, and
a timestamp, denoting the time of compilation.
The time of compilation is measured by the compiling fortress,
and the name of the fortress is provided by the fortress automatically.
Every timestamp issued by a fortress must be unique.
The vendor name typically remains the same throughout a significant
portion of the life of a
user account, and is best provided as a user environment variable.

Every \apiN\ has a unique name that consists of a user-provided identifier.
As with components,
\apiN\ equivalence is determined nominally.
Every \apiN\ also includes a vendor name,
the name of the fortress it is compiled on,
and a timestamp.
\note{
Victor: Is it the \apiN\ name that includes the vendor, etc.?
It used to just say \apiN, and I changed it to \apiN\ name.

Jan: I changed it back, because we don't claim to distinguish APIs
  this way in the remainder of the text, and don't give a
  mechanism for specifying these details in an import.}
Component names must not conflict with \apiN\ names.
For convenience, a compiler can also
produce an \apiN\ directly from a project
with the same name as the component it is derived from.
Such an \apiN\ includes \emph{matching} declarations of the component.
All declarations in the component appear in the API.


A component must include, for every \apiN\ $A$ it exports,
matching definitions for all the declarations in $A$.
A matching definition of a declaration $d$
is a definition $d'$ with the same name as $d$
that includes definitions for all declarations
other than the methods or fields declared \KWD{abstract} in $d$.
The header and type of $d'$ must be the same as the header and type of $d$.
$d'$ may include additional definitions not declared in $d$.

Other than its identity,
the only relevant characteristic of an \apiN\ $\api$
is the set of \apisN\ that it uses,
denoted by $\uses(\api)$.
Because an \apiN\ $\api$ might expose types
defined in $\uses(\api)$,
we require that a component that exports $\api$
also exports all \apisN\ in $\uses(\api)$ that it does not import.
Formally, the following condition holds on the exported \apisN\
of a component $\comp$:
\[
\api \in \exports(\comp) \logicand \api' \in \uses(\api)
        \implies \api' \in \imports(\comp) \union \exports(\comp)
\]
