%%%%%%%%%%%%%%%%%%%%%%%%%%%%%%%%%%%%%%%%%%%%%%%%%%%%%%%%%%%%%%%%%%%%%%%%%%%%%%%%
%   Copyright 2009 Sun Microsystems, Inc.,
%   4150 Network Circle, Santa Clara, California 95054, U.S.A.
%   All rights reserved.
%
%   U.S. Government Rights - Commercial software.
%   Government users are subject to the Sun Microsystems, Inc. standard
%   license agreement and applicable provisions of the FAR and its supplements.
%
%   Use is subject to license terms.
%
%   This distribution may include materials developed by third parties.
%
%   Sun, Sun Microsystems, the Sun logo and Java are trademarks or registered
%   trademarks of Sun Microsystems, Inc. in the U.S. and other countries.
%%%%%%%%%%%%%%%%%%%%%%%%%%%%%%%%%%%%%%%%%%%%%%%%%%%%%%%%%%%%%%%%%%%%%%%%%%%%%%%%

\section{Values}
\seclabel{values}

\note{\TYP{LinearSequence} and \TYP{HeapSequence}
are not yet supported.}

\note{Comments:
1. Operators are bound in an environment just like functions.
    Methods are not values.  They are bound in a type.

2. I'm distinguishing fields from captured variables, even though
   there is no semantic distinction, except that captured variables
   could be shared (but they might not be, and if they are immutable,
   we can't tell anyway).}

\note{Jan: perform promised fixes to definition and usage of environment.
 In particular, environments permit sharing of mutable fields.}

A \emph{value} is the result of normal completion of the evaluation of
an expression.
(See \secref{eval-completion} for a discussion of completion of evaluation.)
A value is an object, a tuple or the void value \EXP{()}.
Every value has a \emph{type},
and every object has an environment (see \secref{environments}).
See \chapref{types} for a description of the types
corresponding to these different values.
An object may be a \emph{value object},
a \emph{reference object},
or a \emph{function}.
A nonfunction object has a finite set of \emph{fields};
the names and types of these fields are specified by the type of the object.
The type of a nonfunction object also specifies its methods.

A field consists of a name, a type, and either a value or a location:
in value objects, fields have values;
in reference objects, locations.
The name of a field is either an identifier or an index.
Only values of type \TYP{LinearSequence}
(defined in \secref{lib:LinearSequence}) or \TYP{HeapSequence}
(defined in \secref{lib:HeapSequence}) have fields named by indices.
Every field in a value object is \emph{immutable}.
Reference objects may have both \emph{mutable} and \emph{immutable} fields.
No two distinct values share any mutable field.

Values are constructed:
\begin{enumerate}
\item by top-level function declarations (see \secref{function-decls})
and singleton declarations (see \secref{object-decls}),
and
\item by evaluating an object expression (see \secref{object-expr}),
a function expression (see \secref{func-expr}),
a local function declaration (see \secref{local-fn-decls}),
a call to an object constructor
(declared by a constructor declaration; see \chapref{objects}),
a literal (see \secref{literals}),
a \KWD{spawn} expression (see \secref{spawn}),
a tuple expression (see \secref{tuple-expr}),
an aggregate expression (see \secref{aggregate-expr}),
or a comprehension (see \secref{comprehensions}).
\end{enumerate}
In the latter case,
the constructed value is the result of
the normal completion of such an evaluation.
