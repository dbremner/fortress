%%%%%%%%%%%%%%%%%%%%%%%%%%%%%%%%%%%%%%%%%%%%%%%%%%%%%%%%%%%%%%%%%%%%%%%%%%%%%%%%
%   Copyright 2009 Sun Microsystems, Inc.,
%   4150 Network Circle, Santa Clara, California 95054, U.S.A.
%   All rights reserved.
%
%   U.S. Government Rights - Commercial software.
%   Government users are subject to the Sun Microsystems, Inc. standard
%   license agreement and applicable provisions of the FAR and its supplements.
%
%   Use is subject to license terms.
%
%   This distribution may include materials developed by third parties.
%
%   Sun, Sun Microsystems, the Sun logo and Java are trademarks or registered
%   trademarks of Sun Microsystems, Inc. in the U.S. and other countries.
%%%%%%%%%%%%%%%%%%%%%%%%%%%%%%%%%%%%%%%%%%%%%%%%%%%%%%%%%%%%%%%%%%%%%%%%%%%%%%%%

\chapter{Type Inference}
\chaplabel{type-inference}

\note{This chapter will include the Fortress static type inference mechanism.}

\note{ \begin{itemize}
 \item
 There seems to be a circular dependency between inference and juxtaposition
 disambiguation -- you have to know if the subexpressions have arrow types
 to disambiguate, but you can't do that without using inference, which
 requires knowing how the variable of unknown type is used.
\item Type Inference (Dan's email titled ``Comments on a first reading of the spec'' on 06/05/07)
\item Do we want to forbid such cases where type inference infers
\TYP{BottomType}s for static parameters?
 \end{itemize}
}