%%%%%%%%%%%%%%%%%%%%%%%%%%%%%%%%%%%%%%%%%%%%%%%%%%%%%%%%%%%%%%%%%%%%%%%%%%%%%%%%
%   Copyright 2009 Sun Microsystems, Inc.,
%   4150 Network Circle, Santa Clara, California 95054, U.S.A.
%   All rights reserved.
%
%   U.S. Government Rights - Commercial software.
%   Government users are subject to the Sun Microsystems, Inc. standard
%   license agreement and applicable provisions of the FAR and its supplements.
%
%   Use is subject to license terms.
%
%   This distribution may include materials developed by third parties.
%
%   Sun, Sun Microsystems, the Sun logo and Java are trademarks or registered
%   trademarks of Sun Microsystems, Inc. in the U.S. and other countries.
%%%%%%%%%%%%%%%%%%%%%%%%%%%%%%%%%%%%%%%%%%%%%%%%%%%%%%%%%%%%%%%%%%%%%%%%%%%%%%%%

\subsection{Ignoring Values}
\seclabel{ignore}

For convenience, the function \VAR{ignore} (equivalent to the
\texttt{ignore} function in the Objective Caml programming language) is
included in \library:
%% ignore(x: Any) = ()
\begin{Fortress}
\(\VAR{ignore}(x\COLON \TYP{Any}) = ()\)
\end{Fortress}
The function discards the value of its argument and
returns \EXP{()}.  For example, the following:
\marginnote{
I manually added space between $f$ and $x$
-- Sukyoung}
%ignore(f x)
\begin{Fortress}
\(\VAR{ignore}(f\ x)\)
\end{Fortress}
is equivalent to:
\marginnote{
I manually added space between $f$ and $x$
-- Sukyoung}
%do f x; () end
\begin{Fortress}
\(\KWD{do} f\ x; () \KWD{end}\)
\end{Fortress}
