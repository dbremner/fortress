%%%%%%%%%%%%%%%%%%%%%%%%%%%%%%%%%%%%%%%%%%%%%%%%%%%%%%%%%%%%%%%%%%%%%%%%%%%%%%%%
%   Copyright 2009, Oracle and/or its affiliates.
%   All rights reserved.
%
%
%   Use is subject to license terms.
%
%   This distribution may include materials developed by third parties.
%
%%%%%%%%%%%%%%%%%%%%%%%%%%%%%%%%%%%%%%%%%%%%%%%%%%%%%%%%%%%%%%%%%%%%%%%%%%%%%%%%

\section{Assignments}
\seclabel{assignment}

\note{Qualified names are not yet supported.}

\begin{Grammar}
\emph{AssignExpr}
&::=& \emph{AssignLefts} \emph{AssignOp} \emph{Expr}\\

\emph{AssignLefts}
&::=& \texttt( \emph{AssignLeft}(\EXP{,} \emph{AssignLeft})$^*$ \texttt)\\
&$|$& \emph{AssignLeft}\\

\emph{AssignLeft}
&::=& \emph{SubscriptExpr}\\
&$|$& \emph{FieldSelection}\\
&$|$& \emph{QualifiedName}\\

\emph{SubscriptExpr} &::=&
\emph{Primary} \emph{LeftEncloser} \option{\emph{StaticArgs}} \option{\emph{ExprList}} \emph{RightEncloser} \\

\emph{FieldSelection} &::=& \emph{Primary}\EXP{.}\emph{Id}\\

\emph{AssignOp} &::=& \EXP{\ASSIGN} $|$ (\emph{Encloser} $|$ \emph{Op})\EXP{=}

\end{Grammar}

An assignment expression consists of a left-hand side
(\emph{AssignLefts}) indicating one or more variables, subscripted
expressions (as described in \secref{subscripted-assignment}), or
field accesses to be updated, an assignment token, and a
right-hand-side expression.  Multiple left-hand sides must be grouped
using tuple notation (comma-separated and parenthesized).  Variables
updated in an assignment expression must already be declared.

The assignment token `\EXP{\ASSIGN}' indicates ordinary assignment.
Ordinary assignment proceeds in two phases.  In the first, the evaluation phase, the
right-hand-side expression is evaluated in parallel with each of the
left-hand-side subexpressions, forming an implicit thread group.
Evaluating a left-hand variable does nothing.  Evaluating a left-hand
field reference evaluates the receiving object.  Evaluating a
left-hand subscripting operation evaluates the receiving object and
the index in parallel in a nested thread group.  After the outer
implicit thread group completes normally, the assignment phase begins.  Each component of the right-hand-side value is
assigned to the corresponding component of the left-hand side in
parallel, forming an implicit thread group.  Assigning a left-hand
variable simply changes the binding for the variable's location to
contain the new value.
Assigning a left-hand field reference calls
the corresponding setter method on the reiver object, passing the new value.
Assigning a left-hand subscript simply calls the subscripted
assignment operation on the corresponding object with the new value.

Any operator (other than `\EXP{\COLONOP}' or `\EXP{=}' or `\EXP{<}' or
`\EXP{>}') followed by `\EXP{=}' with no intervening whitespace
indicates compound (updating) assignment.  This adds an additional
phase to assignment between the two phases of ordinary assignment.
In the first phase, a
left-hand field reference invokes the getter for
the corresponding field after the receiving object has been evaluated.
A left-hand subscripting operation invokes the subscripting operator
on the receiving object once receiver and index have been evaluated.  A
left-hand variable simply returns the current value of the location
associated with that variable.  In the new second phase, the operator
indicated in the assignment is invoked.  The left-hand argument is the
value of the left-hand expression evaluated in the first phase; the
right-hand argument is the value evaluated for the right-hand side of
the assignment expression.  When the operator evaluation completes
normally, the assignment phase is run as in ordinary assignment.

The important point to understand is that compound assignment evaluates
its subexpressions exactly once, and that the parts of assignment
proceed implicitly in parallel where possible.
The value and type of an assignment expression is \EXP{()}.
The type of the right-hand side must be a subtype of the type of the left-hand
side.

Consider the following assignment:
%(a[i], b.x, c) += f(t,u,v)
\input{\home/basic/examples/Expr.Assign.a.tex}
Here in the first phase we evaluate \VAR{a} and \VAR{i} in parallel,
then when this completes we invoke the indexing method of
\VAR{a} to evaluate \EXP{a_i}.
In parallel, we evaluate \VAR{b} and then
call the getter method \EXP{b.x}.
In parallel, we look up the value of variable \VAR{c}.
Finally, we evaluate \EXP{f(t,u,v)} in parallel with all of
these left-hand sides.
In the second phase, we combine the results using the \EXP{+} operator,
to evaluate \EXP{(a_i,b.x,c) \mathrel{+} f(t,u,v)}.
This must return a tuple of three values \EXP{(p,q,r)}.
In the final phase, in parallel we
call the indexed assignment operator \EXP{a_i \ASSIGN p},
call the setter \EXP{b.x\ASSIGN{}q},
and perform the local assignment
\EXP{c\ASSIGN{}r}.

Here are some simpler and more commonplace examples of assignment:
\input{\home/basic/examples/Expr.Assign.b.tex}

\subsection{Definite Assignment}
\seclabel{definite-assignment}

\note{Definite assignment check is not supported yet.}

References to uninitialized variables are statically forbidden. As with the
Java Programming Language, this static constraint is ensured with a specific
conservative flow analysis. In essence, an initialization of a variable
must occur on every possible execution path to each reference to a
variable.
