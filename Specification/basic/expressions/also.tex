%%%%%%%%%%%%%%%%%%%%%%%%%%%%%%%%%%%%%%%%%%%%%%%%%%%%%%%%%%%%%%%%%%%%%%%%%%%%%%%%
%   Copyright 2009 Sun Microsystems, Inc.,
%   4150 Network Circle, Santa Clara, California 95054, U.S.A.
%   All rights reserved.
%
%   U.S. Government Rights - Commercial software.
%   Government users are subject to the Sun Microsystems, Inc. standard
%   license agreement and applicable provisions of the FAR and its supplements.
%
%   Use is subject to license terms.
%
%   This distribution may include materials developed by third parties.
%
%   Sun, Sun Microsystems, the Sun logo and Java are trademarks or registered
%   trademarks of Sun Microsystems, Inc. in the U.S. and other countries.
%%%%%%%%%%%%%%%%%%%%%%%%%%%%%%%%%%%%%%%%%%%%%%%%%%%%%%%%%%%%%%%%%%%%%%%%%%%%%%%%

\subsection{Parallel Do Expressions}
\seclabel{also-block}

\note{Reduction variables are not yet supported.}

A series of blocks may be run in parallel using the \KWD{also}
construct.  Any number of contiguous blocks may be joined together by
\KWD{also}.  Each block is run in a separate
implicit thread; these threads together form a group.  The expression
as a whole completes when the group is complete.
A thread can be placed in a particular region by using an \KWD{at} expression
as described in \secref{parallelism-fundamentals}.
When multiple
expression blocks are separated by \KWD{also}, each expression block must
have type \EXP{()}; the result and type of the parallel \KWD{do}
expression is also \EXP{()}.


For example:
\input{\home/basic/examples/Expr.Do.treeSum.tex}
Note the use of the reduction variable \VAR{accum}
(\secref{reduction-vars}) within the threads in the group.
