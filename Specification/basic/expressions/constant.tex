%%%%%%%%%%%%%%%%%%%%%%%%%%%%%%%%%%%%%%%%%%%%%%%%%%%%%%%%%%%%%%%%%%%%%%%%%%%%%%%%
%   Copyright 2009, Oracle and/or its affiliates.
%   All rights reserved.
%
%
%   Use is subject to license terms.
%
%   This distribution may include materials developed by third parties.
%
%%%%%%%%%%%%%%%%%%%%%%%%%%%%%%%%%%%%%%%%%%%%%%%%%%%%%%%%%%%%%%%%%%%%%%%%%%%%%%%%

\section{Static Expressions}
\seclabel{static-expr}

\note{Static expressions are not yet supported.}

\emph{Static expressions} denote \emph{static values}.  Conceptually,
a static expression is not evaluated while a program is running, and
thus its evaluation
cannot complete abruptly.  Given instantiations of all static
parameters (described in \chapref{trait-parameters}) in scope of a
static expression, the value of the static expression can be
determined statically.  Static expressions which use a restricted subset of operations can occur as
\emph{static arguments} which instantiate static parameters and
\KWD{where}-clause constraints (described in \secref{where-clauses}).
We define the set of static
expressions by first defining the types of static expressions, and
distinguishing static values from the closely related literal values.
We then describe the expressions that evaluate to the various kinds of
static values.


\subsection{Types of Static Expressions}

There are three groups of traits that describe literals
and static expressions:
\begin{enumerate}
\item The \emph{literal} traits, which describe boolean,
  character, string, and numerals.  For example, the literal \VAR{true} has
  trait \EXP{\TYP{BooleanLiteral}\llbracket\VAR{true}\rrbracket} and a
  character literal has trait \TYP{Character}.
See \secref{literals} for a discussion of Fortress literals.

\item The \emph{constant} traits, which describe values denoted by
expressions composed from literals and operators (described in
\secref{constant-values});
the type of a constant expression encodes the value of the expression.
For example, the type of a constant expression \EXP{3+5},
%IntegerConstant[\false, 3+5\]
\EXP{\TYP{IntegerConstant}\llbracket\VAR{false}, 3+5\rrbracket},
encodes the value of the expression; the operator \EXP{+} is
used in the static arguments.  When an operator is not supported in static
arguments, \library\ can still encode the value of the expression by
providing a specific constant trait such as
\EXP{\TYP{RationalValueTimesPi}\llbracket\VAR{false},1,1\rrbracket}
(described in \secref{pi}).



\item The \emph{static} traits, which describe values denoted by
expressions composed from literals, operators (described in
\secref{constant-values}), and \KWD{bool}, \KWD{nat},
and \KWD{int} parameters; here the type
does not encode the value of the expression, but the value of the
expression can nevertheless be known statically if specific
values are specified for the static parameters.  Also, in situations
where the type of an expression composed solely from literals and
operators nevertheless cannot be described by a constant trait, then a
static trait may be used to describe it instead.
For example, a static expression \EXP{2(3+m)} where \VAR{m} is a
\KWD{nat} parameter has trait \TYP{NaturalStatic}.
\end{enumerate}


The only operation on literals that produces a new literal (as opposed
to a constant) is concatenation by using the operator \EXP{\|}.
One may concatenate mixed string and
character literals, producing a string literal.
For example, ``\txt{foo}''\EXP{\|}''\txt{bar}'' yields ``\txt{foobar}''.
One may also concatenate two natural numerals
of the same radix, producing a new natural numeral of that radix.
For example, \EXP{\mathtt{deadc}_{16} \| \mathtt{0de}_{16}} yields \EXP{\mathtt{deadc0de}_{16}}.

Every literal trait extends an appropriate constant trait, and every
constant trait extends an appropriate static trait.  So every
literal is also a constant expression, and every constant expression
is a static expression.




\subsection{Static Expressions and Values}
\seclabel{constant-values}

Static parameters are static expressions.
A \KWD{nat} parameter denotes a value that has type \TYP{NaturalStatic}
(which extends \TYP{IntegerStatic}).
An \KWD{int} parameter denotes a value that has type \TYP{IntegerStatic}.
A \KWD{bool} parameter denotes a value that has type \TYP{BooleanStatic}.


Boolean static expressions may be combined using the operators
\EXP{\wedge}, \EXP{\vee}, \EXP{\oplus},
\EXP{\equiv}, \EXP{\leftrightarrow}, \OPR{NAND}, \OPR{NOR}, \EXP{=},
\EXP{\neq}, and \EXP{\rightarrow} (See \appref{operator-precedence} for a
discussion of Fortress operators.)
to produce other static expressions denoting boolean static expressions.
If both operands are boolean constant expressions, then the result is also
a boolean constant expression.

Character static expressions may be compared using the operators \EXP{<},
\EXP{\leq}, \EXP{\geq}, \EXP{>},
\EXP{=}, and \EXP{\neq} to produce boolean static expressions.  If both
operands are character constant expressions, then the result is a boolean
constant expression.

Numeric static expressions may be compared using the operators \EXP{<},
\EXP{\leq}, \EXP{\geq}, \EXP{>},
\EXP{=}, and \EXP{\neq} to produce boolean static expressions.  If both
operands are numeric constant expressions, then the result is a boolean
constant expression.

Numeric static expressions may be combined using the operators and
functions \EXP{+},
\EXP{-}, \EXP{\times}, \EXP{\cdot}, \EXP{/}, \EXP{!}, \OPR{MIN}, \OPR{MAX},
\EXP{\surd}, \VAR{floor}, \VAR{ceiling},
\VAR{hyperfloor}, \VAR{hyperceiling}, \EXP{\gcd}, \TYP{lcm}, \EXP{\sin},
\EXP{\cos}, \EXP{\tan}, \EXP{\arcsin}, \EXP{\arccos}, and \EXP{\arctan}
to produce new numeric static expressions.  If the result is indeed
a numeric static expression, and both operands are numeric constant
expressions, then the result is also a numeric constant expression
as long as the constant traits provided by \library\ can encode the value
of the expression.
