%%%%%%%%%%%%%%%%%%%%%%%%%%%%%%%%%%%%%%%%%%%%%%%%%%%%%%%%%%%%%%%%%%%%%%%%%%%%%%%%
%   Copyright 2009, Oracle and/or its affiliates.
%   All rights reserved.
%
%
%   Use is subject to license terms.
%
%   This distribution may include materials developed by third parties.
%
%%%%%%%%%%%%%%%%%%%%%%%%%%%%%%%%%%%%%%%%%%%%%%%%%%%%%%%%%%%%%%%%%%%%%%%%%%%%%%%%

\section{Fortress in a Nutshell}

\note{Keyword and varargs parameters, components linking,
distributions are not yet supported.}

Two basic concepts in Fortress are that of
\emph{object} and of \emph{trait}.
An \index{object}object consists of \emph{fields} and \emph{methods}.
The fields of an object are specified in its definition.
An object definition may also include method definitions.

\index{trait}
Traits are named
program constructs that declare sets of methods.
They were introduced in the Self programming language,
and their semantic properties
(and advantages over conventional class inheritance)
were analyzed
by Ducasse, Nierstrasz, Sch{\"a}rli, Wuyts and
  Black \cite{traits}.
In Fortress, a method declared by a trait
may be either \emph{abstract}
or \emph{concrete}:
abstract methods have only
\index{header!method}\index{method!header}\emph{headers};
concrete methods also have
\index{definition!method}\index{method!definition}\emph{definitions}.
A trait may \emph{extend} other traits:
it \emph{inherits}\index{inheritance}
the methods provided by the traits it extends.
\note{
The terminology I'm trying to adopt here is as follows:
traits \emph{declare} methods.
Method declarations are syntactic entities
contained in trait (and object) definitions
(which are also syntactic entities),
so a trait declares those methods
for which its definition contains method declarations,
as well as those methods it inherits.
-- Victor}
A trait provides the methods that it inherits
as well as those explicitly declared in its declaration.

Every object extends a set of traits (its ``supertraits'').
An object inherits the concrete methods of its supertraits
and must include a definition for every method
declared but not defined by its supertraits.

Note that the identifiers used in this example
are not restricted to ASCII character sequences.
Fortress allows the use of Unicode characters~\cite{Unicode}
in program identifiers, as well as subscripts and superscripts
(See \appref{ascii-unicode} for a discussion
of Unicode and support for entering programs in
ASCII.)
Fortress also includes a set of standard formatting rules
that follow the conventions of mathematical notation. For example,
most variable references in Fortress programs are italicized.
Moreover, multiplication
can be expressed by simple juxtaposition.
There is also support for operator overloading,
as well as a facility for  extending
the syntax with domain-specific languages.

Although Fortress is statically and nominally typed,
types are not specified for all fields,
nor for all method parameters and return values.
Instead, wherever possible,
\index{type inference}\emph{type inference}
is used
to reconstruct types.
In the examples throughout this specification,
we often omit the types when they are clear from context.
Additionally, types can be parametric with respect to other
types and values (most notably natural numbers).

These design decisions are motivated in part
by our goal
of making the scientist/programmer's life as easy as possible
without compromising good software engineering.
In particular,
they allow us to write Fortress programs
\index{mathematical notation, preserving}that preserve the look of standard
mathematical notation.

In addition to objects and traits,
Fortress allows the programmer to define top-level functions.
Functions are first-class values:
They can be passed to and returned from functions,
and assigned as values to fields and variables.
Functions and methods can be overloaded, with calls
to overloading methods resolved by multiple dynamic dispatch
similarly to the manner described in \cite{MillsteinChambers}.
\label{nutshellKwd}
Keyword parameters and variable size argument lists
are also supported.



Fortress programs are organized into
\index{component}\emph{components},
which export and import APIs
and can be linked together.
APIs describe the ``shape'' of a component,
specifying the types in traits, objects and functions
provided by a component.
All external references within a component
(i.e., references to traits, objects and functions
implemented by other components)
are to APIs imported by the component.
We discuss components and APIs in detail
in \chapref{components}.

To address the needs of modern high-performance computation,
Fortress also supports a rich set of operations for defining parallel execution
and distribution
of large data structures. This support is built into the core of the language.
For example, \KWD{for} loops in Fortress are parallel by default.
