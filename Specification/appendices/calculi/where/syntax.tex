%%%%%%%%%%%%%%%%%%%%%%%%%%%%%%%%%%%%%%%%%%%%%%%%%%%%%%%%%%%%%%%%%%%%%%%%%%%%%%%%
%   Copyright 2009, Oracle and/or its affiliates.
%   All rights reserved.
%
%
%   Use is subject to license terms.
%
%   This distribution may include materials developed by third parties.
%
%%%%%%%%%%%%%%%%%%%%%%%%%%%%%%%%%%%%%%%%%%%%%%%%%%%%%%%%%%%%%%%%%%%%%%%%%%%%%%%%

\subsection{Syntax}\label{where-syntax}

The syntax for \wherecore\ is provided in Figure \ref{fig:where-syntax}.
For simplicity, we use the following notational conventions:
\begin{itemize}
\renewcommand{\subtype}{\ensuremath{<:}}
\item
We abbreviate a sequence of relations
$\tvone_1 \extends \set{\seq{\tappone_1}}$ $\more$ $\tvone_n \extends
  \set{\seq{\tappone_n}}$ to \seq{\tvone \extends \set{\seq{\tappone}}} and
$\ty_1\subtype\tapptwo_{11}$ \mbox{$\ty_1\subtype\tapptwo_{12}$} \more\
$\ty_1\subtype\tapptwo_{1l}$ $\ty_2\subtype\tapptwo_{21}$ \more\
$\ty_n\subtype\tapptwo_{nm}$ to $\tys\subtype\seq{\seq{\tapptwo}}$.
\renewcommand{\subtype}{\ensuremath{\:<:\:}}

\item
Substitutions of \vname\ with \val\ and \tvone\ with \ty\ are denoted
as \subst{\val}{\vname} and \subst{\ty}{\tvone}, respectively.
A sequence of substitutions represents the composition of those
substitutions where the right-most substitution is applied first.
For example, $S_n \more\ S_1 \ty$ represents
$S_n (\more\ S_2(S_1 \ty)\more)$.
\end{itemize}

\begin{figure}[t]
\[
\begin{array}{llll}
\tvone, \tvtwo&& &  \mbox{type variables}\\
\fname&& &  \mbox{method name}\\
\vname&& &  \mbox{field name}\\
\tname&& &  \mbox{trait name}\\
\oname&& &  \mbox{object name}\\
\cname&& &  \mbox{trait or object name}\\
%%
\ty, \prm\ty, \prm{\prm\tau} &::=& \tvone &\mbox{type}\\
         &|& \tynontv\\
\tynontv&::=& \tappfour& \mbox{type that is not a type variable}\\
%% &\mbox{super trait}\\
         &|& \oapp\\
%% & \mbox{object type}\\
\tappfour, M, L &::=& \tapp
& \mbox{super trait}\\
         &|& \obj \\
\tappone, \tapptwo, \tappthree &::=&\tvone &\mbox{bound on a type variable}\\
         &|& \tappfour \\
\exp   &::=& \vname  & \mbox{expression}\\
         &|& \self   \\% & \mbox{\self\ reference}\\
         &|& \objexp \\% & \mbox{object creation}\\
         &|& \fldacc \\% & \mbox{field access}\\
         &|& \mthinvk \\% & \mbox{method invocation}\\
         &|& \typecase{\exp}{\seq{\ty \Rightarrow \exp}}{\exp}\\
%& \mbox{type-dependent operation}
\fd &::=& \fdsyntax
\quad\quad\quad\quad
\quad\quad\quad\quad
\quad\quad\quad\quad\quad\quad
& \mbox{method definition}\\
\td &::=& \lefteqn{\tdsyntax
\quad\quad\quad\quad
\mbox{trait definition}}\\
\od &::=& \lefteqn{\odsyntax\quad \mbox{object definition}}\\
\d &::=& \td &\mbox{definition}\\
   &|  & \od \\
\pgm &::=& \pd & \mbox{program}\\
\end{array}
\]
\caption{Syntax of \wherecore}
\label{fig:where-syntax}
\end{figure}

Most of the syntax is a straightforward extension of \basiccore.
 %% in Section~\ref{basic-syntax}.
An object or trait definition may include \wc.
Every method invocation is annotated with three kinds of static
types by type inference: the static types of the receiver, the
arguments, and the result.
These type annotations appear in \wherecore\ in a form of \tyAsc\ \ty.
If the annotated types are not enough (to find
``witnesses'' for the where-clauses variables), type checking rejects the
program and requires more type information from the programmer.
