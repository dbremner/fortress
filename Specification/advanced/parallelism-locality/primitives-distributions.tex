%%%%%%%%%%%%%%%%%%%%%%%%%%%%%%%%%%%%%%%%%%%%%%%%%%%%%%%%%%%%%%%%%%%%%%%%%%%%%%%%
%   Copyright 2009 Sun Microsystems, Inc.,
%   4150 Network Circle, Santa Clara, California 95054, U.S.A.
%   All rights reserved.
%
%   U.S. Government Rights - Commercial software.
%   Government users are subject to the Sun Microsystems, Inc. standard
%   license agreement and applicable provisions of the FAR and its supplements.
%
%   Use is subject to license terms.
%
%   This distribution may include materials developed by third parties.
%
%   Sun, Sun Microsystems, the Sun logo and Java are trademarks or registered
%   trademarks of Sun Microsystems, Inc. in the U.S. and other countries.
%%%%%%%%%%%%%%%%%%%%%%%%%%%%%%%%%%%%%%%%%%%%%%%%%%%%%%%%%%%%%%%%%%%%%%%%%%%%%%%%

\section{Placing Threads}
\seclabel{parallelism-fundamentals}

A thread can be placed in a particular region by using an \KWD{at}
expression:
\input{\home/advanced/examples/Parallel.At.a.tex}

In this example, two implicit threads are created; the first computes
\EXP{a_i} locally, the second computes \EXP{a_j} in the region where
the $j^{th}$ element of $a$ resides, specified by
\EXP{a.\VAR{region}(j)}.  The expression after \KWD{at} must return a
value of type \TYP{Region}, and the block immediately following
\KWD{do} is run in that region; the result of the block is the result
of the \KWD{at} expression as a whole.  Often it is more graceful to
use the \EXP{\KWD{also}\;\KWD{do}} construct
(described in\secref{also-block})
in these cases:
\input{\home/advanced/examples/Parallel.At.b.tex}

We can also use \KWD{at} with a \KWD{spawn} expression:
\input{\home/advanced/examples/Parallel.At.c.tex}

Finally, note that it is possible to use an \KWD{at} expression within
a block:
\input{\home/advanced/examples/Parallel.At.d.tex}
We can think of this as the degenerate case of \EXP{\KWD{also}\;
  \KWD{do}}: a thread group is created with a single implicit thread
running the contents of the \KWD{at} expression in the given region; when
this thread completes control returns to the original location.

Note that the regions given in an \KWD{at} expression are non-binding:
the Fortress implementation may choose to run the computations
elsewhere---for example, thread migration might not be possible within
an \KWD{atomic} expression, or load balancing might cause code to be
executed in a different region.  In general, however, implementations
should attempt to respect thread placement annotations when they are
given.
