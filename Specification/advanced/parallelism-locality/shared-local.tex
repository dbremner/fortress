%%%%%%%%%%%%%%%%%%%%%%%%%%%%%%%%%%%%%%%%%%%%%%%%%%%%%%%%%%%%%%%%%%%%%%%%%%%%%%%%
%   Copyright 2009, Oracle and/or its affiliates.
%   All rights reserved.
%
%
%   Use is subject to license terms.
%
%   This distribution may include materials developed by third parties.
%
%%%%%%%%%%%%%%%%%%%%%%%%%%%%%%%%%%%%%%%%%%%%%%%%%%%%%%%%%%%%%%%%%%%%%%%%%%%%%%%%

\section{Shared and Local Data}
\seclabel{threadlocal}

\note{The method \VAR{copy} is not yet supported.}

Every object in a Fortress program is considered to be either
\emph{shared} or \emph{local} (collectively referred to as the
\emph{sharedness} of the object).  A local object must be transitively
reachable (through zero or more object references) from the variables
of at most one running thread.  A local object may be accessed more
cheaply than a shared object, particularly in the case of atomic reads
and writes.  Sharedness is ordinarily managed implicitly by the
Fortress implementation.
Control over sharedness is intended to be a
performance optimization; however, functions such as \VAR{isShared} and
\VAR{localize} provided by the Fortress libraries
can affect program semantics, and must be used with care.

The sharedness of an object must be contrasted with its region.  The
region of an object describes where that object is located on the
machine.  The sharedness of an object describes whether the object is
visible to one thread or to many.  A local object need not actually
reside in a region near the thread to which it is visible (though
ordinarily it will).

The following rules govern sharedness:
\begin{itemize}
\item Reference objects are initially local when they are constructed.
\item The sharedness of an object may change as the program executes.\footnote{Note, for example, that the present Fortress implementation immediately makes every object shared after construction, so that \VAR{isShared()} will always return \VAR{true}.}
\item If an object is currently transitively reachable from more than
  one running thread, it must be shared.
\item When a reference to a local object is stored into a field of a
  shared object, the local object must be \emph{published}: Its
  sharedness is changed to shared, and all of the data to which it
  refers is also published.
\item The value of a local variable referenced by a thread must be
  published before that thread may be run in parallel with the thread
  which created it.  Values assigned to the variable while the
  threads run in parallel must also be published.
\item A field with value type is assigned by copying, and thus has the
   sharedness of the containing object or closure.
\end{itemize}

Publishing can be expensive, particularly if the structure being
broadcast is large and heavily nested; this can cause an apparently
short \KWD{atomic} expression (a single write, say) to run arbitrarily
long.  To avoid this, the library programmer can request that an
object be published by calling the semantically transparent function
\VAR{shared} provided by the Fortress libraries:
\input{\home/advanced/examples/Parallel.Shared.a.tex}
A local copy of an object can be obtained by calling \VAR{copy}, a
method on trait \TYP{Any}:
\note{Example is commented out because it is not yet supported nor run by the interpreter.}
% %localVar := sharedVar.copy()
% \begin{Fortress}
% \(\VAR{localVar} \ASSIGN \VAR{sharedVar}.\VAR{copy}()\)
% \end{Fortress}

Two additional functions are provided which permit different choices of
program behavior based on the sharedness of objects:
\begin{itemize}
\item The function \EXP{\VAR{isShared}(o)} returns \VAR{true} when
\VAR{o} is shared, and \VAR{false} when it is local.  This permits
the program to take different actions based on sharedness.
\item The function \EXP{\VAR{localize}(o)} is provided that attempts to make a local version of object \VAR{o}, by copying if necessary
equivalent to the following expression:
\note{Example is commented out because it is not yet supported nor run by the interpreter.}
%% \input{\home/advanced/examples/Parallel.Shared.b.tex}
%% %% %if o.isShared then o.copy() else o end
%% %% \begin{Fortress}
%% %% \(\KWD{if} o.\VAR{isShared} \KWD{then} o.\VAR{copy}() \KWD{else} o \KWD{end}\)
%% %% \end{Fortress}
\end{itemize}
These functions must be used with extreme caution.  For example,
\VAR{localize} should be used only when there is a unique
reference to the object being localized.  The \VAR{localize} function
can have unexpected behavior if there is a reference to \VAR{o} from
another local object \VAR{p}.  Updates to \VAR{o} will be visible
through \VAR{p}; subsequent publication of \VAR{p} will publish
\VAR{o}.  By contrast, if \VAR{o} was already shared, and referred to
by another shared object, the newly-localized copy will be entirely
distinct; changes to the copy will not be visible through \VAR{p}, and
publishing \VAR{p} will not affect the locality of the copy.
