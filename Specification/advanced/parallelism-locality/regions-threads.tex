%%%%%%%%%%%%%%%%%%%%%%%%%%%%%%%%%%%%%%%%%%%%%%%%%%%%%%%%%%%%%%%%%%%%%%%%%%%%%%%%
%   Copyright 2009, Oracle and/or its affiliates.
%   All rights reserved.
%
%
%   Use is subject to license terms.
%
%   This distribution may include materials developed by third parties.
%
%%%%%%%%%%%%%%%%%%%%%%%%%%%%%%%%%%%%%%%%%%%%%%%%%%%%%%%%%%%%%%%%%%%%%%%%%%%%%%%%

\section{Regions}
\seclabel{regions}

Every thread (either explicit or implicit) and every object in Fortress,
and every element of a Fortress array (the physical storage for that array
element), has an associated \emph{region}.
The Fortress libraries provide a function \VAR{region}
which returns the region in which a given object resides.
Regions abstractly describe the structure of
the machine on which a Fortress program is running.  They are
organized hierarchically to form a tree, the \emph{region hierarchy},
reflecting in an abstract way the degree of locality which those
regions share.  The distinguished region \TYP{Global} represents the root
of the region hierarchy.\footnote{Note: the initial implementation of the Fortress language assumes a single machine with shared memory and exposes only the \TYP{Global} region.}
The different levels of this tree reflect underlying
machine structure, such as execution engines within a CPU, memory
shared by a group of processors, or resources distributed across the
entire machine.  The function \EXP{\VAR{here}()} returns the region in which execution is currently occurring.  Objects which reside in regions near the leaves of
the tree are local entities; those which reside at higher levels of
the region tree are logically spread out.
The method call
\EXP{r.\VAR{isLocalTo}(s)} returns \VAR{true} if \VAR{r} is contained
within the region tree rooted at \VAR{s}.

It is important to understand that regions and the structures
(such as distributions, \secref{distributions})
 built on top of them exist
purely for performance purposes.  The placement of a thread or an
object does not have any semantic effect on the meaning of a program;
it is simply an aid to enable the implementation to make informed
decisions about data placement.

It might not be possible for an object or a thread to reside in a given
region.  Threads of execution reside at the \emph{execution level} of
the region hierarchy, generally the bottommost level in the region
tree.  Each thread is generally associated with some region at the
execution level, indicating where it will preferentially be run.
The programmer can affect the choice of region by using an \KWD{at}
expression (\secref{parallelism-fundamentals}) when the thread is
created.  A thread may have an associated region which is not at
the execution level of the region hierarchy, either because a higher
region was requested with an \KWD{at} expression or because scheduling
decisions permit the thread to run in several possible execution
regions.  The region to which a thread is assigned may also change
over time due to scheduling decisions.
For the object associated with a spawned thread,
the \VAR{region} function provided by the Fortress libraries
returns the region of the associated thread.


The \emph{memory level} of the region hierarchy is where individual
reference objects reside; on a machine with nodes composed of multiple
processor cores sharing a single memory, this generally will not be a
leaf of the region hierarchy.
Imagine a constructor for a reference
object is called by a thread residing in region \VAR{r}, yielding an
object \VAR{o}.  Except in very rare circumstances (for example when a
local node is out of memory) either
%r.isLocalTo(region(o))
\EXP{r.\VAR{isLocalTo}(\VAR{region}(o))} or
%region(o).isLocalTo(r)
\EXP{\VAR{region}(o).\VAR{isLocalTo}(r)} ought to hold: data is
allocated locally to the thread which runs the constructor.  For a value object
\VAR{v} being manipulated by a thread residing in region \VAR{r} either
%region(v).isLocalTo(r)
\EXP{\VAR{region}(v).\VAR{isLocalTo}(r)} or
%r.isLocalTo(region(v))
\EXP{r.\VAR{isLocalTo}(\VAR{region}(v))}
(value objects always appear to be local).

Note that \VAR{region} is a top-level function provided by the Fortress libraries
and can be overridden by any functional method.
The chief example of this is arrays, which are
generally composed from many reference objects; the \VAR{region} function
can be
overridden to return the location of the array as a
whole---the region which contains all of its constituent reference objects.
